In \cref{sec:tri} haben wir gelernt, wie man Stücke stimmt, bei denen jede Note
Teil eines „vertikalen“ Dur- oder Molldreiklangs ist. Akkorde haben jedoch sehr
oft andere Bestandteile als nur Grundton, Terz und Quinte. In diesem Abschnitt
besprechen wir die wichtigsten Variationen von Dur- und Moll-Akkorden und wie
sie gestimmt werden können.

\subsection{Quartvorhalt und erweiterter Nonakkord}
\label{sec:49}

Ein Dreiklang mit \emph{Quartvorhalt} besteht aus dem Grundton, der Quarte und
der Quinte, wobei sich die Quarte in der Regel auf dem nächsten Schlag in die
(große oder kleine) Terz auflöst, vgl. \cite[{}8.2]{Skript}. Dieser Akkord
entspricht in pythagoreischer Stimmung bereits dem sehr übersichtlichen
Verhältnis $6:8:9$ und benötigt keine Korrektur durch ein syntonisches Komma
(was auch daran liegt, dass hier nur Quarte und Quinte vorkommen, Quinten
optimal gestimmt und Quarten dazu komplementär sind).  Wir weisen noch auf zwei
horizontale Intervalle hin:
\begin{enumerate}
\item Die Quarte des Akkordes muss \emph{konsonant eingeführt werden},
  d.\,h. dieselbe Note tauchte im vorherigen regulär Akkord auf und ist durch
  den Harmoniewechsel zur Quarte geworden. Stellt man sich z.\,B. die
  Akkordverbindung T\,–\,D\textsuperscript{$4$–$3$} vor, bei der der Grundton
  der Tonika zur Quarte der Dominante wird, so ist keine Anpassung notwendig, da
  beide Töne, Grundton der Tonika und Quarte der Dominante, auf der gleichen
  Kommahöhe liegen.\todo{Das Wort „Kommahöhe“ scheint mir sehr praktikabel zu
    sein. Man sollte es einmal in §3 einführen. FK}
\item Die Quarte löst sich abwärts mit einem diatonischen Halbton ($15:16$) in
  die große Terz oder mit einem kleinen \acr{GTS} ($9:10$) in die kleine Terz
  auf.
\end{enumerate}

Der \emph{erweiterte Nonakkord} ist ein Vierklang, der sich aus aus einem (Dur-
oder Moll-)Akkord und einer zusätzlichen großen None zusammensetzt.  Die None
liegt eine reine Quinte über der Quinte des Dreiklangs und hat daher die gleiche
Kommahöhe wie der Grundton und die Quinte. Dieser
Akkord entspricht dem Verhältnis $4:5:6:9$ in Dur und $20:24:30:45$ in Moll. Wir
stellen fest, dass die kleine Sekunde zwischen der großen None und der Mollterz
(plus eine Oktave) ein diatonischer Halbtonschritt ($15:16$) ist.

In \cref{fig:chordLines} haben wir die obigen Akkorde als ins Eulerschen Tonnetz
eingezeichnet. Dabei markieren wir nicht nur die Eckpunkte, sondern auch alle
direkten Kanten zwischen ihnen, da jede Kante für eine Beziehung der Form $2:3$,
$4:5$ oder $5:6$ innerhalb des jeweiligen Akkordes steht.

\subsection{Sixte ajoutée}
\label{sec:sixte}

Sehr oft kommt der Subdominantakkord (sowohl in Dur als auch in Moll) mit der
großen Sexte über dem Grundton des Akkords zusammen vor
(z.\,B. f\,:\,a\,:\,c\,:\,d in C-Dur oder f\,:\,\flat a\,:\,c\,:\,d in c-Moll),
vgl. \cite[{}9.3]{Skript}.  Im Fall von Dur ist dieser Akkord eine Verschmelzung
von zwei Dreiklängen, die sich eine Terz teilen (im Beispiel von C-Dur: F-Dur
und d-Moll). Dementsprechend verwenden wir die Ecken der benachbarten Dreiecke
$\vartriangle\!\!\!\triangledown$, um die Stimmung der Einzeltöne abzuleiten
(siehe \cref{fig:chordLines}), und gelangen so zum Verhältnis
$12:15:18:20$. Hier verbirgt sich allerdings wieder eine Kommafalle: Wenn auf
den Sixte ajoutée die Dominante folgt und die Sexte der Subdominante zur Quinte
der Dominante wird (dies sind die gleichen Noten!), so muss die große Sexte beim
Harmoniewechsel um ein syntonisches Komma nach oben angepasst werden (siehe
\cref{fig:496}).  Dies spräche zwar dafür, die Sexte von Anfang an ein Komma
höher zu intonieren, allerdings scheint uns die Verschmelzung der zwei
Dreiklänge und die dadurch entstehende charakteristische Ambiguität des
Vierklangs wichtiger.

In Moll fällt diese Abwägung anders aus, da der Sixte ajoutée in Moll keine
Verschmelzung zweier Dreiecke wie oben beschrieben ist.  Daher wird in diesem
Fall die Sexte verwendet, die auf der gleichen Kommahöhe wie Grundton und Quinte
liegt. Der sich daraus ergebende Akkord hat das Verhältnis $80:96:120:135$.  Die
verminderte Quinte (Tritonus) zwischen großer Sexte und kleiner Terz (plus
Oktave) beträgt $45:64\approx 609{,}78\,\on{ct}$.  Nicht nur wird durch diese
Syntonisierungswahl die oben genannte Kommafalle umgangen; es wird auch ein
melodisches Problem gelöst, vgl. \cref{sec:steps}: Nicht selten wird die Sexte
der Subdominante (in c-Moll: d) von der Terz der Tonika (in c-Moll: es) aus
erreicht.  Nutzte man für die Mollterz \flatp $3$ und für die Sexte \naturalp
$6$, so hätte dieser Halbtonschritt die Größe $25:27\approx 133{,}28\,\om{ct}$,
was viel zu groß ist.\footnote{Dies wäre der auf Wikipedia erwähnte „large
  diatonic semitone“, der sonst nicht auftaucht.}

% Beide Akkorde sind in \cref{fig:chordLines} wieder als Graphen im Tonnetz
% eingetragen.

\begin{figure}
  \centering
  \LY{b-496}
  \includegraphics{ly/b-496}
  \caption{Vier Beispiele, darunter der Quartvorhalt, der Nonakkord und der
    Sixte ajoutée. Der Unterschied zwischen dem zweiten und dem dritten Beispiel
    ist nur eine einzige Note im Tenor: In der ersten Version wird die von uns
    vorgeschlagene große Sexte verwendet, in der zweiten Version die ein
    syntonisches Komma tiefere, die zu dem zu großen Halbtonschritt $25:27$
    im Tenor führt.}\label{fig:496}
\end{figure}

\begin{figure}
  \centering
  \begin{tikzpicture}[scale=.7]
  %% Suspended fourth major
  %  Euler lattice
  \draw[black!50] (0, 0) -- (4, 0);
  \draw[black!50] (1, {sqrt(3)}) -- (3, {sqrt(3)});
  \draw[black!50] (1,-{sqrt(3)}) -- (3,-{sqrt(3)});
  \draw[black!50] (0,0) -- (1, {sqrt(3)}) -- (2,0) -- (3, {sqrt(3)}) -- (4,0);
  \draw[black!50] (0,0) -- (1,-{sqrt(3)}) -- (2,0) -- (3,-{sqrt(3)}) -- (4,0);
  %  Just third and fifth relations
  \draw[black,line width=1mm] (0,0) -- (4,0);
  %  Circles for chord notes labels
  \draw[fill=white] (0,0)          circle (.3);
  \draw[fill=white] (2,0)          circle (.3);
  \draw[fill=white] (4,0)          circle (.3);
  \draw[fill=white] (3,-{sqrt(3)}) circle (.3);
  %  Chord notes labels
  \node at (0,0)              {\scriptsize $4$};
  \node at (2,0)              {\scriptsize $1$};
  \node at (4,0)              {\scriptsize $5$};
  \node at (3,-{sqrt(3)})     {\scriptsize $3$};
  %  Semitone resolutions
  \draw[white,line width=1mm] (.3,{-sqrt(3)*1/10}) -- (2.7,{-sqrt(3)*9/10});
  \draw[-stealth]             (.3,{-sqrt(3)*1/10}) -- (2.7,{-sqrt(3)*9/10});
  %  Label of chord
  \node[rotate=0] at (-1.5,0) {\small $\strut$Dur}; % For rotated text set rotate option to 90 instead of 0
  \node at (2,2.5)            {\small $\strut$Quartvorhalt};
  
  
  %% Extended ninth major
  %  Euler lattice
  \draw[shift={(5.5,0)},black!50] (0, 0) -- (4, 0);
  \draw[shift={(5.5,0)},black!50] (1, {sqrt(3)}) -- (3, {sqrt(3)});
  \draw[shift={(5.5,0)},black!50] (1,-{sqrt(3)}) -- (3,-{sqrt(3)});
  \draw[shift={(5.5,0)},black!50] (0,0) -- (1, {sqrt(3)}) -- (2,0) -- (3, {sqrt(3)}) -- (4,0);
  \draw[shift={(5.5,0)},black!50] (0,0) -- (1,-{sqrt(3)}) -- (2,0) -- (3,-{sqrt(3)}) -- (4,0);
  %  Just third and fifth relations
  \draw[shift={(5.5,0)},line width=1mm] (0,0) -- (4,0);
  \draw[shift={(5.5,0)},line width=1mm] (0,0) -- (1,-{sqrt(3)}) -- (2,0);
  %  Circles for chord notes labels
  \draw[shift={(5.5,0)},fill=white] (0,0)          circle (.3);
  \draw[shift={(5.5,0)},fill=white] (2,0)          circle (.3);
  \draw[shift={(5.5,0)},fill=white] (1,-{sqrt(3)}) circle (.3);
  \draw[shift={(5.5,0)},fill=white] (4,0)          circle (.3);
  %  Chord notes labels
  \node[shift={(3.85,0)}] at (0,0)          {\scriptsize $1$};
  \node[shift={(3.85,0)}] at (1,-{sqrt(3)}) {\scriptsize $3$};
  \node[shift={(3.85,0)}] at (2,0)          {\scriptsize $5$};
  \node[shift={(3.85,0)}] at (4,0)          {\scriptsize $9$};
  %  Label of chord
  \node[shift={(3.85,0)}] at (2,2.5)        {\small $\strut$Hinzugefügte None};
  
  
  %% Sixte ajoutée major
  %  Euler lattice
  \draw[shift={(11,0)},black!50] (0, 0) -- (4, 0);
  \draw[shift={(11,0)},black!50] (1, {sqrt(3)}) -- (3, {sqrt(3)});
  \draw[shift={(11,0)},black!50] (1,-{sqrt(3)}) -- (3,-{sqrt(3)});
  \draw[shift={(11,0)},black!50] (0,0) -- (1, {sqrt(3)}) -- (2,0) -- (3, {sqrt(3)}) -- (4,0);
  \draw[shift={(11,0)},black!50] (0,0) -- (1,-{sqrt(3)}) -- (2,0) -- (3,-{sqrt(3)}) -- (4,0);
  %  Just third and fifth relations
  \draw[shift={(11,0)},line width=1mm] (2,0) -- (4,0);
  \draw[shift={(11,0)},line width=1mm] (1,-{sqrt(3)}) -- (3,-{sqrt(3)});
  \draw[shift={(11,0)},line width=1mm] (1,-{sqrt(3)}) -- (2,0) -- (3,-{sqrt(3)}) -- (4,0);
  %  Circles for chord notes labels
  \draw[shift={(11,0)},fill=white] (1,-{sqrt(3)}) circle (.3);
  \draw[shift={(11,0)},fill=white] (2,0)          circle (.3);
  \draw[shift={(11,0)},fill=white] (3,-{sqrt(3)}) circle (.3);
  \draw[shift={(11,0)},fill=white] (4,0)          circle (.3);
  %  Chord notes labels
  \node[shift={(7.7,0)}] at (2,0)          {\scriptsize $1$};
  \node[shift={(7.7,0)}] at (3,-{sqrt(3)}) {\scriptsize $3$};
  \node[shift={(7.7,0)}] at (4,0)          {\scriptsize $5$};
  \node[shift={(7.7,0)}] at (1,-{sqrt(3)}) {\scriptsize $6$};
  %  Label of chord
  \node[shift={(7.7,0)}] at (2,2.5)        {\small $\strut$Sixte ajoutée};
  
  %% Suspended fourth minor
  %  Euler lattice
  \draw[shift={(0,-4.5)},black!50] (0, 0) -- (4, 0);
  \draw[shift={(0,-4.5)},black!50] (1, {sqrt(3)}) -- (3, {sqrt(3)});
  \draw[shift={(0,-4.5)},black!50] (1,-{sqrt(3)}) -- (3,-{sqrt(3)});
  \draw[shift={(0,-4.5)},black!50] (0,0) -- (1, {sqrt(3)}) -- (2,0) -- (3, {sqrt(3)}) -- (4,0);
  \draw[shift={(0,-4.5)},black!50] (0,0) -- (1,-{sqrt(3)}) -- (2,0) -- (3,-{sqrt(3)}) -- (4,0);
  %  Just third and fifth relations
  \draw[shift={(0,-4.5)},black,line width=1mm] (0,0) -- (4,0);
  %  Circles for chord notes labels
  \draw[shift={(0,-4.5)},fill=white] (0,0)          circle (.3);
  \draw[shift={(0,-4.5)},fill=white] (2,0)          circle (.3);
  \draw[shift={(0,-4.5)},fill=white] (4,0)          circle (.3);
  \draw[shift={(0,-4.5)},fill=white] (3,+{sqrt(3)}) circle (.3);
  %  Semitone resolutions
  \draw[shift={(0,-4.5)},white,line width=1mm] (.3,{+sqrt(3)*1/10}) -- (2.7,{+sqrt(3)*9/10});
  \draw[shift={(0,-4.5)},-stealth]             (.3,{+sqrt(3)*1/10}) -- (2.7,{+sqrt(3)*9/10});
  %  Chord notes labels
  \node[shift={(0,-3.15)}] at (0,0)          {\scriptsize $4$};
  \node[shift={(0,-3.15)}] at (2,0)          {\scriptsize $1$};
  \node[shift={(0,-3.15)}] at (4,0)          {\scriptsize $5$};
  \node[shift={(0,-3.15)}] at (3,+{sqrt(3)}) {\scriptsize \flat $3$};
  %  Label of chord
  \node[rotate=0]          at (-1.5,-4.5)    {\small $\strut$Moll}; % For rotated text set rotate option to 90 instead of 0
  
  
  %% Extended ninth minor
  %  Euler lattice
  \draw[shift={(5.5,-4.5)},black!50] (0, 0) -- (4,0);
  \draw[shift={(5.5,-4.5)},black!50] (1, {sqrt(3)}) -- (3, {sqrt(3)});
  \draw[shift={(5.5,-4.5)},black!50] (1,-{sqrt(3)}) -- (3,-{sqrt(3)});
  \draw[shift={(5.5,-4.5)},black!50] (0,0) -- (1, {sqrt(3)}) -- (2,0) -- (3, {sqrt(3)}) -- (4,0);
  \draw[shift={(5.5,-4.5)},black!50] (0,0) -- (1,-{sqrt(3)}) -- (2,0) -- (3,-{sqrt(3)}) -- (4,0);
  %  Just third and fifth relations
  \draw[shift={(5.5,-4.5)},line width=1mm] (0,0) -- (4,0);
  \draw[shift={(5.5,-4.5)},line width=1mm] (0,0) -- (1,+{sqrt(3)}) -- (2,0);
  %  Circles for chord notes labels
  \draw[shift={(5.5,-4.5)},fill=white] (0,0)          circle (.3);
  \draw[shift={(5.5,-4.5)},fill=white] (2,0)          circle (.3);
  \draw[shift={(5.5,-4.5)},fill=white] (1,+{sqrt(3)}) circle (.3);
  \draw[shift={(5.5,-4.5)},fill=white] (4,0)          circle (.3);
  %  Chord notes labels
  \node[shift={(3.85,-3.15)}] at (0,0)          {\scriptsize $1$};
  \node[shift={(3.85,-3.15)}] at (1,+{sqrt(3)}) {\scriptsize \flat $3$};
  \node[shift={(3.85,-3.15)}] at (2,0)          {\scriptsize $5$};
  \node[shift={(3.85,-3.15)}] at (4,0)          {\scriptsize $9$};
  
  
  %% Sixte ajoutée minor
  %  Euler lattice
  \draw[shift={(11,-4.5)},black!50] (0, 0) -- (6, 0);
  \draw[shift={(11,-4.5)},black!50] (1, {sqrt(3)}) -- (5, {sqrt(3)});
  \draw[shift={(11,-4.5)},black!50] (1,-{sqrt(3)}) -- (5,-{sqrt(3)});
  \draw[shift={(11,-4.5)},black!50] (0,0) -- (1, {sqrt(3)}) -- (2,0) -- (3, {sqrt(3)}) -- (4,0) -- (5, {sqrt(3)}) -- (6,0);
  \draw[shift={(11,-4.5)},black!50] (0,0) -- (1,-{sqrt(3)}) -- (2,0) -- (3,-{sqrt(3)}) -- (4,0) -- (5,-{sqrt(3)}) -- (6,0);
  %  Just third and fifth relations
  \draw[shift={(11,-4.5)},line width=1mm]                (2,0) -- (0,0);
  \draw[shift={(11,-4.5)},line width=1mm]                (2,0) -- (1,+{sqrt(3)}) -- (0,0);
  %  8:9 whole tone relations
  \draw[shift={(11,-4.5)},line width=1mm,densely dashed] (2,0) -- (6,0);
  %  Circles for chord notes labels
  \draw[shift={(11,-4.5)},fill=white] (6,0)          circle (.3);
  \draw[shift={(11,-4.5)},fill=white] (0,0)          circle (.3);
  \draw[shift={(11,-4.5)},fill=white] (1,+{sqrt(3)}) circle (.3);
  \draw[shift={(11,-4.5)},fill=white] (2,0)          circle (.3);
  %  Chord notes labels
  \node[shift={(7.7,-3.15)}] at (0,0)          {\scriptsize $1$};
  \node[shift={(7.7,-3.15)}] at (1,+{sqrt(3)}) {\scriptsize \flat $3$};
  \node[shift={(7.7,-3.15)}] at (2,0)          {\scriptsize $5$};
  \node[shift={(7.7,-3.15)}] at (6,0)          {\scriptsize $6$};
\end{tikzpicture}%

%%% Local Variables:
%%% mode: latex
%%% TeX-master: "../main"
%%% End:

  \caption{Quartvorhalt, erweiterter Nonakkord und Sixte ajoutée in Dur (oben)
    und Moll (unten), visualisiert als Unterkomplexe des Eulerschen
    Tonnetzes.}\label{fig:chordLines}
\end{figure}

\subsection{Dominantseptakkord}
\label{sec:dom7syn}

Sehr oft kommt der Dominantakkord zusammen mit der kleinen Septime über dem
Grundton des Akkords zusammen vor (z.\,B. g\,:\,h\,:\,d\,:\,f in C-Dur wie in
c-Moll), vgl. \cite[{}9.1]{Skript}.  Bei der Frage, wie die kleine Septime zu
intonieren ist, fällt uns zuerst der Vierklang aus \cref{sec:int} ein, der dem
Verhältnis $4:5:6:7$ entsprach. Wir stellen schnell fest, dass sich dieser
Vierklang nicht im Eulerschen Tonnetz wiederfindet, was schlichtweg daran liegt,
dass dieses nur Primzahlen bis $5$ kennt, vgl. \cref{sec:primes}.  Wir werden in
\cref{sec:sept} über die Möglichkeit sprechen, diesen Vierklang dennoch zu
verwenden, bemerken dabei aber, dass er nur in ganz wenigen Situationen im
Kontext klassischer Musik sinnvoll ist.

Stattdessen ist eine im Eulerschen Tonnetz beheimatete Intonationsmöglichkeit
leichter in einen musikalischen Gesamtkontext zu integrieren: In der Nähe des
Dominantakkordes (g\,:\,\naturalm h\,:\,d) gibt es im Eulerschen Tonnetz zwar
zwei unterschiedliche kleine Septimen (\natural f und \naturalp f), die in
Betracht kommen.  Allerdings hat die auf der Kommahöhe des Grundtons (\natural
f) entscheidende Vorteile:
\begin{enumerate}
\item Sie kommt auch in anderen Dreiklängen der Tonart (z.\,B. als Grundton der
  Subdominante!) vor.  Die andere könnte man (mit viel Wohlwollen) vielleicht
  noch als Terz der Doppeldominantvariante verwenden.
\item Nicht selten bleibt die Septime der Dominante beim Harmoniewechsel
  D\textsuperscript{$7$}\,–\,T\textsuperscript{$4$–$3$} liegen und wird zur
  Quarte der Tonika.  Wird hierbei die empfohlene Intonation der Septime
  verwendet, so ist hierbei keine Anpassung nötig.
\item Gleiches gilt, wenn beim Harmoniewechsel S\,–\,D\textsuperscript{$7$} der
  Grundton der Subdominante liegen bleibt und zur Septime der Dominante wird.
\end{enumerate}
Der so entstehende Vierklang hat das Frequenzverhältnis $36:45:54:64$. Die
verminderte Quinte (Tritonus) zwischen der großen Terz und der kleinen Septime
ist $45:64\approx 609{,}78\,\on{ct}$ groß, also genauso groß wie der im Sixte
ajoutée in Moll. Die Auflösung der Septime der Dominante in die große bzw.
kleine Terz der Tonika erfolgt wie beim Quartvorhalt (siehe \cref{sec:49}) mit
einem diatonischen Halbton bzw. einem kleinen Ganzton. In
\cref{fig:chordLinessevenths} findet sich eine Verortung dieses Akkords im
Tonnetz.

\subsection{Verminderter Septakkord}
\label{sec:dim7syn}

% Auch für den Fall des verminderten Septakkordes werden wir in \cref{sec:sept}
% eine Alternative entwickeln, die wir aber nicht immer einsetzen wollen.

Der verminderte Septakkord, kurz D$^v$, besteht aus der großen Terz, reinen
Quinte, kleinen Septime und kleinen None des Dominantakkordes\footnote{Wie jede
  Variation der Dominante kann dies auch bei einer Zwischendominante auftreten.}
(z.\,B. h\,:\,d\,:\,f\,:\,\flat a in C-Dur wie in c-Moll),
vgl. \cite[{}11.1]{Skript}. Es handelt sich also im Wesentlichen um einen
Dominantseptakkord, bei dem der Grundton durch eine kleine None ausgetauscht
wird. Es ergibt also Sinn, Terz, Quinte und Septime so zu intonieren wie im
vorherigen Abschnitt beschrieben. Für die None kommt in der harmonischen Nähe
nur ein Kandidat in Frage, nämlich \flatp $9$.  Wird diese None beim
Akkordwechsel D$^v$\,–\,t in die Quinte der Tonika geführt, so hat dieser
Schritt wieder die Größe $15:16$.

Insgesamt erhalten wir also in obigem Beispiel \naturalm
h\,:\,d\,:\,f\,:\,\flatp a, was dem Verhältnis $225:270:320:384$ entspricht,
siehe auch \cref{fig:chordLinessevenths}, inklusive aller Auflösungsrichtungen.
Dieses Verhältnis mag auf den ersten Blick wenig überzeugen, spiegelt aber viele
Besonderheiten dieses Vierklangs wider:
\begin{enumerate}
\item Der verminderte Septakkord enthält zwei Tritoni und ist vom Wesen her eine
  Dominante mit zusätzlichen Strebetönen (hin zu Terz und Quinte der Tonika); es
  ist also gar nicht zu erwarten, dass er stabil klingt.  (In Bachs Musiksprache
  tritt er an besonders schmerzvollen und dramatischen Stellen auf.)
\item Die None liegt eine reine kleine Terz über der Septime (so wie auch die
  Terz eine reine kleine Terz unter der Quinte ist) und bildet zur Quinte den
  gleichen Tritonus, den auch die Septime zur Terz bildet.  Der Akkord ist also
  in symmetrischen Terzen geschichtet: Terz zu Quinte $5:6$, Quinte zu Septime
  $27:32$ und Septime zu None $5:6$.
\item In \cite[S.\,92ff.]{deLaMotte} wird der verminderte Septakkord beschrieben
  als Verschmelzung zweier (unvollständiger) Funktionen: Er enthält den Grundton
  und die kleine Terz der Mollsubdominante (hier: f und \flat a) sowie die große
  Terz und die Quinte der Dominante (hier: h und d).  Unsere Verortung dieser
  Töne im Eulerschen Tonnetz ist im Einklang mit dieser Beschreibung.
\end{enumerate}

\subsection{Septnonakkord}
\label{sec:dom79syn}

Der Dominantseptnonakkord in Dur ist ein Vierklang, der entsteht, indem dem
Dominantseptakkord die (skaleneigene) große None hinzugefügt und der Grundton
ausgelassen wird (in \flat E-Dur: d\,:\,f\,:\,\flat a\,:\,c),
vgl. \cite[{}11.3]{Skript}.  Da alle Bestandteile bis auf die None bereits in
\cref{sec:dom7syn} besprochen wurden, bleibt nur die Stimmung der None zu
klären.  Die None ist entweder als große Terz über der Septime oder als Quinte
über der Quinte (und damit pythagoreische None über dem Grundton)
interpretierbar.  Zweiteres ist aus folgenden Gründen sinnvoller:
\begin{enumerate}
  \item Diese None ist verträglich mit der aus \cref{sec:49}.
  \item Die Septime ist in dem Dominantakkord ein spannungsreicher Ton.  Ein
  reines Intervall zu ihr erzeugt tendenziell weniger Konsonanz als reine
  Quinten über dem Grundton.
  \item Die Auflösung der None erfolgt mit einem Ganzton abwärts in den Grundton
  der Tonika; hierbei ist der große, pythagoreische Ganzton ($8:9$) dem kleinen
  Ganzton ($9:10$) melodisch vorzuziehen, siehe \cref{sec:melody}.
\end{enumerate}
Der Dominantseptnonakkord entspricht also dem Verhältnis $45:54:64:81$ (in \flat
E-Dur: \naturalm d\,:\,f\,:\,\flat a\,:\,c) und lässt sich im Eulerschen Tonnetz
wie in \cref{fig:chordLinessevenths} zeichnen (wir haben den Grundton der
Übersichtlichkeit halber auch eingezeichnet).

Wir merken an, dass unter Verwendung der Primzahl $7$ auch die Ausstimmung
$5:6:7:9$ in Betracht kommt, die vor allem deswegen witzig ist, weil Septime und
None den Faktoren $7$ und $9$ entsprechen.  Wie schon beim Dominantseptakkord
ist dieses Verhältnis aber eher von akademischem Interesse und im Kontext
klassischer Musik meistens unbrauchbar. Wir klammern diese Möglichkeit hier also
aus und verweisen wieder auf \cref{sec:sept}.

Abschließend weisen wir noch auf eine Besonderheit des Dominantseptnonakkords
hin: Er setzt sich aus den gleichen Tönen zusammen wie der Sixte ajoutée der
Mollparallele (in \flat E-Dur: f\,:\,\flat a\,:\,c\,:\,d),
vgl. \cref{sec:sixte}. Weil diese Noten sogar enharmonisch identisch sind,
werden sie pythagoreisch gleich intoniert. Unsere Kommaverteilung liefert
allerdings verschiedene Ausstimmungen, wie \cref{tab:79-6} zeigt.  Wie wir
sehen, teilen sie sich den Tritonus \naturalm d\,:\,\flat a, doch die Quinten
f\,:\,c und \naturalm f\,:\,\naturalm c unterscheiden sich um ein Komma.

\begin{table}[h]
  \centering
  \begin{tabular}{lrrrr}
    \toprule
    Dominantseptnonakkord (Terz oben): & f & \flat a & c & \naturalm d\\
    Sixte ajoutée der Parallele (Sexte oben): & \naturalm f & \flat a & \naturalm c & \naturalm d\\
    \bottomrule
  \end{tabular}
  \caption{Die Bestandteile zweier sehr ähnlicher Akkorde in \flat E-Dur}\label{tab:79-6}
\end{table}

\subsection{Verminderter Septakkord mit tiefalterierter Quinte}
\label{sec:herzsyn}

Der letzte Akkord, den wir in diesem Abschnitt behandeln, ist eine spezielle
Abwandlung des verminderten Septakkordes (\cref{sec:dim7syn}), die (fast)
ausschließlich als Variation der Doppeldominante auftritt: Bei ihr ist
zusätzlich die Quinte tiefalteriert (in C-Dur oder c-Moll: \sharp f\,:\,\flat
a\,:\,c\,:\,\flat e), vgl. \cite[{}11.2]{Skript}.  Alle Töne haben eine
Spannung, die sie mit einem Halbtonschritt zur Dominante auflösen: die Terz
(\sharp f) nach oben und alle anderen Töne nach unten.

Wenn wir die Intonation der Terz, Septime und None von \cref{sec:dim7syn}
übernehmen, so gibt es eine unausweichliche Intonation für die tiefalterierte
Quinte: eine syntonische Ebene höher als das tonale Zentrum, denn dann ist der
Durdreiklang aus Quinte, Septime und None rein und alle Töne lösen sich mit
einem diatonischen Halbtonschritt $15:16$ auf. In C-Dur (oder c-Moll) ist der
Akkord also \sharpm f\,:\,\flatp a\,:\,c\,:\,\flatp e, was dem Verhältnis
$225:256:320:384$ entspricht. Die verminderte Terz zwischen großer Terz und
verminderter Quinte ist $225:256 \approx 223{,}46\,\on{ct}$ groß.  Auch diesen
Akkord mitsamt seinen Auflösungsrichtungen haben wir in
\cref{fig:chordLinessevenths} ins Tonnetz eingezeichnet.

Wir bemerken abschließend, dass dieser Vierklang enharmonisch umgedeutet werden
kann zu einem Dominantseptkkord in einer völlig anderen Tonart (im Beispiel:
zum Dominantseptakkord in \flatp D-Dur). Die darin enthaltene Septime (\flatp g)
ist im Vergleich zur Terz der Doppeldominante (\sharpm f) ein pythagoreisches
Komma tiefer und zwei syntonische Kommata höher, insgesamt also etwa
$19{,}55\,\om{ct}$ höher.

\begin{figure}
  \centering
  \begin{tikzpicture}[scale=.7]
  %% Dominant Seventh in major
  %  Euler lattice
  \draw[black!50] (0, 0)         -- (6, 0);
  \draw[black!50] (1, {sqrt(3)}) -- (5, {sqrt(3)});
  \draw[black!50] (1,-{sqrt(3)}) -- (5,-{sqrt(3)});
  \draw[black!50] (0,0)          -- (1, {sqrt(3)}) -- (2,0) -- (3, {sqrt(3)}) -- (4,0) -- (5, {sqrt(3)}) -- (6,0);
  \draw[black!50] (0,0)          -- (1,-{sqrt(3)}) -- (2,0) -- (3,-{sqrt(3)}) -- (4,0) -- (5,-{sqrt(3)}) -- (6,0);
  %  Just third and fifth relations
  \draw[black,line width=1mm]                (4,0) -- (6,0) -- (5,-{sqrt(3)}) -- (4,0);
  %  8:9 whole tone relations
  \draw[black,line width=1mm,densely dashed] (0,0) -- (4,0);
  %  Circles for chord notes labels
  \draw[fill=white] (0,0)          circle (.3);
  \draw[fill=white] (4,0)          circle (.3);
  \draw[fill=white] (6,0)          circle (.3);
  \draw[fill=white] (5,-{sqrt(3)}) circle (.3);
  %  Chord notes labels
  \node at (0,0)          {\scriptsize $7$};
  \node at (4,0)          {\scriptsize $1$};
  \node at (6,0)          {\scriptsize $5$};
  \node at (5,-{sqrt(3)}) {\scriptsize $3$};
  %  Semitone resolutions
  \draw[white,line width=1mm] (.3,{-sqrt(3)*1/10})  -- (2.7,{-sqrt(3)*9/10});
  \draw[white,line width=1mm] (4.7,{-sqrt(3)*9/10}) -- (2.3,{-sqrt(3)*1/10});
  \draw[-stealth]             (.3,{-sqrt(3)*1/10})  -- (2.7,{-sqrt(3)*9/10});
  \draw[-stealth]             (4.7,{-sqrt(3)*9/10}) -- (2.3,{-sqrt(3)*1/10});
  %  Label of chord
  \node at (-1.5,0)       {\small $\strut$Dur};
  \node at (3,2.5)        {\small $\strut$Septakkord};
  
  
  %% Diminished seventh in major
  %  Euler lattice
  \draw[shift={(7.5,0)},black!50] (0, 0)         -- (6, 0);
  \draw[shift={(7.5,0)},black!50] (1, {sqrt(3)}) -- (5, {sqrt(3)});
  \draw[shift={(7.5,0)},black!50] (1,-{sqrt(3)}) -- (5,-{sqrt(3)});
  \draw[shift={(7.5,0)},black!50] (0,0)          -- (1, {sqrt(3)}) -- (2,0) -- (3, {sqrt(3)}) -- (4,0) -- (5, {sqrt(3)}) -- (6,0);
  \draw[shift={(7.5,0)},black!50] (0,0)          -- (1,-{sqrt(3)}) -- (2,0) -- (3,-{sqrt(3)}) -- (4,0) -- (5,-{sqrt(3)}) -- (6,0);
  %  Just third and fifth relations
  \draw[shift={(7.5,0)},black,line width=1mm] (6,0) -- (5,-{sqrt(3)});
  \draw[shift={(7.5,0)},black,line width=1mm] (0,0) -- (1,+{sqrt(3)});
  %  Circles for chord notes labels
  \draw[shift={(7.5,0)},fill=white] (0,0)          circle (.3);
  \draw[shift={(7.5,0)},fill=white] (1,{sqrt(3)})  circle (.3);
  \draw[shift={(7.5,0)},fill=white] (6,0)          circle (.3);
  \draw[shift={(7.5,0)},fill=white] (5,-{sqrt(3)}) circle (.3);
  %  Chord notes labels
  \node at (7.5,0)           {\scriptsize $7$};
  \node at (8.5,{sqrt(3)})   {\scriptsize \flat $9$};
  \node at (13.5,0)          {\scriptsize $5$};
  \node at (12.5,-{sqrt(3)}) {\scriptsize $3$};
  %  Semitone resolutions
  \draw[shift={(7.5,0)},white,line width=1mm] (.3,{-sqrt(3)*1/10})  -- (2.7,{-sqrt(3)*9/10});
  \draw[shift={(7.5,0)},white,line width=1mm] (4.7,{-sqrt(3)*9/10}) -- (2.3,{-sqrt(3)*1/10});
  \draw[shift={(7.5,0)},white,line width=1mm] (1.3,{+sqrt(3)*9/10}) -- (3.7,{+sqrt(3)*1/10});
  \draw[shift={(7.5,0)},-stealth]             (.3,{-sqrt(3)*1/10})  -- (2.7,{-sqrt(3)*9/10});
  \draw[shift={(7.5,0)},-stealth]             (4.7,{-sqrt(3)*9/10}) -- (2.3,{-sqrt(3)*1/10});
  \draw[shift={(7.5,0)},-stealth]             (1.3,{+sqrt(3)*9/10}) -- (3.7,{+sqrt(3)*1/10});
  %  Label of chord
  \node at (10.5,2.5)        {\small $\strut$Verminderter Septakkord};
  
  
  %% Dominant seventh in minor
  %  Euler lattice
  \draw[shift={(0,-4.5)},black!50] (0, 0)         -- (6, 0);
  \draw[shift={(0,-4.5)},black!50] (1, {sqrt(3)}) -- (5, {sqrt(3)});
  \draw[shift={(0,-4.5)},black!50] (1,-{sqrt(3)}) -- (5,-{sqrt(3)});
  \draw[shift={(0,-4.5)},black!50] (0,0)          -- (1, {sqrt(3)}) -- (2,0) -- (3, {sqrt(3)}) -- (4,0) -- (5, {sqrt(3)}) -- (6,0);
  \draw[shift={(0,-4.5)},black!50] (0,0)          -- (1,-{sqrt(3)}) -- (2,0) -- (3,-{sqrt(3)}) -- (4,0) -- (5,-{sqrt(3)}) -- (6,0);
  %  Just third and fifth relations
  \draw[shift={(0,-4.5)},black,line width=1mm]                (4,0) -- (6,0) -- (5,-{sqrt(3)}) -- (4,0);
  %  8:9 whole tone relations
  \draw[shift={(0,-4.5)},black,line width=1mm,densely dashed] (0,0) -- (4,0);
  %  Circles for chord notes labels
  \draw[shift={(0,-4.5)},fill=white] (0,0)          circle (.3);
  \draw[shift={(0,-4.5)},fill=white] (4,0)          circle (.3);
  \draw[shift={(0,-4.5)},fill=white] (6,0)          circle (.3);
  \draw[shift={(0,-4.5)},fill=white] (5,-{sqrt(3)}) circle (.3);
  %  Chord notes labels
  \node at (0,-4.5)           {\scriptsize $7$};
  \node at (4,-4.5)           {\scriptsize $1$};
  \node at (6,-4.5)           {\scriptsize $5$};
  \node at (5,-{sqrt(3)}-4.5) {\scriptsize $3$};
  %  Semitone resolutions
  \draw[shift={(0,-4.5)},white,line width=1mm] (.3,{+sqrt(3)*1/10})  -- (2.7,{+sqrt(3)*9/10});
  \draw[shift={(0,-4.5)},white,line width=1mm] (4.7,{-sqrt(3)*9/10}) -- (2.3,{-sqrt(3)*1/10});
  \draw[shift={(0,-4.5)},-stealth]             (.3,{+sqrt(3)*1/10})  -- (2.7,{+sqrt(3)*9/10});
  \draw[shift={(0,-4.5)},-stealth]             (4.7,{-sqrt(3)*9/10}) -- (2.3,{-sqrt(3)*1/10});
  %  Label of chord
  \node at (-1.5,-4.5)        {\small $\strut$Moll};
  
  
  %% Diminished seventh in minor
  %  Euler lattice
  \draw[shift={(7.5,-4.5)},black!50] (0, 0)         -- (6, 0);
  \draw[shift={(7.5,-4.5)},black!50] (1, {sqrt(3)}) -- (5, {sqrt(3)});
  \draw[shift={(7.5,-4.5)},black!50] (1,-{sqrt(3)}) -- (5,-{sqrt(3)});
  \draw[shift={(7.5,-4.5)},black!50] (0,0)          -- (1, {sqrt(3)}) -- (2,0) -- (3, {sqrt(3)}) -- (4,0) -- (5, {sqrt(3)}) -- (6,0);
  \draw[shift={(7.5,-4.5)},black!50] (0,0)          -- (1,-{sqrt(3)}) -- (2,0) -- (3,-{sqrt(3)}) -- (4,0) -- (5,-{sqrt(3)}) -- (6,0);
  %  Just third and fifth relations
  \draw[shift={(7.5,-4.5)},black,line width=1mm] (6,0) -- (5,-{sqrt(3)});
  \draw[shift={(7.5,-4.5)},black,line width=1mm] (0,0) -- (1,+{sqrt(3)});
  %  Circles for chord notes labels
  \draw[shift={(7.5,-4.5)},fill=white] (0,0)          circle (.3);
  \draw[shift={(7.5,-4.5)},fill=white] (6,0)          circle (.3);
  \draw[shift={(7.5,-4.5)},fill=white] (1,{sqrt(3)})  circle (.3);
  \draw[shift={(7.5,-4.5)},fill=white] (5,-{sqrt(3)}) circle (.3);
  %  Chord notes labels
  \node at (7.5,-4.5)            {\scriptsize $7$};
  \node at (8.5,{sqrt(3)-4.5})   {\scriptsize \flat $9$};
  \node at (13.5,-4.5)           {\scriptsize $5$};
  \node at (12.5,-{sqrt(3)-4.5}) {\scriptsize $3$};
  %  Semitone resolutions
  \draw[shift={(7.5,-4.5)},white,line width=1mm] (.3,{+sqrt(3)*1/10})  -- (2.7,{+sqrt(3)*9/10});
  \draw[shift={(7.5,-4.5)},white,line width=1mm] (4.7,{-sqrt(3)*9/10}) -- (2.3,{-sqrt(3)*1/10});
  \draw[shift={(7.5,-4.5)},white,line width=1mm] (1.3,{+sqrt(3)*9/10}) -- (3.7,{+sqrt(3)*1/10});
  \draw[shift={(7.5,-4.5)},-stealth]             (.3,{+sqrt(3)*1/10})  -- (2.7,{+sqrt(3)*9/10});
  \draw[shift={(7.5,-4.5)},-stealth]             (4.7,{-sqrt(3)*9/10}) -- (2.3,{-sqrt(3)*1/10});
  \draw[shift={(7.5,-4.5)},-stealth]             (1.3,{+sqrt(3)*9/10}) -- (3.7,{+sqrt(3)*1/10});
\end{tikzpicture}

%%% Local Variables:
%%% mode: latex
%%% TeX-master: "../main"
%%% End:

  \caption{Verschiedene Septakkorde im Eulerschen Tonnetz. Die Pfeile zeigen die
    Auflösungsrichtung einzelner Töne an.}\label{fig:chordLinessevenths}
\end{figure}

%%% Local Variables:
%%% mode: latex
%%% TeX-master: "../main"
%%% End:
