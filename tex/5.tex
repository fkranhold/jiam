\subsection{Schritte als horizontale Intervalle}

\subsection{Durchgangs- und Wechselnoten}

Durchgänge, Wechselnoten

\subsection{Leittöne}
\label{sec:ln}

Wir weisen darauf hin, dass die Terz im Dreiklang, die zur Dominante (oder einer
Zwischendominante) gehört, ebenfalls um ein syntonisches Komma abgesenkt wird,
was der gängigen Praxis widerspricht, Leittöne etwas höher zu intonieren, um die
Spannung zu erhöhen, siehe z.\,B. […]\todo{Literatur FK}. Wir sind der
Überzeugung, dass die Dominante ein eigenständiger, stabiler Dur-Dreiklang sein
sollte, aber in dieser Frage kann man anderer Meinung sein als wir. In diesem
Fall schlagen wir vor, den Leitton um ein syntonisches Komma zu erhöhen, wodurch
der Dreiklang die pythagoreische Stimmung $64:81:96$ erhält. (Im Hörbeispiel
hören wir zwei Versionen einer Kadenz, eine mit einem rein intonierten Leitton,
eine mit einem pythagoreischen Leitton). Wir halten uns jedoch an die
Konvention, alle Dreiklänge, auch solche, die eine Dominantenfunktion haben, so
zu intonieren, dass sie eine reine Terz haben.

\subsection{Halbtonschritte}
\label{sec:semi}

\begin{figure}
  \centering
  \LY{b-lead}
  \includegraphics{ly/b-lead}
  \caption{Zwei Arten, eine Vollkadenz zu intonieren: Einmal mit reiner
  	Intonation aller Dur-Dreiklänge und einmal mit einem pythagoreischen
  	Leitton.}\label{fig:lead}
\end{figure}

%%% Local Variables:
%%% mode: latex
%%% TeX-master: "../main"
%%% End:
