Von nun an interpretieren wir geschriebene Noten in pythagoreischer Stimmung und
bezeichnen alle Abweichungen davon mit bestimmten Symbolen, die wir einführen,
sobald sie notwendig werden.

\subsection{Das syntonische Komma}

Wir beginnen mit der Beobachtung, dass in pythagoreischer Stimmung ein
Dur-Dreiklang dem Verhältnis $64:81:96$ statt $4:5:6=64:80:96$ entspricht,
d.\,h. die große Terz etwas zu hoch ist (im Hörbeispiel hören wir zuerst einen
pythagoreischen Dur-Dreiklang und dann $4:5:6$).\LY{m-pyth456} Der Unterschied,
$\frac{81}{80}\approx 21{,}51\ct$, heißt \emph{syntonisches
  Komma}.\footnote{Das syntonische Komma ist also \emph{fast} genauso groß wie
  das pythagoreische. Der Unterschied, nämlich
  $23{,}46\ct-21{,}51\ct\approx 1{,}95\ct$, wird
  \emph{Schisma} genannt und ist \emph{fast} so groß wie ein Grad, aber nicht exakt,
  was wir auch ohne konkretes Nachrechnen begründen können: Das Verhältnis eines
  Schismas ist rational, während ein Grad $\smash{\frac32\cdot 2^{-\frac7{12}}}$
  und damit irrational ist.} Mit anderen Worten: Die große pythagoreische Terz
$64:81$ ist um ein syntonisches Komma größer als $4:5$, und die kleine
pythagoreische Terz $27:32$ ist ein um ein syntonisches Komma kleiner als $5:6$,
denn $5:6$ ist das Komplement von $4:5$ in $2:3$ und damit die gewünschte kleine
Terz. Von nun an nennen wir das Verhältnis $4:5$ eine \emph{reine} große Terz
und $5:6$ eine \emph{reine} kleine Terz. Folglich nennen wir ihre zur Oktave
komplementären Intervalle $5:8$ und $3:5$ die \emph{reine} kleine bzw. große
Sexte.\looseness-1

\subsection{Helmholtz-Ellis-Notation}

Wenn man versucht, Tonhöhen um ein syntonisches Komma zu alterieren, um die
Stimmung zu verbessern, stößt man auf das Problem, dass jeder Ton prinzipiell
die große Terz, aber auch der Grundton eines Dreiklangs sein kann. Das bedeutet:
Es hängt vom harmonischen Kontext ab, ob eine Tonhöhe alteriert werden muss
oder nicht, und das lässt sich leider nicht aus dem bloßen (enharmonisch
präzisen) Tonnamen herauslesen.\looseness-1

Bei einer Partitur ergänzen wir die zusätzliche Information, dass bestimmte
Noten um ein oder mehrere syntonische Kommata abgesenkt oder angehoben werden
müssen, durch das Hinzufügen von Pfeilen an die schon existenten
Versetzungszeichen in der Tradition von Helmholtz und Ellis \cite{HE} mit den
Glyphen von \cite{HEJI}: Die Versetzungszeichen \dsharpp, \sharpp, \naturalp,
\flatp, und \dflatp\ bedeuten also die Erhöhung des Tons um ein syntonisches
Komma; genauso bedeuten \dsharppp, \sharppp, \naturalpp, \flatpp, und \dflatpp\
zwei Kommata und so weiter. In die andere Richtung benutzen wir \dsharpm,
\sharpm, \naturalm, \flatm, und \dflatm\ für Noten, die um ein syntonisches
Komma tiefer erklingen sollen. Ein Beispiel ist in \cref{fig:HE} zu sehen. Wie
üblich gelten Versetzungszeichen für einen ganzen Takt; wenn also ein einzelner
Takt erst \naturalm a' und dann \natural a' enthält, so müssen beide
Versetzungszeichen angegeben werden. Aus Gründen, die im nächsten Unterabschnitt
erläutert werden, verwenden wir überhaupt keine Generalvorzeichen.\looseness-1

\begin{figure}\centering
  \includegraphics{ly/p-HE}
  \LY{p-HE}
  \caption{Im ersten Takt ist das \sharp f’ um ein syntonisches Komma
    tiefalteriert, sodass wir einen Dur-Dreiklang $4:5:6$ erhalten. Im zweiten
    Takt wird das g’ um ein, das \flat e’’ um zwei syntonische Kommata
    hochalteriert, sodass sich eine reine kleine Sexte $5:8$
    ergibt.}\label{fig:HE}
\end{figure}

\subsection{Die ptolemäische Skala}
\label{subsec:ptol}

Ein klassischer Ansatz, syntonische Kommata konsequent in einem
ganzen Musikstück einzusetzen, besteht darin, die pythagoreisch gestimmte
Dur-Tonleiter zu verändern, indem die Töne, die am ehesten als große Terzen
eines Dreiklangs auftreten, nämlich Terz, Sexte und Septime, um ein syntonisches
Komma erniedrigt werden. Das Ergebnis nennt man die \emph{ptolemäische
  Skala}. Legt man die Frequenz des Grundtons der Skala als $1$ fest, lassen
sich die Frequenzen aller in der Skala vorkommenden Noten als
$1,\frac98,\frac54,\frac43,\frac32,\frac53,\frac{15}8,2$ berechnen. Dieser Logik
folgend müsste etwa C-Dur drei Generalvorzeichen haben, nämlich \naturalm e,
\naturalm a\ und \naturalm h.

Wir haben uns aus den folgenden Gründen gegen die Verwendung dieser Skala
entschieden (wir vermeiden das Wort „Skala“ sogar ganz):
\begin{enumerate}
\item Der zweite Ton (in C-Dur: das d) der Skala kommt in zwei wichtigen
  „Rollen“ vor: Als Quinte der Dominante (g\,:\,h\,:\,d) und als Grundton der
  Subdominantparallele (d\,:\,f\,:\,a). Im Falle der Subdominantparallele muss auch
  dieser zweite Ton um ein syntonisches Komma tiefalteriert werden. Da beide
  Funktionen mit ähnlicher Häufigkeit auftreten, ist unklar, welcher Alteration
  im Rahmen einer Skala der Vorzug zu geben ist.
\item Die ptolemäische Skala weist nur einigen Tonhöhen eine Frequenz zu. Es
  gibt einige Tonhöhen, die mit einer Tonleiter „eng verwandt“ sind, auch wenn
  sie technisch gesehen nicht zu ihr gehören (z.\,B. das \sharp f in C-Dur, was
  die große Terz der Doppeldominante ist).
\item Wie wir in \cref{sec:balance} auseinandersetzen werden wirken melodische
  Linien mit unterschiedlich großen Ganztonschritten unausgewogen, weswegen man
  in melodisch gedachten Passagen diese Skala nicht verwenden würde.
% \item Unser Stimmungssystem sollte nicht von der Tonart des Stücks, sondern nur
%   vom Kammerton abhängen.\todo{Ich bin mir nicht sicher ob ich diesen Punkt verstehe. Das Stimmungssystem ändert sich ja durch die Vorzeichen nicht, oder? Ich bin verwirrt. –T}
\end{enumerate}

\subsection{Das Eulersche Tonnetz}

Stattdessen wollen wir alle möglichen Frequenzen eines Stücks in einer Weise
anordnen, die ihre harmonische Nähe zueinander widerspiegelt. Dies wird
typischerweise durch das \emph{Eulersche Tonnetz} erreicht, siehe
\cref{fig:latticeExcerpt}: Wir beginnen mit der Quintenspirale aus
\cref{sec:pyth}, die wir jetzt als horizontale Gerade schreiben, und unterteilen
jeden Schritt ($2:3$) in eine reine große Terz ($4:5$) und eine reine kleine
Terz ($5:6$). Dieser Zwischenton wird zwischen dem ersten und dem zweiten, aber
etwas weiter unten notiert. Wenn wir alle Schritte unterteilen, erhalten wir
eine vollständige neue Zeile, die wieder aus reinen Quinten besteht, aber jetzt
sind die Frequenzen in dieser Zeile um ein syntonisches Komma
erniedrigt. Dasselbe lässt sich auch in der anderen Richtung machen, indem man
die Reihenfolge von kleiner und großer Terz vertauscht und die Zwischentöne
etwas höher als die Ursprungsgerade schreibt. Wir erhalten eine neue
Quintenreihe, in der alle Frequenzen um ein syntonisches Komma erhöht sind. Wenn
wir diesen Prozess in beide Richtungen wiederholen, erhalten wir ein
Tonhöhengitter (siehe \cref{fig:latticeExcerpt}), in dem benachtbarte Töne im
Verhältnis einer reinen Quinte, reinen großen Terz oder reinen kleinen Terz
stehen.

Jeder (enharmonisch präzise angegebene) Ton kommt (modulo Oktave) in jeder Zeile
genau einmal vor, sodass ein Knoten im Netz einem Ton zusammen mit einer Anzahl
von darauf angewendeten syntonischen Kommata entspricht. Diese Anzahl der
syntonischen Kommata nennen wir die \emph{Kommahöhe} des Knotens. Beispielsweise
hat \sharppp c die Kommahöhe $+2$. Zwei Knoten haben genau dann die gleiche
Kommahöhe, wenn sie in der gleichen „Zeile“ des Tonnetzes stehen.

\begin{figure}[h]
  \[
  \begin{tikzcd}[column sep=-1em,labels=description]
    & \text{\flatp D}\,(\frac{16}{15})\ar[rr,dash]\ar[dr,dash] && \text{\flatp A}\,(\frac45)\ar[rr,dash]\ar[dr,dash] && \text{\flatp E}\,(\frac{6}{5})\ar[dr,dash]\ar[rr,dash] && \text{\flatp B}\,(\frac95)\ar[dr,dash]\ar[rr,dash] && \text{\naturalp F}\,(\frac{27}{20})\ar[dr,dash]\\
    \text{\flat B}\,(\frac{16}9)\ar[rr,dash]\ar[dr,dash]\ar[ur,dash] && \rF\,(\frac43)\ar[ur,dash]\ar[rr,dash]\ar[dr,dash] && \rC\,(1)\ar[dash]{ur}{5:6}\ar[dash]{rr}{2:3}\ar[dash]{dr}{4:5} && \rG\,(\frac32)\ar[ur,dash]\ar[rr,dash]\ar[dr,dash] && \rD\,(\frac98)\ar[ur,dash]\ar[rr,dash]\ar[dr,dash] && \rA\,(\frac{27}{16})\\
    & \text{\naturalm D}\,(\frac{10}9)\ar[rr,dash]\ar[dr,dash]\ar[ur,dash] && \text{\naturalm A}\,(\frac53)\ar[rr,dash]\ar[dr,dash]\ar[ur,dash] && \text{\naturalm E}\,(\frac54)\ar[rr,dash]\ar[dr,dash]\ar[ur,dash] && \text{\naturalm B}\,(\frac{15}{16})\ar[rr,dash]\ar[dr,dash]\ar[ur,dash] && \text{\sharpm F}\,(\frac{45}{32})\ar[dr,dash]\ar[ur,dash]\\
    \text{\naturalmm B}\,(\frac{27}{25})\ar[rr,dash]\ar[ur,dash] && \text{\sharpmm F}\,(\frac{50}{36})\ar[rr,dash]\ar[ur,dash] && \text{\sharpmm C}\,(\frac{25}{24})\ar[rr,dash]\ar[ur,dash] && \text{\sharpmm G}\,(\frac{25}{16})\ar[rr,dash]\ar[ur,dash] && \text{\sharpmm D}\,(\frac{75}{64})\ar[rr,dash]\ar[ur,dash] && \text{\sharpmm A}\,(\frac{225}{128})
  \end{tikzcd}
\]

  \caption{Ein Ausschnitt des Eulerschen Tonnetzes. Wir berechnen
    Frequenzfaktoren relativ zum c, und „oktav-normiert“, sodass sie zwischen
    $1$ und $2$ liegen.}\label{fig:latticeExcerpt}
\end{figure}

\subsection{Geometrie im Tonnetz}

Im Eulerschen Tonnetz umfasst ein Dreieck der Form $\triangledown$ die
Bestandteile eines rein intonierten Dur-Akkords ($4:5:6$), während ein Dreieck
$\vartriangle$ die Bestandteile eines Mollakkords ($10:12:15$)
erfasst. Benachbarte Dreiecke teilen sich entweder eine Quinte, d.\,h. sie sind
\emph{Varianten} voneinander, oder eine Terz, d.\,h. sie stehen in
\emph{paralleler} ($\vartriangle\!\!\!\triangledown$) oder
\emph{gegenparalleler} ($\triangledown\!\!\!\vartriangle$) Beziehung
zueinander. Legt man ein Dreieck als Tonika eines Stücks fest, so lassen sich
die Standardfunktionen wie \mbox{(Sub-)}Dominanten sowie ihre (Gegen-)Parallelen
und sogar Zwischendominanten (vgl. \cite[§\,10]{Skript}) leicht ermitteln: Für
ein gegebenes Dreieck ist seine Zwischendominante das Dur-Dreieck, das die
Quinte rechts neben dem gegebenen Dreieck enthält (z.\,B.  ist die
Doppeldominante von C-Dur der D-Dur-Dreiklang in der Mitte rechts in
\cref{fig:latticeExcerpt}, nicht der Dreiklang unten links, der ein syntonisches
Komma tiefer liegt.\looseness-1

\subsection{Mikroalterationen funktionaler Harmonien}

Bei einem Musikstück (z.\,B. \cref{fig:triadPiece}) bestimmt man zunächst seine
Tonart (hier: e-Moll). Dann wird das Tonika-Dreieck im Eulerschen Tonnetz
eindeutig durch den Modus (Dur oder Moll) und die Bedingung bestimmt, dass der
Grundton (hier: e) nicht durch ein Komma verändert wird. Für jeden Ton bestimmen
wir dann den Dreiklang, zu dem er gehört (der vierte Ton des Tenors, a, ist Teil
eines D-Dur-Dreiklangs) und die Funktion des Dreiklangs (Zwischendominante zur
Tonikaparallele), und finden dann das entsprechende Dreieck im Eulerschen
Tonnetz (siehe \cref{fig:chordsLattice} für die Verortung aller in
\cref{fig:triadPiece} vorkommen Dreiklänge im Eulerschen Tonnetz). Daraus ergibt
sich dann, wie viele syntonische Kommata wir auf die Note anwenden müssen (hier:
a muss um eines erhöht werden). Seine absolute Frequenz beträgt
$\frac{81}{80}\cdot 220\,\text{Hz}=222{,}75\,\text{Hz}$, und relativ zum
nächsttieferen Grundtons, e, entspricht er dem Faktor
$\frac{81}{80}\cdot\frac{4}{3}=\frac{27}{20}$.\looseness-1

Wir weisen darauf hin, dass bei einem Stück in C-Dur der Grundton eines
a-Moll-Dreiklangs um ein syntonisches Komma herabgesetzt wird, während dies bei
einem Stück in a-Moll nicht der Fall ist – auch wenn das Stück dieselben
Vorzeichen hat. Daher hat bei einem Stück mit gegebenen Generalvorzeichen die
Entscheidung, welcher der beiden möglichen Dreiklänge die Tonika ist
(z.\,B. C-Dur oder a-Moll), Auswirkungen darauf, wie hoch das Stück intoniert
werden soll. Das mag seltsam klingen, ergibt sich aber zwingend, wenn wir uns
darauf einigen, dass der Grundton der Tonika als Zentrum des Stücks direkt vom
Kammerton abgeleitet werden sollte. Im Übrigen ist dies auch aus praktischen
Gesichtspunkten sinnvoll, wenn die Anzahl der Mikroalterationszeichen möglichst
klein gehalten werden soll.

\wrap{\enlargethispage{\baselineskip}}

Wir bemerken abschließend, dass die „richtige“ Intonation von der Interpretation der
Funktion, die der Dreiklang darstellt, abhängt. Ein D-Dur-Dreiklang in C-Dur
z.\,B.  wird wohl in den meisten Fällen als Doppeldominante verstanden werden,
in seltenen Fällen aber auch als „dorische“ Subdominante der Tonikaparallele
(a-Moll). Im zweiten Fall müsst der Akkord ein syntonisches Komma tiefer
gespielt werden.

\begin{figure}
  \centering
  \LY{b-triadPiece}
  \includegraphics{ly/b-triadPiece}
  \caption{Ein schematisches vierstimmiges Stück in e-Moll (nur aus
  	Dreiklängen), mit eingebauten syntonischen Kommata. Es gibt zwei
  	Audioausgaben des Stücks, eine in reiner pythagoreischer Stimmung und eine,
  	die die syntonischen Kommata zur Korrektur
  	verwendet.\looseness-1}\label{fig:triadPiece}
\end{figure}

\begin{figure}
  \[
  \begin{tikzcd}[row sep=.7em,column sep=.7em]
    & \text{\naturalp c}\ar[rr,dash]\ar[ddr,dash] && \text{\naturalp g}\ar[rr,dash]\ar[ddr,dash] && \text{\naturalp d}\ar[rr,dash]\ar[ddr,dash] && \text{\naturalp a}\ar[ddr,dash] &\\
    & 7 & 10 & 1,3  & 5,9 & & 4,8 & & \\
    \text{a}\ar[rr,dash]\ar[uur,dash]\ar[ddr,dash] && \text{e}\ar[rr,dash]\ar[uur,dash]\ar[ddr,dash] && \text{h}\ar[rr,dash]\ar[uur,dash]\ar[ddr,dash] && \text{\sharp f}\ar[rr,dash]\ar[uur,dash]\ar[ddr,dash] && \text{\sharp c}\\
    &   &    & 6,13 &     & 2,12     & & 11\\
    & \text{\sharpm c}\ar[rr,dash]\ar[uur,dash] && \text{\sharpm g}\ar[rr,dash]\ar[uur,dash] &&
    \text{\sharpm d}\ar[rr,dash]\ar[uur,dash] && \text{\sharpm a}\ar[uur,dash]
  \end{tikzcd}
\]

  \caption{Die Dreiklänge die im Notenbeispiel von \cref{fig:triadPiece}
    auftauchen, verortet im Eulerschen Tonnetz.}\label{fig:chordsLattice}
\end{figure}

\subsection{Die Kommafalle}

Werfen wir einen erneuten Blick auf den Ausschnitt des Eulerschen Tonnetzes in
\cref{fig:latticeExcerpt}, so finden wir den Ton d an zwei Stellen in der Nähe
von c, nämlich \naturalm $\text{d}\,(\frac{10}9)$ and
$\text{d}\,(\frac98)$. Beide erscheinen als Eckpunkte von Dreiecken, die
Funktionen in C-Dur entsprechen: Der erste als Grundton von d-Moll
(Subdominantparallele, kurz Sp) und der zweite als Quinte von G-Dur (Dominante,
kurz D) – dies ist genau das Phänomen, das wir in \cref{subsec:ptol} im Zuge
unserer Kritik an einer festen mikroalterierten Skala erwähnt haben.\looseness-1

Man stelle sich nun ein Stück vor, das die Funktionen Sp und D direkt
hintereinander enthält, sowie eine Stimme, die den Grundton der Sp hat und hält,
während er zur Quinte von D wird (siehe \cref{fig:drift}). Dann gibt es
zwei Möglichkeiten:

\begin{enumerate}[itemsep=0em]
\item Einer der Akkorde wird um ein syntonisches Komma transponiert.
\item Die haltende Stimme muss die Tonhöhe beim Harmoniewechsel 
  \emph{nachjustieren}.
\end{enumerate}%
%
In der Vokalmusik geschieht typischerweise ersteres, weil die übliche
Interpretation des Haltens einer Note darin besteht, sie – \emph{per
  definitionem} – nicht zu verändern, und die übrigen Stimmen, die sich bewegen
müssen, ihre richtige Intonation im Verhältnis zum gehaltenen Ton suchen. Die
Folge ist, dass der zweite Akkord (D) um ein syntonisches Komma absinkt
und das Ensemble an Intonation verliert, d.\,h. es \emph{detoniert}, siehe den
ersten Teil von \cref{fig:drift}. Dieser Effekt wird als \emph{Kommafalle}
bezeichnet. Wenn dieses Phänomen fünfmal auftritt, beträgt die Detonation
$5\cdot 21{,}51\ct=107{,}55\ct$, d.\,h. das Ensemble ist in Summe um
mehr als einen gleichstufigen Halbton gesackt.\looseness-1

Die wünschenswertere Lösung, die jedoch vom Ensemble ein Verständnis der
harmonischen Situation verlangt, ist die zweite, die im zweiten Teil von
\cref{fig:drift} dargestellt ist. Dieses Beispiel zeigt, dass das theoretische
Bewusstsein für reine Intonation (die, wie wir behauptet haben, Chorsänger:innen
intuitiv nutzen) dabei helfen kann, die Ursache für ein sehr praktisches und
häufiges Problem aufzuspüren, nämlich das des Intonationsverlusts.

Die Akkordfolge Sp\kern1px–\kern1px D ist nicht die einzige, die
Kommaabweichungen verursachen kann. Der dritte Teil von \cref{fig:drift} zeigt
eine modulierende Akkordfolge, die die Doppeldominante einbezieht. Hier müssen
sogar zwei Stimmen, die eine Quinte voneinander entfernt sind, ihre Intonation
anpassen. Die gleichzeitigen An\-passungen ergeben zusammen eine „Bewegung“ in
parallelen Quinten.\footnote{Das Parallelenverbot im klassischen Kontrapunkt ist
  nicht für solche „Mikroparallelen“ ausgelegt, weswegen diese Bewegungen
  satztechnisch unproblematisch sind.}\looseness-1

\begin{figure}
  \centering
  \LY{b-drift}
  \includegraphics{ly/b-drift}
  \caption{(1) Eine Akkordfolge, bei der alle wiederholten Tonhöhen auf die   
  	gleiche Weise intoniert werden, wodurch eine Detonation um ein syntonisches
  	Komma entsteht.%
  	\quad (2) Dieselbe Akkordfolge mit der notwendigen Anpassung (durch eine
  	Zick-Zack-Linie gekennzeichnet): Zwischen dem dritten und dem vierten Akkord
  	muss der Sopran das d’’ anpassen.%
  	\quad (3) Eine ähnliche Situation, in der sogar zwei Stimmen eine Anpassung
  	vornehmen müssen, was eine Komma-Bewegung in „parallelen Quinten“
  	verursacht.\looseness-1}\label{fig:drift}
\end{figure}

%%% Local Variables:
%%% mode: latex
%%% TeX-master: "../main"
%%% End:
