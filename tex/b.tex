Blicken darauf zurück, in welcher Reihenfolge wir unseren Tonvorrat schrittweise
erweitert haben, so bemerken wir, dass in jedem Schritt die nächstgrößere
Primzahl ins Spiel gekommen ist. Um dies präzise zu machen, fixieren wir die
Frequenz $f_0$ des Kammertons a’ (z.\,B. $440\Hz$) und nennen für einen Ton mit
Frequenz $f$ den Quotienten $f/f_0$ den \emph{Faktor} des Tons. Dann sehen wir
folgendes:
\begin{enumerate}
	\item[2.] Für jede Oktavierung des Kammertons gibt es eine eindeutig bestimmte
	ganze Zahl $a_1\in\Z$, für die der Faktor des gegebenen Tons genau $2^{a_1}$
	beträgt. Mathematisch gesprochen definiert $a_1\mapsto 2^{a_1}$ eine Bijektion
	zwischen $\Z$ und den Faktoren der Oktavierungen des Kammertons.
	\item[3.] Für jeden Ton in pythagoreischer Stimmung gibt es eindeutig bestimmte
	ganze Zahlen $a_1,a_2\in\Z$, für die der Faktor des gegebenen Tons genau
	$2^{a_1}\cdot 3^{a_2}$ beträgt. Mathematisch gesprochen definiert
	$(a_1,a_2)\mapsto 2^{a_1}\cdot 3^{a_2}$ eine Bijektion zwischen $\Z^2$ und den
	Faktoren der Töne in pythagoreischer Stimmung.\footnote{Dass die
		Quintenspirale selbst $1$-dimensional ist, liegt daran, dass wir aus ihr die
		Oktave „rausgerechnet“ haben, also z.\,B. nicht zwischen \flat a und \flat
		a’ unterscheiden.}
	\item[5.] Für jeden Ton im Eulerschen Tonnetz (zusammen mit einer Oktavlage,
	z.\,B. \flatp a’) gibt es eindeutig bestimmte ganze Zahlen $a_1,a_2,a_3\in\Z$,
	für die der Faktor des gegebenen Tons genau
	$2^{a_1}\cdot 3^{a_2}\cdot 5^{a_3}$ beträgt.  Mathematisch gesprochen
	definiert $(a_1,a_2,a_3)\mapsto 2^{a_1}\cdot 3^{a_2}\cdot 5^{a_3}$ eine
	Bijektion zwischen $\Z^3$ und den Faktoren der Töne im Eulerschen
	Tonnetz.\footnote{Dass das Tonnetz selbst $2$-dimensional ist, liegt wieder
		daran, dass wir aus ihm die Oktave „rausgerechnet“ haben, also z.\,B. nicht
		zwischen \flatp a und \flatp a’ unterscheiden.}
	\item[7.] Das septimale Kommas bringt die Primzahl $7$ ins Spiel. Wir haben in
	\cref{sec:sept} nur die $0$- oder $1$-fache Verwendung des Kommas
	eingeführt. Würde man dies für jede ganze Zahl $a_4$ tun, so ergäbe sich ein
	$4$-dimensionales Gitter an Tönen, das durch
	$(a_1,a_2,a_3,a_4)\mapsto 2^{a_1}\cdot 3^{a_2}\cdot 5^{a_3}\cdot 7^{a_4}$
	parametrisiert ist.
\end{enumerate}
Allgemein bezeichnen wir eine Menge $T\subset \Q_{>0}$ von positiven Brüchen als
\emph{rationalen Tonvorrat}.  Unabhängig davon, wie wir bestimmte Töne nennen,
kann man für ein gegebenes Stück mathematisch beschreiben, welcher rationale
Tonvorrat verwendet wird.  Sei $p$ eine Primzahl und seien $p_1<\dotsb<p_r=p$
alle Primzahlen bis $p$. Dann ist der \emph{durch $p$ begrenzte} Tonvorrat
(engl. „$p$-limit tuning“) definiert als
\[T_p\coloneqq\{p_1^{a_1}\dotsm p_r^{a_r};\,a_i\in\Z\}\subset \Q_{>0}.\]%
Wir bemerken, dass $(a_1,\dotsc,a_r)\mapsto p_1^{a_1}\dotsm p_r^{a_r}$ eine
Bijektion zwischen $\Z^r$ und $T_p$ ist, sodass wir die Töne in $T_p$ in einem
$r$-dimensionalen Gitter anordnen können. Ferner sind im durch $p$ begrenzten
Tonvorrat alle auftretenden Frequenzverhältnisse zwischen $s$ gegebenen Tönen
der Form
\[p_1^{a_{11}}\dotsm p_r^{a_{r1}}:\dotsb:p_1^{a_{1s}}\dotsm p_r^{a_{rs}},\]%
wobei wir durch Multiplikation mit dem Hauptnenner erreichen können, dass
$a_{ij}\ge 0$ gilt. Ein Verhältnis von $s$ Tönen ist also beschrieben durch eine
$(r\times s)$-Matrix mit nicht-negativen ganzzahligen
Einträgen.\footnote{Gekürzt ist das Verhältnis genau dann, wenn es für jedes
	$1\le i\le r$ ein $1\le j\le s$ gibt mit $a_{ij}=0$, also wenn in jeder Zeile
	der Matrix eine $0$ auftaucht.} Dies erklärt zum Beispiel, warum das
Verhältnis $4:7$ im Eulerschen Tonnetz nicht auftreten kann.

Der für klassische Musik geeignete ist im Wesentlichen der durch $5$ begrenzte
Tonvorrat $T_5$, mit vereinzelten, in \cref{sec:sept} vorgestellten, Ausnahmen,
in denen man einen Schritt nach $T_7$ erwägen kann.  Dies macht natürlich die
daruffolgenden Primzahlen und die durch sie beschreibbaren Tonvorräte nicht
weniger interessant. Zum einen gibt es für jede Primzahl $p$ ein entsprechendes
Komma, um sie in unser Notensystem zu integrieren. Für viele davon wurden in
\cite{HEJI} Glyphen entworfen, die ähnlich wie \septimal\ die vorgestelle
Helmholtz-Ellis-Notation fortsetzen. Zum anderen wurde in den zurückliegenden
Jahrzehnten eine ganze Reihe sogenannter „mikrotonaler Musik“ geschrieben, die
auf einen durch eine große Primzahl (z.\,B. $67$) begrenzten Tonvorrat
zurückgreift, z.\,B. \cite{Nicholson}.

%%% Local Variables:
%%% mode: latex
%%% TeX-master: "../main"
%%% End:
 
