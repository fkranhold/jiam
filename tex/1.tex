Wir setzen im Nachfolgenden einige Grundkonzepte klassischer Musiktheorie als
bekannt voraus, z.\,B. Töne, Intervallnamen, diatonische Skalen, Akkorde,
Tonarten und der Quintenzirkel. Dies wird im Wesentlichen von
\cite[§\,1–6]{Skript} abgedeckt.

Wenn ein Instrument (z.\,B. die menschliche Stimme) einen Ton erzeugt, schwingt
ein „Material“ (z.\,B. eine Saite oder eine Luftsäule) mit einer gewissen
Frequenz.  Je höher die Frequenz ist, desto höher ist der Ton. Ein
\emph{Stimmungssystem} weist jeder Note eine Frequenz zu. Typischerweise
beginnen wir mit der Fixierung des \emph{Kammertons}, also der Frequenz des a’,
heutzutage bei $440$\,Hz. Von dort aus leiten wir die anderen Tonhöhen wie folgt
ab: Die einfachste Beziehung zwischen Tönen ist die der \emph{Oktave}, die einem
Frequenzverhältnis von $1:2$ entspricht, z.\,B. bekommt das a’’ die Frequenz
$880$\,Hz zugewiesen. In der heutzutage üblichen \emph{gleichstufigen Stimmung}
wird die Oktave in zwölf gleich große Schritte, genannt \emph{gleichstufige
  Halbtöne}, aufgeteilt. Ein gleichstufiger Halbton entspricht also dem
Frequenzverhältnis $1:2^{\smash{\frac1{12}}}$. Mit diesem Verhältnis kann nun
jedem Ton eine Frequenz zugewiesen werden; so wird etwa dem e’’ sieben
gleichstufige Halbtöne über dem a’ eine Frequenz von \looseness-1
\[440\,\text{Hz}\cdot 2^{\frac7{12}} \approx 659{,}26\,\text{Hz}\]%
zugewiesen.  An dieser Stelle könnte man erwarten, dass die Erzählung, und damit
auch dieses Skript, ihr Ende haben. Allerdings stellt sich dieses
Stimmungssystem als suboptimal heraus, sobald mehrere Töne gleichzeitig
erklingen. Die Physik™ lehrt uns, dass mehrere Töne konsonieren, wenn ihre
Frequenzen im Verhältnis kleiner, paarweise teilerfremder Ganzzahlen
stehen,\footnote{Eine präzise physikalisch-anatomische Begründung dieses
  Phänomens würde den Rahmen des Kurses sprengen, nur so viel: Helmholtz’
  Schwebungstheorie \cite[§\,10–12]{HE} zufolge ist ein Hauptkriterium für
  wahrgenommene Konsonanz zwischen zwei Tönen der Anteil an Überschneidungen
  ihrer Obertonspektren. Stehen nun die Frequenzen zweier Töne im Verhältnis
  kleiner Brüche zueinander, so gibt es in den entsprechenden Obertonspektren
  der jeweiligen Töne viele solcher Überschneidungen. Unsere Interpretation
  des Wortes „viele“ in diesem Kontext haben wir in \cref{sec:kgv} ausgeführt.
  \vspace*{.4px}}
z.\,B. $4:5:6:7$. Sehen wir uns nun einen Durakkord mit kleiner Septime,
bestehend aus Grundton, großer Terz, reiner Quinte und kleiner Septime, an. Da
in gleichstufiger Stimmung eine große Terz aus $4$, eine reine Quinte aus $7$
und eine kleine Septime aus $10$ gleichstufigen Halbtönen bestehen, ist das
Frequenverhältnis dieser vier gleichzeitig erklingenden Töne \looseness-1%
\[1:2^{\frac4{12}}:2^{\frac7{12}}:2^{\frac{10}{12}}\approx
  1:1{,}260:1{,}498:1{,}781.\]%
Die Beobachtung,\footnote{Streng genommen erzählen wir hier die Geschichte nicht
  in historisch korrekter Reihenfolge, denn es ist mitnichten so, dass im Laufe
  der Musikgeschichte Defizite der gleichstufigen Stimmung erkannt und angepasst
  wurden. Vielmehr kann die Idee ganzzahliger Frequenzverhältnisse als Beginn
  der Musiktheorie gesehen werden, während heutzutage z.\,B. für
  Tasteninstrumente die gleichstufige Stimmung aus Gründen der Praktikabilität
  verwendet wird. Wir haben uns trotzdem für diese Darstellung entschieden, weil
  sie systematischer und zudem leichter zu verstehen ist.} die der Theorie der
reinen Stimmung zugrunde liegt, ist nun, dass dieses Verhältnis zwar
\emph{ungefähr}, aber nicht \emph{exakt} $4:5:6:7$ ist. Die „Unreinheiten“, die
wir hier sehen, sind deutlich kleiner als ein Halbton, weswegen wir ihre
Größe mit Hilfe der Einheit \emph{Cent} ausdrücken, wobei $100\ct$ einem
gleichstufigen Halbton entsprechen. Anders ausgedrückt: Die Oktave (d.\,h. die
Multiplikation mit $2$) umfasst $1{.}200\ct$. Wir können damit
beispielsweise sehen, dass das Verhältnis $4:5$ einem Unterschied von
\[1{.}200\ct\cdot \log_2(\tfrac54) \approx 386{,}31\ct\]%
entspricht, es also von dem vier gleichstufige Halbtöne umfassenden Intervall
($400\ct$) um $13{,}69\ct$ abweicht. Diese Unterschiede „korrigiert“ die
\emph{reine Intonation}, indem sie Akkordbestandteile systematisch
mikroalteriert (engl. „microtune“), also die Töne je nach harmonischem Kontext
ein kleines bisschen höher oder tiefer spielt. Das ist natürlich auf
Musikinstrumenten mit zwölf Tasten pro Oktave unmöglich; dort müssen andere
Stimmungssysteme eingesetzt werden. Nur an der Theorie und nicht an praktischen
Lösungsansätzen interessiert, werden wir letztere hier nicht
ausführen. Unabhängig davon \emph{ist} reine Intonation in der musikalischen
Praxis durchaus einsetzbar, z.\,B. im A-cappella-Gesang: Professionell
ausgebildete Chorsänger:innen hören aufeinander, um möglichst konsonant zu
singen, und stimmen daher ihre Akkorde \emph{intuitiv} aus, damit sie
„einrasten“, siehe etwa \cite[§\,2.4]{Maria} für eine ausführliche Übersicht zu
diversen empirischen Untersuchungen.

Dieses einführende Skript leitet die fürs Ausstimmen notwendigen
Mikroalterationen systematisch her und erklärt, wie sie konsistent eingesetzt
werden können, zeigt aber auch auf, an welchen Stellen dieses System an seine
Grenzen stößt. Dabei gliedert es sich in folgende Abschnitte:\looseness-1

\begin{enumerate}
  \setcounter{enumi}{1}
\item Mit der Einführung der \emph{pythagoreischen Stimmung} erreichen wir, dass
  alle reinen Quinten das Frequenzverhältnis von $2:3$ haben. Dies wird dazu
  führen, dass beispielsweise das \sharp g etwas höher wird als das \flat a;
  dieser Unterschied heißt \emph{pythagoreisches Komma}. Da wir aber nicht an
  ein Tasteninstument gebunden sind, ist das kein Problem; es zwingt uns nur
  dazu, Enharmonik ernst zu nehmen. Der Quintenzirkel wird zu einer unendlichen
  \emph{Quintenspirale}.  Von hier an ist unser Standardstimmungssystem das
  pythagoreische; das e’’ hat also eine Frequenz von
  $\frac32\cdot 440\,\text{Hz}=660\,\text{Hz}$ und nicht mehr
  $659{,}26\,\text{Hz}$.\looseness-1
\item Selbst im pythagoreischen System sind Dreiklänge nicht optimal gestimmt,
  z.\,B. wäre ein Dur-Dreiklang $64:81:96$ statt $4:5:6$. Die große Terz ist ein
  bisschen zu hoch; der Unterschied $\frac{81}{80}$ wird als \emph{syntonisches
    Komma} bezeichnet. Folglich bedeutet das Ausstimmen des Akkords eine
  Absenkung der großen Terz um ein syntonisches Komma. Wir erhalten eine zweite
  Quintenreihe, die um ein syntonisches Komma tiefer liegt. Wenn wir diesen
  Prozess in beide Richtungen wiederholen, können die entstehenden Reihen zum
  \emph{Eulerschen Tonnetz} zusammengesetzt werden, in dem die Bestandteile von
  (Dur- und Moll-)Dreiklängen nebeneinander liegen und Dreiecke bilden. Da
  \mbox{(gegen-)}parallele Dreiklänge benachbarten Dreiecken entsprechen, eignet
  sich dieses Gitter gut für die klassische Harmonik, erklärt aber auch die
  sogenannte \emph{Kommafalle} für bestimmte Akkordfolgen: eine häufige Quelle
  für Detonationen (das ist der Verlust der ursprünglichen Grundfrequenz).
\item Während das Eulersche Tonnetz gut geeignet ist, um \emph{vertikale}
  Intervalle und Klänge zu stimmen, hat es auf der anderen Seite auch
  Auswirkungen auf \emph{horizontale} Intervalle und dadurch auf die
  Melodieführung jeder einzelnen beteiligten Stimme.  Wir wollen uns dieser
  Effekte bewusst werden, um sie bei unseren darauffolgenden Überlegungen zu
  berücksichtigen.
\item Mithilfe des Eulerschen Tonnetzes lassen sich auch andere für die
  klassische Musik übliche Akkorde (etwa die Subdominante mit Sixte ajoutée)
  sinnvoll ausstimmen. Wir werden den Versuch unternehmen, dies konsistent für
  die meisten relevanten Harmonien zu tun.
\item Selbst innerhalb des Eulerschen Tonnetzes kann der oben beschriebene
  Vierklang $4:5:6:7$ nicht realisiert werden – im Wesentlichen weil das Tonnetz
  nur die Primzahlen $2$, $3$ und $5$ kennt. Ein Dur-Dreiklang mit kleiner
  Septime im Eulerschen Tonnetz hat stattdessen das Verhältnis $36:45:54:64$.
  Die kleine Septime ist also ein bisschen zu hoch; der Fehler $\frac{64}{63}$
  wird als \emph{septimales Komma} bezeichnet. Wir werden einige (allerdings rar
  gesäte) Situationen in der klassischen Vokalmusik finden, in denen die
  Verwendung des septimalen Kommas erwogen werden kann.
\end{enumerate}
Die Perspektive, von der aus wir große Teile dieses Skripts schreiben, ist
folgende: \emph{Wie müsste man für einen gegebenen mehrstimmigen Satz jeden
  einzelnen Ton mikroalterieren, um sowohl vertikal als auch horizontal das
  „bestmögliche“ Klangergebnis zu erhalten?} Unsere Formulierungen sind daher
zumeist präskriptiv. Die Behauptung, dass einige dieser Anpassungen intuitiv
umgesetzt werden, steht dazu allerdings in keinem Widerspruch: An den Stellen,
wo eine Abwägung getroffen werden muss, ist auch vom Ensemble ein theoretisches
Verständnis der Situation erforderlich.

% Dieser Ansatz unterscheidet sich in zwei Aspekten von der
% Standardliteratur:\todo{Welche? – \{Das würde mich auch interessieren.\} –T} 
% Erstens machen wir das pythagoreische System zum bevorzugten und verfolgen alle
% Abweichungen davon durch syntonische und septimale Kommata, während
% klassischerweise für jede Tonart eine sogenannte ptolemäische Skala verwendet
% wird, in der die Terz, die Sexte und die Septime standardmäßig um ein
% syntonisches Komma erniedrigt werden. Die Ptolemäische Skala hat den Vorteil,
% dass weniger Kommata verfolgt werden müssen; allerdings, werden wir darlegen,
% aus welchen systematischen Gründen wir uns gegen ihre Verwendung entschieden
% haben. Zweitens wird das septimale Komma in der Regel weggelassen,
% wahrscheinlich weil die durch den Tritonus verursachte Spannung (zwischen der
% kleinen Septime und der großen Terz) hörbar sein soll. Wir sind jedoch
% überzeugt, dass die Beachtung des septimalen Kommas zu einem saubereren System
% führt.

Diese Abhandlung ist mit mehreren musikalischen Beispielen versehen, die am
Seitenrand dieses Dokuments angegeben sind\LY{m-4567} (hier ist ein erstes
Beispiel, in dem ein Dur-Akkord mit kleiner Septime zu hören ist, einmal in
gleichstufiger Stimmung und einmal genau \mbox{$4:5:6:7$}, wie oben
beschrieben). Diese Beispiele wurden mit dem Notensatzprogramm \verb!lilypond!
geschrieben und kodieren entweder eine \acr{PDF}-Datei (dann beginnt ihr
Name mit \verb!p-!), eine \acr{MIDI}-Datei (\verb!m-!) oder beides
(\verb!b-!). Auf\looseness-1
\begin{quote}
  \href{https://github.com/fkranhold/jiam/}{\textsf{github.com/fkranhold/jiam}}
\end{quote}
finden sich sämtliche Quelldateien.  Wir ermutigen alle, mit diesen Dateien
selbst zu experimentieren. Darüber hinaus finden sich die von unserem Quelltext
erzeugten Dateien (\acr{PDF}s, \acr{MIDI}s und \acr{OGG}s) auf:
\begin{quote}
  \href{https://files.fkranhold.de/jiam.tar.gz}{\textsf{files.fkranhold.de/jiam.tar.gz}}
\end{quote}
Eine weitere sehr hilfreiche Ressource beim Nachdenken und Experimentieren ist
folgende Webanwendung, die alle in diesem Skript behandelten (nicht
gleichstufigen) Töne und Intervalle sowie noch unzählige weitere berechnen kann:
\begin{quote}
	\href{https://hejicalc.plainsound.org/}{\textsf{hejicalc.plainsound.org}}
\end{quote}


%%% Local Variables:
%%% mode: latex
%%% TeX-master: "../main"
%%% End:
