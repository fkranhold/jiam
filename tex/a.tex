Wir wollen die \emph{Konsonanz} eines gegebenen Akkords noch besser
quantifizieren, denn bisher ist unser einziges qualitatives Maß „kleine
ganzzahlige Verhältnisse“. Folgen wir Helmholtz’ Schwebungstheorie
\cite[§\,10–12]{HE}, so sollte Konsonanz quantitativ ausgedrückt werden durch die
Menge an Überschneidungen der Obertonspektren der gegebenen Töne. Für eine
gegebene Frequenz $f\in\R_{>0}$ ist sein Obertonspektrum die Menge aller Töne der
Naturtonreihe, also alle positiven ganzzahligen Vielfachen von $f$:
\[\on{Spek}(f) \coloneqq \{n\cdot f;\,n\in\Z_{>0}\}\subset \R_{>0}.\]
Haben wir also einen Akkord, der dem Zahlenverhältnis $k_1:\dotsb:k_r$
entspricht, so heißt dies, dass es eine Grundfrequenz $f_0$ (auch
\emph{imaginierter Grundton} genannt) gibt, für die die Töne des vorliegenden
Akkords genau $k_1\cdot f_0,\dotsc,k_r\cdot f_0$ sind. Der Schnitt ihrer
Obertonspektren ist also gegeben durch
\[{\text{Spek}(k_1\cdot f_0)}\cap\dotsb\cap{\text{Spek}(k_r\cdot f_0)}\]%
und unendlich groß: Es enthält genau die Vielfachen von
$\on{kgV}(k_1,\dotsc,k_r)\cdot f_0$, wobei $\on{kgV}$ das kleinste gemeinsame
Vielfache bezeichnet. Wir können also nicht einfach zählen, wie viele
Überschneidungen es gibt. Stattdessen können wir schauen, wie \emph{dicht}
dieser Schnitt ist; d.\,h. die Menge der Überschneidungen ist größer, wenn die
Schrittweite $\on{kgV}(k_1,\dotsc,k_r)\cdot f_0$ möglichst klein ist.  Dies kann
natürlich leicht erreicht werden, indem man den Akkord sehr tief spielt und
dadurch $f_0$ sehr klein macht. Allerdings sollte unser Maß für Konsonanz
\emph{translationsinvariant} sein, also nicht von der Höhe, in der der Akkord
gespielt wird, abhängen (z.\,B. sollen c’\,:\,\naturalm e’\,:\,g’ und
c’’\,:\,\naturalm e’’\,:\,g’’ gleich konsonant sein). Weil Translation den
imaginierten Grundton $f_0$ verschiebt, aber das kgV invariant lässt, scheint
uns also das kleinste gemeinsame Vielfache der am Verhältnis beteiligten Töne
ein geeignetes Maß für Konsonanz zu sein, und je kleiner diese Zahl ist, desto
konsonanter der Akkord. Ähnliche Überlegungen, bei denen die Konsonanz in
Abhängigkeit des kgV ausgedrückt wird, finden sich bereits bei Euler
\cite{Euler}.

\begin{table}
  \centering
  \begin{tabular}{lcr}
    \toprule
    \thl{Akkord} & \thl{Verhältnis} & \thl{kgV}\\
    \midrule
    Quartvorhalt & $6:8:9$ & $72$\\
    pythagoreischer Durdreiklang & $64:81:96$ & $5{.}184$\\
    pythagoreischer Molldreiklang & $54:64:81$ & $5{.}184$\\
    \midrule
    Durdreiklang & $4:5:6$ & $60$\\
    Molldreiklang & $10:12:15$ & $60$\\
    Erweiterter Nonakkord (Dur) & $4:5:6:9$ & $180$\\
    Erweiterer Nonakkord (Moll) & $20:24:30:45$ & $360$\\
    Sixte ajoutée (Dur) & $12:15:18:20$ & $180$\\
    Septakkord  & $36:45:54:64$ & $8{.}640$\\
    Septnonakkord & $45:54:64:81$ & ~~$25{.}920$\\
    \midrule
    Septakkord mit Naturseptime & $4:5:6:7$ & $420$\\
    Septnonakkord mit Naturseptime & $5:6:7:9$ &  $630$\\
    \bottomrule
  \end{tabular}
  \caption{Einige im Skript besprochene Akkorde und das kgV ihrer Faktoren. Im
    ersten Block verwenden wir nur die pythagoreische Stimmung; im zweiten das
    Tonnetz und im dritten zusätzlich noch die Naturseptime.}\label{tab:kgV}
\end{table}
Berechnen wir für einige im Skript vorgestellten Akkorde das kleinste gemeinsame
Vielfache ihrer Faktoren, so ergibt sich \cref{tab:kgV}. Hierbei fällt ins Auge,
dass die kgV von Dur- und Molldreiklängen (sowohl pythagoreisch als auch im
Tonnetz) gleich sind. Dies ist kein Zufall: Diese Akkorde stimmen bis auf
vertikale Spiegelung überein. Auf der Ebene von Frequenzverhältnisse schickt
vertikale Spiegelung $k_1:\dotsb:k_r$ auf das ebenfalls gekürzte Verhältnis
$k_r':\dotsb:k_1'$, wobei
\[k_i'=\frac{\on{kgV}(k_1,\dotsc,k_r)}{k_{i}}\]%
gilt. Es ist nun eine einfache mathematische Überlegung,\footnote{Seien
  $c\coloneqq\on{kgV}(k_1,\dotsc,k_r)$ und
  $c'\coloneqq \on{kgV}(k_1',\dotsc,k_r')$. Zunächst bemerken wir, dass auch
  $k_1':\dotsb:k_r'$ vollständig gekürzt ist, denn falls ein $d>1$ alle $k_i'$
  teilte, so wäre $k_i'/d\cdot k_i=c/d<c$ ein kleineres gemeinsames Vielfaches
  von $k_1,\dotsc,k_r$, Widerspruch. Ferner ist $c$ ein Vielfaches von allen
  $k_i'=c/k_i$, weswegen $c'\mid c$ gilt. Daher gibt es ein $l\ge 1$ mit
  $c=l\cdot c'$, und wir sehen $k_i=c/k_i'=l\cdot c'/k_i'$. Also sind alle $k_i$ durch
  $l$ teilbar. Weil $k_1:\dotsb:k_r$ vollständig gekürzt war, folgt $l=1$ und
  $c=c'$.} dass die so berechneten $k_1',\dotsc,k_r'$ dasselbe kgV haben.

Abschließend sehen wir, dass „kleine Zahlen“ trügerisch sein können: Nutzte man
im Mollakkord die „septimale kleine Terz“ $6:7$, so entstünde der Dreiklang
$6:7:9$, der zwar übersichtlicher als $10:12:15$ ist, allerdings ein kgV von
$126$ hat.

\enlargethispage{\baselineskip}

%%% Local Variables:
%%% mode: latex
%%% TeX-master: "../main"
%%% End:
