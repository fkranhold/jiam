\todo{Dieser Abschnitt wurde geschrieben, als er noch Teil von §3 war. Wir
  müssen das wohl noch anpassen. FK} Wir bemerken, dass es in pythagoreischer
Stimmung für jeden Ton (z.\,B. \flat a’) eindeutig bestimmte ganze Zahlen
$a,b\in \Z$ gibt, für die das Frequenzverhältnis des gegebenen Tones zum
Kammerton a’ genau $2^a\cdot 3^b$ ist. Umgekehrt gibt es auch für jede dieser
Zahlen $a$ und $b$ einen Ton, der dieses Verhältnis
realisiert.\footnote{Mathematisch gesprochen ist also die Abbildung
  $\Z^2\to \{\text{Töne in pythagoreischer Stimmung}\}$ mit
  $(a,b)\mapsto 2^a\cdot 3^b$ eine Bijektion. Dass die Quintenspirale selbst
  $1$-dimensional ist, liegt daran, dass wir aus ihr die Oktave „rausgerechnet“
  haben, also z.\,B. nicht zwischen \flat a\ und \flat a'\ unterscheiden.}  Um
anzudeuten, dass in pythagoreischer Stimmung nur die Primzahlen bis
einschließlich $3$ vorkommen, bezeichnet man sie auch als \emph{durch $3$
  begrenzt} (engl. „$3$-limit“).  Diese Eigenschaft impliziert, dass
Verhältnisse $k_1:\dotsb:k_r$ in pythagoreischer Stimmung stets der Form
$2^{a_1}3^{b_1}:\dotsb:2^{a_r}3^{b_r}$ mit $a_i,b_i\in\Z_{\ge 0}$ sind. So ist
zum Beispiel $4:5:6$ in pythagoreischer Stimmung nicht realisierbar, da hier die
Primzahl $5$ auftaucht.

Dieser Logik folgend sind die Töne, die unter Zuhilfenahme des syntonischen
Kommas (also mit Helmholtz-Ellis-Versetzungszeichen) beschrieben werden können,
\emph{durch $5$ begrenzt}: Hier gibt es für jeden Ton (z.\,B. \flatp a')
eindeutig bestimmte ganze Zahlen $a,b,c\in\Z$, für die das Frequenzverhältnis
des gegebenen Tones zum Kammerton a’ genau $2^a\cdot 3^b\cdot 5^c$ ist, und
umgekehrt gibt es auch für jede dieser Zahlen $a$, $b$ und $c$ einen Ton, der
dieses Verhältnis realisiert.\footnote{Mathematisch gesprochen ist also die
  Abbildung $\Z^3\to \{\text{Töne mit syntotischem Komma}\}$ mit
  $(a,b,c)\mapsto 2^a\cdot 3^b\cdot 5^c$ eine Bijektion. Dass das Tonnetz selbst
  $2$-dimensional ist, liegt wieder daran, dass wir aus ihm die Oktave
  „rausgerechnet“ haben, also z.\,B. nicht zwischen \flatp a und \flatp a’
  unterscheiden.} Wieder impliziert dies, dass in Frequenzverhältnissen nur
Zahlen der Form $2^a\cdot 3^b\cdot 5^c$ auftreten können, z.\,B. ist $4:5:6:7$ im Tonnetz
nicht realisierbar, da hier die nächste Primzahl $7$ auftaucht.

Man kann sich nun fragen, warum man bei $5$ aufhören sollte. Mathematisch
besteht hierfür kein Grund, und tatsächlich gibt es moderne Musik, die z.\,B. in
„$31$-limit tuning“ geschrieben ist.\todo{Evtl. auf etwas verweisen? War nicht
  z.\,B. dieses eine saggital.org Stück \emph{high limit}? TD}  Für die Belange
klassischer Vokalmusik hingegen genügen die Primzahlen $2$, $3$ und $5$ fast
vollständig; nur ganz selten wollen wir die nächste Primzahl $7$ benutzen, um
die schon angedeutete (sehr) kleine Septime $4:7$ verwenden zu können – dies ist
Gegenstand von \cref{sec:sept}.
