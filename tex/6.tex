Mit der Einführung des syntonischen Kommas können wir nicht nur Dreiklänge
(\cref{sec:tri}) sondern auch viele Vierklänge (\cref{sec:quad}) rein stimmen.
Allerdings haben wir immer noch keine Möglichkeit die bereits mehrfach erwähnte
kleine Septime mit der Größe $4:7=36:63\approx 968.83\,\on{ct}$ darzustellen.
Dieses Intervall nennen wir \emph{Naturseptime}, da es in der Naturtonfolge als
erste Septime vorkommt; es ist $31.18\,\on{ct}$ also fast ein Drittel \acr{HTS}
kleiner als die gleichstufige kleine Septime und klingt deshalb auch deutlich
anders, wie in dem Audiobeispiel nachzuhören ist\LY{m-4567} (zuerst erklingt
ein gleichstufiger Dominantseptakkord, dann einer mit dem Frequenzverhältnis
$4:5:6:7$). Die pythagoreische kleine Septime
$9:16=36:64\approx 996.09\,\on{ct}$ ist $\frac{64}{63}\approx 27.26\,\on{ct}$
größer als die Naturseptime, diesen Unterschied nennen wir das
\emph{septimale Komma}.

Eine Tiefalteration um ein septimales Komma notieren wir mit dem zusätzlichen
\septimal Vorzeichen, es wird vor andere Versetzungszeichen gesetzt, wenn sie
nötig sind. Wir werden keinen Ton um ein septimales Komma \emph{erhöhen},
deshalb müssen wir dafür auch kein Zeichen einführen. Der Dominantseptakkord in
D-Dur besteht also aus a, \sharpm c, e und \septimal g. Der erste und zweite
Teil von \cref{fig:sept} zeigt ein Beispiel für einen Dominantseptakkord ohne
und mit Naturseptime. Bei der Auflösung der Naturseptime um einen Halbton nach
unten kommt es zu einem $20:21\approx 84.47\,\on{ct}$, bei einer Auflösung um
einen Ganzton nach unten zu einem $32:35\approx 155.14\,\on{ct}$ großen Schritt.

\begin{figure}
	\centering
	\LY{b-sept}
	\includegraphics{ly/b-sept}
	\caption{(1) Eine Schlusskadenz mit Dominantseptakkord. Die Septime im
		Tenor wird pythagoreisch intoniert.
		\quad(2) Die gleiche Kadenz mit Naturseptime im Tenor.
		\quad(3) Ein Beispiel mit vermindertem Septakkord ohne Naturseptime.
		\quad(4) Das gleiche Beispiel mit Naturseptime; die kleine None wird
		auch um ein septimales Komma alteriert.}\label{fig:sept}
\end{figure}

\subsection{Verminderter Septakkord}

Wie in \cref{sec:quad} stimmen wir die None als reine kleine Terz über der
Septime, sodass der Akkord dem Frequenzverhältnis $5:6:7:\frac{42}5 = 25:30:35:42$
entspricht. Der Unterschied zwischen den beiden Varianten ist im dritten und
viertes Beispiel in \cref{fig:sept} zu sehen und zu hören. Diese Version des
Akkords ist immer noch symmetrisch geschichtet, aber die mittlere der drei
Terzen ist nicht mehr pythagoreisch ($27:32$), sondern septimal ($6:7$). In C besteht dieser Akkord beispielsweise aus \naturalm h, d, \septimal f und \septimal \flatp a.

\subsection{Dominantseptnonakkord}

\begin{table}
	\centering
	\begin{tabular}{lrrrrr}
		\toprule
		\textsc{Akkord} & \multicolumn{4}{l}{\textsc{Frequenzverhältnis}}\\
		\midrule
		Dur                         & $4$  & $5$  & $6$\\
		Moll                        & $10$ & $12$ & $15$\\
		Quartvorhalt                & $6$  & $8$  & $9$\\
		Hinzugefügte None (Dur)     & $4$  & $5$  & $6$  & $9$\\
		Hinzugefügte None (Moll)    & $20$ & $24$ & $30$ & $45$\\
		Sixte ajoutée (Dur)         & $12$ & $15$ & $18$ & $20$\\
		Sixte ajoutée (Moll)        & $80$ & $96$ & $120$ & $135$\\
		Dominantseptakkord          & $4$  & $5$  & $6$  & $7$\\
		Verminderter Septakkord     &      & $25$ & $30$ & $35$ & $42$\\
		Septnonakkord               &      & $5$  & $6$  & $7$  & $9$\\
		\bottomrule
	\end{tabular}
\end{table}

%%% Local Variables:
%%% mode: latex
%%% TeX-master: "../main"
%%% End:
