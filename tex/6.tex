Wir haben noch immer keine Möglichkeit gesehen, die bereits mehrfach erwähnte
kleine Septime der Größe $4:7\approx 968{,}83\,\on{ct}$ darzustellen (was wie
gesagt daran liegt, dass das Eulersche Tonnetz nur Primzahlen bis $5$ kennt,
vgl. \cref{sec:primes}). Dieses Intervall nennen wir \emph{Naturseptime}, da es
in der „Naturtonfolge“\footnote{Die Naturtonfolge einer gegebenen Grundfrequenz
  ist gegeben durch ihre positiven ganzzahligen Vielfachen. Das Intervall $4:7$
  ist also das zwischen dem $4$. und $7$. Naturton.} als erste Septime vorkommt.
Sie beträgt $968{,}83\,\on{ct}$, ist also fast um \pazofrac13 eines
gleichstufigen Halbtons kleiner als die gleichstufige kleine Septime und klingt
deshalb auch deutlich anders, wie in dem bereits in \cref{sec:int} erwähnten
Audiobeispiel nachzuhören ist\LY{m-4567}.\looseness-1

\subsection{Das septimale Komma}

Um die Naturseptime in die bestehende Helmholtz-Ellis-Notation zu integrieren,
müssen wir ihre Abweichung zur pythagoreischen kleinen Septime betrachten.  Hier
sehen wir, dass die pythagoreische kleine Septime
$9:16=36:64\approx 996{,}09\,\on{ct}$ um $\frac{64}{63}\approx 27{,}26\,\on{ct}$
größer ist als die Naturseptime. Diesen Unterschied bezeichnen wir als
\emph{septimales Komma}.

\enlargethispage{\baselineskip}

Die Tiefalteration einer gegebenen Note um ein septimales Komma notieren wir wie
in \cite{HEJI} mit dem Symbol \septimal.  Dieses wird wie ein Versetzungszeichen
gebraucht. Hat die zu modifizierende Note bereits ein Versetzungszeichen, so
wird \septimal\ links von diesem hinzugefügt. So ist zum Beispiel
\septimal\naturalp f ein f, das um ein syntonisches Komma erhöht und um ein
septimales Komma gesenkt wurde. Da wir keinen Ton um ein septimales Komma
\emph{erhöhen} werden, müssen wir dafür auch kein Zeichen
einführen.% (Die naheliegende Wahl wäre, das Symbol einer
% horizontalen Achse zu spiegeln; so wird das auch in Situationen, wo dies
% erforderlich ist, getan.)

\begin{figure}[h]
  \centering
  \LY{b-harm7}
  \includegraphics{ly/b-harm7}
  \caption{(1+2)~Eine Schlusskadenz mit Dominantseptakkord; einmal mit
    pythagoreischer Septime und einmal mit Naturseptime. \quad(3+4)~Ein Beispiel
    mit Septnonakkord, wieder einmal ohne und einmal mit ohne
    Naturseptime.}\label{fig:harm7}
\end{figure}

\subsection{Septakkorde}

Bei zwei der in \cref{sec:quad} vorgestellten Septakkorden kann es unter
Umständen harmonisch sinnvoll sein, sie unter Verwendung der Naturseptime
auszustimmen.  Dabei müssen wir gar nicht so viel ändern:
% Wenn wir in den Septakkorden aus \cref{sec:quad} nun die pythagoreische durch
% eine Naturseptime ersetzen wollen, müssen wir oft nicht viel ändern:

\begin{enumerate}
\item Der Dominantseptakkord (\cref{sec:dom7syn}) bedarf nach dem Absenken der
  pythagoreischen Septime um ein septimales Komma keiner Anpassung mehr, denn er
  hat schon das Verhältnis $4:5:6:7$. In C-Dur hat dieser Akkord dann die Töne
  g\,:\,\naturalm h\,:\,d\,:\,\septimal f.  Die verminderte Quinte \naturalm
  h\,:\,\septimal f entspricht $5:7 \approx 582{,}51\ct$.  Diese setzt sich aus
  zwei kleinen Terzen unterschiedlicher Größe zusammen, nämlich
  $5:6 \approx 315{,}64\ct$ (\naturalm h\,:\,d) und $6:7\approx 266{,}87\ct$
  (d\,:\,\septimal f).  Dieser Akkord ist berüchtigt für seine Omnipräsenz im
  Barbershop-Gesang (wobei dieses Gerücht einer empirischen Untersuchung
  \cite{Barbershop} nur bedingt standhalten kann).  \looseness-1
  % Wird die Septime abwärts in
  % die Dur- oder Mollterz der Tonika aufgelöst, so entsteht als horizontales
  % Intervall $20:21\approx 84{,}47\ct$ im Falle von Dur bzw.
  % $32:35\approx 155{,}20\ct$ im Falle von Moll,
  % vgl. \cref{sec:septmel}.\looseness-1
% \item\todo{Evtl. weglassen. Zu kompliziert, taucht zu selten auf, und wir lassen
%     ja schon den Herzchenakkord weg.} Im verminderten Septakkord
%   (\cref{sec:dim7syn}) haben wir die kleine None als reine kleine Terz über der
%   Septime definiert. Es liegt also nahe, die None gemeinsam mit der Septime um
%   ein septimales Komma abzusenken. Der Akkord hat das Verhältnis $25:30:35:42$
%   und besteht in C-Dur aus den Tönen \naturalm h\,:\,d\,:\,\septimal
%   f\,:\,\septimal\flatp a. Wie in \cref{sec:dim7syn} erwähnt wurde handelt es
%   sich (laut \cite[S.\,92ff.]{deLaMotte}) bei diesem Akkord in Moll um eine
%   Verschmelzung von s und D. Mit der septimalen Alterierung der Septime und None
%   verlieren wir diese Eigenschaft. In Moll ist dies daher als weiterer Faktor
%   gegen die Naturseptime in der Abwägung für und gegen ihren Einsatz anzuführen.
\item Im Dominantseptnonakkord (\cref{sec:dom79syn}) muss ebenfalls lediglich
  die Septime gesenkt werden; dann entsteht das Verhältnis $5:6:7:9$. In C-Dur
  entspricht dieser Akkord dann \naturalm h\,:\,d\,:\,\septimal f\,:\,a.  Hier
  taucht nun eine sehr große große Terz auf, nämlich $7:9\approx 435{,}08\ct$
  (\septimal f\,:\,a).
\end{enumerate}
Wir bemerken, dass die Absenkung der Septime in diesen beiden Akkorden deutlich
hörbar ist, weil die Naturseptime wirklich \emph{viel} tiefer als die
pythagoreische und erst recht als die mittlerweile wohl sehr gewohnte
gleichstufige ist.

\subsection{Septimale Schritte}
\label{sec:septSteps}

Durch den Einsatz des septimalen Kommas können natürlich auch neue horizontale
Intervalle und vor allem interessante bis unbrauchbare Schritte entstehen:

\begin{enumerate}
\item \emph{Septimaler (diatonischer) Halbton}\\
  Wird die Naturseptime des Dominantseptakkords abwärts in die Durterz der Tonika
  aufgelöst (in C-Dur: \septimal f\,–\,\naturalm e), so entsteht ein
  diatonischer Halbton der Größe $20:21\approx 84{,}47\ct$.  Dies ist der
  kleinste diatonische Halbton, den wir kennen; allerdings ist er größer als der
  kleine chromatische Halbton.
\item \emph{Kleiner septimaler Ganzton}\\
  Wird die Naturseptime des Dominantseptakkords abwärts in die Mollterz der Tonika
  aufgelöst (in c-Moll: \septimal f\,–\,\flatp e), so entsteht ein Ganzton der
  Größe $32:35\approx 155{,}14\ct$.  Dieser ist extrem klein; er liegt fast in der
  \emph{Mitte} zwischen einem gleichstufigen Halbton und einem gleichstufigen
  Ganzton.\looseness-1
\item \emph{Großer septimaler Ganzton}\\
  Wird die Naturseptime von oben (z.\,B. vom Grundton der Dominante) aus
  erreicht (in C-Dur: g\,–\,\septimal f), so entsteht ein Ganzton der Größe
  $7:8\approx 231{,}17\ct$. Dies ist der größte Ganzton, den wir kennen, und für
  die klassische Musik als Schritt eigentlich unbrauchbar,
  siehe \cref{sec:septmel}.
\end{enumerate}

\subsection{Brauchbarkeit}
\label{sec:septmel}

% Da die Naturseptime \emph{sehr} tief ist, können bei ihrem Einsatz melodische
% Probleme auftreten (siehe \cref{sec:melody}).
% Wichtige horizontale Intervalle die vorkommen können sind diese:
% \begin{enumerate}
%   \item Wird die Naturseptime mit einem Halbton von unten erreicht so ist das
%   horizontale Intervall der \emph{septimale (diatonische) Halbton}.\todo{Das
%     \href{https://en.xen.wiki/w/Gallery_of_just_intervals}{XenWiki} führt dieses
%     Intervall als „minor semitone, large septimal chroma“. Ersteres ist nicht
%     wirklich spezifisch und zweiteres ist falsch, weil das Intervall ja
%     diatonisch und nicht chromatisch ist. Weil diese Namen nichts taugen benutze
%     ich diese für uns sinnvolle Bezeichung. TD} Er ist beispielsweise
%   \naturalm e\,–\,\septimal f, hat die Größe $20:21 = 84{,}47\ct$ und ist damit
%   der kleinste diatonische Halbton den wir kennen. Dies ist auch das Intervall
%   mit dem die Septime in Dur (nach unten) aufgelöst wird.
%   \item Wird die Naturseptime mit einem Ganzton von unten erreicht so ist das
%   horizontale Intervall der \emph{kleine septimale Ganzton}.\todo{Auch hier hat
%     das XenWiki eine andere Meinung: „neutral second“. Da aber dieses Intervall
%     für uns ein Ganzton ist und es (auch laut XenWiki) sonst nur den großen
%     $7:8$ septimalen Ganzton gibt erscheint mir „kleiner septimaler Ganzton“
%     sinnvoll. TD} Er ist beispielsweise \flatp e\,–\,\septimal f, hat die Größe
%   $32:35 = 155{,}14\ct$ und ist damit extrem klein für einen Ganzton. Dies ist
%   auch das Intervall mit dem die Septime in Moll (nach unten) aufgelöst wird.
%   \item Erreichen wir die Naturseptime von oben (z.\,B. vom Grundton der
%   Dominante) so erhalten wir den \emph{großen septimalen Ganzton}. Dieser hat
%   die Größe $7:8 = 231{,}17\ct$, ist damit der größte Ganzton den wir kennen und
%   ist z.\,B. \septimal f\,–\,g.
% \end{enumerate}

So schön die beiden oben erwähnten Septakkorde mit Naturseptime vertikal und
mathematisch sein mögen, so problematisch ist ihr Einsatz im Verlauf eines
Stücks. Dies hat im Wesentlichen drei Gründe:
\begin{enumerate}
\item Der Einsatz des septimalen Kommas verstärkt in melodisch gedachten
  Passagen das in \cref{sec:balance} besprochene Problem unterschiedlich großer
  Schritte.
\item Septimale Ganztöne sind melodisch sehr unpassend, weil unser Gehör sie nicht
  einfach als das erkennen kann, was sie sein sollen. Der kleine septimale
  Ganzton ist so klein, dass er fast einen Viertelton tiefer als der
  gleichstufige Ganzton und damit weder als Halb- noch als Ganzton erkennbar
  ist. Der große septimale Ganzton ist $231{,}17\ct$ groß, was sehr nah an den
  $274{,}58\ct$ der reinen übermäßigen Sekunde (in a-Moll etwa \naturalp
  f\,–\,\sharpm g zwischen der Terz von d-Moll und der Terz von E-Dur)
  liegt.\looseness-1%  Zum Vergleich: Der jeweils kleinstmögliche Unterschied zwischen reiner
  % Prime und Halbton, Halbton und Ganzton bzw. kleiner und großer Terz beträgt
  % wenigstens $70\ct$\footnote{Die einzige Ausnahme zu dieser Regel ist die
  %   Differenz zwischen dem großen diatonischen Halbton ($133{,}24\ct$) und dem
  %   kleinen Ganzton ($182{,}40\ct$). Dies ist aber auch der Grund weshalb wir
  %   den großen diatonischen Halbton möglichst nicht verwenden wollen.}; hier
  % sind es gerade einmal $43,41\ct$.  Daher sollten septimale Ganztöne nach
  % Möglichkeit vermieden werden.\looseness-1
\item Wenn der häufige Fall auftritt, in der die Auflösung der Septime
  vorenthalten und ohne Anpassung zum Quartvorhalt wird, ist die entstehende
  Quarte $16:21 \approx 470{,}78\ct$ sehr unrein.  Es ist dann also
  unerlässlich, die Tonhöhe um ein septimales Komma nach \emph{oben}
  anzupassen. Wie schon in \cref{sec:49} argumentiert unterminiert dies aber die
  Idee einer konsonanten Einführung von Vorhaltetönen und sollte daher vermieden
  werden.\looseness-1
\end{enumerate}
Im Gegensatz zu septimalen Ganztönen ist der septimale Halbton hingegen zwar
klein, aber nicht so klein, dass er nicht mehr als solcher erkennbar ist. Die
Auflösung der Naturseptime mit diesem Halbton in eine Durterz ist, weil der
Schritt kleiner ist als sonst, eventuell etwas spannungsreicher (vergleichbar
mit dem höheren Leitton in \cref{sec:ln}, nur von oben und ohne harmonische
Abstriche) und deshalb bei eindeutiger Kadenzbildung womöglich sogar willkommen.
Wenn wir Naturseptimen einsetzen wollen, sollten also folgende Bedingungen
allesamt erfüllt sein:\looseness-1

\enlargethispage{\baselineskip}

\begin{enumerate}[itemsep=4px]
\item Wir befinden uns in einer eindeutigen vertikal gedachten Kadenz.
\item Die Septime wird durch einen Halbton von unten oder durch einen Sprung
  erreicht. (Dies heißt, dass die Naturseptime nicht durch einen septimalen
  Ganzton oder eine septimale Nachjustierung erreicht wird.)
\item Die Septime wird in eine Durterz aufgelöst. (Das heißt, dass durch den
  Einsatz der Naturseptime bei der Auflösung kein septimaler Ganzton
  entsteht.)\looseness-1
\item Die Septime wird nicht zu einem Quartvorhalt gehalten.
\item Die zu stimmende Septime befindet sich nicht in der Melodiestimme.
\end{enumerate}
%
Als besonders abschreckendes Beispiel möge \cref{fig:bad7} dienen, wo fast alle
diese Regeln missachtet werden. Insgesamt stellen wir also fest, dass es in der
klassischen Musik nur sehr wenige Situationen gibt, in denen man den Einsatz der
Naturseptime erwägen könnte.  Es wird wohl nicht wenige Musiker:innen geben, die
von ihrer Verwendung im klassischen Kontext komplett abraten.\looseness-1

\begin{figure}
  \centering
  \LY{b-bad7}
  \includegraphics{ly/b-bad7}
  \caption{(1) Eine einfache Melodie im Sopran, mit vielen Zwischendominanten in
    den Begleitstimmen.\quad (2) Derselbe Satz mit einigen
    Naturseptimen und seltsamen Schritten.}\label{fig:bad7}
\end{figure}

%%% Local Variables:
%%% mode: latex
%%% TeX-master: "../main"
%%% End:
