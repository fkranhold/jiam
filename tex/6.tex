Mit der Einführung des syntonischen Kommas können wir nicht nur Dreiklänge
(\cref{sec:tri}) sondern auch viele Vierklänge (\cref{sec:quad}) rein stimmen.
Allerdings haben wir immer noch keine Möglichkeit die bereits mehrfach erwähnte
kleine Septime mit der Größe $4:7=36:63\approx 968{,}83\,\on{ct}$ darzustellen.
Dieses Intervall nennen wir \emph{Naturseptime}, da es in der Naturtonfolge als
erste Septime vorkommt; es ist $31{,}18\,\on{ct}$ also fast
$\frac13\,\text{Hts}$ kleiner als die gleichstufige kleine Septime und
klingt deshalb auch deutlich anders, wie in dem Audiobeispiel nachzuhören
ist\LY{m-4567} (zuerst erklingt ein gleichstufiger Dominantseptakkord, dann
einer mit dem Frequenzverhältnis $4:5:6:7$). Die pythagoreische kleine Septime
$9:16=36:64\approx 996{,}09\,\on{ct}$ ist
$\frac{64}{63}\approx 27{,}26\,\on{ct}$ größer als die Naturseptime, diesen
Unterschied nennen wir das \emph{septimale Komma}.

Eine Tiefalteration um ein septimales Komma notieren wir mit dem zusätzlichen
\septimal\ Vorzeichen; es wird vor andere Versetzungszeichen gesetzt, wenn sie
nötig sind. Wir werden keinen Ton um ein septimales Komma \emph{erhöhen},
deshalb müssen wir dafür auch kein Zeichen einführen.

\begin{figure}
	\centering
	\LY{b-sept}
	\includegraphics{ly/b-sept}
	\caption{(1)~Eine Schlusskadenz mit Dominantseptakkord. Die Septime im
		Tenor wird pythagoreisch intoniert.
		\quad(2)~Die gleiche Kadenz mit Naturseptime im Tenor.
		\quad(3)~Ein Beispiel mit vermindertem Septakkord ohne Naturseptime.
		\quad(4)~Das gleiche Beispiel mit Naturseptime; die kleine None wird
		auch um ein septimales Komma alteriert.}\label{fig:sept}
\end{figure}

\subsection{Septakkorde}

Wenn wir in den Septakkorden aus \cref{sec:quad} nun die pythagoreische durch
eine Naturseptime ersetzen wollen, müssen wir oft nicht sehr viel ändern:

\begin{enumerate}
  \item Der Dominantseptakkord (\cref{sec:dom7syn}) bedarf nach dem ersetzen der
  pythagoreischen durch die septimale Septime keiner Anpassung mehr, er hat dann
  das Verhältnis $4:5:6:7$ und der Tritonus ist $5:7 = 582{,}51\ct$ groß. In
  C-Dur hat dieser Akkord dann die Töne g\,:\,\naturalm h\,:\,d\,:\,\septimal f.
  \item In dem verminderten Septakkord (\cref{sec:dim7syn}) haben wir die kleine
  None als reine kleine Terz über der Septime definiert. Es ergibt also Sinn
  auch die None um ein septimales Komma zu alterieren. Dies passt auch dazu,
  dass die None wie die Septime mit einem Halbton nach unten aufgelöst wird,
  sodass diese Auflösung in $5:6$ Terzparallelen erfolgt. Der Akkord hat das
  Verhältnis $25:30:35:42$ und besteht in C-Dur aus den Tönen
  \naturalm h\,:\,d\,:\,\septimal f\,:\,\septimal\flatp a. Wie in \cref{sec:dim7syn} erwähnt wurde handelt es sich (laut \cite[S.\,92ff.]{deLaMotte}) bei diesem Akkord in Moll um eine Verschmelzung von s und D. Mit der septimalen Alterierung der Septime und None verlieren wir diese Eigenschaft. In Moll ist dies daher als weiterer Faktor gegen die Naturseptime in der Abwägung für und gegen ihren Einsatz anzuführen.
  \item Der Dominantseptnonakkord (\cref{sec:dom79syn}) muss auch nicht weiter
  verändert werden. Er hat das Verhältnis $5:6:7:9$ und besteht in C-Dur aus
  diesen Tönen: \naturalm h\,:\,d\,:\,\septimal f\,:\,a.
  \item Herzchenakkord?
\end{enumerate}

\subsection{Melodische Perspektive}

Da die Naturseptime \emph{sehr} tief ist, können bei ihrem Einsatz melodische
Probleme auftreten (siehe \cref{sec:melody}).
Wichtige horizontale Intervalle die vorkommen können sind diese:
\begin{enumerate}
  \item Wird die Naturseptime mit einem Halbton von unten erreicht so ist das
  horizontale Intervall der \emph{septimale (diatonische) Halbton}.\todo{Das
    \href{https://en.xen.wiki/w/Gallery_of_just_intervals}{XenWiki} führt dieses
    Intervall als „minor semitone, large septimal chroma“. Ersteres ist nicht
    wirklich spezifisch und zweiteres ist falsch, weil das Intervall ja
    diatonisch und nicht chromatisch ist. Weil diese Namen nichts taugen benutze
    ich diese für uns sinnvolle Bezeichung. TD} Er ist beispielsweise
  \naturalm e\,–\,\septimal f, hat die Größe $20:21 = 84{,}47\ct$ und ist damit
  der kleinste diatonische Halbton den wir kennen. Dies ist auch das Intervall
  mit dem die Septime in Dur (nach unten) aufgelöst wird.
  \item Wird die Naturseptime mit einem Ganzton von unten erreicht so ist das
  horizontale Intervall der \emph{kleine septimale Ganzton}.\todo{Auch hier hat
    das XenWiki eine andere Meinung: „neutral second“. Da aber dieses Intervall
    für uns ein Ganzton ist und es (auch laut XenWiki) sonst nur den großen
    $7:8$ septimalen Ganzton gibt erscheint mir „kleiner septimaler Ganzton“
    sinnvoll. TD} Er ist beispielsweise \flatp e\,–\,\septimal f, hat die Größe
  $32:35 = 155{,}14\ct$ und ist damit extrem klein für einen Ganzton. Dies ist
  auch das Intervall mit dem die Septime in Moll (nach unten) aufgelöst wird.
  \item Erreichen wir die Naturseptime von oben (z.\,B. vom Grundton der
  Dominante) so erhalten wir den \emph{großen septimalen Ganzton}. Dieser hat
  die Größe $7:8 = 231{,}17\ct$, ist damit der größte Ganzton den wir kennen und
  ist z.\,B. \septimal f\,–\,g.
\end{enumerate}

Die septimalen Ganztöne sind melodisch sehr unpassend, da unser Gehör sie nicht
einfach als das erkennen kann, was sie sein sollen. Der kleine septimale Ganzton
ist so klein, dass er fast einen Viertelton tiefer als der gleichstufige Ganzton
und damit weder als Halb- noch als Ganzton erkennbar ist. Der große septimale
Ganzton ist $231{,}17\ct$ groß, was sehr nah an den $274{,}58\ct$ der
übermäßigen Sekunde zwischen None und Terz im (syntonischen) verminderten
Septakkord liegt.todo{\#hartamHiatus TD} Zum Vergleich: Der minimale Unterschied
zwischen Prime und Halbton, Halbton und Ganzton und kleiner und großer Terz sind
immer wenigsten $70\ct$, hier sind es gerade einmal $43,41\ct$. Nach Möglichkeit
wollen wir die septimalen Ganztöne also vermeiden.

Der septimale Halbton hingegen ist zwar klein, aber nicht so klein, dass er nicht mehr als solcher erkennbar ist. Die Auflösung der Naturseptime mit diesem Halbton in eine Durterz ist, weil der Schritt kleiner ist als sonst, eventuell etwas spannungsreicher (vergleichbar mit dem höheren Leitton in \cref{sec:ln} nur von oben und ohne harmonische Abstriche) und deshalb bei eindeutiger Kadenzbildung (je nach Geschmack) sogar willkommen. Wenn der häufige Fall auftritt, in der die Auflösung der Septime vorenthalten und ohne Anpassung zum Quartvorhalt wird, ist die entstehende Quarte $16:21 = 470{,}78\ct$ sehr unrein. Es ist also unerlässlich die Tonhöhe um ein septimales Komma nach \emph{oben} anzupassen\todo{was meiner Meinung nach gegen die Idee von Vorhalten verstößt und deshalb sehr sorgfältig abgewogen werden muss}.

Wenn wir Naturseptimen einsetzen wollen, sollten also folgende Bedingungen möglichst erfüllt sein (priorisiert von oben nach unten):

\begin{enumerate}
  \item Wir befinden uns in einer eindeutigen vertikal gedachten (Phrasen-)Endkadenz.
  \item Die Septime (und kleine None) wird mit einem Halbton von unten oder durch einen Sprung erreicht.
  \item Die Stimme mit der Septime (oder kleinen None) ist nicht die Melodiestimme.
  \item Die Septime wird in eine Durterz aufgelöst (in Moll z.\,B. am Ende in eine \emph{picardische Terz}\todo{aka \acr{PSG} (nicht etwa Paris Saint-Germain, sondern Philip-Schwartz-Gedächtnisterz) TD}).
  \item Es handelt sich nicht um den verminderten Septakkord in Moll.
  \item Die Septime (und kleine None) wird nicht gehalten und zum Quart(-sext-)vorhalt.
\end{enumerate}

\begin{table}
  \centering
  \begin{tabular}{lrrrrr}
    \toprule
    \textsc{akkord} & \multicolumn{4}{l}{\textsc{frequenzverhältnis}}\\
    \midrule
    Dur                               & $4$  & $5$   & $6$                   \\
    Moll                              & $10$ & $12$  & $15$                  \\
    Quartvorhalt                      & $6$  & $8$   & $9$                   \\
    Erweiterter Nonakkord (Dur)       & $4$  & $5$   & $6$    & $9$          \\
    Erweiterter Nonakkord (Moll)      & $20$ & $24$  & $30$   & $45$         \\
    Sixte ajoutée (Dur)               & $12$ & $15$  & $18$   & $20$         \\
    Sixte ajoutée (Moll)              & $80$ & $96$  & $120$  & $135$        \\
    Dominantseptakkord (7-limit)      & $4$  & $5$   & $6$    & $7$          \\
    Dominantseptakkord (5-limit)      & $36$ & $45$  & $54$   & $64$         \\
    Verminderter Septakkord (7-limit) &      & $25$  & $30$   & $35$  & $42$ \\
    Verminderter Septakkord (5-limit) &      & $225$ & $270$  & $320$ & $384$\\
    Septnonakkord (7-limit)           &      & $5$   & $6$    & $7$   & $9$  \\
    Septnonakkord (5-limit)           &      & $45$  & $54$   & $64$  & $81$ \\
    \bottomrule
  \end{tabular}
\end{table}

%%% Local Variables:
%%% mode: latex
%%% TeX-master: "../main"
%%% End:
