Das kontextabhängige Mikroalterieren von Tönen in einem Stück hat nicht nur
Einfluss auf die vertikale Struktur eines Satzes (also die aus den Tönen
bestehenden \emph{Akkorde}), sondern auch auf die horizontale Strukur (also die
aus den Tönen bestehenden \emph{Melodien}). Wir wollen diese Effekte explizit
machen.

\subsection{Schritte}
\label{sec:steps}

In der klassischen Musiktheorie ist ein \emph{Schritt} ein horizontales
Intervall, das 1\,\acr{HTS} oder 2\,\acr{HTS} umfasst.  Klammern wir den
Extremfall einer verminderten Terz aus, kann dies auf drei Weisen enharmonisch
präzise realisiert werden:
\begin{itemize}[itemsep=0em]
\item als übermäßige Prime (etwa c\,–\,\sharp c, 1\,\acr{HTS}), genannt
  \emph{chromatischer \acr{HTS}},
\item als kleine Sekunde (etwa h\,–\,c, 1\,\acr{HTS}), genannt
  \emph{diatonischer \acr{HTS}}.
\item als große Sekunde (etwa c\,–\,d, 2\,\acr{HTS}), genannt
  \emph{Ganztonschritt (\acr{GTS})}.
\end{itemize}
In gleichstufiger Stimmung sind chromatischer und diatonischer Halbtonschritt
beide genau $100\ct$ und der Ganztonschritt $200\ct$ groß.  Schon in
pythagoreischer Stimmung spielt die Enharmonik eine Rolle; hier ist der
chromatische \acr{HTS} ($113{,}69\ct$) ein pythagoreisches Komma größer als der
diatonische ($90{,}22\ct$).

Im harmonischen Kontext und unter Verwendung des Eulerschen Tonnetzes sieht das
nochmal anders aus: \Cref{fig:steps} zeigt einige typische Akkordfolgen in
C-Dur, bei denen verschiedene Schritte in der obersten Stimme
auftreten:\footnote{Die
  \href{https://en.wikipedia.org/wiki/Semitone\#Just_intonation}{englische
    Wikipedia} ist an dieser Stelle irreführend und stellenweise mathematisch
  inkorrekt. Ferner ist uns (jedenfalls mit Dreiklängen) keine Situation
  eingefallen, in der der dort genannte „large diatonic semitone“ realisiert
  würde.}
\begin{enumerate}
\item \emph{Kleiner chromatischer Halbtonschritt}\\
  Wechselt man vom Grundton der Tonika (c) mit einem chromatischen \acr{HTS} in
  die Terz der Zwischendominante zur Subdominantparallele (\sharpmm c), so
  verliert man im Vergleich zum pythagoreischen chromatischen \acr{HTS} ganze
  zwei syntonische Kommata. Der entstehende Halbtonschritt ($70{,}67\ct$)
  entspricht dem Verhältnis $24:25$ und ist sehr klein. Dieser Halbtonschritt
  ist der gleiche wie der zwischen Mollterz (\flatp e) und Durterz
  (\naturalm e) zueinander varianter Dreiklänge.
\item \emph{Großer chromatischer Halbtonschritt}\\
  Wechselt man vom Grundton der Subdominante (f) mit einem chromatischen
  \acr{HTS} in die Terz der Doppeldominante (\sharpm f), so verliert man diesmal
  nur ein syntonisches Komma.  Der entstehende Halbtonschritt
  ($92{,}18\,\on{ct}$) ist recht nah am gleichstufigen, allerdings
  noch immer etwas kleiner.\looseness-1
\item \emph{Diatonischer Halbtonschritt}\\
  Wechselt man von der Terz der Dominante (\naturalm h) mit einem diatonischen
  \acr{HTS} in den Grundton der Tonika (c), so gewinnt man im Vergleich zum
  pythagoreischen diatonischen \acr{HTS} ein syntonisches Komma.  Der
  entstehende Halbtonschritt entspricht dem recht einfachen Verhältnis $15:16$
  und beträgt $111{,}73\ct$.  Wir bemerken, dass dieser Schritt in jedem
  Dominante-Tonika-Verhältnis, also auch bei Zwischendominanten, so auftaucht,
  siehe auch \cref{sec:ln}.
\item \emph{Kleiner Ganztonschritt}\\
  Wechselt man von der Quinte der Tonika mit einem Ganztonschritt in die Terz
  der Subdominante (\naturalm a), so verliert man im Vergleich zur
  pythagoreischen großen Sekunde ein syntonisches Komma. Der entstehende Schritt
  entspricht dem recht einfachen Verhältnis \mbox{$9:10$} und beträgt $182{,}40\ct$,
  ist also recht deutlich unterhalb des gleichstufigen.
\item \emph{Großer Ganztonschritt}\\
  Wechselt man zum Beispiel von der Terz der Subdominante (\naturalm a) zur Terz
  der Dominante (\naturalm h), so bleibt man auf derselben Kommahöhe. Es
  entsteht also ein rein pythagoreischer Ganztonschritt, der dem Verhältnis
  $8:9$ entspricht und $203{,}91\ct$ (also gleichstufiger Ganztonschritt plus
  $2$ Grad) entspricht.  Selbiges gilt für den Wechsel vom Grundton der
  Subdominante zum Grundton der Dominante, eine typische Bassbewegung.
\end{enumerate}
Diese fünf Schritte, ihre Verhältnisse und Großen sind in \cref{tab:steps} noch
einmal kurz und bündig zusammengefasst.  Wir bemerken, dass in allen
auftretenden Situationen chromatische Halbtonschritte anders im pythagoreischen
Fall stets \emph{kleiner} als diatonische Halbtonschritte sind.  Dies liegt
daran, dass in den Situationen, in denen sie auftreten, die Verteilung der
syntonischer Kommata genau entgegengesetzt ist und das pythagoreische Komma
überkompensiert.

Als eine Moral dieses Unterabschnitts könnte man festhalten:
\emph{Halbtonschritt ist nicht gleich Halbtonschritt.}  Dies liefert eine
mathematische Begründung für das Phänomen, dass Laienchöre bei chromatischen
Linien oft detonieren: Es liegt nahe, zu glauben, dass Halbtonschritte stets
gleich groß seien, und auch, dass die Stimme, die die Chromatik singt, die
einfachste Aufgabe habe – schließlich bewegt sie sich ja immer um die „kleinste
Einheit“. Alles an dieser auf den ersten Blick einleuchtenden Vermutung
widerspricht obiger Berechnung.  Stattdessen muss sich die sich chromatisch
bewegende Stimme an den sie umgebenden Harmonien orientieren.\looseness-1

\begin{figure}
  \centering
  \includegraphics{ly/b-steps}\LY{b-steps}
  \caption{Verschiedene Schritte in der Oberstimme: (1) Kleiner chromatischer
    \acr{HTS} c\,–\,\sharpmm c vom Grundton der T zur Terz der Zwischendominante
    zur Sp.\quad (2) Großer chromatischer \acr{HTS} f\,–\,\sharpm f vom Grundton
    der S zur Terz der DD.\quad (3) Diatonischer \acr{HTS} \naturalm h\,–\,c von
    der Terz der D zum Grundton der T.\quad (4) Kleiner \acr{GTS}
    g\,–\,\naturalm a von der Quinte der T zur Terz der S.\quad (5) Großer
    \acr{GTS} \naturalm a\,–\,\naturalm h von der Terz der S zur Terz der
    D.}\label{fig:steps}
\end{figure}

\begin{table}
  \centering
  \begin{tabular}{lr@{\hspace*{1.6px}}lr@{\hspace*{2.4px}}lr}
    \toprule
    \thl{Bezeichnung} & \multicolumn{2}{c}{\thl{Beispiel}} & \multicolumn{2}{c}{\thl{Verhältnis}} & \thl{Größe in ct}\\
    \midrule
    kleiner chromatischer \acr{HTS} & \hspace{3mm}c\,&–\,\sharpmm c & $24$ & $:25$ & $70{,}67$\\
    großer chromatischer \acr{HTS} & f\,&–\,\sharpm f & \hspace*{3mm}$128$ & $:135$ & $92{,}18$\\
    diatonischer \acr{HTS} & \naturalm h\,&–\,c & $15$ & $:16$ & $111{,}73$\\
    kleiner \acr{GTS} & g\,&–\,\naturalm a & $9$ & $:10$ & $182{,}40$\\
    großer \acr{GTS} & f\,&–\,g & $8$ & $:9$ & $203{,}91$\\
    \bottomrule
  \end{tabular}
  \caption{Verschiedene Schritte und ihre Größen}\label{tab:steps}
\end{table}

\subsection{Quintfälle}

Ein weiteres wichtiges horizontales Intervall ist die Quinte (bzw. komplementär
dazu: die Quarte), denn oft setzen sich Basslinien aus ihnen in Form sogenannter
\emph{Quintfälle} zusammen, vgl. \cite[{}12.3]{Skript}. Wir bemerken, dass in
einer üblichen Akkordfolge so ein Quintfall nicht ausschließlich das für
vertikale Intervalle so erstrebenswerte Verhältnis $2:3$ nutzt, sondern zumeist
wenigstens einmal ein Komma abgezogen (bzw. zur Quarte hinzuaddiert) wird, siehe
\cref{fig:5fall}.

Dies ist aber auch nicht besonders tragisch: Ein üblicher Quintfall muss sogar
wenigstens eine \emph{verminderte} Quinte enthalten, um nicht zu modulieren
(hier: von \naturalp f auf h), weswegen man derartige Abweichungen gewohnt ist.

\begin{figure}
  \centering
  \LY{b-5fall}
  \includegraphics{ly/b-5fall}
  \caption{Ein Quintfall im Bass, bei der der Sprung d\,–\,\naturalp g um ein
    syntonisches Komma größer als die reine Quarte $3:4$ ist.  Der Akkord
    h\,:\,\naturalp f\,:\,d\,:\,a im letzten Takt ist die Subdominante mit Sexte
    im Bass, deren Ausstimmung in \cref{sec:sixte} erklärt wird.}\label{fig:5fall}
\end{figure}

\subsection{Leittöne}
\label{sec:ln}

Wir weisen darauf hin, dass die Terz im Dreiklang, die zur Dominante (oder einer
Zwischendominante) gehört, ebenfalls um ein syntonisches Komma abgesenkt wird,
was dem häufig im Instrumentalunterricht vermittelten Ideal widerspricht, den
Leitton (also den Ton, der eine kleine Sekunde unter dem Grundton liegt) höher
zu intonieren, um die „Spannung zu erhöhen“: In unserer mathematischen
Beschreibung liegt der Leitton um einen diatonischen \acr{HTS} und damit
$111{,}73\ct$ unter dem Grundton, was \emph{tiefer} als der gleichstufige
Leitton ist.

Dieser Widerspruch wurde auch in \cite[211]{viitasaari} benannt und
diskutiert, mit dem Ergebnis, dass es stark auf den Kontext ankommt: In
homophonen Sätzen, bei denen die \emph{vertikale} Struktur der Musik im
Vordergrund steht, sollte die Dominante ein eigenständiger, stabiler
Dur-Dreiklang sein und der Leitton somit tief intoniert werden. Steht hingegen
die melodische Bewegung einzelner Stimmen, also die \emph{horizontale} Struktur
im Vordergrund (wir merken an, dass dies in keinem unserer bisherigen Beispiele
der Fall war), so kann man in Erwägung ziehen, den Leitton pythagoreisch zu
intonieren, wodurch er nur noch $90{,}22\ct$ unter dem Grundton und damit über
dem gleichstufigen Leitton läge. Der entstehende vertikale Dominantdreiklang
hätte dann das Verhältnis $64:81:96$. In \cref{fig:lead} hören wir zwei
Versionen einer Kadenz, eine mit einem rein intonierten Leitton, eine mit einem
pythagoreischen Leitton.

\begin{figure}
  \centering
  \LY{b-lead}
  \includegraphics{ly/b-lead}
  \caption{Zwei Arten, eine Kadenz zu intonieren: Einmal mit reiner
    Intonation aller Dreiklänge und einmal, in einem bewegteren Kontext, mit
    einem pythagoreischen Leitton.}\label{fig:lead}
\end{figure}

\subsection{Sinnvoll abwägen}

Die Diskussion um den Leitton aus dem vorherigen Abschnitt ist nur ein Beispiel
eines allgemeineren Priorisierungsproblems, das wir bisher absichtsvoll umgangen
sind: Alle bisherigen Beispiele haben homophone Sätze betrachtet, bei denen der
Aspekt der \emph{Bewegung} der einzelnen Stimmen komplett nachrangig
war. Außerdem haben wir unsere Beispiele stets langsam genug gespielt, um uns
Zeit zu geben, wahrzunehmen, wie die Akkorde einrasten. Dies ist dadurch
gerechtfertigt, dass tatsächlich ein gehöriger Teil klassischer Chormusik auf
diese Weise gedacht ist und musiziert wird; insbesondere \emph{enden} viele
Stücke auf diese Weise. Es blendet allerdings aus, dass auch Vokalmusik ganz
anders konzipiert sein kann.

Als extremes Gegenbeispiel mag \cref{fig:melody} dienen, bei der – obwohl auf
den einzelnen Schlägen vertikal funktionale Harmonien klar erkennbar sind – die
Stimmen ständig in Bewegung sind und dem Akkord gar keine Zeit gegeben wird,
einzurasten. Außerdem bekommen die aufsteigenden Linien eine „Richtung“, durch
die (zumindest unserer\todo{so far: meiner. FK} Meinung nach) eine abgesenkte
Sexte oder Septime künstlich wirkt.  In solchen Situationen ist es
empfehlenswert, auf entsprechende syntonische Kommata zu verzichten und
stattdessen stellenweise komplett pythagoreisch zu intonieren.  Wir empfehlen,
die Alt- oder Bassstimme einmal selbst zu singen und mit den Hörbeispielen zu
vergleichen.  Man könnte nun fragen, warum wir dann nicht direkt gleichstufig
intonieren.  Unser Hauptgrund, pythagoreisch zu intonieren, ist, dass es uns
ermöglicht, an \emph{ausgewählten} Stellen doch Kommata zu verwenden, um sie
rein zu stimmen – hier die letzten beiden Akkorde, bei denen eine Kadenz
erkennbar ist und wo deutlich weniger Bewegung stattfindet.

\begin{figure}
  \centering
  \LY{b-melody}
  \includegraphics{ly/b-melody}
  \caption{Eine melodisch gedachter Abschnitt, einmal
    vollständig rein ausgestimmt (nicht empfohlen) und einmal über weite
    Strecken pythagoreisch gestimmt.}\label{fig:melody}
\end{figure}

Wieder anders sieht die Situation in \cref{fig:inter} aus: Dieses Beispiel ist
im Wesentlichen homophon und langsam – nur vereinzelt gibt es Zwischennoten
(hier: Durchgangsnoten und angesprungene Nebennoten). Hier empfiehlt es sich,
zunächst die Akkorde rein zu stimmen und sich erst dann mit den Zwischennoten zu
befassen. Um ihre Kommahöhe zu ermitteln, scheinen uns folgende Kriterien
sinnvoll, mit absteigender Priorisierung:
\begin{enumerate}
\item Alle auftretenden Schritte sollten entweder pythagoreisch oder in \cref{tab:steps}
  gelistet sein.
\item Es sollte keine Abfolge $x$\,–\,$y$\,–\,$x$ von Noten entstehen, bei der
  die beiden Instanzen von $x$ unterschiedlich syntonisiert sind.
\item Die Zwischennote sollte von der vorherigen Note aus durch ein
  pythagoreisches Intervall erreicht werden.
\end{enumerate}
So würde Kriterium 3 in obigem Beispiel für die erste Durchgangsnote h im Bass
die gleiche Kommahöhe wie beim c nahelegen; dies wird aber von Kriterium 1
verunmöglicht,weil dadurch der Ganztonschritt zum \naturalm a zu groß würde. Die
zweite Zwischennote, das g im Alt, wird vom \naturalm e aus erreicht, weswegen
Kriterium 3 \naturalm g nahelegt. Dies ist vereinbar mit den Kriterien 1 und 2,
da der darauffolgende Schritt zum f ein kleiner Ganztonschritt ist. Die dritte
Zwischennote, das c im Sopran, taucht nach zur einem weiteren Ton (dem \naturalm
h) ohne Mikroalteration auf. Daher legt Kriterium 2 \natural c nahe, was mit
Kriterium 1 vereinbar ist.

\begin{figure}
  \centering
  \LY{b-inter}
  \includegraphics{ly/b-inter}
  \caption{Ein im Wesentlichen homophoner Satz, mit vereinzelten
    Zwischennoten. Die Akkorde sind rein gestimmt; für die Zwischennoten gibt
    es unterschiedliche Anhaltspunkte.}\label{fig:inter}
\end{figure}

%%% Local Variables:
%%% mode: latex
%%% TeX-master: "../main"
%%% End:
