In \cref{sec:tri} haben wir gelernt, wie man Stücke stimmt, bei denen jede Note
Teil eines Dur- oder Molldreiklangs ist. Akkorde haben jedoch sehr oft andere
Bestandteile als nur Grundton, Terz und Quinte. In diesem Abschnitt besprechen
wir die wichtigsten Variationen von Dur- und Moll-Akkorden und wie sie gestimmt
werden können.

\subsection{Quartvorhalt and hinzugefügte None}
\label{sec:49}

Der Akkord mit Quartvorhalt besteht aus dem Grundton, der Quarte und der Quinte,
wobei sich die Quarte in der Regel auf dem nächsten Schlag in die (große oder
kleine) Terz auflöst. Der Quartvorhalt braucht keine Korrektur durch ein
syntonisches Komma relativ zum Grundton; der Akkord entspricht $6:8:9$ in der
pythagoreischen Stimmung.\todo{Man beachte, dass die Quarte in der
  pythagoreischen Stimmung ($3:4$) \emph{rein} ist, außerdem bedarf es keiner
  Korrektur, wenn die vorenthaltene Quarte aus der Subdominante  entsteht, wo
  sie der Grundton war. TD}
Die Auflösung der Quarte in die große Terz ist ein Halbton im Verhältnis
$15:16\approx 111{,}73\,\on{ct}$,\todo{Genau der Schritt den wir auch bei der
  Auflösung des Leittons haben; auf \cref{sec:ln} verweisen? TD}
die Auflösung der Quarte in die kleine Terz ist ein Ganzton im Verhältnis
$9:10\approx 182{,}40\,\on{ct}$.\todo{Überbindung ohne korrektur. FK}


Der \emph{erweiterte Nonakkord}\todo{Ich habe den jetzt manchmal mit
  „erweiterter Nonakkord“, manchmal mit „[Akkord mit] hinzugefügte[r] None“ und
  manchmal nur mit „Nonakkord“ übersetzt, weil ich tatsächlich nicht weiß wie
  der im Deutschen heißt. Wenn irgendwas davon falsch ist, kann ich das noch mal
  korrigieren. TD}
besteht aus einem (Dur- oder Moll-)Akkord und einer zusätzlichen großen None.
Die None liegt eine reine Quinte über der Quinte des Dreiklangs und hat daher
die gleiche Anzahl an Kommata wie der Grundton und die Quinte. Dieser Akkord
entspricht dem Verhältnis $4:5:6:9$ in Dur und $20:24:30:45$ in Moll. Wir
stellen fest, dass der Halbton zwischen der großen None und der kleinen Terz
(plus eine Oktave) wieder $15:16$ ist.

In \cref{fig:chordLines} haben wir die obigen Akkorde als Unterkomplexe des
Eulerschen Tonnetzes dargestellt. Wir markieren nicht nur die Eckpunkte, sondern
auch alle direkten Kanten zwischen ihnen, da jede Kante für eine Beziehung der
Form $2:3$, $4:5$ oder $5:6$ innerhalb des jeweiligen Akkordes steht.

\subsection{Sixte ajoutée}

\todo{Auf \cite[Abschnitt 9.3]{Skript} verweisen? TD}
Sehr oft kommt der Subdominantakkord (sowohl in Dur als auch in Moll) mit der
großen Sexte über dem Grundton des Akkords zusammen vor
(z.\,B. f\,-\,a\,-\,c\,-\,d in C-Dur oder f\,-\,\flat a\,-\,c\,-\,d in c-Moll).
Im Fall von Dur ist dieser Akkord eine Verschmelzung von zwei Dreiklängen, die
sich eine Terz teilen (im Beispiel: F-Dur und d-Moll). Dementsprechend verwenden
wir die Ecken der benachbarten Dreiecke $\vartriangle\!\!\!\triangledown$, um
die Stimmung der Einzeltöne abzuleiten (siehe \cref{fig:chordLines}). Wir
erhalten das Verhältnis $12:15:18:20$. Wenn auf den Sixte ajoutée die Dominante
folgt, muss die große Sexte beim Harmoniewechsel angepasst werden (siehe
\cref{fig:496}).\todo{Das andere aber wichtiger. FK}

In Moll ist die Situation komplizierter, da der Sixte ajoutée keine
Verschmelzung wie oben ist. Wir argumentieren, warum wir die große Sexte
verwenden sollten die auf der \emph{gleichen} Kommastufe wie der Grundton und
die Quinte und nicht ein Komma tiefer:

\begin{itemize}
\item \naturalm $6$ ist nicht in einem Dreiklang in der harmonischen Nähe der 
  Tonika enthalten; dies war beim Dur-Fall\footnote{Höhö Dur-„Kadenz“} anders.
\item Die \natural $6$ der Subdominante ist genau die \natural $5$ der 
  Dominante, also ist sogar\todo{Warum „sogar“? Habe ich irgendwas verpasst? TD}
  wenn auf den Sixte ajoutée die Dominante folgt, bei der eine Stimme den
  gemeinsamen Ton hält, keine Anpassung nötig.
\item The Tritonus zwischen \flatp $3$ und \naturalm $6$ ist sehr klein, nämlich
  $18:25\approx 568{,}71\,\on{ct}$, also $31{,}29\,\on{ct}$ kleiner als der
  gleichstufige.\todo{Warum ist das ein Argument? Gleichstufige Stimmung war
    vorher noch nie ein Argument. Ist das evtl. weil der Tritonus anders als
    andere Intervalle dissonant sein \emph{soll} und deshalb Nähe zu
    $1:\sqrt{2}$ erwünscht ist? TD}
\item Der Halbton zwischen der \flatp $3$ der Tonika und der \naturalm $6$
  der Subdominante ist $25:27\approx 133{,}28\,\on{ct}$, was viel zu groß ist.
  In dem dritten Teil von \cref{fig:496}, taucht dieser große Halbton als
  horizontales Intervall im Tenor auf.
\end{itemize}
Der sich daraus ergebende Akkord hat das Verhältnis $80:96:120:135$, und der
Tritonus zwischen der kleinen Terz und der großen Sexte beträgt
$32:45\approx 590{,}22\,\on{ct}$, was der gleichstufigen Stimmung
($600\,\on{ct}$) viel näher kommt als $18:25$.\looseness-1

\begin{figure}
  \centering
  \LY{b-496}
  \includegraphics{ly/b-496}
  \caption{Vier Beispiele, darunter der Quartvorhalt, der Nonakkord und der
  	Sixte ajoutée. Der Unterschied zwischen dem zweiten und dem dritten Beispiel
  	ist nur eine einzige Note im Tenor: In der ersten Version wird die von uns
  	vorgeschlagene große Sexte verwendet, in der zweiten Version die ein
  	syntonisches Komma tiefere, die zu dem horizontalen Intervall $25:27$
  	führt.}\label{fig:496}
\end{figure}

\begin{figure}
  \centering
  \begin{tikzpicture}[scale=.7]
  %% Suspended fourth major
  %  Euler lattice
  \draw[black!50] (0, 0) -- (4, 0);
  \draw[black!50] (1, {sqrt(3)}) -- (3, {sqrt(3)});
  \draw[black!50] (1,-{sqrt(3)}) -- (3,-{sqrt(3)});
  \draw[black!50] (0,0) -- (1, {sqrt(3)}) -- (2,0) -- (3, {sqrt(3)}) -- (4,0);
  \draw[black!50] (0,0) -- (1,-{sqrt(3)}) -- (2,0) -- (3,-{sqrt(3)}) -- (4,0);
  %  Just third and fifth relations
  \draw[black,line width=1mm] (0,0) -- (4,0);
  %  Circles for chord notes labels
  \draw[fill=white] (0,0)          circle (.3);
  \draw[fill=white] (2,0)          circle (.3);
  \draw[fill=white] (4,0)          circle (.3);
  \draw[fill=white] (3,-{sqrt(3)}) circle (.3);
  %  Chord notes labels
  \node at (0,0)              {\scriptsize $4$};
  \node at (2,0)              {\scriptsize $1$};
  \node at (4,0)              {\scriptsize $5$};
  \node at (3,-{sqrt(3)})     {\scriptsize $3$};
  %  Semitone resolutions
  \draw[white,line width=1mm] (.3,{-sqrt(3)*1/10}) -- (2.7,{-sqrt(3)*9/10});
  \draw[-stealth]             (.3,{-sqrt(3)*1/10}) -- (2.7,{-sqrt(3)*9/10});
  %  Label of chord
  \node[rotate=0] at (-1.5,0) {\small $\strut$Dur}; % For rotated text set rotate option to 90 instead of 0
  \node at (2,2.5)            {\small $\strut$Quartvorhalt};
  
  
  %% Extended ninth major
  %  Euler lattice
  \draw[shift={(5.5,0)},black!50] (0, 0) -- (4, 0);
  \draw[shift={(5.5,0)},black!50] (1, {sqrt(3)}) -- (3, {sqrt(3)});
  \draw[shift={(5.5,0)},black!50] (1,-{sqrt(3)}) -- (3,-{sqrt(3)});
  \draw[shift={(5.5,0)},black!50] (0,0) -- (1, {sqrt(3)}) -- (2,0) -- (3, {sqrt(3)}) -- (4,0);
  \draw[shift={(5.5,0)},black!50] (0,0) -- (1,-{sqrt(3)}) -- (2,0) -- (3,-{sqrt(3)}) -- (4,0);
  %  Just third and fifth relations
  \draw[shift={(5.5,0)},line width=1mm] (0,0) -- (4,0);
  \draw[shift={(5.5,0)},line width=1mm] (0,0) -- (1,-{sqrt(3)}) -- (2,0);
  %  Circles for chord notes labels
  \draw[shift={(5.5,0)},fill=white] (0,0)          circle (.3);
  \draw[shift={(5.5,0)},fill=white] (2,0)          circle (.3);
  \draw[shift={(5.5,0)},fill=white] (1,-{sqrt(3)}) circle (.3);
  \draw[shift={(5.5,0)},fill=white] (4,0)          circle (.3);
  %  Chord notes labels
  \node[shift={(3.85,0)}] at (0,0)          {\scriptsize $1$};
  \node[shift={(3.85,0)}] at (1,-{sqrt(3)}) {\scriptsize $3$};
  \node[shift={(3.85,0)}] at (2,0)          {\scriptsize $5$};
  \node[shift={(3.85,0)}] at (4,0)          {\scriptsize $9$};
  %  Label of chord
  \node[shift={(3.85,0)}] at (2,2.5)        {\small $\strut$Hinzugefügte None};
  
  
  %% Sixte ajoutée major
  %  Euler lattice
  \draw[shift={(11,0)},black!50] (0, 0) -- (4, 0);
  \draw[shift={(11,0)},black!50] (1, {sqrt(3)}) -- (3, {sqrt(3)});
  \draw[shift={(11,0)},black!50] (1,-{sqrt(3)}) -- (3,-{sqrt(3)});
  \draw[shift={(11,0)},black!50] (0,0) -- (1, {sqrt(3)}) -- (2,0) -- (3, {sqrt(3)}) -- (4,0);
  \draw[shift={(11,0)},black!50] (0,0) -- (1,-{sqrt(3)}) -- (2,0) -- (3,-{sqrt(3)}) -- (4,0);
  %  Just third and fifth relations
  \draw[shift={(11,0)},line width=1mm] (2,0) -- (4,0);
  \draw[shift={(11,0)},line width=1mm] (1,-{sqrt(3)}) -- (3,-{sqrt(3)});
  \draw[shift={(11,0)},line width=1mm] (1,-{sqrt(3)}) -- (2,0) -- (3,-{sqrt(3)}) -- (4,0);
  %  Circles for chord notes labels
  \draw[shift={(11,0)},fill=white] (1,-{sqrt(3)}) circle (.3);
  \draw[shift={(11,0)},fill=white] (2,0)          circle (.3);
  \draw[shift={(11,0)},fill=white] (3,-{sqrt(3)}) circle (.3);
  \draw[shift={(11,0)},fill=white] (4,0)          circle (.3);
  %  Chord notes labels
  \node[shift={(7.7,0)}] at (2,0)          {\scriptsize $1$};
  \node[shift={(7.7,0)}] at (3,-{sqrt(3)}) {\scriptsize $3$};
  \node[shift={(7.7,0)}] at (4,0)          {\scriptsize $5$};
  \node[shift={(7.7,0)}] at (1,-{sqrt(3)}) {\scriptsize $6$};
  %  Label of chord
  \node[shift={(7.7,0)}] at (2,2.5)        {\small $\strut$Sixte ajoutée};
  
  %% Suspended fourth minor
  %  Euler lattice
  \draw[shift={(0,-4.5)},black!50] (0, 0) -- (4, 0);
  \draw[shift={(0,-4.5)},black!50] (1, {sqrt(3)}) -- (3, {sqrt(3)});
  \draw[shift={(0,-4.5)},black!50] (1,-{sqrt(3)}) -- (3,-{sqrt(3)});
  \draw[shift={(0,-4.5)},black!50] (0,0) -- (1, {sqrt(3)}) -- (2,0) -- (3, {sqrt(3)}) -- (4,0);
  \draw[shift={(0,-4.5)},black!50] (0,0) -- (1,-{sqrt(3)}) -- (2,0) -- (3,-{sqrt(3)}) -- (4,0);
  %  Just third and fifth relations
  \draw[shift={(0,-4.5)},black,line width=1mm] (0,0) -- (4,0);
  %  Circles for chord notes labels
  \draw[shift={(0,-4.5)},fill=white] (0,0)          circle (.3);
  \draw[shift={(0,-4.5)},fill=white] (2,0)          circle (.3);
  \draw[shift={(0,-4.5)},fill=white] (4,0)          circle (.3);
  \draw[shift={(0,-4.5)},fill=white] (3,+{sqrt(3)}) circle (.3);
  %  Semitone resolutions
  \draw[shift={(0,-4.5)},white,line width=1mm] (.3,{+sqrt(3)*1/10}) -- (2.7,{+sqrt(3)*9/10});
  \draw[shift={(0,-4.5)},-stealth]             (.3,{+sqrt(3)*1/10}) -- (2.7,{+sqrt(3)*9/10});
  %  Chord notes labels
  \node[shift={(0,-3.15)}] at (0,0)          {\scriptsize $4$};
  \node[shift={(0,-3.15)}] at (2,0)          {\scriptsize $1$};
  \node[shift={(0,-3.15)}] at (4,0)          {\scriptsize $5$};
  \node[shift={(0,-3.15)}] at (3,+{sqrt(3)}) {\scriptsize \flat $3$};
  %  Label of chord
  \node[rotate=0]          at (-1.5,-4.5)    {\small $\strut$Moll}; % For rotated text set rotate option to 90 instead of 0
  
  
  %% Extended ninth minor
  %  Euler lattice
  \draw[shift={(5.5,-4.5)},black!50] (0, 0) -- (4,0);
  \draw[shift={(5.5,-4.5)},black!50] (1, {sqrt(3)}) -- (3, {sqrt(3)});
  \draw[shift={(5.5,-4.5)},black!50] (1,-{sqrt(3)}) -- (3,-{sqrt(3)});
  \draw[shift={(5.5,-4.5)},black!50] (0,0) -- (1, {sqrt(3)}) -- (2,0) -- (3, {sqrt(3)}) -- (4,0);
  \draw[shift={(5.5,-4.5)},black!50] (0,0) -- (1,-{sqrt(3)}) -- (2,0) -- (3,-{sqrt(3)}) -- (4,0);
  %  Just third and fifth relations
  \draw[shift={(5.5,-4.5)},line width=1mm] (0,0) -- (4,0);
  \draw[shift={(5.5,-4.5)},line width=1mm] (0,0) -- (1,+{sqrt(3)}) -- (2,0);
  %  Circles for chord notes labels
  \draw[shift={(5.5,-4.5)},fill=white] (0,0)          circle (.3);
  \draw[shift={(5.5,-4.5)},fill=white] (2,0)          circle (.3);
  \draw[shift={(5.5,-4.5)},fill=white] (1,+{sqrt(3)}) circle (.3);
  \draw[shift={(5.5,-4.5)},fill=white] (4,0)          circle (.3);
  %  Chord notes labels
  \node[shift={(3.85,-3.15)}] at (0,0)          {\scriptsize $1$};
  \node[shift={(3.85,-3.15)}] at (1,+{sqrt(3)}) {\scriptsize \flat $3$};
  \node[shift={(3.85,-3.15)}] at (2,0)          {\scriptsize $5$};
  \node[shift={(3.85,-3.15)}] at (4,0)          {\scriptsize $9$};
  
  
  %% Sixte ajoutée minor
  %  Euler lattice
  \draw[shift={(11,-4.5)},black!50] (0, 0) -- (6, 0);
  \draw[shift={(11,-4.5)},black!50] (1, {sqrt(3)}) -- (5, {sqrt(3)});
  \draw[shift={(11,-4.5)},black!50] (1,-{sqrt(3)}) -- (5,-{sqrt(3)});
  \draw[shift={(11,-4.5)},black!50] (0,0) -- (1, {sqrt(3)}) -- (2,0) -- (3, {sqrt(3)}) -- (4,0) -- (5, {sqrt(3)}) -- (6,0);
  \draw[shift={(11,-4.5)},black!50] (0,0) -- (1,-{sqrt(3)}) -- (2,0) -- (3,-{sqrt(3)}) -- (4,0) -- (5,-{sqrt(3)}) -- (6,0);
  %  Just third and fifth relations
  \draw[shift={(11,-4.5)},line width=1mm]                (2,0) -- (0,0);
  \draw[shift={(11,-4.5)},line width=1mm]                (2,0) -- (1,+{sqrt(3)}) -- (0,0);
  %  8:9 whole tone relations
  \draw[shift={(11,-4.5)},line width=1mm,densely dashed] (2,0) -- (6,0);
  %  Circles for chord notes labels
  \draw[shift={(11,-4.5)},fill=white] (6,0)          circle (.3);
  \draw[shift={(11,-4.5)},fill=white] (0,0)          circle (.3);
  \draw[shift={(11,-4.5)},fill=white] (1,+{sqrt(3)}) circle (.3);
  \draw[shift={(11,-4.5)},fill=white] (2,0)          circle (.3);
  %  Chord notes labels
  \node[shift={(7.7,-3.15)}] at (0,0)          {\scriptsize $1$};
  \node[shift={(7.7,-3.15)}] at (1,+{sqrt(3)}) {\scriptsize \flat $3$};
  \node[shift={(7.7,-3.15)}] at (2,0)          {\scriptsize $5$};
  \node[shift={(7.7,-3.15)}] at (6,0)          {\scriptsize $6$};
\end{tikzpicture}%

%%% Local Variables:
%%% mode: latex
%%% TeX-master: "../main"
%%% End:

  \caption{Quartvorhalt, Nonakkord und Sixte ajoutée in Dur
    (oben) und Moll (unten), visualisiert als Unterkomplexe des Eulerschen
    Tonnetzes.}\label{fig:chordLines}
\end{figure}

\subsection{Dominantseptakkord}
\label{sec:dom7syn}

\todo{Auf \cite[Abschnitt 9.1]{Skript} verweisen? TD}
Kein Vierklang im Eulerschen Tonnetz kann das gewünschte Intervall $4:7$
enthalten (siehe \cref{sec:primes}).\todo{Allerdings haben wir (insbesondere du)
  auch mal gesagt, dass die Naturseptime nicht immer gut klingt. Deshalb sollten
  wir trotzdem die $9:16$ Septime als Alternative einführen. Ich schreibe mal
  ein Beispiel hier: TD}

Dennoch wollen wir eine rein im Eulerschen Tonnetz beheimatete
Intonationsmöglichkeit vorstellen. Der Dominantseptakkord besteht aus der
Dominante mit einer kleinen Septime, die unabhängig vom Tongeschlecht
leitereigen ist. In der harmonischen Nähe des Dominantakkordes (in C-Dur oder
-Moll wäre das G-Dur) gibt es zwar zwei unterschiedliche kleine Septimen (im
Beispiel \natural f und \naturalp f), aber nur die auf der syntonischen Höhe
des Grundtons ist sinnvoll, denn falls die Septime beim Harmoniewechsel in
einem Quartvorhalt vorenthalten wird ist keine Anpassung nötig und die andere
kommt (als Terz der Doppeldominantvariante) in der Tonart praktisch nicht vor.
Der so entstehende Akkord hat das Frequenzverhältnis $36:45:54:64$. Der Tritonus
zwischen der großen Terz und der kleinen Septime ist
$45:64\approx 609{,}78\,\on{ct}$ groß. Die Auflösung der Septime der Dominante
in die große oder kleine Terz der Tonika erfolgt wie beim Quartvorhalt (siehe
\cref{sec:49}) mit einem diatonischen Halbton $15:16$, oder einem kleinen
Ganzton $9:10$.

\subsection{Verminderter Septakkord}

\todo{Auf \cite[Abschnitt 11.1]{Skript} verweisen? TD}
Auch für den Fall des verminderten Septakkordes werden wir in \cref{sec:sept}
eine Alternative entwickeln, die wir aber nicht immer einsetzen wollen.

Der verminderte Septakkord besteht aus der großen Terz, reinen Quinte, kleinen
Septime und kleinen None eines (Zwischen-)Dominantakkordes (am Beispiel C-Dur
oder -Moll: h\,-\,d\,-\,f\,-\,\flat a). Es ist also im Wesentlichen ein
Dominantseptakkord, bei dem der Grundton (= die Oktave) durch eine kleine
None ausgetauscht wird. Es ergibt also Sinn Terz, Quinte und Septime wie im
vorherigen Fall zu intonieren. Für die None kommt in der harmonischen Nähe nur
ein Kandidat in Frage, nämlich \flatp $9$. Diese None ist eine reine kleine Terz
über der Septime (so, wie auch die Terz eine reine kleine Terz unter der Quinte
ist) und bildet zur Quinte den gleichen Tritonus, den auch die Septime zur Terz
bildet, der Akkord ist also in symmetrischen Terzen geschichtet: Terz zu Quinte
$5:6$, Quinte zu Septime $27:32$ und Septime zu None $5:6$; insgesamt
$45:54:64:\frac{384}5 = 225:270:320:384$. Die None wird ebenfalls mit dem
diatonischen $15:16$ Halbton nach unten aufgelöst.

Im Moll-Fall ist dieser Akkord ähnlich dem Sixte ajoutée in Dur auch eine
Vermischung von zwei Dreiklängen (in c-Moll wäre der Akkord \naturalm
h\,-\,d\,-\,f\,-\,\flatp a). Die Terz und die Quinte sind die Terz und die Quinte der
Dominante (\naturalm h und d) und die Septime und None sind der Grundton und
die Terz der Subdominante (f und \flatp a), was unsere Wahl der syntonischen
Ebenen für die Septime und None ebenfalls bestätigt.

\subsection{Septnonakkord}

\todo{Auf \cite[Abschnitt 11.3]{Skript} verweisen? TD}
Der Dominantseptnonakkord in Dur entsteht indem dem Dominantseptakkord die (skaleneigene) große None hinzugefügt wird, oft wird dabei der Grundton, seltener die Quinte ausgelassen (in \flat E-Dur ist er \flat b\,-\,d\,-\,f\,-\,\flat a\,-\,c).  Da alle Bestandteile bis auf die None bereits in \cref{sec:dom7syn} besprochen wurden, bleibt nur die Stimmung der None zu klären.  Die None ist entweder als große Terz über der Septime oder als Quinte über der Quinte (und damit pythagoreische None über dem Grundton) interpretierbar.  Zweiteres ist aus folgenden Gründen sinnvoller:

\begin{itemize}
	\item Die None ist verträglich mit \cref{sec:49}.
	\item Die Septime ist in dem Dominantakkord ein spannungsreicher Ton.  Ein reines Intervall zu ihr erzeugt tendenziell weniger Konsonanz als reine Quinten über dem Grundton.
	\item Die Auflösung der None erfolgt mit einem Ganzton abwärts in den Grundton der Tonika; hierbei ist der $8:9$ Ganzton dem $9:10$ Ganzton melodisch (leicht) vorzuziehen (siehe \cref{sec:melody}).
\end{itemize}
Wir stellen den Dominantseptnonakkord innerhalb des Eulerschen Tonnetzes also mit dem Verhältnis $36:45:54:64:81$ (in \flat E-Dur also \flat b\,-\,\naturalm d\,-\,f\,-\,\flat a\,-\,c) dar.

In gleichstufiger Stimmung ist die häufigste Form des Dominantseptnonakkordes (ohne Grundton) genau der Sixte ajoutée der Mollparallele. In unserer Stimmung sind die beiden Akkorde aber von einander zu unterscheiden, wie das Beispiel \flat E-Dur zeigt:
\begin{alignat*}{5}
\text{D}^{9 \atop 7}:\quad&\text{\naturalm d}\quad&\text{f}\quad&\text{\flat a}\quad&\text{c}\quad&\quad\\
\text{s}^{6 \atop 5}_{[\text{Tp}]}:\quad&\quad&\text{\naturalm f}\quad&\text{\flat a}\quad&\text{\naturalm c}\quad&\text{\naturalm d}
\end{alignat*}
Wie wir sehen, teilen sie sich den Tritonus \naturalm d\,-\,\flat a, doch die Quinte f\,-\,c beziehungsweise \naturalm f\,-\,\naturalm c ist verschoben.



\subsection{Verminderter Septakkord mit tiefalterierter Quinte}

\todo{Auf \cite[Abschnitt 11.2]{Skript} verweisen? TD}

%%% Local Variables:
%%% mode: latex
%%% TeX-master: "../main"
%%% End:
