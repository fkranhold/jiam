Das kontextabhängige Mikroalterieren von Tönen in einem Stück hat nicht nur
Einfluss auf die vertikale Struktur eines Satzes (also die aus den Tönen
bestehenden \emph{Akkorde}), sondern auch auf die horizontale Strukur (also die
aus den Tönen bestehenden \emph{Melodien}). Wir wollen diese Effekte explizit
machen.

\subsection{Schritte}
\label{sec:steps}

In der klassischen Musiktheorie ist ein \emph{Schritt} ein horizontales
Intervall der Größe einer Sekunde, vgl. \cite[{}7.1]{Skript}.  Für
intonationstheoretische Belange ist es sinnvoll, den Begriff etwas anders zu
gebrauchen: Für uns ist ein \emph{Schritt} eines der drei folgenden enharmonisch
präzisen Intervalle:\looseness-1
\begin{itemize}[itemsep=0em]
\item übermäßige Prime (etwa c\,–\,\sharp c), genannt
  \emph{chromatischer Halbton},
\item kleine Sekunde (etwa h\,–\,c), genannt
  \emph{diatonischer Halbton}.
\item große Sekunde (etwa c\,–\,d), genannt
  \emph{Ganzton}.
\end{itemize}
In gleichstufiger Stimmung sind sowohl chromatischer als auch diatonischer
Halbton $100\ct$ und der Ganzton $200\ct$ groß.  Schon in pythagoreischer
Stimmung spielt die Enharmonik eine Rolle; hier ist der chromatische Halbton
($113{,}69\ct$) ein pythagoreisches Komma größer als der diatonische
($90{,}22\ct$). Man kann sich dies am Beispiel aus \cref{sec:pythTuning}
klarmachen: Der chromatische Halbton g\,–\,\sharp g unterscheidet sich vom
diatonischen Halbton g\,–\,\flat a genau um den Unterschied zwischen \sharp g
und \flat a, der wie in \cref{sec:pythComma} festgestellt genau ein
pythagoreisches Komma ist.

Im harmonischen Kontext und unter Verwendung des Eulerschen Tonnetzes sieht das
nochmal anders aus: \Cref{fig:steps} zeigt fünf typische Akkordfolgen in C-Dur,
bei denen verschiedene Schritte in der obersten Stimme auftreten, geordnet nach
Größe des jeweiligen Schrittes:\footnote{Die
  \href{https://en.wikipedia.org/wiki/Semitone\#Just_intonation}{englische
    Wikipedia} ist an dieser Stelle irreführend und stellenweise mathematisch
  inkorrekt.}
\begin{enumerate}
\item \emph{Kleiner chromatischer Halbton}\\
  Wechselt man vom Grundton der Tonika (c) mit einem chromatischen Halbton in
  die Terz der Zwischendominante zur Subdominantparallele (\sharpmm c), so
  verliert man im Vergleich zum pythagoreischen chromatischen Halbton ganze zwei
  syntonische Kommata. Der entstehende Halbton ($70{,}67\ct$) entspricht dem
  Verhältnis $24:25$ und ist sehr klein. Dieser Halbton ist der gleiche wie der
  zwischen Mollterz (\flatp e) und Durterz (\naturalm e) zueinander varianter
  Dreiklänge.
\item \emph{Großer chromatischer Halbton}\\
  Wechselt man vom Grundton der Subdominante (f) mit einem chromatischen Halbton
  in die Terz der Doppeldominante (\sharpm f), so verliert man diesmal nur ein
  syntonisches Komma.  Der entstehende Halbton ($128:135\approx 92{,}18\ct$)
  ist recht nah am gleichstufigen, allerdings noch immer etwas
  kleiner.\looseness-1
\item \emph{Reiner diatonischer Halbton}\\
  Wechselt man von der Terz der Dominante (\naturalm h) mit einem diatonischen
  Halbton in den Grundton der Tonika (c), so gewinnt man im Vergleich zum
  pythagoreischen diatonischen Halbton ein syntonisches Komma.  Der entstehende
  Halbton entspricht dem Verhältnis $15:16$ und beträgt $111{,}73\ct$.  Wir
  bemerken, dass dieser Schritt in jedem Dominante-Tonika-Verhältnis, also auch
  bei Zwischendominanten, so auftaucht, siehe auch \cref{sec:ln}.\looseness-1
\item \emph{Kleiner Ganzton}\\
  Wechselt man von der Quinte der Tonika mit einem Ganzton in die Terz der
  Subdominante (\naturalm a), so verliert man im Vergleich zur pythagoreischen
  großen Sekunde ein syntonisches Komma. Der entstehende Schritt entspricht dem
  recht einfachen Verhältnis \mbox{$9:10$} und beträgt $182{,}40\ct$, ist also
  recht deutlich unterhalb des gleichstufigen.
\item \emph{Großer Ganzton}\\
  Wechselt man zum Beispiel von der Terz der Subdominante (\naturalm a) zur Terz
  der Dominante (\naturalm h), so bleibt man auf derselben Kommahöhe. Es
  entsteht also ein reiner pythagoreischer Ganzton, der dem Verhältnis $8:9$
  und damit $203{,}91\ct$ (also gleichstufiger Ganzton plus $2$ Grad)
  entspricht.  Selbiges gilt für den Wechsel vom Grundton der Subdominante (f)
  zum Grundton der Dominante (g), eine typische Bassbewegung.
\end{enumerate}

\begin{figure}
  \centering
  \LY{b-steps}
  \includegraphics{ly/b-steps}
  \caption{Verschiedene Schritte in der Oberstimme:\\
    \hspace*{3mm}(1) Kleiner chromatischer  Halbton c\,–\,\sharpmm c
    vom Grundton der T zur Terz der (D)\textsubscript{[Sp]},\\
    \hspace*{3mm}(2) Großer chromatischer Halbton f\,–\,\sharpm f
    vom Grundton  der S zur Terz der \raisebox{1px}{D}\hspace*{-4.5px}D,\\
    \hspace*{3mm}(3) Reiner diatonischer Halbton \naturalm h\,–\,c
    von der Terz der D zum Grundton der T,\\
    \hspace*{3mm}(4) Kleiner Ganzton g\,–\,\naturalm a
    von der Quinte der T zur Terz der S,\\
    \hspace*{3mm}(5) Großer Ganzton \naturalm a\,–\,\naturalm h
    von der Terz der S zur Terz der D.}\label{fig:steps}
\end{figure}

\noindent Wir bemerken, dass in allen auftretenden Situationen chromatische
Halbtöne anders als im pythagoreischen Fall \emph{kleiner} als diatonische
Halbtöne sind.  Dies liegt daran, dass im harmonischen Kontext die Verteilung
der syntonischen Kommata dem pythagoreischen Komma genau entgegengesetzt ist
(z.\,B. g\,–\,\sharpm g, aber g\,–\,\flatp a).

Zusätzlich zu diesen fünf harmonisch begründbaren Schritten wollen wir noch zwei
weitere Halbtöne erwähnen, die im späteren Verlauf an noch nicht
diskutierten Stellen auftreten können. Um unsere die Größe der Schritte
berücksichtigende Nummerierung nicht kaputtzumachen, quetschen wir sie dazwischen:

\begin{enumerate}
\item[2\pazofrac12.] \emph{Limma} (griechisch „Überrest“)\\
  Ein diatonischer Halbton, bei dem beide Töne auf derselben Kommahöhe liegen.
  Dieser Schritt kommt in vertikal gedachten Passagen (fast) nicht vor, dafür
  aber an melodisch gedachten, z.\,B. wenn in einer Tonleiter nur große Ganztöne
  verwendet werden, vgl. \cref{sec:balance}. Wie schon in \cref{sec:pythInt}
  festgestellt entspricht er dem Verhältnis $243:256$, also $90{,}22\ct$.
\item[3\pazofrac12.] \emph{Großer diatonischer Halbton}\\
  Ein diatonischer Halbton, bei dem wir zwei syntonische Kommata gewinnen, wie
  z.\,B. \naturalm d\,–\,\flatp e. Dieser Schritt entspricht dem Verhältnis
  $25:27$, also $133{,}24\ct$. Weil er selten auftritt und wegen seiner
  ungewöhnlichen Größe auch nur begrenzt gut klingt, sehen wir davon ab, den
  schon besprochenen reinen diatonischen Halbton (also $15:16$) zur Abgrenzung
  „klein“ zu nennen – jener ist in harmonisch gedachten Passagen der
  Standardfall.
\end{enumerate}
Diese sieben Schritte, ihre Verhältnisse und Größen sind in \cref{tab:steps}
noch einmal kurz und bündig zusammengefasst.

\begin{table}
  \centering
  \begin{tabular}{lr@{\hspace*{1.6px}}lr@{\hspace*{2.4px}}lr}
    \toprule
    \thl{Bezeichnung} & \multicolumn{2}{c}{\thl{Beispiel}} & \multicolumn{2}{c}{\thl{Verhältnis}} & \thl{Größe in ct}\\
    \midrule
    kleiner chromatischer Halbton & \hspace{3mm}c\,&–\,\sharpmm c & $24$ & $:25$ & $70{,}67$\\
    großer chromatischer Halbton & f\,&–\,\sharpm f & \hspace*{3mm}$128$ & $:135$ & $92{,}18$\\
    Limma & a\,&–\,\flat h & $243$ & $:256$ & $90{,}22$\\
    reiner diatonischer Halbton & \naturalm h\,&–\,c & $15$ & $:16$ & $111{,}73$\\
    großer diatonischer Halbton & \naturalm d\,&–\,\flatp e & $25$ & $:27$ & $133{,}24$\\
    kleiner Ganzton & g\,&–\,\naturalm a & $9$ & $:10$ & $182{,}40$\\
    großer Ganzton & f\,&–\,g & $8$ & $:9$ & $203{,}91$\\
    \bottomrule
  \end{tabular}
  \caption{Verschiedene Schritte und ihre Größen}\label{tab:steps}
\end{table}

Als eine Moral dieses Unterabschnitts könnte man festhalten:
\emph{Halbton ist nicht gleich Halbton.}  Dies liefert eine
mathematische Begründung für das Phänomen, dass Laienchöre bei chromatischen
Linien oft detonieren: Es liegt nahe, zu glauben, dass Halbtöne stets
gleich groß seien, und auch, dass die Stimme, die die Chromatik singt, die
einfachste Aufgabe habe – schließlich bewegt sie sich ja immer um die „kleinste
Einheit“. Alles an dieser auf den ersten Blick einleuchtenden Vermutung
widerspricht obiger Berechnung.  Stattdessen muss sich die sich chromatisch
bewegende Stimme an den sie umgebenden Harmonien orientieren.\looseness-1

\subsection{Quintfälle}

Ein weiteres wichtiges horizontales Intervall ist die Quinte (bzw. komplementär
dazu: die Quarte), denn oft setzen sich Basslinien aus ihnen in Form sogenannter
\emph{Quintfälle} zusammen, vgl. \cite[{}12.3]{Skript}. Wir bemerken, dass in
einer üblichen Akkordfolge so ein Quintfall nicht ausschließlich das für
vertikale Intervalle so erstrebenswerte Verhältnis $2:3$ nutzt, sondern zumeist
wenigstens einmal ein Komma abgezogen (bzw. zur Quarte hinzuaddiert) wird, siehe
\cref{fig:5fall}. Dieses Intervall wird oft \emph{unreine Quinte} (oder
\emph{unreine Quarte}) genannt und hat das Verhältnis $27:40 \approx 680{,}45\ct$
(oder komplementär dazu $20:27 \approx 519{,}55\ct$).

Dies ist aber auch nicht besonders tragisch: Ein üblicher Quintfall muss sogar
wenigstens eine \emph{verminderte} Quinte enthalten, um nicht zu modulieren
(hier: von \naturalp f auf h), weswegen man derartige Abweichungen gewohnt ist.

\begin{figure}[h]
  \centering
  \LY{b-5fall}
  \includegraphics{ly/b-5fall}
  \caption{Ein Quintfall im Bass, bei dem der Sprung d\,–\,\naturalp g um ein
    syntonisches Komma größer als die reine Quarte $3:4$ ist.  Der Akkord
    h\,:\,\naturalp f\,:\,d\,:\,a im letzten Takt ist die Subdominante mit Sexte
    im Bass, deren Ausstimmung in \cref{sec:sixte} erklärt wird.}\label{fig:5fall}
\end{figure}

\subsection{Leittöne}
\label{sec:ln}

Wir bemerken, dass die Terz im (Zwischen-)Dominantdreiklang so wie in allen
Durdreiklängen auch um ein syntonisches Komma abgesenkt wird, was dem häufig im
Instrumentalunterricht vermittelten Ideal widerspricht, den Leitton (also den
Ton, der eine kleine Sekunde unter dem Grundton liegt) höher zu intonieren, um
die „Spannung zu erhöhen“: In unserer mathematischen Beschreibung liegt der
Leitton um einen diatonischen Halbton und damit $111{,}73\ct$ unter dem
Grundton, was \emph{tiefer} als der gleichstufige Leitton ist.

Dieser Widerspruch wurde auch in \cite[211]{viitasaari} benannt und
diskutiert, mit dem Ergebnis, dass es stark auf den Kontext ankommt: In
homophonen Sätzen, bei denen die \emph{vertikale} Struktur der Musik im
Vordergrund steht, sollte die Dominante ein eigenständiger, stabiler
Dur-Dreiklang sein und der Leitton somit tief intoniert werden. Steht hingegen
die melodische Bewegung einzelner Stimmen, also die \emph{horizontale} Struktur
im Vordergrund (wir merken an, dass dies in keinem unserer bisherigen Beispiele
der Fall war), so kann man in Erwägung ziehen, den Leitton pythagoreisch zu
intonieren, wodurch er nur noch $90{,}22\ct$ unter dem Grundton und damit über
dem gleichstufigen Leitton läge. Der entstehende vertikale Dominantdreiklang
hätte dann das Verhältnis $64:81:96$. In \cref{fig:lead} hören wir zwei
Versionen einer Kadenz, eine mit einem rein intonierten Leitton, eine mit einem
pythagoreischen Leitton.

\begin{figure}
  \centering
  \LY{b-lead}
  \includegraphics{ly/b-lead}
  \caption{Zwei Arten, eine Kadenz zu intonieren: Einmal mit reiner
    Intonation aller Dreiklänge und einmal, in einem bewegteren Kontext, mit
    einem pythagoreischen Leitton.}\label{fig:lead}
\end{figure}

\subsection{Sinnvoll abwägen}
\label{sec:balance}

Die Diskussion um den Leitton aus dem vorherigen Abschnitt ist nur ein Beispiel
eines allgemeineren Priorisierungsproblems, das wir bisher absichtsvoll umgangen
sind: Alle bisherigen Beispiele haben homophone Sätze betrachtet, bei denen der
Aspekt der \emph{Bewegung} der einzelnen Stimmen zumeist nachrangig
war. Außerdem haben wir unsere Beispiele stets langsam genug gespielt, um uns
Zeit zu geben, wahrzunehmen, wie die Akkorde einrasten. Dies ist dadurch
gerechtfertigt, dass tatsächlich ein gehöriger Teil klassischer Chormusik auf
diese Weise gedacht ist und musiziert wird; insbesondere \emph{enden} viele
Stücke auf diese Weise. Es blendet allerdings aus, dass auch Vokalmusik ganz
anders konzipiert sein kann.

Als extremes Gegenbeispiel mag \cref{fig:melody} dienen, bei der – obwohl auf
den einzelnen Schlägen vertikal funktionale Harmonien klar erkennbar sind – die
Stimmen ständig in Bewegung sind und dem Akkord gar keine Zeit gegeben wird,
einzurasten. Außerdem klingen die melodische Linien „schief“, wenn viele
unterschiedlich große Schritte (insbesondere zwei unterschiedliche Ganztöne) in
ihr vorkommen.  In solchen Situationen ist es empfehlenswert, auf entsprechende
syntonische Kommata zu verzichten und stattdessen stellenweise komplett
pythagoreisch (relativ zum lokalen tonalen Zentrum) zu intonieren.  Wir
empfehlen, die Alt- oder Bassstimme des Beispiels einmal selbst zu singen und
mit den Hörbeispielen zu vergleichen.  Unser Hauptgrund, pythagoreisch und nicht
direkt gleichstufig zu intonieren, ist, dass es uns dieser Ansatz ermöglicht, an
\emph{ausgewählten} Stellen doch Kommata zu verwenden, um sie rein zu stimmen –
hier die letzten beiden Akkorde, bei denen eine Kadenz erkennbar ist und wo
deutlich weniger Bewegung stattfindet.\looseness-1

\begin{figure}
  \centering
  \LY{b-melody}
  \includegraphics{ly/b-melody}
  \caption{Eine melodisch gedachter Abschnitt, einmal
    vollständig rein ausgestimmt (nicht empfohlen) und einmal über weite
    Strecken pythagoreisch gestimmt.}\label{fig:melody}
\end{figure}

Wieder anders sieht die Situation in \cref{fig:inter} aus: Dieses Beispiel ist
im Wesentlichen homophon und langsam – nur vereinzelt gibt es Zwischennoten
(hier: Durchgangsnoten und angesprungene Nebennoten). Hier empfiehlt es sich,
zunächst die Akkorde rein zu stimmen und sich erst dann mit den Zwischennoten zu
befassen. Um ihre Kommahöhe zu ermitteln, scheinen uns folgende Kriterien
sinnvoll, mit absteigender Priorisierung:
\begin{enumerate}
\item Alle auftretenden Schritte sollten in \cref{tab:steps} gelistet sein.
\item Taucht ein Ton auf zwei unterschiedlichen Kommahöhen auf, so sollten
  dazwischen mindestens zwei andere Töne liegen.
\item Die Zwischennote sollte von der vorherigen Note aus durch ein
  pythagoreisches Intervall erreicht werden.
\end{enumerate}
So würde Kriterium 3 in obigem Beispiel für die erste Durchgangsnote h im Bass
die gleiche Kommahöhe wie beim c nahelegen; dies wird allerdings von Kriterium 1
verunmöglicht, weil dadurch der Ganzton zum \naturalm a zu groß würde. Die
zweite Zwischennote, also das g im Alt, wird vom \naturalm e aus erreicht,
weswegen Kriterium 3 \naturalm g nahelegt. Dies ist vereinbar mit den Kriterien
1 und 2, da der darauffolgende Schritt zum f ein kleiner Ganzton ist. Die dritte
Zwischennote, das c im Sopran, taucht nach zur einem weiteren Ton (dem \naturalm
h) ohne Mikroalteration auf. Daher legt Kriterium 2 die Syntonisierung \natural
c nahe, was mit Kriterium 1 vereinbar ist.\looseness-1

\begin{figure}[h]
  \centering
  \LY{b-inter}
  \includegraphics{ly/b-inter}
  \caption{Ein im Wesentlichen homophoner Satz, mit vereinzelten
    Zwischennoten. Die Akkorde sind rein gestimmt; für die Zwischennoten gibt
    es unterschiedliche Anhaltspunkte.}\label{fig:inter}
\end{figure}

Wieder anders sieht das Ganze aus, wenn in einer Melodie \emph{gebrochene
  Akkorde} auftreten. Dies sind zwar horizontale Intervalle, die allerdings
nicht melodisch, sondern harmonisch gedacht sind. Hier scheint es uns sinnvoll
zu sein, sie genau wie vertikale Akkorde zu stimmen, siehe \cref{fig:arp}.


\begin{figure}
  \centering
  \LY{b-arp}
  \includegraphics{ly/b-arp}
  \caption{Der Beginn von Bachs Präludium C-Dur (\acr{BWV} 846) aus dem
    \emph{Wohltemperierten Klavier 1}, rein ausgestimmt.}\label{fig:arp}
\end{figure}

%%% Local Variables:
%%% mode: latex
%%% TeX-master: "../main"
%%% End:
