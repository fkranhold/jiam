\documentclass[ngerman,11pt]{scrartcl}

\title     {Theorie der reinen Intonation}
\author    {Thomas Drögemüller\and Florian Kranhold}
\subtitle  {Skript}

%% KOMA-Optionen (scrlttr2, sonst: scrartcl, scrbook)
\KOMAoptions{headings=standardclasses,numbers=enddot}

%% Encoding
\usepackage[utf8]{inputenc}
\usepackage[T1]{fontenc}

%% Sprache
\usepackage{babel}
\usepackage{csquotes} % Correct quotiation marks „…“ bzw. ‘…’

%% Datumsformat (ngerman, british)
\usepackage[cleanlook]{isodate}

%% LModern ist ein besserer Fallback als Typewriter 
\usepackage{lmodern}

%% Palatino als Brotschrift
%% (Um die Serifenlose kümmern wir uns unten)
\usepackage[osf,sc]{mathpazo}

%% In einigen Fällen passen Lining figures besser
\newcommand{\LF}[1]{{\fontfamily{pplx}\selectfont #1}}

%% Classico als Serifenlose, falls vorhanden
%% Andernfalls ist der Fallback die LM Sans Serif,
%% die wir unten nach Möglichkeit vermeiden werden.
\IfFileExists{classico.sty}{\usepackage{classico}}{}

%% Beramono als Dicktengleiche
\usepackage[scaled=.85]{beramono}

%% Palatino braucht mehr Platz
\usepackage{setspace}
\setstretch{1.07}
\renewcommand{\arraystretch}{1.07} % Für tabular- und array-Umgebungen

%% Mikrotypographie!
\usepackage{ellipsis} % More space after dots
\usepackage[babel,tracking=true]{microtype}

%% microtype: Disable Footnote-Patching for ≥3.0
\makeatletter
\@ifpackagelater{microtype}{2020/12/08}
  {\microtypesetup{nopatch=footnote}}
  {}
\makeatother

%% microtype: Settings
\UseMicrotypeSet[tracking]{smallcaps}
\SetTracking{encoding=*,shape=sc}{30}

%% Hurenkinder verbieten
\makeatletter
  \widowpenalty=\@M
\makeatother

%% Inline-Formeln NIE umbrechen
\relpenalty=10000
\binoppenalty=10000

%% Less Looseness
\everypar=\expandafter{\the\everypar\looseness=-1}

%% Text nicht vertikal strecken oder stauchen
\raggedbottom

%% Akronyme
\usepackage{relsize}
\newcommand{\acr}[1]{\texorpdfstring{\textsmaller{\LF{#1}}}{#1}}

%% Eigene Listing styles mit kurzer Definition
\usepackage[shortlabels]{enumitem}

% hyperref<7.00h replaces \C with the cyrillic letter \CYRDZE
% See https://github.com/latex3/hyperref/issues/170
% We want it to be the complex numbers, as previously defined.
\let\COld\C

%% Figures
\usepackage{graphicx}
\usepackage[margin    =20pt,
            font      ={footnotesize,stretch=1.06},
            format    =plain,
            labelsep  =period,
            labelfont =bf]{caption}
  
%% See above
\let\C\COld

%% Fancy tables
\usepackage{booktabs}
  
%% Colours, e.g. for References
\usepackage{xcolor}

%% Einbinden anderen Präambeln (die auch Pakete einbinden),
%% außer -env, was cleveref einbindet und deshalb zuletzt kommt.
\IfFileExists{preamble-math.tex}  {%%% commit: dc6a172

%% Grundlegende mathematische Symbole
\usepackage{amssymb,amsmath} % Standard
\usepackage{mathtools}       % \coloneqq, \eqqcolon
\usepackage{stmaryrd}        % \sslash, \Ydown, \llbracket

%% TikZ and TikZ-CD
\usepackage{tikz-cd}
\usepackage{tikz}
\usetikzlibrary{arrows,calc}

%% Mini-Indizes (\scaleto{…}{3pt})
\usepackage{scalerel}

%% Some choices
\let \le       \leqslant
\let \ge       \geqslant
\let \leq      \leqslant
\let \geq      \geqslant
\let \setminus \smallsetminus
\let \emptyset \varnothing
\let \epsilon  \varepsilon
%\let \theta    \vartheta

%% Bold
\newcommand{\bm} [1]{\mathbold{#1}}

%% Short for overline and operatorname
\newcommand{\ol} [1]{\overline{#1}}
\newcommand{\on} [1]{\mathrm{#1}}

%% Fix the spacing of \left and \right
\let\originalleft\left
\let\originalright\right
\renewcommand{\left}{\mathopen{}\mathclose\bgroup\originalleft}
\renewcommand{\right}{\aftergroup\egroup\originalright}
\renewcommand{\bigl}{\mathopen\big}
\renewcommand{\bigr}{\mathclose\big}

%% Brackets
\newcommand{\lla}[1]{\mathopen{}\left\langle #1 \right\rangle\mathclose{}}      % <…>
\newcommand{\ula}[1]{\langle #1\rangle}                                         % <…>   sm
\newcommand{\llb}[1]{\mathopen{}\left\llbracket #1\right\rrbracket\mathclose{}} % [[…]]
\newcommand{\ulb}[1]{\mathopen{}\left\llbracket #1\right\rrbracket\mathclose{}} % [[…]] sm

%% Some operatornames
\newcommand{\GL} {\on{GL}}
\newcommand{\pr} {\on{pr}}
\newcommand{\ob} {\on{ob}}
\newcommand{\sg} {\on{sg}}
\newcommand{\im} {\on{im}}
\newcommand{\op} {{\on{op}}} % It is nearly always an exponent
\newcommand{\Aut}{\on{Aut}}
\newcommand{\Hom}{\on{Hom}}
\newcommand{\map}{\on{map}}
\newcommand{\Id} {\on{id}}   % The only one where the capitalisation has changed

% colon on the right side
\newcommand*\lon{%
        \nobreak
        \mskip6mu plus1mu
        \mathpunct{}%
        \nonscript
        \mkern-\thinmuskip
        {:}%
        \mskip2mu
        \relax
}

%% Some categories
\newcommand{\Top}{\mathbf{Top}}
\newcommand{\Sig}{\mathbf{\Sigma}}
\newcommand{\Del}{\mathbf{\Delta}}
\newcommand{\Set}{\mathbf{Set}}
\newcommand{\Mod}{\mathbf{Mod}}
\newcommand{\Alg}{\mathbf{Alg}}

%% Miscellanea 
\newcommand{\bEn}{\mathbb{1}}
\newcommand{\RP}{\R\mathrm{P}}
\newcommand{\CP}{\C\mathrm{P}}
\newcommand{\Diff}{\on{Diff}{}}
\newcommand{\hAut}{\on{hAut}}
\newcommand{\dcup}{\,\dot\cup\,}
\newcommand{\swr}{\kern.6px\wr\kern.6px}

%% Integral sign
\DeclareSymbolFont{eulargesymbols}{U}{zeuex}{m}{n}
\DeclareMathSymbol{\intop}{\mathop}{eulargesymbols}{"52}

%% TikZ: centerarc
\def\centerarc[#1](#2)(#3:#4:#5)% Syntax: [draw options] (center) (initial angle:final angle:radius)
    { \draw[#1] ($(#2)+({#5*cos(#3)},{#5*sin(#3)})$) arc (#3:#4:#5); }

%\usepackage[scr=boondoxo,scrscaled=1,cal=euler]{mathalfa} % Calligraphic fonts

% Zahlenbereiche
\newcommand{\N}{\mathbb{N}}
\newcommand{\Z}{\mathbb{Z}}
\newcommand{\Q}{\mathbb{Q}}
\newcommand{\R}{\mathbb{R}}
\newcommand{\C}{\mathbb{C}}

%% Small \vdots
\usepackage{letltxmacro}
\LetLtxMacro\orgvdots\vdots
\LetLtxMacro\orgddots\ddots
\makeatletter
\DeclareRobustCommand\vdots{%
  \mathpalette\@vdots{}%
}
\newcommand*{\@vdots}[2]{%
  % #1: math style
  % #2: unused
  \sbox0{$#1\cdotp\cdotp\cdotp\m@th$}%
  \sbox2{$#1.\m@th$}%
  \vbox{%
    \dimen@=\wd0 %
    \advance\dimen@ -3\ht2 %
    \kern.5\dimen@
    % remove side bearings
    \dimen@=\wd2 %
    \advance\dimen@ -\ht2 %
    \dimen2=\wd0 %
    \advance\dimen2 -\dimen@
    \vbox to \dimen2{%
      \offinterlineskip
      \copy2 \vfill\copy2 \vfill\copy2 %
    }%
  }%
}
\makeatother

% Correct fracs
\DeclareFontFamily{OMS}{zplm}{
   \skewchar\font=48 % kept from the original file
   \fontdimen 9\font=0.34\fontdimen6\font
   \fontdimen 8\font=0.84\fontdimen6\font
}
\DeclareFontShape{OMS}{zplm}{m}{n}{<-> zplmr7y}{}
\DeclareFontShape{OMS}{zplm}{b}{n}{<-> zplmb7y}{}
\DeclareFontShape{OMS}{zplm}{bx}{n}{<->ssub * zplm/b/n}{}

%%% The rest of this header file shortens the following letter sets:
%%% bbA, bA, caA, scA, sca, bfA, bmA, frA, fra, tA

%% Blackboard bold
\newcommand{\bbA}{\mathbb{A}}
\newcommand{\bbB}{\mathbb{B}}
\newcommand{\bbC}{\mathbb{C}}
\newcommand{\bbD}{\mathbb{D}}
\newcommand{\bbE}{\mathbb{E}}
\newcommand{\bbF}{\mathbb{F}}
\newcommand{\bbG}{\mathbb{G}}
\newcommand{\bbH}{\mathbb{H}}
\newcommand{\bbI}{\mathbb{I}}
\newcommand{\bbJ}{\mathbb{J}}
\newcommand{\bbK}{\mathbb{K}}
\newcommand{\bbL}{\mathbb{L}}
\newcommand{\bbM}{\mathbb{M}}
\newcommand{\bbN}{\mathbb{N}}
\newcommand{\bbO}{\mathbb{O}}
\newcommand{\bbP}{\mathbb{P}}
\newcommand{\bbQ}{\mathbb{Q}}
\newcommand{\bbR}{\mathbb{R}}
\newcommand{\bbS}{\mathbb{S}}
\newcommand{\bbT}{\mathbb{T}}
\newcommand{\bbU}{\mathbb{U}}
\newcommand{\bbV}{\mathbb{V}}
\newcommand{\bbW}{\mathbb{W}}
\newcommand{\bbX}{\mathbb{X}}
\newcommand{\bbY}{\mathbb{Y}}
\newcommand{\bbZ}{\mathbb{Z}}

%% Blackboard bold (shorter)
\newcommand{\bA}{\mathbb{A}}
\newcommand{\bB}{\mathbb{B}}
\newcommand{\bC}{\mathbb{C}}
\newcommand{\bD}{\mathbb{D}}
\newcommand{\bE}{\mathbb{E}}
\newcommand{\bF}{\mathbb{F}}
\newcommand{\bG}{\mathbb{G}}
\newcommand{\bH}{\mathbb{H}}
\newcommand{\bI}{\mathbb{I}}
\newcommand{\bJ}{\mathbb{J}}
\newcommand{\bK}{\mathbb{K}}
\newcommand{\bL}{\mathbb{L}}
\newcommand{\bM}{\mathbb{M}}
\newcommand{\bN}{\mathbb{N}}
\newcommand{\bO}{\mathbb{O}}
\newcommand{\bP}{\mathbb{P}}
\newcommand{\bQ}{\mathbb{Q}}
\newcommand{\bR}{\mathbb{R}}
\newcommand{\bS}{\mathbb{S}}
\newcommand{\bT}{\mathbb{T}}
\newcommand{\bU}{\mathbb{U}}
\newcommand{\bV}{\mathbb{V}}
\newcommand{\bW}{\mathbb{W}}
\newcommand{\bX}{\mathbb{X}}
\newcommand{\bY}{\mathbb{Y}}
\newcommand{\bZ}{\mathbb{Z}}

%% Calligraphic
\newcommand{\caA}{\mathcal{A}}
\newcommand{\caB}{\mathcal{B}}
\newcommand{\caC}{\mathcal{C}}
\newcommand{\caD}{\mathcal{D}}
\newcommand{\caE}{\mathcal{E}}
\newcommand{\caF}{\mathcal{F}}
\newcommand{\caG}{\mathcal{G}}
\newcommand{\caH}{\mathcal{H}}
\newcommand{\caI}{\mathcal{I}}
\newcommand{\caJ}{\mathcal{J}}
\newcommand{\caK}{\mathcal{K}}
\newcommand{\caL}{\mathcal{L}}
\newcommand{\caM}{\mathcal{M}}
\newcommand{\caN}{\mathcal{N}}
\newcommand{\caO}{\mathcal{O}}
\newcommand{\caP}{\mathcal{P}}
\newcommand{\caQ}{\mathcal{Q}}
\newcommand{\caR}{\mathcal{R}}
\newcommand{\caS}{\mathcal{S}}
\newcommand{\caT}{\mathcal{T}}
\newcommand{\caU}{\mathcal{U}}
\newcommand{\caV}{\mathcal{V}}
\newcommand{\caW}{\mathcal{W}}
\newcommand{\caX}{\mathcal{X}}
\newcommand{\caY}{\mathcal{Y}}
\newcommand{\caZ}{\mathcal{Z}}

%% Script large
\newcommand{\scA}{\mathscr{A}}
\newcommand{\scB}{\mathscr{B}}
\newcommand{\scC}{\mathscr{C}}
\newcommand{\scD}{\mathscr{D}}
\newcommand{\scE}{\mathscr{E}}
\newcommand{\scF}{\mathscr{F}}
\newcommand{\scG}{\mathscr{G}}
\newcommand{\scH}{\mathscr{H}}
\newcommand{\scI}{\mathscr{I}}
\newcommand{\scJ}{\mathscr{J}}
\newcommand{\scK}{\mathscr{K}}
\newcommand{\scL}{\mathscr{L}}
\newcommand{\scM}{\mathscr{M}}
\newcommand{\scN}{\mathscr{N}}
\newcommand{\scO}{\mathscr{O}}
\newcommand{\scP}{\mathscr{P}}
\newcommand{\scQ}{\mathscr{Q}}
\newcommand{\scR}{\mathscr{R}}
\newcommand{\scS}{\mathscr{S}}
\newcommand{\scT}{\mathscr{T}}
\newcommand{\scU}{\mathscr{U}}
\newcommand{\scV}{\mathscr{V}}
\newcommand{\scW}{\mathscr{W}}
\newcommand{\scX}{\mathscr{X}}
\newcommand{\scY}{\mathscr{Y}}
\newcommand{\scZ}{\mathscr{Z}}

%% Script small
\newcommand{\sca}{\mathscr{a}}
\newcommand{\scb}{\mathscr{b}}
\newcommand{\scc}{\mathscr{c}}
\newcommand{\scd}{\mathscr{d}}
\newcommand{\sce}{\mathscr{e}}
\newcommand{\scf}{\mathscr{f}}
\newcommand{\scg}{\mathscr{g}}
\newcommand{\sch}{\mathscr{h}}
\newcommand{\sci}{\mathscr{i}}
\newcommand{\scj}{\mathscr{j}}
\newcommand{\sck}{\mathscr{k}}
\newcommand{\scl}{\mathscr{l}}
\newcommand{\scm}{\mathscr{m}}
\newcommand{\scn}{\mathscr{n}}
\newcommand{\sco}{\mathscr{o}}
\newcommand{\scp}{\mathscr{p}}
\newcommand{\scq}{\mathscr{q}}
\newcommand{\scr}{\mathscr{r}}
\newcommand{\scs}{\mathscr{s}}
\newcommand{\sct}{\mathscr{t}}
\newcommand{\scu}{\mathscr{u}}
\newcommand{\scv}{\mathscr{v}}
\newcommand{\scw}{\mathscr{w}}
\newcommand{\scx}{\mathscr{x}}
\newcommand{\scy}{\mathscr{y}}
\newcommand{\scz}{\mathscr{z}}

%% Bold (but not italic)
\newcommand{\bfA}{\mathbf{A}}
\newcommand{\bfB}{\mathbf{B}}
\newcommand{\bfC}{\mathbf{C}}
\newcommand{\bfD}{\mathbf{D}}
\newcommand{\bfE}{\mathbf{E}}
\newcommand{\bfF}{\mathbf{F}}
\newcommand{\bfG}{\mathbf{G}}
\newcommand{\bfH}{\mathbf{H}}
\newcommand{\bfI}{\mathbf{I}}
\newcommand{\bfJ}{\mathbf{J}}
\newcommand{\bfK}{\mathbf{K}}
\newcommand{\bfL}{\mathbf{L}}
\newcommand{\bfM}{\mathbf{M}}
\newcommand{\bfN}{\mathbf{N}}
\newcommand{\bfO}{\mathbf{O}}
\newcommand{\bfP}{\mathbf{P}}
\newcommand{\bfQ}{\mathbf{Q}}
\newcommand{\bfR}{\mathbf{R}}
\newcommand{\bfS}{\mathbf{S}}
\newcommand{\bfT}{\mathbf{T}}
\newcommand{\bfU}{\mathbf{U}}
\newcommand{\bfV}{\mathbf{V}}
\newcommand{\bfW}{\mathbf{W}}
\newcommand{\bfX}{\mathbf{X}}
\newcommand{\bfY}{\mathbf{Y}}
\newcommand{\bfZ}{\mathbf{Z}}

%% Bold (but not italic)
\newcommand{\bfa}{\mathbf{a}}
\newcommand{\bfb}{\mathbf{b}}
\newcommand{\bfc}{\mathbf{c}}
\newcommand{\bfd}{\mathbf{d}}
\newcommand{\bfe}{\mathbf{e}}
\newcommand{\bff}{\mathbf{f}}
\newcommand{\bfg}{\mathbf{g}}
\newcommand{\bfh}{\mathbf{h}}
\newcommand{\bfi}{\mathbf{i}}
\newcommand{\bfj}{\mathbf{j}}
\newcommand{\bfk}{\mathbf{k}}
\newcommand{\bfl}{\mathbf{l}}
\newcommand{\bfm}{\mathbf{m}}
\newcommand{\bfn}{\mathbf{n}}
\newcommand{\bfo}{\mathbf{o}}
\newcommand{\bfp}{\mathbf{p}}
\newcommand{\bfq}{\mathbf{q}}
\newcommand{\bfr}{\mathbf{r}}
\newcommand{\bfs}{\mathbf{s}}
\newcommand{\bft}{\mathbf{t}}
\newcommand{\bfu}{\mathbf{u}}
\newcommand{\bfv}{\mathbf{v}}
\newcommand{\bfw}{\mathbf{w}}
\newcommand{\bfx}{\mathbf{x}}
\newcommand{\bfy}{\mathbf{y}}
\newcommand{\bfz}{\mathbf{z}}

%% Bold and italic
\newcommand{\bmA}{\bm{A}}
\newcommand{\bmB}{\bm{B}}
\newcommand{\bmC}{\bm{C}}
\newcommand{\bmD}{\bm{D}}
\newcommand{\bmE}{\bm{E}}
\newcommand{\bmF}{\bm{F}}
\newcommand{\bmG}{\bm{G}}
\newcommand{\bmH}{\bm{H}}
\newcommand{\bmI}{\bm{I}}
\newcommand{\bmJ}{\bm{J}}
\newcommand{\bmK}{\bm{K}}
\newcommand{\bmL}{\bm{L}}
\newcommand{\bmM}{\bm{M}}
\newcommand{\bmN}{\bm{N}}
\newcommand{\bmO}{\bm{O}}
\newcommand{\bmP}{\bm{P}}
\newcommand{\bmQ}{\bm{Q}}
\newcommand{\bmR}{\bm{R}}
\newcommand{\bmS}{\bm{S}}
\newcommand{\bmT}{\bm{T}}
\newcommand{\bmU}{\bm{U}}
\newcommand{\bmV}{\bm{V}}
\newcommand{\bmW}{\bm{W}}
\newcommand{\bmX}{\bm{X}}
\newcommand{\bmY}{\bm{Y}}
\newcommand{\bmZ}{\bm{Z}}

%% Fraktur large
\newcommand{\frA}{\mathfrak{A}}
\newcommand{\frB}{\mathfrak{B}}
\newcommand{\frC}{\mathfrak{C}}
\newcommand{\frD}{\mathfrak{D}}
\newcommand{\frE}{\mathfrak{E}}
\newcommand{\frF}{\mathfrak{F}}
\newcommand{\frG}{\mathfrak{G}}
\newcommand{\frH}{\mathfrak{H}}
\newcommand{\frI}{\mathfrak{I}}
\newcommand{\frJ}{\mathfrak{J}}
\newcommand{\frK}{\mathfrak{K}}
\newcommand{\frL}{\mathfrak{L}}
\newcommand{\frM}{\mathfrak{M}}
\newcommand{\frN}{\mathfrak{N}}
\newcommand{\frO}{\mathfrak{O}}
\newcommand{\frP}{\mathfrak{P}}
\newcommand{\frQ}{\mathfrak{Q}}
\newcommand{\frR}{\mathfrak{R}}
\newcommand{\frS}{\mathfrak{S}}
\newcommand{\frT}{\mathfrak{T}}
\newcommand{\frU}{\mathfrak{U}}
\newcommand{\frV}{\mathfrak{V}}
\newcommand{\frW}{\mathfrak{W}}
\newcommand{\frX}{\mathfrak{X}}
\newcommand{\frY}{\mathfrak{Y}}
\newcommand{\frZ}{\mathfrak{Z}}

%% Fraktur small
\newcommand{\fra}{\mathfrak{a}}
\newcommand{\frb}{\mathfrak{b}}
\newcommand{\frc}{\mathfrak{c}}
\newcommand{\frd}{\mathfrak{d}}
\newcommand{\fre}{\mathfrak{e}}
\newcommand{\frf}{\mathfrak{f}}
\newcommand{\frg}{\mathfrak{g}}
\newcommand{\frh}{\mathfrak{h}}
\newcommand{\fri}{\mathfrak{i}}
\newcommand{\frj}{\mathfrak{j}}
\newcommand{\frk}{\mathfrak{k}}
\newcommand{\frl}{\mathfrak{l}}
\newcommand{\frm}{\mathfrak{m}}
\newcommand{\frn}{\mathfrak{n}}
\newcommand{\fro}{\mathfrak{o}}
\newcommand{\frp}{\mathfrak{p}}
\ifdefined\frq\renewcommand{\frq}{\mathfrak{q}}\else \newcommand{\frq}{\mathfrak{q}}\fi
\newcommand{\frr}{\mathfrak{r}}
\newcommand{\frs}{\mathfrak{s}}
\newcommand{\frt}{\mathfrak{t}}
\newcommand{\fru}{\mathfrak{u}}
\newcommand{\frv}{\mathfrak{v}}
\newcommand{\frw}{\mathfrak{w}}
\newcommand{\frx}{\mathfrak{x}}
\newcommand{\fry}{\mathfrak{y}}
\newcommand{\frz}{\mathfrak{z}}

%% Tilde
\newcommand{\tA}{\tilde{A}}
\newcommand{\tB}{\tilde{B}}
\newcommand{\tC}{\tilde{C}}
\newcommand{\tD}{\tilde{D}}
\newcommand{\tE}{\tilde{E}}
\newcommand{\tF}{\tilde{F}}
\newcommand{\tG}{\tilde{G}}
\newcommand{\tH}{\tilde{H}}
\newcommand{\tI}{\tilde{I}}
\newcommand{\tJ}{\tilde{J}}
\newcommand{\tK}{\tilde{K}}
\newcommand{\tL}{\tilde{L}}
\newcommand{\tM}{\tilde{M}}
\newcommand{\tN}{\tilde{N}}
\newcommand{\tO}{\tilde{O}}
\newcommand{\tP}{\tilde{P}}
\newcommand{\tQ}{\tilde{Q}}
\newcommand{\tR}{\tilde{R}}
\newcommand{\tS}{\tilde{S}}
\newcommand{\tT}{\tilde{T}}
\newcommand{\tU}{\tilde{U}}
\newcommand{\tV}{\tilde{V}}
\newcommand{\tW}{\tilde{W}}
\newcommand{\tX}{\tilde{X}}
\newcommand{\tY}{\tilde{Y}}
\newcommand{\tZ}{\tilde{Z}}
}  {}
\IfFileExists{preamble-bib.tex}   {%%% commit: dc6a172

\usepackage[backend      = biber,
            style        = numeric,
            maxbibnames  = 9,
            maxcitenames = 9,
            %doi          = false,
            bibencoding  = utf8,
            datamodel    = ext-eprint]{biblatex}

% Oxford comma in Bibliography
\DefineBibliographyExtras{british}{%
  \renewcommand{\finalandcomma}{,}
}

\DeclareFieldFormat{doi}{\textsf{\href{https://doi.org/#1}{[DOI]}}}
%\DeclareFieldFormat[misc]{title}{‘#1’}  % plain article title for arXiv

%% Falls die Serifenlose zur Verfügung steht,
%% dann setze auch die arXiv-Klasse [math.AT] damit.
\makeatletter
\IfFileExists{classico.sty}{
\DeclareFieldFormat{eprint:arxiv}{%
  arXiv\addcolon\space
  \ifhyperref
    {\href{http://arxiv.org/\abx@arxivpath/##1}{%
       \nolinkurl{##1}%
       \iffieldundef{eprintclass}
     {}
     {\addspace\textsf{\mkbibbrackets{\thefield{eprintclass}}}}}}
    {\nolinkurl{##1}
     \iffieldundef{eprintclass}
       {}
       {\addspace\textsf{\mkbibbrackets{\thefield{eprintclass}}}}}}
}{}
\makeatother

%% Nicht in Von-Bis-Strichen umbrechen,
%% z.B. bei den Seitenzahlen
\DefineBibliographyExtras{british}{\protected\def\bibrangedash{\nobreakdash--}}
\DefineBibliographyExtras{ngerman}{\protected\def\bibrangedash{\nobreakdash--}}

%\AtBeginBibliography{\small}

\bibliography{bibliography.bib}
}   {}
\IfFileExists{preamble-music.tex} {\input{preamble-music.tex}} {}
\IfFileExists{preamble-custom.tex}{\usepackage[size=tiny]{todonotes}
\usepackage{multicol}
%\setlength\columnseprule{.4pt}

\usepackage{marginnote}
\newcommand{\AS}[1]{\marginnote{\texttt{#1}}}
\newcommand{\LY}[1]{\marginnote{\texttt{#1}}}

%\usepackage{pgfplots}

\usepackage{xparse}% http://ctan.org/pkg/xparse
\usepackage{etoolbox}% http://ctan.org/pkg/etoolbox
\newcounter{listtotal2}\newcounter{listcntr}%
\NewDocumentCommand{\pitch}{o}{%
  \setcounter{listtotal2}{0}\setcounter{listcntr}{-1}%
  \renewcommand*{\do}[1]{\stepcounter{listtotal2}}%
  \expandafter\docsvlist\expandafter{\pitcharray}%
  \IfNoValueTF{#1}
    {\pitcharray}% \pitch
    {% \pitch[<index>]
     \renewcommand*{\do}[1]{\stepcounter{listcntr}\ifnum\value{listcntr}=#1\relax##1\fi}%
     \expandafter\docsvlist\expandafter{\pitcharray}}%
}

\newcommand{\rC}{\text{C}}
\newcommand{\rD}{\text{D}}
\newcommand{\rE}{\text{E}}
\newcommand{\rF}{\text{F}}
\newcommand{\rG}{\text{G}}
\newcommand{\rA}{\text{A}}
\newcommand{\rB}{\text{B}}

\newcommand{\uP}{{\text{+}}}
\newcommand{\uM}{{\text{\textminus}}}
\newcommand{\uMM}{{\text{\textminus\textminus}}}

\newcommand{\Hz}{\,\text{Hz}}
\newcommand{\ct}{\,\text{ct}}

\usepackage{heji2}
\newcommand{\om}[1]{\mathrm{#1}}

% Format for the table headline
\newcommand{\thl}[1]{\textsc{\textls[40]{\MakeLowercase{#1}}}}
}{}

%% Footnote-Design
\usepackage{footmisc}
\renewcommand{\footnotelayout}{\setstretch{1.06}}
\setlength{\footnotemargin}{1em}

%% Kopfzeilen in geneigtem statt kursivem Palatino
\addtokomafont{pagehead}{\fontfamily{ppl}\selectfont}

%% Description mit SC, vor allem nicht als Sans
\setkomafont{descriptionlabel}{\scshape}

%% pazofrac (ähnlich nicefrac, aber besser auf mathpazo abgestimmt)
%% z.B. um „4/4-Takt“ oder „1/2 Liter“ zu schreiben, deshalb
%% von allgemeinem Interesse und nicht in header-math.tex
\newcommand{\pazofrac}[2]%
  {\leavevmode\kern.1em\raise.65ex\hbox{\scriptsize #1}%
   \kern-.125em/\kern-.125em\lower-.05ex\hbox{\scriptsize #2}}

%% Abstract ohne Einrückung des ersten Absatzes
\usepackage{etoolbox}
\ifundef{\abstract}{}{
  \patchcmd{\abstract}{\quotation}{\quotation\noindent\ignorespaces}{}{}}

%% hyperref should be the last package, except for cleveref.
%% We load cleveref and amsthm later.
\usepackage[pdfusetitle,%
            bookmarksnumbered=true,%
            unicode,%
            hidelinks]{hyperref}
            
%% Wenn Classico nicht verfügbar ist, dann umgehen wir Serifenlose
%% für URLs und Mailadressen und weichen auf Dicktengleiche aus.
\IfFileExists{classico.sty}{
  \let\mailfont\textsf
  \urlstyle{sf}
}{
  \let\mailfont\texttt
  \urlstyle{tt}
}

%% Mails
\newcommand{\mail}[1]{\upshape\href{mailto:#1}{\mailfont{#1}}}

%% Farben für Referenzen
\hypersetup{
  colorlinks,
  linkcolor={black!45!red},   % Interne Referenzen
  citecolor={black!55!green}, % Bibliographie
  urlcolor ={black!45!blue}   % Externe Referenzen
}

%% Umgebungen und cleveref (lädt: amsthm, cleveref)
\IfFileExists{preamble-env.tex}{%%% commit: dc6a172

%% Spätestens hier brauchen wir das amsthm-Paket.
%% Womöglich wurde es aber schon über die header-custom
%% eingebunden, etwa wenn individuelle Umgebungen definiert wurden.
\makeatletter
  \@ifpackageloaded{amsthm}{}{\usepackage{amsthm}}
\makeatother

%% cleveref has to be the last package,
%% but it has to be introduced before we define
%% individual theorem environments.
\usepackage[capitalise,noabbrev]{cleveref}

\theoremstyle{definition}
\newtheorem{aufg}{Aufgabe}[section]
\crefname{aufg}{Aufgabe}{Aufgaben}

%% A command between two environments, which takes effect
%% somewhere else (e.g. a figure that is placed on top of
%% a page, or an \enlargethispage) will cause additional
%% space between the environments. The following modified
%% commands prevent this.

%% Measure the intended distance
\newlength{\intendedDistance}

%% Wrap (e.g. \wrap{\enlargethispage{\baselineskip}})
\newcommand{\wrap}[1]{
  \setlength{\intendedDistance}{\lastskip}\addvspace{-\lastskip}
  #1
  \addvspace{\intendedDistance}
}

%% Repair top figures
\newenvironment{t-figure}{
  \setlength{\intendedDistance}{\lastskip}\addvspace{-\lastskip}
  \begin{figure}[t]
  }{
  \end{figure}
  \addvspace{\intendedDistance}
}

%% Repair top tables
\newenvironment{t-table}{
  \setlength{\intendedDistance}{\lastskip}\addvspace{-\lastskip}  
  \begin{table}[t]
  \centering
  }{
  \end{table}
  \addvspace{\intendedDistance}
}
}{}

%% Adressen am Ende eines Artikels mit Lining Figures
\newcommand{\addr}[2]{%
  \paragraph{#1}
  \LF{#2}
} 

%% Wir lassen KOMA den Satzspiegel selbst errechnen.
\KOMAoptions{DIV=calc}


\usepackage{scrlayer-scrpage}
\lohead{T. Drögemüller, F. Kranhold}
\cohead{}
\rohead{Theorie der reinen Intonation}
\pagestyle{scrheadings}

\begin{document}

\maketitle

\begin{abstract}
  In diesem Skript entwickeln wir die mathematischen und musikalischen
  Grundlagen der reinen Intonation in der klassischen Musik, einschließlich des
  pythagoreischen, syntonischen und septimalen Kommas sowie des Eulerschen
  Tonnetzes. Darauf aufbauend stellen wir eine Methode zum systematischen
  „Ausstimmen“ mehrstimmiger Vokalmusik vor. Unsere Herangehensweise ist dabei
  theoretisch, d.\,h. wir wollen uns nicht auf ein Instrument mit einer
  endlichen Anzahl an Tasten beschränken. Dieser Aufsatz wird von mehreren
  Audiobeispielen begleitet, die für den Open-Source-Interpreter
  \texttt{lilypond} geschrieben sind.\looseness-1
\end{abstract}

\section{Einleitung und Überblick}
\label{sec:int}
Wir setzen im Nachfolgenden einige Grundkonzepte klassischer Musiktheorie als
bekannt voraus, z.\,B. Töne, Intervallnamen, diatonische Skalen, Akkorde,
Tonarten und der Quintenzirkel. Dies wird im Wesentlichen von
\cite[§\,1–6]{Skript} abgedeckt.

Wenn ein Instrument (z.\,B. die menschliche Stimme) einen Ton erzeugt, schwingt
ein „Material“ (wie eine Saite oder eine Luftsäule) mit einer gewissen Frequenz.
Je höher die Frequenz ist, desto höher ist der Ton. Ein \emph{Stimmungssystem}
weist jedem Ton eine Frequenz zu. Typischerweise beginnen wir mit der Fixierung
des \emph{Kammertons}, der Frequenz des a’, heutzutage bei $440$\,Hz. Von dort
leiten wir die anderen Tonhöhen wie folgt ab: Die einfachste Beziehung zwischen
Tönen ist die der \emph{Oktave}, die einem Frequenzverhältnis von $1:2$
entspricht, z.\,B. bekommt das a’’ die Frequenz $880$\,Hz zugewiesen. In dem
modernen Stimmungssystem der \emph{gleichstufigen Stimmung} wird die Oktave in
zwölf gleich große Schritte – genannt \emph{Halbtonschritte} oder \emph{Hts} – 
aufgeteilt. Ein Halbtonschritt entspricht also dem Frequenzverhältnis
$1:2^{\smash{\frac1{12}}}$. Mit diesem Verhältnis kann nun jedem Ton eine
Frequenz zugewiesen werden; beispielsweise wird dem e’’ $7$\,Hts
über dem a’ eine Frequenz von \looseness-1
\[440\,\text{Hz}\cdot 2^{\frac7{12}} \approx 659{,}26\,\text{Hz}\]%
zugewiesen. Hier könnte man erwarten, dass die Erzählung, und damit auch dieses
Skript™, ihr Ende haben. Allerdings stellt sich dieses Stimmungssystem als
suboptimal heraus, sobald mehrere Töne gleichzeitig erklingen. Die Physik™ lehrt
uns,\todo{„Das ist halt Quatsch, außer man argumentiert mit Differenztönen“
  pks40} dass mehrere Wellen konsonieren, wenn ihre Frequenzen im Verhältnis
kleiner, paarweise teilerfremder Ganzzahlen stehen, z.\,B. $4:5:6:7$. Sehen wir
uns nun einen Durakkord mit kleiner Septime, bestehend aus Grundton, großer
Terz, reiner Quinte und kleiner Septime, an. Da eine große Terz aus $4$, eine
reine Quinte aus $7$ und eine kleine Septime aus $10$\,Hts bestehen,
ist das Frequenverhältnis dieser vier gleichzeitig erklingenden Töne
\looseness-1%
\[1:2^{\frac4{12}}:2^{\frac7{12}}:2^{\frac{10}{12}}\approx
  1:1{,}260:1{,}498:1{,}781.\]%
Die Beobachtung,\footnote{Streng genommen erzählen wir hier die Geschichte nicht
  in historisch korrekter Reihenfolge, denn es ist mitnichten so, dass im Laufe
  der Musikgeschichte Defizite der gleichstufigen Stimmung erkannt und angepasst
  wurden. Vielmehr kann die Idee ganzzahliger Frequenzverhältnisse als Beginn
  der Musiktheorie gesehen werden, während heutzutage z.\,B. für
  Tasteninstrumente die gleichstufige Stimmung aus Gründen der Praktikabilität
  verwendet wird. Wir haben uns dennoch für diese Darstellung entschieden, weil
  sie systematischer und zudem leichter zu verstehen ist.} die der Theorie der
reinen Stimmung zugrunde liegt, ist nun, dass dieses Verhältnis zwar
\emph{ungefähr}, aber nicht \emph{exakt} $4:5:6:7$ ist. Die „Unreinheiten“, die
wir hier sehen, sind deutlich kleiner als ein Halbtonschritt, weswegen wir ihre
Größe mit Hilfe der Einheit \emph{Cent} ausdrücken, wobei $100\ct$ einem
gleichstufigen Halbtonschritt entsprechen. Anders ausgedrückt: Die Oktave (d.\,h. die
Multiplikation mit $2$) umfasst $1{.}200\ct$. Wir können damit
beispielsweise sehen, dass das Verhältnis $4:5$ einem Unterschied von
\[1{.}200\ct\cdot \log_2(\tfrac54) \approx 386,31\ct\]%
entspricht, es also von dem vier Halbtonschritte ($400\ct$) großen Intervall um
$13{,}69\ct$ abweicht. Diese Unterschiede „korrigiert“ \emph{reine
  Intonation}, indem sie Akkordbestandteile systematisch mikroalteriert
(engl. „microtune“), also die Töne je nach harmonischem Kontext ein kleines
bisschen höher oder tiefer spielt. Das ist natürlich auf Musikinstrumenten mit
zwölf Tasten pro Oktave unmöglich; dort müssen andere Stimmungssysteme
eingesetzt werden. Nur an der Theorie und nicht an praktischen Lösungsansätzen
interessiert, werden wir letztere hier nicht ausführen. Unabhängig davon
\emph{ist} reine Intonation in der musikalischen Praxis durchaus einsetzbar,
z.\,B. im A-cappella-Gesang: Professionell ausgebildete Chorsänger:innen hören
aufeinander, um möglichst konsonant zu singen, und microtunen daher ihre Akkorde
\emph{intuitiv}, damit sie einrasten, siehe etwa \cite[§\,2.4]{Maria} für eine
ausführliche Übersicht zu diversen empirischen Untersuchungen.

Dieses einführende Skript leitet diese Microtunings systematisch her und zeigt,
wie sie konsistent eingesetzt werden können, zeigt aber auch auf, an welchen
Stellen dieses System an seine Grenzen stößt. Dabei gliedert es sich in folgende
Abschnitte:\looseness-1

\begin{enumerate}
  \setcounter{enumi}{1}
\item Mit der Einführung der \emph{pythagoreischen Stimmung} erreichen wir, dass
  alle reinen Quinten das Frequenzverhältnis von $2:3$ haben. Dies wird dazu
  führen, dass beispielsweise das \sharp g etwas höher wird als das \flat a;
  dieser Unterschied heißt \emph{pythagoreisches Komma}. Da wir aber nicht an
  ein Tasteninstument gebunden sind, ist das kein Problem; es zwingt uns nur
  dazu, Enharmonik ernst zu nehmen. Der Quintenzirkel wird zu einer unendlichen
  \emph{Quintenspirale}.  Von hier an ist unser Standardstimmungssystem das
  pythagoreische; das e’’ hat also eine Frequenz von
  $\frac32\cdot 440\,\text{Hz}=660\,\text{Hz}$ und nicht mehr
  $659{,}26\,\text{Hz}$.\looseness-1
\item Selbst im pythagoreischen System sind Dreiklänge nicht optimal gestimmt,
  z.\,B. wäre ein Dur-Dreiklang $64:81:96$ statt $4:5:6$. Die große Terz ist ein
  bisschen zu hoch; der Unterschied $\frac{81}{80}$ wird als \emph{syntonisches
    Komma} bezeichnet. Folglich bedeutet das Microtunen des Akkords eine
  Absenkung der großen Terz um ein syntonisches Komma. Wir erhalten eine zweite
  Quintenreihe, die um ein syntonisches Komma tiefer liegt. Wenn wir diesen
  Prozess in beide Richtungen wiederholen, können die entstehenden Reihen zum
  \emph{Eulerschen Tonnetz} zusammengesetzt werden, in dem die Bestandteile von
  (Dur- und Moll-)Dreiklängen nebeneinander liegen und Dreiecke bilden. Da
  \mbox{(gegen-)}parallele Dreiklänge benachbarten Dreiecken entsprechen, eignet
  sich dieses Gitter gut für die klassische Harmonik, erklärt aber auch die so
  genannte \emph{Kommafalle} für bestimmte Akkordfolgen: eine häufige Quelle für
  Detonationen™ (das ist der Verlust der ursprünglichen Grundfrequenz).
\item Während das Eulersche Tonnetz gut geeignet ist, um \emph{vertikale}
  Intervalle und Klänge zu stimmen, hat es auf der anderen Seite auch
  Auswirkungen auf \emph{horizontale} Intervalle und dadurch auf die
  Melodieführung jeder einzelnen beteiligten Stimme.  Wir wollen uns dieser
  Effekte bewusst werden, um sie bei unseren darauffolgenden Überlegungen zu
  berücksichtigen.
\item Mithilfe des Eulerschen Tonnetzes lassen sich auch andere für die
  klassische Musik übliche Akkorde (etwa die Subdominante mit Sixte ajoutée)
  sinnvoll ausstimmen. Wir werden den Versuch unternehmen, dies konsistent für
  die meisten relevanten Harmonien zu tun.
\item Selbst innerhalb des Eulerschen Tonnetzes kann der oben beschriebene
  Vierklang $4:5:6:7$ nicht realisiert werden – im Wesentlichen weil das Tonnetz
  nur die Primzahlen $2$, $3$ und $5$ kennt. Ein Dur-Dreiklang mit kleiner
  Septime im Eulerschen Tonnetz hat stattdessen das Verhältnis $36:45:54:64$.
  Die kleine Septime ist also ein bisschen zu hoch; der Fehler $\frac{64}{63}$
  wird als \emph{septimales Komma} bezeichnet. Wir werden einige (allerdings rar
  gesäte) Situationen in der klassischen Vokalmusik finden, in denen durch die
  Zuhilfenahme des septimalen Kommas ein „besseres“ Klangergebnis erreicht
  werden kann.
\end{enumerate}

% Dieser Ansatz unterscheidet sich in zwei Aspekten von der
% Standardliteratur:\todo{Welche? – \{Das würde mich auch interessieren.\} –T} 
% Erstens machen wir das pythagoreische System zum bevorzugten und verfolgen alle
% Abweichungen davon durch syntonische und septimale Kommata, während
% klassischerweise für jede Tonart eine sogenannte ptolemäische Skala verwendet
% wird, in der die Terz, die Sexte und die Septime standardmäßig um ein
% syntonisches Komma erniedrigt werden. Die Ptolemäische Skala hat den Vorteil,
% dass weniger Kommata verfolgt werden müssen; allerdings, werden wir darlegen,
% aus welchen systematischen Gründen wir uns gegen ihre Verwendung entschieden
% haben. Zweitens wird das septimale Komma in der Regel weggelassen,
% wahrscheinlich weil die durch den Tritonus verursachte Spannung (zwischen der
% kleinen Septime und der großen Terz) hörbar sein soll. Wir sind jedoch
% überzeugt, dass die Beachtung des septimalen Kommas zu einem saubereren System
% führt.

\noindent Bevor wir diesen Prozess beginnen, weisen wir darauf hin, dass diese
Abhandlung mit mehreren musikalischen Beispielen versehen ist, die am Seitenrand
dieses Dokuments angegeben sind\LY{m-4567} (hier ist ein erstes Beispiel, in dem
ein Dur-Akkord mit kleiner Septime zu hören ist, einmal in gleichstufiger
Stimmung und einmal genau \mbox{$4:5:6:7$}, wie oben beschrieben). Diese
Beispiele wurden mit dem Notensatzprogramm \verb!lilypond! geschrieben und
kodieren entweder eine \acr{PDF}-Datei (dann beginnt ihr Dateiname mit
\verb!p-!), eine \acr{MIDI}-Datei (\verb!m-!) oder beides (\verb!b-!). Auf
\href{https://github.com/fkranhold/jiam/}{\textsf{github.com/fkranhold/jiam}}
sind die Quelldateien, die Kompilate und, falls eine \acr{MIDI}-Datei unter
diesen ist, eine Audio-Konvertierung nach \acr{OGG} Vorbis zu finden. Wir
ermutigen alle, mit diesen Quelldateien selbst zu experimentieren.

%%% Local Variables:
%%% mode: latex
%%% TeX-master: "../main"
%%% End:


\section{Reine Quinten und die pythagoreische Stimmung}
\label{sec:pyth}
\subsection{Das Grad}

Wie in \cref{sec:int} erwähnt wollen wir uns zunächst mit dem genauen Verhältnis
einer reinen Quinte ($2:3$) beschäftigen. Die Differenz zur gleichstufigen
Quinte,
\[1{.}200\,\on{ct}\cdot \log_2(\tfrac32) - 700\,\on{ct} \approx
  1{,}96\,\on{ct}\]%
wird \emph{Grad} genannt und ist sehr klein (etwa $\frac1{50}$ eines
gleichstufigen Halbtonschrittes), aber dennoch etwas, das wir korrigieren
wollen.

\subsection{Enharmonik}

Ohne die Einschränkungen eines Instruments mit diskreten Tonhöhen kann das Grad
leicht korrigiert werden, sofern wir akzeptieren, dass enharmonisch
unterschiedene Tonhöhen (wie etwa \sharp g und \flat a) verschiedenen Frequenzen
entsprechen können.  Dies mag auf den ersten Blick seltsam erscheinen, aber das
Beispiel aus \cref{fig:enhContext} mag skeptische Leser:innen vielleicht
überzeugen (das Audiobeispiel ist in gleichstufiger Stimmung): In beiden Fällen
ist das resultierende Intervall acht Halbtonschritte groß (\flat e’\,–\,h’ und
\sharp d’\,–\,h’), aber im ersten Fall klingt es dissonant, da der Kontext
nahelegt, dass es aus vier Ganztönen besteht, während es im zweiten Fall
konsonant klingt.

\begin{figure}[h]
  \centering
  \LY{b-enhContext}
  \includegraphics{ly/b-enhContext.pdf}
  \caption{Eine übermäßige Quinte ist keine kleine Sexte.}\label{fig:enhContext}
\end{figure}

\subsection{Die pythagoreische Stimmung}

\noindent Akzeptierend, dass Enharmonik eine Rolle spielt, machen wir uns die
Tatsache zunutze, dass es für jede enharmonisch verschiedene Tonhöhe (c’, \sharp
h und \dflat d’ sind drei unterschiedliche Beispiele!) eindeutig bestimmte ganze
Zahlen $p,q\in\Z$ gibt, für die die gegebene Tonhöhe von a’ aus erreicht werden
kann, indem man $p$ reine Quinten und $q$ Oktaven nach oben geht (eine negative
Zahl bedeutet, dass man nach unten geht). Setzen wir den Kammerton auf $440\Hz$,
so ordnen wir der Tonhöhe die Frequenz $(\frac32)^p\cdot 2^q\cdot 440\Hz$
zu. Ein Beispiel:
\begin{align*}
  \text{\sharp g’}&\equiv(\tfrac32)^5\cdot 2^{-3}\cdot 440\Hz\approx 417.66\Hz,\\
  \text{\flat a’} &\equiv(\tfrac32)^{-7}\cdot 2^4\cdot 440\Hz\approx 412.03\Hz.
\end{align*}

\subsection{Das pythagoreische Komma und die Wolfsquinte}

Der Unterschied zwischen den obigen Frequenzen für \sharp g' und \flat a' heißt
\emph{pythagoreisches Komma} und beträgt
$3^{12}\cdot 2^{-19}\approx 23{,}46\,\on{ct}$. Wir bemerken, dass das
pythagoreische Komma genau $12$ Grad groß ist
($12\cdot 1{,}96\,\om{ct}\approx 23{,}46\om{ct}$), was daran liegt, dass die
beiden enharmonisch verwechselten Töne genau $12$ Quinten (und einige Oktaven)
auseinanderliegen.

Eine verminderte Sexte unterscheidet sich von einer reinen Quinte
notwendigerweise um genau dieses pythagoreische Komma. Eine verminderte Sexte
wird in der pythagoreischen Stimmung manchmal als „Wolfsintervall“ bezeichnet,
da sie sehr verstimmt klingt, siehe \cref{fig:pythComma}:2.

\subsection{Die Quintenspirale}

In \cref{fig:spiral5} ist eine enharmonische Variante des Quintenzirkels zu
sehen, die wir als \emph{Quintenspirale} bezeichnen können. Mit jeder vollen
Runde in dieser Spirale verschiebt sich die Tonhöhe um ein pythagoreisches
Komma, mit jedem Schritt (also $\frac1{12}$ einer Runde) entfernen wir uns um
$1$ zusätzliches Grad von der gleichstufigen Stimmung (oder nähern uns ihr; je
nach dem, in welche Richtung wir gehen).

Da die Töne einer heptatonischen Skala (z.\,B. Dur) einen zusammenhängenden
Abschnitt in dieser Quintenspirale bilden (z.\,B. sind die Töne von C-Dur genau
die von f bis h), sind diese Skalen „in sich“ (also relativ zum Grundton)
beinahe gleichstufig gestimmt: Die größte Abweichung in Dur ist wohl der Leitton
(also der 7. Ton), der pythagoreisch ganze $5$ Grad höher als gleichstufig ist.

\begin{figure}[h]
  \centering
  \includegraphics{ly/b-pythComma}
  \LY{b-pythComma}
  \caption{(1) Der Unterschied zwischen \sharp g’ and \flat a’. (2) Das
    Wolfsintervall \sharp g’\,–\,\flat e’, gefolgt von der reinen Quinte \sharp
    g’\,–\,\sharp d’.}\label{fig:pythComma}
\end{figure}

\subsection{Intervalle in pythagoreischer Stimmung}

Für jedes (enharmonisch präzise angegebenes) Intervall können wir das
entsprechende Frequenzverhältnis berechnen und angeben, um wie viel Grad das
Verhältnis vom gleichstufigen Fall abweicht, siehe \cref{tab:1}.

Wir bemerken, dass diese Tabelle fast alle relevanten Intervalle abdeckt: Wenn
ein bestimmtes Intervall das Verhältnis $a:b$ hat, so hat nämlich sein
\emph{Kom\-plementär\-intervall} das Verhältnis $b:2a$.  Wenn es die Größe $c$ in
Cent hat, so hat sein Komplementärintervall die Größe $1{.}200\,\on{ct}-c$, und
die Abweichung in Grad wechselt das Vorzeichen.  So entspricht eine kleine Sexte
in pythagoreischer Stimmung dem Verhältnis $81:128$, was dasselbe ist wie
$792{,}18\,\on{ct}$ und von der gleichstufigen kleinen Sexte um $-4$ Grad
abweicht.\looseness-1

\begin{figure}
  \centering%
  \newcommand{\pitcharray}{%
  \dflat d, \dflat a, \dflat e, \dflat h, \flat f,
  \flat c, \flat g, \flat d, \flat a, \flat e,
  \flat h, f, c, g, d, a, e, h,
  \sharp f, \sharp c, \sharp g, \sharp d, \sharp a, \sharp e, \sharp h}
\begin{tikzpicture}[xscale=-1]
  \def\xA{1.4}
  \def\xB{.1}
  \draw[dotted,domain=.35*pi:.4*pi,smooth,samples=100,variable=\t]
  plot ({(\xA+\xB*\t)*cos(\t r)},{(\xA+\xB*\t)*sin(\t r)});
  \draw[domain=.4*pi:4.6*pi,smooth,samples=100,variable=\t]
  plot ({(\xA+\xB*\t)*cos(\t r)},{(\xA+\xB*\t)*sin(\t r)});
  \draw[dotted,domain=4.6*pi:4.63*pi,smooth,samples=100,variable=\t]
  plot ({(\xA+\xB*\t)*cos(\t r)},{(\xA+\xB*\t)*sin(\t r)});
  \foreach \i in {0,...,24}{
    \node[fill=white,inner sep=1.5px]
    at ({(\xA+\xB*(\i+3)*pi/6)*cos((\i+3)*pi/6 r)},
    {(\xA+\xB*(\i+3)*pi/6)*sin((\i+3)*pi/6 r)})
    {\footnotesize \pitch[\i]$\strut$};
  };
\end{tikzpicture}

  \caption{Die enharmonische Quintenspirale. Mit jeder vollen Rotation im
  	Uhrzeigersinn gewinnen wir ein pythagoreisches Komma.}\label{fig:spiral5}
\end{figure}

\begin{table}
  \centering
  \begin{tabular}{lr@{\hspace*{2.4px}}lrr}
    \toprule
    \textsc{intervall} & \multicolumn{2}{c}{\textsc{verhältnis}} & %
    \textsc{größe in ct} & \textsc{abweichung in grad}\\
    \midrule
    kleine Sekunde  & ~~~$243$ & $:256$ &  $90{,}22$ & $-5$\\
    große Sekunde   &      $8$ & $:9$   & $203{,}91$ & $+2$\\
    kleine Terz     &     $27$ & $:32$  & $294{,}13$ & $-3$\\
    große Terz      &     $64$ & $:81$  & $407{,}82$ & $+4$\\
    reine Quarte    &      $3$ & $:4$   & $498{,}04$ & $-1$\\
    \bottomrule
  \end{tabular}
  \caption{Ausgewählte Intervalle in pythagoreischer Stimmung und ihre Differenz
    zur gleichstufigen Stimmung in Grad.}\label{tab:1}
\end{table}

\subsection{Einordnung}

Während alle noch kommenden Mikroalterationen (die darauf abzielen, ganze
Akkorde zu stimmen) den Preis haben werden, dass es mehrere Frequenzen für
denselben Tonnamen gibt, verwendet die bloße pythagoreische Stimmung nur die
Informationen, die uns durch die geschriebenen Noten gegeben werden.
Insbesondere haben in pythagoreischer Stimmung diatonische Halb- und
Ganztonschritte (also kleine und große Sekunde) feste Größen; eine diatonische
Skala in pythagoreischer Stimmung kennt also nur zwei Schrittweiten – eine
Tugend, die durch kontextabhängige Mikroalterationen leider verloren gehen wird.
Daher liegt es bei melodisch gedachten Passagen mit vielen horizontalen
Schritten oft nahe, sie pythagoreisch zu intonieren.

Der entscheidende Nachteil der pythagoreischen Stimmung, der nichts mit der
Realisierbarkeit auf einem diskretisierenden Instrument zu tun hat, sondern rein
theoretischer Natur ist, ist allerdings die Tatsache, dass sie die Unreinheit
der Terzen in einem Dreiklang im Vergleich zur gleichstufigen Stimmung sogar
noch erhöht: Der Unterschied zwischen einer pythagoreischen großen Terz
($407{,}82\,\on{ct}$) und dem oben erwähnten Verhältnis $4:5$
($386{,}31\,\on{ct}$) beträgt $21{,}51\,\on{ct}$. Dieser Fehler ist Gegenstand
des nächsten Abschnitts.

%%% Local Variables:
%%% mode: latex
%%% TeX-master: "../main"
%%% End:


\section{Dreiklänge und das Eulersche Tonnetz}
\label{sec:tri}
Von nun an interpretieren wir geschriebene Noten in pythagoreischer Stimmung und
bezeichnen alle Abweichungen davon mit bestimmten Symbolen, die wir einführen,
sobald sie notwendig werden.

\subsection{Das syntonische Komma}

Wir beginnen mit der Beobachtung, dass in pythagoreischer Stimmung ein
Dur-Dreiklang dem Verhältnis $64:81:96$ statt $4:5:6=64:80:96$ entspricht,
d.\,h. die große Terz etwas zu hoch ist (im Hörbeispiel hören wir zuerst einen
pythagoreischen Dur-Dreiklang und dann $4:5:6$).\LY{m-pyth456} Der Unterschied,
$\frac{81}{80}\approx 21{,}51\,\om{ct}$, heißt \emph{syntonisches
  Komma}.\footnote{Das syntonische Komma ist also \emph{fast} genauso groß wie
  das pythagoreische. Der Unterschied, nämlich
  $23{,}46\,\on{ct}-21{,}51\,\om{ct}\approx 1{,}95\,\om{ct}$, wird
  \emph{Schisma} genannt und ist \emph{fast} so groß wie ein Grad, aber nicht exakt,
  was wir auch ohne konkretes Nachrechnen begründen können: Das Verhältnis eines
  Schismas ist rational, während ein Grad $\smash{\frac32\cdot 2^{-\frac7{12}}}$
  und damit irrational ist.} Mit anderen Worten: Die große pythagoreische Terz
$64:81$ ist um ein syntonisches Komma größer als $4:5$, und die kleine
pythagoreische Terz $27:32$ ist ein um ein syntonisches Komma kleiner als $5:6$,
denn $5:6$ ist das Komplement von $4:5$ in $2:3$ und damit die gewünschte kleine
Terz. Von nun an nennen wir das Verhältnis $4:5$ eine \emph{reine} große Terz
und $5:6$ eine \emph{reine} kleine Terz. Folglich nennen wir ihre zur Oktave
komplementären Intervalle $5:8$ und $3:5$ die \emph{reine} kleine bzw. große
Sexte.\looseness-1

\subsection{Helmholtz-Ellis-Notation}

Wenn man versucht, Tonhöhen um ein syntonisches Komma zu alterieren, um die
Stimmung zu verbessern, stößt man auf das Problem, dass jeder Ton prinzipiell
die große Terz, aber auch der Grundton eines Dreiklangs sein kann. Das bedeutet:
Es hängt vom harmonischen Kontext ab, ob eine Tonhöhe alteriert werden muss
oder nicht, und das lässt sich leider nicht aus dem bloßen (enharmonisch
präzisen) Tonnamen herauslesen.\looseness-1

Bei einer Partitur fügen wir die zusätzliche Information, dass bestimmte Noten
um ein oder mehrere syntonische Kommata abgesenkt oder angehoben werden müssen,
durch das Hinzufügen von Pfeilen an die schon existenten Versetzungszeichen in
der Tradition von Helmholtz und Ellis \cite{HE} hinzu: Die Versetzungszeichen
\dsharpp, \sharpp, \naturalp, \flatp, und \dflatp\ bedeuten die Erhöhung des
Tons um ein syntonisches Komma; genauso bedeuten \dsharppp, \sharppp,
\naturalpp, \flatpp, und \dflatpp\ zwei Kommata und so weiter. In die andere
Richtung benutzen wir \dsharpm, \sharpm, \naturalm, \flatm, und \dflatm\ für
Noten, die um ein syntonisches Komma tiefer erklingen sollen. Ein Beispiel ist
in \cref{fig:HE} zu sehen. Wie üblich gelten Versetzungszeichen für einen ganzen
Takt; wenn also ein einzelner Takt erst \naturalm a' und dann \natural a'
enthält, so müssen beide Versetzungszeichen angegeben werden. Aus Gründen, die
im nächsten Unterabschnitt erläutert werden, verwenden wir überhaupt keine
Generalvorzeichen.\looseness-1

\begin{figure}\centering
  \includegraphics{ly/p-HE}
  \LY{p-HE}
  \caption{Im ersten Takt ist das \sharp f’ um ein syntonisches Komma
    tiefalteriert, sodass wir einen Dur-Dreiklang $4:5:6$ erhalten. Im zweiten
    Takt wird das g’ um ein, das \flat e’’ um zwei syntonische Kommata
    hochalteriert, sodass sich eine reine kleine Sexte $5:8$
    ergibt.}\label{fig:HE}
\end{figure}

\subsection{Die ptolemäische Skala}
\label{subsec:ptol}

Ein klassischer Ansatz, syntonische Kommata konsequent in einem
ganzen Musikstück einzusetzen, besteht darin, die pythagoreisch gestimmte
Dur-Tonleiter zu verändern, indem die Töne, die am ehesten als große Terzen
eines Dreiklangs auftreten, nämlich Terz, Sexte und Septime, um ein syntonisches
Komma erniedrigt werden. Das Ergebnis nennt man die \emph{ptolemäische
  Skala}. Legt man die Frequenz des Grundtons der Skala als $1$ fest, lassen
sich die Frequenzen aller in der Skala vorkommenden Noten als
$1,\frac98,\frac54,\frac43,\frac32,\frac53,\frac{15}8,2$ berechnen. Dieser Logik
folgend müsste etwa C-Dur drei Generalvorzeichen haben, nämlich \naturalm e,
\naturalm a\ und \naturalm h.

Wir haben uns aus den folgenden Gründen gegen die Verwendung dieser Skala
entschieden (wir vermeiden das Wort „Skala“ sogar ganz):
\begin{enumerate}
\item Der zweite Ton (in C-Dur: das d) der Skala kommt in zwei wichtigen
  „Rollen“ vor: Als Quinte der Dominante (g–h–d) und als Grundton der
  Subdominantparallele (d–f–a). Im Falle der Subdominantparallele muss auch
  dieser zweite Ton um ein syntonisches Komma tiefalteriert werden. Da beide
  Funktionen mit ähnlicher Häufigkeit auftreten, ist unklar, welcher Alteration
  im Rahmen einer Skala der Vorzug zu geben ist.
\item Die ptolemäische Skala weist nur einigen Tonhöhen eine Frequenz zu. Es
  gibt einige Tonhöhen, die mit einer Tonleiter „eng verwandt“ sind, auch wenn
  sie technisch gesehen nicht zu ihr gehören (z.\,B. das \sharp f in C-Dur, was
  die große Terz der Doppeldominante ist).
% \item Unser Stimmungssystem sollte nicht von der Tonart des Stücks, sondern nur
%   vom Kammerton abhängen.\todo{Ich bin mir nicht sicher ob ich diesen Punkt verstehe. Das Stimmungssystem ändert sich ja durch die Vorzeichen nicht, oder? Ich bin verwirrt. –T}
\end{enumerate}

\subsection{Das Eulersche Tonnetz}

Stattdessen wollen wir alle möglichen Frequenzen eines Stücks in einer Weise
anordnen, die ihre harmonische Nähe zueinander widerspiegelt. Dies wird
typischerweise durch das \emph{Eulersche Tonnetz} erreicht, siehe
\cref{fig:latticeExcerpt}: Wir beginnen mit der Quintenspirale aus
\cref{sec:pyth}, die wir jetzt als horizontale Gerade schreiben, und unterteilen
jeden Schritt ($2:3$) in eine reine große Terz ($4:5$) und eine reine kleine
Terz ($5:6$). Dieser Zwischenton wird zwischen dem ersten und dem zweiten, aber
etwas weiter unten notiert. Wenn wir alle Schritte unterteilen, erhalten wir
eine vollständige neue Zeile, die wieder aus reinen Quinten besteht, aber jetzt
sind die Frequenzen in dieser Zeile um ein syntonisches Komma
erniedrigt. Dasselbe lässt sich auch in der anderen Richtung machen, indem man
die Reihenfolge von kleiner und großer Terz vertauscht und die Zwischentöne
etwas höher als die Ursprungsgerade schreibt. Wir erhalten eine neue
Quintenreihe, in der alle Frequenzen um ein syntonisches Komma erhöht sind. Wenn
wir diesen Prozess in beide Richtungen wiederholen, erhalten wir ein
Tonhöhengitter (siehe \cref{fig:latticeExcerpt}), bei dem eine Bewegung der Form
‚$\to$‘ einer reinen Quinte, ‚$\searrow$‘ einer reinen großen Terz und
‚$\nearrow$’ einer reinen kleinen Terz entspricht. Jede (enharmonisch präzise)
Tonklasse (d.\,h. ihre Oktave ignorierend) kommt in jeder Zeile genau einmal
vor, sodass ein Knoten im Gitter einer Tonklasse zusammen mit einer Anzahl von
darauf angewandten syntonischen Kommata entspricht.%

\begin{figure}[h]
  \[
  \begin{tikzcd}[column sep=-1em,labels=description]
    & \text{\flatp D}\,(\frac{16}{15})\ar[rr,dash]\ar[dr,dash] && \text{\flatp A}\,(\frac45)\ar[rr,dash]\ar[dr,dash] && \text{\flatp E}\,(\frac{6}{5})\ar[dr,dash]\ar[rr,dash] && \text{\flatp B}\,(\frac95)\ar[dr,dash]\ar[rr,dash] && \text{\naturalp F}\,(\frac{27}{20})\ar[dr,dash]\\
    \text{\flat B}\,(\frac{16}9)\ar[rr,dash]\ar[dr,dash]\ar[ur,dash] && \rF\,(\frac43)\ar[ur,dash]\ar[rr,dash]\ar[dr,dash] && \rC\,(1)\ar[dash]{ur}{5:6}\ar[dash]{rr}{2:3}\ar[dash]{dr}{4:5} && \rG\,(\frac32)\ar[ur,dash]\ar[rr,dash]\ar[dr,dash] && \rD\,(\frac98)\ar[ur,dash]\ar[rr,dash]\ar[dr,dash] && \rA\,(\frac{27}{16})\\
    & \text{\naturalm D}\,(\frac{10}9)\ar[rr,dash]\ar[dr,dash]\ar[ur,dash] && \text{\naturalm A}\,(\frac53)\ar[rr,dash]\ar[dr,dash]\ar[ur,dash] && \text{\naturalm E}\,(\frac54)\ar[rr,dash]\ar[dr,dash]\ar[ur,dash] && \text{\naturalm B}\,(\frac{15}{16})\ar[rr,dash]\ar[dr,dash]\ar[ur,dash] && \text{\sharpm F}\,(\frac{45}{32})\ar[dr,dash]\ar[ur,dash]\\
    \text{\naturalmm B}\,(\frac{27}{25})\ar[rr,dash]\ar[ur,dash] && \text{\sharpmm F}\,(\frac{50}{36})\ar[rr,dash]\ar[ur,dash] && \text{\sharpmm C}\,(\frac{25}{24})\ar[rr,dash]\ar[ur,dash] && \text{\sharpmm G}\,(\frac{25}{16})\ar[rr,dash]\ar[ur,dash] && \text{\sharpmm D}\,(\frac{75}{64})\ar[rr,dash]\ar[ur,dash] && \text{\sharpmm A}\,(\frac{225}{128})
  \end{tikzcd}
\]

  \caption{Ein Ausschnitt des Eulerschen Tonnetzes. Wir berechnen
    Frequenzfaktoren relativ zum c, und „oktav-normiert“, sodass sie zwischen
    $1$ und $2$ liegen.}\label{fig:latticeExcerpt}
\end{figure}

\subsection{Geometrie im Tonnetz}

In Eulerschen Tonnetz umfasst ein Dreieck der Form $\triangledown$ die
Bestandteile eines rein intonierten Dur-Akkords ($4:5:6$), während ein Dreieck
$\vartriangle$ die Bestandteile eines Mollakkords ($10:12:15$)
erfasst. Benachbarte Dreiecke teilen sich entweder eine Quinte, d.\,h. sie sind
\emph{Varianten} voneinander, oder eine Terz, d.\,h. sie stehen in
\emph{paralleler} ($\vartriangle\!\!\!\triangledown$) oder
\emph{gegenparalleler} ($\triangledown\!\!\!\vartriangle$) Beziehung
zueinander. Legt man ein Dreieck als Tonika eines Stücks fest, so lassen sich
die Standardfunktionen wie \mbox{(Sub-)}Dominanten sowie ihre (Gegen-)Parallelen
und sogar Zwischendominanten (vgl. \cite[§\,10]{Skript}) leicht ermitteln: Für
ein gegebenes Dreieck ist seine Zwischendominante das Dur-Dreieck, das die
Quinte rechts neben dem gegebenen Dreieck enthält (z.\,B.  ist die
Doppeldominante von C-Dur der D-Dur-Dreiklang in der Mitte rechts in
\cref{fig:latticeExcerpt}, nicht der Dreiklang unten links, der ein syntonisches
Komma tiefer liegt.\looseness-1

\subsection{Mikroalterationen funktionaler Harmonien}

Bei einem Musikstück (z.\,B. \cref{fig:triadPiece}) bestimmt man zunächst seine
Tonart (hier: e-Moll). Dann wird das Tonika-Dreieck im Eulerschen Tonnetz
eindeutig durch den Modus (Dur oder Moll) und die Bedingung bestimmt, dass der
Grundton (hier: e) nicht durch ein Komma verändert wird. Für jeden Ton bestimmen
wir dann den Dreiklang, zu dem er gehört (der vierte Ton des Tenors, a, ist Teil
eines D-Dur-Dreiklangs) und die Funktion des Dreiklangs (Zwischendominante zur
Tonikaparallele), und finden dann das entsprechende Dreieck im Eulerschen
Tonnetz (siehe \cref{fig:chordsLattice} für die Verortung aller in
\cref{fig:triadPiece} vorkommen Dreiklänge im Eulerschen Tonnetz). Daraus ergibt
sich dann, wie viele syntonische Kommata wir auf die Note anwenden müssen (hier:
a muss um eines erhöht werden). Seine absolute Frequenz beträgt
$\frac{81}{80}\cdot 220\,\text{Hz}=222{,}75\,\text{Hz}$, und relativ zum
nächsttieferen Grundtons, e, entspricht er dem Faktor
$\frac{81}{80}\cdot\frac{4}{3}=\frac{27}{20}$.\looseness-1

Wir weisen darauf hin, dass bei einem Stück in C-Dur der Grundton eines
a-Moll-Dreiklangs um ein syntonisches Komma herabgesetzt wird, während dies bei
einem Stück in a-Moll nicht der Fall ist – auch wenn das Stück dieselben
Vorzeichen hat. Daher hat bei einem Stück mit gegebenen Generalvorzeichen die
Entscheidung, welcher der beiden möglichen Dreiklänge die Tonika ist
(z.\,B. C-Dur oder a-Moll), Auswirkungen darauf, wie hoch das Stück intoniert
werden soll. Das mag seltsam klingen, ergibt sich aber zwingend, wenn wir uns
darauf einigen, dass der Grundton der Tonika als Zentrum des Stücks direkt vom
Kammerton abgeleitet werden sollte. Im Übrigen ist dies auch aus praktischen
Gesichtspunkten sinnvoll, wenn die Anzahl der Mikroalterationszeichen möglichst
klein gehalten werden soll.

Wir weisen außerdem darauf hin, dass die „richtige“ Intonation von der
Interpretation der Funktion abhängt, die der Dreiklang darstellt. Ein
D-Dur-Dreiklang in C-Dur z.\,B.  wird wohl in den allermeisten Fällen als
Doppeldominante verstanden werden, in seltenen Fällen aber auch als „dorische“
Subdominante der Tonikaparallele (a-Moll). Im zweiten Fall müsst der Akkord ein
syntonisches Komma tiefer gespielt werden.

\begin{figure}
  \centering
  \LY{b-triadPiece}
  \includegraphics{ly/b-triadPiece}
  \caption{Ein schematisches vierstimmiges Stück in e-Moll (nur aus
  	Dreiklängen), mit eingebauten syntonischen Kommata. Es gibt zwei
  	Audioausgaben des Stücks, eine in reiner pythagoreischer Stimmung und eine,
  	die die syntonischen Kommata zur Korrektur
  	verwendet.\looseness-1}\label{fig:triadPiece}
\end{figure}

\begin{figure}
  \[
  \begin{tikzcd}[row sep=.7em,column sep=.7em]
    & \text{\naturalp c}\ar[rr,dash]\ar[ddr,dash] && \text{\naturalp g}\ar[rr,dash]\ar[ddr,dash] && \text{\naturalp d}\ar[rr,dash]\ar[ddr,dash] && \text{\naturalp a}\ar[ddr,dash] &\\
    & 7 & 10 & 1,3  & 5,9 & & 4,8 & & \\
    \text{a}\ar[rr,dash]\ar[uur,dash]\ar[ddr,dash] && \text{e}\ar[rr,dash]\ar[uur,dash]\ar[ddr,dash] && \text{h}\ar[rr,dash]\ar[uur,dash]\ar[ddr,dash] && \text{\sharp f}\ar[rr,dash]\ar[uur,dash]\ar[ddr,dash] && \text{\sharp c}\\
    &   &    & 6,13 &     & 2,12     & & 11\\
    & \text{\sharpm c}\ar[rr,dash]\ar[uur,dash] && \text{\sharpm g}\ar[rr,dash]\ar[uur,dash] &&
    \text{\sharpm d}\ar[rr,dash]\ar[uur,dash] && \text{\sharpm a}\ar[uur,dash]
  \end{tikzcd}
\]

  \caption{Die Dreiklänge die im Notenbeispiel von \cref{fig:triadPiece} auftauchen, verortet im
  	Eulerschen Tonnetz.}\label{fig:chordsLattice}
\end{figure}

\subsection{Die Kommafalle}

Wenn wir einen erneuten Blick auf den Ausschnitt des Eulerschen Tonnetzes in
\cref{fig:latticeExcerpt} werfen, so finden wir den Ton d an zwei Stellen in der
Nähe von c, nämlich \naturalm $\text{d}\,(\frac{10}9)$ and
$\text{d}\,(\frac98)$. Beide erscheinen als Eckpunkte von Dreiecken, die
Funktionen in C-Dur entsprechen: Der erste als Grundton von d-Moll
(Subdominantparallele, kurz Sp) und der zweite als Quinte von G-Dur (Dominante,
kurz D) – dies ist genau das Phänomen, das wir in \cref{subsec:ptol} im Zuge
unserer Kritik an einer festen mikroalterierten Skala erwähnt haben.

Man stelle sich nun ein Stück vor, das die Funktionen Sp und D direkt
hintereinander enthält, sowie eine Stimme, die den Grundton der Sp hat und hält,
während er zur Quinte von D wird (siehe \cref{fig:drift}). Dann gibt es
zwei Möglichkeiten:

\begin{enumerate}[itemsep=0em]
\item Einer der Akkorde wird um ein syntonisches Komma transponiert.
\item Die haltende Stimme muss die Tonhöhe beim Harmoniewechsel 
  \emph{nachjustieren}.
\end{enumerate}%
%
In der Vokalmusik geschieht typischerweise ersteres, weil die übliche
Interpretation des Haltens einer Note darin besteht, sie – \emph{per
  definitionem} – nicht zu verändern, und die übrigen Stimmen, die sich bewegen
müssen, ihre richtige Intonation im Verhältnis zum gehaltenen Ton suchen. Die
Folge ist, dass der zweite Akkord (D) um ein syntonisches Komma absinkt
und das Ensemble an Intonation verliert, d.\,h. es \emph{detoniert}, siehe den
ersten Teil von \cref{fig:drift}. Dieser Effekt wird als \emph{Kommafalle}
bezeichnet. Wenn dieses Phänomen fünfmal auftritt, beträgt die Detonation
$5\cdot 21{,}51\,\on{ct}=107{,}55\,\on{ct}$, d.\,h. das Ensemble ist in Summe um
mehr als einen gleichstufigen Halbton gesackt.\looseness-1

Die wünschenswertere Lösung, die jedoch vom Ensemble ein Verständnis der
harmonischen Situation verlangt, ist die zweite, die im zweiten Teil von
\cref{fig:drift} dargestellt ist. Dieses Beispiel zeigt, dass das theoretische
Bewusstsein für reine Intonation (die, wie wir behauptet haben, Chorsänger:innen
intuitiv nutzen) dabei helfen kann, die Ursache für ein sehr praktisches und
häufiges Problem aufzuspüren, nämlich das des Intonationsverlusts.

Die Akkordfolge Sp\kern1px–\kern1px D ist nicht die einzige, die
Kommaabweichungen verursachen kann. Der dritte Teil von \cref{fig:drift} zeigt
eine modulierende Akkordfolge, die die Doppeldominante einbezieht. Hier müssen
sogar zwei Stimmen, die eine Quinte voneinander entfernt sind, ihre Intonation
anpassen. Die gleichzeitigen An\-passungen ergeben zusammen eine „Bewegung“ in
parallelen Quinten.\footnote{Das Parallelenverbot im klassischen Kontrapunkt ist
  nicht für solche „Mikroparallelen“ ausgelegt, weswegen diese Bewegungen
  satztechnisch unproblematisch sind.}\looseness-1

\begin{figure}
  \centering
  \LY{b-drift}
  \includegraphics{ly/b-drift}
  \caption{(1) Eine Akkordfolge, bei der alle wiederholten Tonhöhen auf die   
  	gleiche Weise intoniert werden, wodurch eine Detonation um ein syntonisches
  	Komma entsteht. %
  	(2) Dieselbe Akkordfolge mit der notwendigen Anpassung (durch eine
  	Zick-Zack-Linie gekennzeichnet): Zwischen dem dritten und dem vierten Akkord
  	muss der Sopran das d’’ anpassen. %
  	(3) Eine ähnliche Situation, in der sogar zwei Stimmen eine Anpassung
  	vornehmen müssen, was eine Komma-Bewegung in „parallelen Quinten“
  	verursacht.\looseness-1}\label{fig:drift}
\end{figure}

%%% Local Variables:
%%% mode: latex
%%% TeX-master: "../main"
%%% End:


\section{Die melodische Perspektive}
\label{sec:melody}
In \cref{sec:tri} haben wir gelernt, wie man Stücke stimmt, bei denen jede Note
Teil eines „vertikalen“ Dur- oder Molldreiklangs ist. Akkorde haben jedoch sehr
oft andere Bestandteile als nur Grundton, Terz und Quinte. In diesem Abschnitt
besprechen wir die wichtigsten Variationen von Dur- und Moll-Akkorden und wie
sie gestimmt werden können.

\subsection{Quartvorhalt und erweiterter Nonakkord}
\label{sec:49}

Ein Dreiklang mit \emph{Quartvorhalt} besteht aus dem Grundton, der Quarte und
der Quinte, wobei sich die Quarte in der Regel auf dem nächsten Schlag in die
(große oder kleine) Terz auflöst, vgl. \cite[{}8.2]{Skript}. Dieser Akkord
entspricht in pythagoreischer Stimmung bereits dem sehr übersichtlichen
Verhältnis $6:8:9$ und benötigt keine Korrektur durch ein syntonisches Komma
(was auch daran liegt, dass hier nur Quarte und Quinte vorkommen, Quinten
optimal gestimmt und Quarten dazu komplementär sind).  Wir weisen noch auf zwei
horizontale Intervalle hin:
\begin{enumerate}
\item Die Quarte des Akkordes muss \emph{konsonant eingeführt werden},
  d.\,h. dieselbe Note tauchte im vorherigen regulär Akkord auf und ist durch
  den Harmoniewechsel zur Quarte geworden. Stellt man sich z.\,B. die
  Akkordverbindung T\,–\,D\textsuperscript{$4$–$3$} vor, bei der der Grundton
  der Tonika zur Quarte der Dominante wird, so ist keine Anpassung notwendig, da
  beide Töne, Grundton der Tonika und Quarte der Dominante, auf der gleichen
  Kommahöhe liegen.\todo{Das Wort „Kommahöhe“ scheint mir sehr praktikabel zu
    sein. Man sollte es einmal in §3 einführen. FK}
\item Die Auflösung der Quarte in die große Terz ist ein Halbton abwärts, dessen
  Größe $15:16\approx 111{,}73\,\on{ct}$ beträgt, die Auflösung der Quarte in
  die kleine Terz ist ein Ganzton im Verhältnis $9:10\approx 182{,}40\,\on{ct}$.
\end{enumerate}

Der \emph{erweiterte Nonakkord} ist ein Vierklang, der sich aus aus einem (Dur-
oder Moll-)Akkord und einer zusätzlichen großen None zusammensetzt.  Die None
liegt eine reine Quinte über der Quinte des Dreiklangs und hat daher die gleiche
Kommahöhe wie der Grundton und die Quinte. Dieser
Akkord entspricht dem Verhältnis $4:5:6:9$ in Dur und $20:24:30:45$ in Moll. Wir
stellen fest, dass die kleine Sekunde zwischen der großen None und der Mollterz
(plus eine Oktave) wieder dem Verhältnis $15:16\approx 111{,}73\,\on{ct}$
entspricht.

In \cref{fig:chordLines} haben wir die obigen Akkorde als ins Eulerschen Tonnetz
eingezeichnet. Dabei markieren wir nicht nur die Eckpunkte, sondern auch alle
direkten Kanten zwischen ihnen, da jede Kante für eine Beziehung der Form $2:3$,
$4:5$ oder $5:6$ innerhalb des jeweiligen Akkordes steht.

\subsection{Sixte ajoutée}
\label{sec:sixte}

Sehr oft kommt der Subdominantakkord (sowohl in Dur als auch in Moll) mit der
großen Sexte über dem Grundton des Akkords zusammen vor
(z.\,B. f\,:\,a\,:\,c\,:\,d in C-Dur oder f\,:\,\flat a\,:\,c\,:\,d in c-Moll),
vgl. \cite[{}9.3]{Skript}.  Im Fall von Dur ist dieser Akkord eine Verschmelzung
von zwei Dreiklängen, die sich eine Terz teilen (im Beispiel von C-Dur: F-Dur
und d-Moll). Dementsprechend verwenden wir die Ecken der benachbarten Dreiecke
$\vartriangle\!\!\!\triangledown$, um die Stimmung der Einzeltöne abzuleiten
(siehe \cref{fig:chordLines}), und gelangen so zum Verhältnis
$12:15:18:20$. Hier verbirgt sich allerdings wieder eine Kommafalle: Wenn auf
den Sixte ajoutée die Dominante folgt und die Sexte der Subdominante zur Quinte
der Dominante wird (dies sind die gleichen Noten!), somuss die große Sexte beim
Harmoniewechsel um ein syntonisches Komma nach oben angepasst werden (siehe
\cref{fig:496}).  Dies spräche zwar dafür, die Sexte von Anfang an ein Komma
höher zu intonieren, allerdings scheint uns die Verschmelzung der zwei
Dreiklänge und die dadurch entstehende charakteristische Ambiguität des
Vierklangs wichtiger.

In Moll fällt diese Abwägung anders aus, da der Sixte ajoutée in Moll keine
Verschmelzung zweier Dreiecke wie oben beschrieben ist.  Daher wird in diesem
Fall die Sexte verwendet, die auf der gleichen Kommahöhe wie Grundton und Quinte
liegt. Der sich daraus ergebende Akkord hat das Verhältnis $80:96:120:135$.  Die
verminderte Quinte (Tritonus) zwischen großer Sexte und kleiner Terz (plus
Oktave) beträgt $45:64\approx 609{,}78\,\on{ct}$.  Nicht nur wird durch diese
Syntonisierungswahl die oben genannte Kommafalle umgangen; es wird auch ein
melodisches Problem gelöst, vgl. \cref{sec:semi}: Nicht selten wird die Sexte
der Subdominante (in c-Moll: d) von der Terz der Tonika (in c-Moll: es) aus
erreicht.  Nutzte man für die Mollterz \flatp $3$ und für die Sexte \naturalp
$6$, so hätte dieser Halbtonschritt die Größe $25:27\approx 133{,}28\,\om{ct}$,
was viel zu groß ist.

% Beide Akkorde sind in \cref{fig:chordLines} wieder als Graphen im Tonnetz
% eingetragen.

\begin{figure}
  \centering
  \LY{b-496}
  \includegraphics{ly/b-496}
  \caption{Vier Beispiele, darunter der Quartvorhalt, der Nonakkord und der
    Sixte ajoutée. Der Unterschied zwischen dem zweiten und dem dritten Beispiel
    ist nur eine einzige Note im Tenor: In der ersten Version wird die von uns
    vorgeschlagene große Sexte verwendet, in der zweiten Version die ein
    syntonisches Komma tiefere, die zu dem zu großen Halbtonschritt $25:27$
    im Tenor führt.}\label{fig:496}
\end{figure}

\begin{figure}
  \centering
  \begin{tikzpicture}[scale=.7]
  %% Suspended fourth major
  %  Euler lattice
  \draw[black!50] (0, 0) -- (4, 0);
  \draw[black!50] (1, {sqrt(3)}) -- (3, {sqrt(3)});
  \draw[black!50] (1,-{sqrt(3)}) -- (3,-{sqrt(3)});
  \draw[black!50] (0,0) -- (1, {sqrt(3)}) -- (2,0) -- (3, {sqrt(3)}) -- (4,0);
  \draw[black!50] (0,0) -- (1,-{sqrt(3)}) -- (2,0) -- (3,-{sqrt(3)}) -- (4,0);
  %  Just third and fifth relations
  \draw[black,line width=1mm] (0,0) -- (4,0);
  %  Circles for chord notes labels
  \draw[fill=white] (0,0)          circle (.3);
  \draw[fill=white] (2,0)          circle (.3);
  \draw[fill=white] (4,0)          circle (.3);
  \draw[fill=white] (3,-{sqrt(3)}) circle (.3);
  %  Chord notes labels
  \node at (0,0)              {\scriptsize $4$};
  \node at (2,0)              {\scriptsize $1$};
  \node at (4,0)              {\scriptsize $5$};
  \node at (3,-{sqrt(3)})     {\scriptsize $3$};
  %  Semitone resolutions
  \draw[white,line width=1mm] (.3,{-sqrt(3)*1/10}) -- (2.7,{-sqrt(3)*9/10});
  \draw[-stealth]             (.3,{-sqrt(3)*1/10}) -- (2.7,{-sqrt(3)*9/10});
  %  Label of chord
  \node[rotate=0] at (-1.5,0) {\small $\strut$Dur}; % For rotated text set rotate option to 90 instead of 0
  \node at (2,2.5)            {\small $\strut$Quartvorhalt};
  
  
  %% Extended ninth major
  %  Euler lattice
  \draw[shift={(5.5,0)},black!50] (0, 0) -- (4, 0);
  \draw[shift={(5.5,0)},black!50] (1, {sqrt(3)}) -- (3, {sqrt(3)});
  \draw[shift={(5.5,0)},black!50] (1,-{sqrt(3)}) -- (3,-{sqrt(3)});
  \draw[shift={(5.5,0)},black!50] (0,0) -- (1, {sqrt(3)}) -- (2,0) -- (3, {sqrt(3)}) -- (4,0);
  \draw[shift={(5.5,0)},black!50] (0,0) -- (1,-{sqrt(3)}) -- (2,0) -- (3,-{sqrt(3)}) -- (4,0);
  %  Just third and fifth relations
  \draw[shift={(5.5,0)},line width=1mm] (0,0) -- (4,0);
  \draw[shift={(5.5,0)},line width=1mm] (0,0) -- (1,-{sqrt(3)}) -- (2,0);
  %  Circles for chord notes labels
  \draw[shift={(5.5,0)},fill=white] (0,0)          circle (.3);
  \draw[shift={(5.5,0)},fill=white] (2,0)          circle (.3);
  \draw[shift={(5.5,0)},fill=white] (1,-{sqrt(3)}) circle (.3);
  \draw[shift={(5.5,0)},fill=white] (4,0)          circle (.3);
  %  Chord notes labels
  \node[shift={(3.85,0)}] at (0,0)          {\scriptsize $1$};
  \node[shift={(3.85,0)}] at (1,-{sqrt(3)}) {\scriptsize $3$};
  \node[shift={(3.85,0)}] at (2,0)          {\scriptsize $5$};
  \node[shift={(3.85,0)}] at (4,0)          {\scriptsize $9$};
  %  Label of chord
  \node[shift={(3.85,0)}] at (2,2.5)        {\small $\strut$Hinzugefügte None};
  
  
  %% Sixte ajoutée major
  %  Euler lattice
  \draw[shift={(11,0)},black!50] (0, 0) -- (4, 0);
  \draw[shift={(11,0)},black!50] (1, {sqrt(3)}) -- (3, {sqrt(3)});
  \draw[shift={(11,0)},black!50] (1,-{sqrt(3)}) -- (3,-{sqrt(3)});
  \draw[shift={(11,0)},black!50] (0,0) -- (1, {sqrt(3)}) -- (2,0) -- (3, {sqrt(3)}) -- (4,0);
  \draw[shift={(11,0)},black!50] (0,0) -- (1,-{sqrt(3)}) -- (2,0) -- (3,-{sqrt(3)}) -- (4,0);
  %  Just third and fifth relations
  \draw[shift={(11,0)},line width=1mm] (2,0) -- (4,0);
  \draw[shift={(11,0)},line width=1mm] (1,-{sqrt(3)}) -- (3,-{sqrt(3)});
  \draw[shift={(11,0)},line width=1mm] (1,-{sqrt(3)}) -- (2,0) -- (3,-{sqrt(3)}) -- (4,0);
  %  Circles for chord notes labels
  \draw[shift={(11,0)},fill=white] (1,-{sqrt(3)}) circle (.3);
  \draw[shift={(11,0)},fill=white] (2,0)          circle (.3);
  \draw[shift={(11,0)},fill=white] (3,-{sqrt(3)}) circle (.3);
  \draw[shift={(11,0)},fill=white] (4,0)          circle (.3);
  %  Chord notes labels
  \node[shift={(7.7,0)}] at (2,0)          {\scriptsize $1$};
  \node[shift={(7.7,0)}] at (3,-{sqrt(3)}) {\scriptsize $3$};
  \node[shift={(7.7,0)}] at (4,0)          {\scriptsize $5$};
  \node[shift={(7.7,0)}] at (1,-{sqrt(3)}) {\scriptsize $6$};
  %  Label of chord
  \node[shift={(7.7,0)}] at (2,2.5)        {\small $\strut$Sixte ajoutée};
  
  %% Suspended fourth minor
  %  Euler lattice
  \draw[shift={(0,-4.5)},black!50] (0, 0) -- (4, 0);
  \draw[shift={(0,-4.5)},black!50] (1, {sqrt(3)}) -- (3, {sqrt(3)});
  \draw[shift={(0,-4.5)},black!50] (1,-{sqrt(3)}) -- (3,-{sqrt(3)});
  \draw[shift={(0,-4.5)},black!50] (0,0) -- (1, {sqrt(3)}) -- (2,0) -- (3, {sqrt(3)}) -- (4,0);
  \draw[shift={(0,-4.5)},black!50] (0,0) -- (1,-{sqrt(3)}) -- (2,0) -- (3,-{sqrt(3)}) -- (4,0);
  %  Just third and fifth relations
  \draw[shift={(0,-4.5)},black,line width=1mm] (0,0) -- (4,0);
  %  Circles for chord notes labels
  \draw[shift={(0,-4.5)},fill=white] (0,0)          circle (.3);
  \draw[shift={(0,-4.5)},fill=white] (2,0)          circle (.3);
  \draw[shift={(0,-4.5)},fill=white] (4,0)          circle (.3);
  \draw[shift={(0,-4.5)},fill=white] (3,+{sqrt(3)}) circle (.3);
  %  Semitone resolutions
  \draw[shift={(0,-4.5)},white,line width=1mm] (.3,{+sqrt(3)*1/10}) -- (2.7,{+sqrt(3)*9/10});
  \draw[shift={(0,-4.5)},-stealth]             (.3,{+sqrt(3)*1/10}) -- (2.7,{+sqrt(3)*9/10});
  %  Chord notes labels
  \node[shift={(0,-3.15)}] at (0,0)          {\scriptsize $4$};
  \node[shift={(0,-3.15)}] at (2,0)          {\scriptsize $1$};
  \node[shift={(0,-3.15)}] at (4,0)          {\scriptsize $5$};
  \node[shift={(0,-3.15)}] at (3,+{sqrt(3)}) {\scriptsize \flat $3$};
  %  Label of chord
  \node[rotate=0]          at (-1.5,-4.5)    {\small $\strut$Moll}; % For rotated text set rotate option to 90 instead of 0
  
  
  %% Extended ninth minor
  %  Euler lattice
  \draw[shift={(5.5,-4.5)},black!50] (0, 0) -- (4,0);
  \draw[shift={(5.5,-4.5)},black!50] (1, {sqrt(3)}) -- (3, {sqrt(3)});
  \draw[shift={(5.5,-4.5)},black!50] (1,-{sqrt(3)}) -- (3,-{sqrt(3)});
  \draw[shift={(5.5,-4.5)},black!50] (0,0) -- (1, {sqrt(3)}) -- (2,0) -- (3, {sqrt(3)}) -- (4,0);
  \draw[shift={(5.5,-4.5)},black!50] (0,0) -- (1,-{sqrt(3)}) -- (2,0) -- (3,-{sqrt(3)}) -- (4,0);
  %  Just third and fifth relations
  \draw[shift={(5.5,-4.5)},line width=1mm] (0,0) -- (4,0);
  \draw[shift={(5.5,-4.5)},line width=1mm] (0,0) -- (1,+{sqrt(3)}) -- (2,0);
  %  Circles for chord notes labels
  \draw[shift={(5.5,-4.5)},fill=white] (0,0)          circle (.3);
  \draw[shift={(5.5,-4.5)},fill=white] (2,0)          circle (.3);
  \draw[shift={(5.5,-4.5)},fill=white] (1,+{sqrt(3)}) circle (.3);
  \draw[shift={(5.5,-4.5)},fill=white] (4,0)          circle (.3);
  %  Chord notes labels
  \node[shift={(3.85,-3.15)}] at (0,0)          {\scriptsize $1$};
  \node[shift={(3.85,-3.15)}] at (1,+{sqrt(3)}) {\scriptsize \flat $3$};
  \node[shift={(3.85,-3.15)}] at (2,0)          {\scriptsize $5$};
  \node[shift={(3.85,-3.15)}] at (4,0)          {\scriptsize $9$};
  
  
  %% Sixte ajoutée minor
  %  Euler lattice
  \draw[shift={(11,-4.5)},black!50] (0, 0) -- (6, 0);
  \draw[shift={(11,-4.5)},black!50] (1, {sqrt(3)}) -- (5, {sqrt(3)});
  \draw[shift={(11,-4.5)},black!50] (1,-{sqrt(3)}) -- (5,-{sqrt(3)});
  \draw[shift={(11,-4.5)},black!50] (0,0) -- (1, {sqrt(3)}) -- (2,0) -- (3, {sqrt(3)}) -- (4,0) -- (5, {sqrt(3)}) -- (6,0);
  \draw[shift={(11,-4.5)},black!50] (0,0) -- (1,-{sqrt(3)}) -- (2,0) -- (3,-{sqrt(3)}) -- (4,0) -- (5,-{sqrt(3)}) -- (6,0);
  %  Just third and fifth relations
  \draw[shift={(11,-4.5)},line width=1mm]                (2,0) -- (0,0);
  \draw[shift={(11,-4.5)},line width=1mm]                (2,0) -- (1,+{sqrt(3)}) -- (0,0);
  %  8:9 whole tone relations
  \draw[shift={(11,-4.5)},line width=1mm,densely dashed] (2,0) -- (6,0);
  %  Circles for chord notes labels
  \draw[shift={(11,-4.5)},fill=white] (6,0)          circle (.3);
  \draw[shift={(11,-4.5)},fill=white] (0,0)          circle (.3);
  \draw[shift={(11,-4.5)},fill=white] (1,+{sqrt(3)}) circle (.3);
  \draw[shift={(11,-4.5)},fill=white] (2,0)          circle (.3);
  %  Chord notes labels
  \node[shift={(7.7,-3.15)}] at (0,0)          {\scriptsize $1$};
  \node[shift={(7.7,-3.15)}] at (1,+{sqrt(3)}) {\scriptsize \flat $3$};
  \node[shift={(7.7,-3.15)}] at (2,0)          {\scriptsize $5$};
  \node[shift={(7.7,-3.15)}] at (6,0)          {\scriptsize $6$};
\end{tikzpicture}%

%%% Local Variables:
%%% mode: latex
%%% TeX-master: "../main"
%%% End:

  \caption{Quartvorhalt, erweiterter Nonakkord und Sixte ajoutée in Dur (oben)
    und Moll (unten), visualisiert als Unterkomplexe des Eulerschen
    Tonnetzes.}\label{fig:chordLines}
\end{figure}

\subsection{Dominantseptakkord}
\label{sec:dom7syn}

Sehr oft kommt der Dominantakkord zusammen mit der kleinen Septime über dem
Grundton des Akkords zusammen vor (z.\,B. g\,:\,h\,:\,d\,:\,f in C-Dur wie in
c-Moll), vgl. \cite[{}9.1]{Skript}.  Bei der Frage, wie die kleine Septime zu
intonieren ist, fällt uns zuerst der Vierklang aus \cref{sec:int} ein, der dem
Verhältnis $4:5:6:7$ entsprach. Wir stellen schnell fest, dass sich dieser
Vierklang nicht im Eulerschen Tonnetz wiederfindet, was schlichtweg daran liegt,
dass dieses nur Primzahlen bis $5$ kennt, vgl. \cref{sec:primes}.  Wir werden in
\cref{sec:sept} über die Möglichkeit sprechen, diesen Vierklang dennoch zu
verwenden, bemerken dabei aber, dass er nur in ganz wenigen Situationen im
Kontext klassischer Musik sinnvoll ist.

Stattdessen ist eine im Eulerschen Tonnetz beheimatete Intonationsmöglichkeit
leichter in einen musikalischen Gesamtkontext zu integrieren: In der Nähe des
Dominantakkordes (g\,:\,\naturalm h\,:\,d) gibt es im Eulerschen Tonnetz zwar
zwei unterschiedliche kleine Septimen (\natural f und \naturalp f), die in
Betracht kommen.  Allerdings hat die auf der Kommahöhe des Grundtons (\natural
f) entscheidende Vorteile:
\begin{itemize}
\item Sie kommt auch in anderen Dreiklängen der Tonart (z.\,B. als Grundton der
  Subdominante!) vor.  Die andere könnte man (mit viel Wohlwollen) vielleicht
  noch als Terz der Doppeldominantvariante verwenden.
\item Nicht selten bleibt die Septime der Dominante beim Harmoniewechsel
  D\textsuperscript{$7$}\,–\,T\textsuperscript{$4$–$3$} liegen und wird zur
  Quarte der Tonika.  Wird hierbei die empfohlene Intonation der Septime
  verwendet, so ist hierbei keine Anpassung nötig.
\item Gleiches gilt, wenn beim Harmoniewechsel S\,–\,D\textsuperscript{$7$} der
  Grundton der Subdominante liegen bleibt und zur Septime der Dominante wird.
\end{itemize}
Der so entstehende Vierklang hat das Frequenzverhältnis $36:45:54:64$. Die
verminderte Quinte (Tritonus) zwischen der großen Terz und der kleinen Septime
ist $45:64\approx 609{,}78\,\on{ct}$ groß, also genauso groß wie der im Sixte
ajoutée in Moll. Die Auflösung der Septime der Dominante in die große bzw.
kleine Terz der Tonika erfolgt wie beim Quartvorhalt (siehe \cref{sec:49}) mit
einem diatonischen Halbton $15:16\approx 111{,}73\,\on{ct}$ bzw. einem kleinen
Ganzton $9:10\approx 182{,}40\,\om{ct}$.\todo{Beide Begriffe, „diatonischer
  Halbton“ und „kleiner Ganzton“ wurden noch nicht eingeführt. Wo tut man das am
  besten? Man könnte sogar erwägen, §4 und §5 zu tauschen, denn im Abschnitt zur
  Melodik würde sich sowas sehr gut anbieten. FK} In
\cref{fig:chordLinessevenths} findet sich eine Verortung dieses Akkords im
Tonnetz.

\subsection{Verminderter Septakkord}
\label{sec:dim7syn}

% Auch für den Fall des verminderten Septakkordes werden wir in \cref{sec:sept}
% eine Alternative entwickeln, die wir aber nicht immer einsetzen wollen.

Der verminderte Septakkord, kurz D$^v$, besteht aus der großen Terz, reinen
Quinte, kleinen Septime und kleinen None des Dominantakkordes\footnote{Wie jede
  Variation der Dominante kann dies auch bei einer Zwischendominante auftreten.}
(z.\,B. h\,:\,d\,:\,f\,:\,\flat a in C-Dur wie in c-Moll),
vgl. \cite[{}11.1]{Skript}. Es handelt sich also im Wesentlichen um einen
Dominantseptakkord, bei dem der Grundton durch eine kleine None ausgetauscht
wird. Es ergibt also Sinn, Terz, Quinte und Septime so zu intonieren wie im
vorherigen Abschnitt beschrieben. Für die None kommt in der harmonischen Nähe
nur ein Kandidat in Frage, nämlich \flatp $9$.  Wird diese None beim
Akkordwechsel D$^v$\,–\,t in die Quinte der Tonika geführt, so hat dieser
Schritt wieder die Größe $15:16$.

Insgesamt erhalten wir also in obigem Beispiel \naturalm
h\,:\,d\,:\,f\,:\,\flatp a, was dem Verhältnis $225:270:320:384$ entspricht,
siehe auch \cref{fig:chordLinessevenths}, inklusive aller Auflösungsrichtungen.
Dieses Verhältnis mag auf den ersten Blick wenig überzeugen, spiegelt aber viele
Besonderheiten dieses Vierklangs wider:
\begin{itemize}
\item Der verminderte Septakkord enthält zwei Tritoni und ist vom Wesen her eine
  Dominante mit zusätzlichen Strebetönen (hin zu Terz und Quinte der Tonika); es
  ist also gar nicht zu erwarten, dass er stabil klingt.  (In Bachs Musiksprache
  tritt er an besonders schmerzvollen und dramatischen Stellen auf.)
\item Die None liegt eine reine kleine Terz über der Septime (so wie auch die
  Terz eine reine kleine Terz unter der Quinte ist) und bildet zur Quinte den
  gleichen Tritonus, den auch die Septime zur Terz bildet.  Der Akkord ist also
  in symmetrischen Terzen geschichtet: Terz zu Quinte $5:6$, Quinte zu Septime
  $27:32$ und Septime zu None $5:6$.
\item In \cite[S.\,92ff.]{deLaMotte} wird der verminderte Septakkord beschrieben
  als Verschmelzung zweier (unvollständiger) Funktionen: Er enthält den Grundton
  und die kleine Terz der Mollsubdominante (hier: f und \flat a) sowie die große
  Terz und die Quinte der Dominante (hier: h und d).  Unsere Verortung dieser
  Töne im Eulerschen Tonnetz ist im Einklang mit dieser Beschreibung.
\end{itemize}

\subsection{Septnonakkord}

Der Dominantseptnonakkord in Dur ist ein Vierklang, der entsteht, indem dem
Dominantseptakkord die (skaleneigene) große None hinzugefügt und der Grundton
ausgelassen wird (in \flat E-Dur: d\,:\,f\,:\,\flat a\,:\,c),
vgl. \cite[{}11.3]{Skript}.  Da alle Bestandteile bis auf die None bereits in
\cref{sec:dom7syn} besprochen wurden, bleibt nur die Stimmung der None zu
klären.  Die None ist entweder als große Terz über der Septime oder als Quinte
über der Quinte (und damit pythagoreische None über dem Grundton)
interpretierbar.  Zweiteres ist aus folgenden Gründen sinnvoller:
\begin{itemize}
\item Diese None ist verträglich mit der aus \cref{sec:49}.
  \item Die Septime ist in dem Dominantakkord ein spannungsreicher Ton.  Ein
  reines Intervall zu ihr erzeugt tendenziell weniger Konsonanz als reine
  Quinten über dem Grundton.
\item Die Auflösung der None erfolgt mit einem Ganzton abwärts in den Grundton
  der Tonika; hierbei ist der große Ganzton ($8:9$) dem kleinen Ganzton ($9:10$)
  melodisch vorzuziehen, siehe \cref{sec:melody}.
\end{itemize}
Der Dominantseptnonakkord entspricht also dem Verhältnis $45:54:64:81$ (in \flat
E-Dur: \naturalm d\,:\,f\,:\,\flat a\,:\,c) und lässt sich im Eulerschen Tonnetz
wie in \cref{fig:chordLinessevenths} zeichnen (wir haben den Grundton der
Übersichtlichkeit halber auch eingezeichnet).

Wir merken an, dass unter Verwendung der Primzahl $7$ auch die Ausstimmung
$5:6:7:9$ in Betracht kommt, die vor allem deswegen witzig ist, weil Septime und
None den Faktoren $7$ und $9$ entsprechen.  Wie schon beim Dominantseptakkord
ist dieses Verhältnis aber eher von akademischem Interesse und im Kontext
klassischer Musik meistens unbrauchbar. Wir klammern diese Möglichkeit hier also
aus und verweisen wieder auf \cref{sec:sept}.

Abschließend weisen wir noch auf eine Besonderheit des Dominantseptnonakkords
hin: Er setzt sich aus den gleichen Tönen zusammen wie der Sixte ajoutée der
Mollparallele (in \flat E-Dur: f\,:\,\flat a\,:\,c\,:\,d),
vgl. \cref{sec:sixte}. Weil diese Noten sogar enharmonisch identisch sind,
werden sie pythagoreisch gleich intoniert. Unsere Kommaverteilung liefert
allerdings verschiedene Ausstimmungen, wie \cref{tab:79-6} zeigt.  Wie wir
sehen, teilen sie sich den Tritonus \naturalm d\,:\,\flat a, doch die Quinten
f\,:\,c und \naturalm f\,:\,\naturalm c unterscheiden sich um ein Komma.

\begin{table}[h]
  \centering
  \begin{tabular}{lrrrr}
    \toprule
    Dominantseptnonakkord (Terz oben): & f & \flat a & c & \flat d\\
    Sixte ajoutée der Parallele (Sexte oben): & \naturalm f & \flat a & \naturalm c & \naturalm d\\
    \bottomrule
  \end{tabular}
  \caption{Die Bestandteile zweier sehr ähnlicher Akkorde in \flat E-Dur}\label{tab:79-6}
\end{table}

\subsection{Verminderter Septakkord mit tiefalterierter Quinte}

Der letzte Akkord, den wir in diesem Abschnitt behandeln, ist eine spezielle
Abwandlung des verminderten Septakkordes (\cref{sec:dim7syn}), die (fast)
ausschließlich als Variation der Doppeldominante auftritt: Bei ihr ist
zusätzlich die Quinte tiefalteriert (in C-Dur oder c-Moll: \sharp f\,:\,\flat
a\,:\,c\,:\,\flat e), vgl. \cite[{}11.2]{Skript}.  Alle Töne haben eine
Spannung, die sie mit einem Halbtonschritt zur Dominante auflösen: die Terz
(\sharp f) nach oben und alle anderen Töne nach unten.

Wenn wir die Intonation der Terz, Septime und None von \cref{sec:dim7syn}
übernehmen, so gibt es eine unausweichliche Intonation für die tiefalterierte
Quinte: eine syntonische Ebene höher als das tonale Zentrum, denn dann ist der
Durdreiklang aus Quinte, Septime und None rein und alle Töne lösen sich mit
einem diatonischen Halbtonschritt $15:16$ auf. In C-Dur (oder c-Moll) ist der
Akkord also \sharpm f\,:\,\flatp a\,:\,c\,:\,\flatp e, was dem Verhältnis
$225:256:320:384$ entspricht. Die verminderte Terz zwischen großer Terz und
verminderter Quinte ist $225:256 \approx 223{,}46\,\on{ct}$ groß.  Auch diesen
Akkord mitsamt seinen Auflösungsrichtungen haben wir in
\cref{fig:chordLinessevenths} ins Tonnetz eingezeichnet.

Wir bemerken abschließend, dass dieser Vierklang enharmonisch umgedeutet werden
kann zu einem Dominantseptkkord in einer völlig anderen Tonart (im Beispiel:
zum Dominantseptakkord in \flatp D-Dur). Die darin enthaltene Septime (\flatp g)
ist im Vergleich zur Terz der Doppeldominante (\sharpm f) ein pythagoreisches
Komma tiefer und zwei syntonische Kommata höher, insgesamt also etwa
$19{,}55\,\om{ct}$ höher.

\begin{figure}
  \centering
  \begin{tikzpicture}[scale=.7]
  %% Dominant Seventh in major
  %  Euler lattice
  \draw[black!50] (0, 0)         -- (6, 0);
  \draw[black!50] (1, {sqrt(3)}) -- (5, {sqrt(3)});
  \draw[black!50] (1,-{sqrt(3)}) -- (5,-{sqrt(3)});
  \draw[black!50] (0,0)          -- (1, {sqrt(3)}) -- (2,0) -- (3, {sqrt(3)}) -- (4,0) -- (5, {sqrt(3)}) -- (6,0);
  \draw[black!50] (0,0)          -- (1,-{sqrt(3)}) -- (2,0) -- (3,-{sqrt(3)}) -- (4,0) -- (5,-{sqrt(3)}) -- (6,0);
  %  Just third and fifth relations
  \draw[black,line width=1mm]                (4,0) -- (6,0) -- (5,-{sqrt(3)}) -- (4,0);
  %  8:9 whole tone relations
  \draw[black,line width=1mm,densely dashed] (0,0) -- (4,0);
  %  Circles for chord notes labels
  \draw[fill=white] (0,0)          circle (.3);
  \draw[fill=white] (4,0)          circle (.3);
  \draw[fill=white] (6,0)          circle (.3);
  \draw[fill=white] (5,-{sqrt(3)}) circle (.3);
  %  Chord notes labels
  \node at (0,0)          {\scriptsize $7$};
  \node at (4,0)          {\scriptsize $1$};
  \node at (6,0)          {\scriptsize $5$};
  \node at (5,-{sqrt(3)}) {\scriptsize $3$};
  %  Semitone resolutions
  \draw[white,line width=1mm] (.3,{-sqrt(3)*1/10})  -- (2.7,{-sqrt(3)*9/10});
  \draw[white,line width=1mm] (4.7,{-sqrt(3)*9/10}) -- (2.3,{-sqrt(3)*1/10});
  \draw[-stealth]             (.3,{-sqrt(3)*1/10})  -- (2.7,{-sqrt(3)*9/10});
  \draw[-stealth]             (4.7,{-sqrt(3)*9/10}) -- (2.3,{-sqrt(3)*1/10});
  %  Label of chord
  \node at (-1.5,0)       {\small $\strut$Dur};
  \node at (3,2.5)        {\small $\strut$Septakkord};
  
  
  %% Diminished seventh in major
  %  Euler lattice
  \draw[shift={(7.5,0)},black!50] (0, 0)         -- (6, 0);
  \draw[shift={(7.5,0)},black!50] (1, {sqrt(3)}) -- (5, {sqrt(3)});
  \draw[shift={(7.5,0)},black!50] (1,-{sqrt(3)}) -- (5,-{sqrt(3)});
  \draw[shift={(7.5,0)},black!50] (0,0)          -- (1, {sqrt(3)}) -- (2,0) -- (3, {sqrt(3)}) -- (4,0) -- (5, {sqrt(3)}) -- (6,0);
  \draw[shift={(7.5,0)},black!50] (0,0)          -- (1,-{sqrt(3)}) -- (2,0) -- (3,-{sqrt(3)}) -- (4,0) -- (5,-{sqrt(3)}) -- (6,0);
  %  Just third and fifth relations
  \draw[shift={(7.5,0)},black,line width=1mm] (6,0) -- (5,-{sqrt(3)});
  \draw[shift={(7.5,0)},black,line width=1mm] (0,0) -- (1,+{sqrt(3)});
  %  Circles for chord notes labels
  \draw[shift={(7.5,0)},fill=white] (0,0)          circle (.3);
  \draw[shift={(7.5,0)},fill=white] (1,{sqrt(3)})  circle (.3);
  \draw[shift={(7.5,0)},fill=white] (6,0)          circle (.3);
  \draw[shift={(7.5,0)},fill=white] (5,-{sqrt(3)}) circle (.3);
  %  Chord notes labels
  \node at (7.5,0)           {\scriptsize $7$};
  \node at (8.5,{sqrt(3)})   {\scriptsize \flat $9$};
  \node at (13.5,0)          {\scriptsize $5$};
  \node at (12.5,-{sqrt(3)}) {\scriptsize $3$};
  %  Semitone resolutions
  \draw[shift={(7.5,0)},white,line width=1mm] (.3,{-sqrt(3)*1/10})  -- (2.7,{-sqrt(3)*9/10});
  \draw[shift={(7.5,0)},white,line width=1mm] (4.7,{-sqrt(3)*9/10}) -- (2.3,{-sqrt(3)*1/10});
  \draw[shift={(7.5,0)},white,line width=1mm] (1.3,{+sqrt(3)*9/10}) -- (3.7,{+sqrt(3)*1/10});
  \draw[shift={(7.5,0)},-stealth]             (.3,{-sqrt(3)*1/10})  -- (2.7,{-sqrt(3)*9/10});
  \draw[shift={(7.5,0)},-stealth]             (4.7,{-sqrt(3)*9/10}) -- (2.3,{-sqrt(3)*1/10});
  \draw[shift={(7.5,0)},-stealth]             (1.3,{+sqrt(3)*9/10}) -- (3.7,{+sqrt(3)*1/10});
  %  Label of chord
  \node at (10.5,2.5)        {\small $\strut$Verminderter Septakkord};
  
  
  %% Dominant seventh in minor
  %  Euler lattice
  \draw[shift={(0,-4.5)},black!50] (0, 0)         -- (6, 0);
  \draw[shift={(0,-4.5)},black!50] (1, {sqrt(3)}) -- (5, {sqrt(3)});
  \draw[shift={(0,-4.5)},black!50] (1,-{sqrt(3)}) -- (5,-{sqrt(3)});
  \draw[shift={(0,-4.5)},black!50] (0,0)          -- (1, {sqrt(3)}) -- (2,0) -- (3, {sqrt(3)}) -- (4,0) -- (5, {sqrt(3)}) -- (6,0);
  \draw[shift={(0,-4.5)},black!50] (0,0)          -- (1,-{sqrt(3)}) -- (2,0) -- (3,-{sqrt(3)}) -- (4,0) -- (5,-{sqrt(3)}) -- (6,0);
  %  Just third and fifth relations
  \draw[shift={(0,-4.5)},black,line width=1mm]                (4,0) -- (6,0) -- (5,-{sqrt(3)}) -- (4,0);
  %  8:9 whole tone relations
  \draw[shift={(0,-4.5)},black,line width=1mm,densely dashed] (0,0) -- (4,0);
  %  Circles for chord notes labels
  \draw[shift={(0,-4.5)},fill=white] (0,0)          circle (.3);
  \draw[shift={(0,-4.5)},fill=white] (4,0)          circle (.3);
  \draw[shift={(0,-4.5)},fill=white] (6,0)          circle (.3);
  \draw[shift={(0,-4.5)},fill=white] (5,-{sqrt(3)}) circle (.3);
  %  Chord notes labels
  \node at (0,-4.5)           {\scriptsize $7$};
  \node at (4,-4.5)           {\scriptsize $1$};
  \node at (6,-4.5)           {\scriptsize $5$};
  \node at (5,-{sqrt(3)}-4.5) {\scriptsize $3$};
  %  Semitone resolutions
  \draw[shift={(0,-4.5)},white,line width=1mm] (.3,{+sqrt(3)*1/10})  -- (2.7,{+sqrt(3)*9/10});
  \draw[shift={(0,-4.5)},white,line width=1mm] (4.7,{-sqrt(3)*9/10}) -- (2.3,{-sqrt(3)*1/10});
  \draw[shift={(0,-4.5)},-stealth]             (.3,{+sqrt(3)*1/10})  -- (2.7,{+sqrt(3)*9/10});
  \draw[shift={(0,-4.5)},-stealth]             (4.7,{-sqrt(3)*9/10}) -- (2.3,{-sqrt(3)*1/10});
  %  Label of chord
  \node at (-1.5,-4.5)        {\small $\strut$Moll};
  
  
  %% Diminished seventh in minor
  %  Euler lattice
  \draw[shift={(7.5,-4.5)},black!50] (0, 0)         -- (6, 0);
  \draw[shift={(7.5,-4.5)},black!50] (1, {sqrt(3)}) -- (5, {sqrt(3)});
  \draw[shift={(7.5,-4.5)},black!50] (1,-{sqrt(3)}) -- (5,-{sqrt(3)});
  \draw[shift={(7.5,-4.5)},black!50] (0,0)          -- (1, {sqrt(3)}) -- (2,0) -- (3, {sqrt(3)}) -- (4,0) -- (5, {sqrt(3)}) -- (6,0);
  \draw[shift={(7.5,-4.5)},black!50] (0,0)          -- (1,-{sqrt(3)}) -- (2,0) -- (3,-{sqrt(3)}) -- (4,0) -- (5,-{sqrt(3)}) -- (6,0);
  %  Just third and fifth relations
  \draw[shift={(7.5,-4.5)},black,line width=1mm] (6,0) -- (5,-{sqrt(3)});
  \draw[shift={(7.5,-4.5)},black,line width=1mm] (0,0) -- (1,+{sqrt(3)});
  %  Circles for chord notes labels
  \draw[shift={(7.5,-4.5)},fill=white] (0,0)          circle (.3);
  \draw[shift={(7.5,-4.5)},fill=white] (6,0)          circle (.3);
  \draw[shift={(7.5,-4.5)},fill=white] (1,{sqrt(3)})  circle (.3);
  \draw[shift={(7.5,-4.5)},fill=white] (5,-{sqrt(3)}) circle (.3);
  %  Chord notes labels
  \node at (7.5,-4.5)            {\scriptsize $7$};
  \node at (8.5,{sqrt(3)-4.5})   {\scriptsize \flat $9$};
  \node at (13.5,-4.5)           {\scriptsize $5$};
  \node at (12.5,-{sqrt(3)-4.5}) {\scriptsize $3$};
  %  Semitone resolutions
  \draw[shift={(7.5,-4.5)},white,line width=1mm] (.3,{+sqrt(3)*1/10})  -- (2.7,{+sqrt(3)*9/10});
  \draw[shift={(7.5,-4.5)},white,line width=1mm] (4.7,{-sqrt(3)*9/10}) -- (2.3,{-sqrt(3)*1/10});
  \draw[shift={(7.5,-4.5)},white,line width=1mm] (1.3,{+sqrt(3)*9/10}) -- (3.7,{+sqrt(3)*1/10});
  \draw[shift={(7.5,-4.5)},-stealth]             (.3,{+sqrt(3)*1/10})  -- (2.7,{+sqrt(3)*9/10});
  \draw[shift={(7.5,-4.5)},-stealth]             (4.7,{-sqrt(3)*9/10}) -- (2.3,{-sqrt(3)*1/10});
  \draw[shift={(7.5,-4.5)},-stealth]             (1.3,{+sqrt(3)*9/10}) -- (3.7,{+sqrt(3)*1/10});
\end{tikzpicture}

%%% Local Variables:
%%% mode: latex
%%% TeX-master: "../main"
%%% End:

  \caption{Verschiedene Septakkorde im Eulerschen Tonnetz. Die Pfeile zeigen die
    Auflösungsrichtung einzelner Töne an.}\label{fig:chordLinessevenths}
\end{figure}

%%% Local Variables:
%%% mode: latex
%%% TeX-master: "../main"
%%% End:


\section{Komplexere Akkorde}
\label{sec:quad}
\subsection{Schritte als horizontale Intervalle}

\subsection{Durchgangs- und Wechselnoten}

Durchgänge, Wechselnoten

\subsection{Leittöne}
\label{sec:ln}

Wir weisen darauf hin, dass die Terz im Dreiklang, die zur Dominante (oder einer
Zwischendominante) gehört, ebenfalls um ein syntonisches Komma abgesenkt wird,
was der gängigen Praxis widerspricht, Leittöne etwas höher zu intonieren, um die
Spannung zu erhöhen, siehe z.\,B. […]\todo{Literatur FK}. Wir sind der
Überzeugung, dass die Dominante ein eigenständiger, stabiler Dur-Dreiklang sein
sollte, aber in dieser Frage kann man anderer Meinung sein als wir. In diesem
Fall schlagen wir vor, den Leitton um ein syntonisches Komma zu erhöhen, wodurch
der Dreiklang die pythagoreische Stimmung $64:81:96$ erhält. (Im Hörbeispiel
hören wir zwei Versionen einer Kadenz, eine mit einem rein intonierten Leitton,
eine mit einem pythagoreischen Leitton). Wir halten uns jedoch an die
Konvention, alle Dreiklänge, auch solche, die eine Dominantenfunktion haben, so
zu intonieren, dass sie eine reine Terz haben.

\begin{figure}
  \centering
  \LY{b-lead}
  \includegraphics{ly/b-lead}
  \caption{Zwei Arten, eine Vollkadenz zu intonieren: Einmal mit reiner
  	Intonation aller Dur-Dreiklänge und einmal mit einem pythagoreischen
  	Leitton.}\label{fig:lead}
\end{figure}

%%% Local Variables:
%%% mode: latex
%%% TeX-master: "../main"
%%% End:


\section{Naturseptimen}
\label{sec:sept}
Wir haben im letzten Abschnitt gesehen, wie wir innerhalb des Eulerschen
Tonnetzes diverse Akkorde rein stimmen können.  Allerdings haben wir immer noch
keine Möglichkeit gesehen, die bereits mehrfach erwähnte kleine Septime der
Größe $4:7\approx 968{,}83\,\on{ct}$ darzustellen (was, wie gesagt, daran liegt,
dass das Eulersche Tonnetz nur Primzahlen bis $5$ kennt,
vgl. \cref{sec:primes}). Dieses Intervall nennen wir \emph{Naturseptime}, da es
in der „Naturtonfolge“\footnote{Die Naturtonfolge einer gegebenen Grundfrequenz
  ist gegeben durch ihre positiven ganzzahligen Vielfachen. Das Intervall $4:7$
  ist also das zwischen dem $4$. und $7$. Naturton.} als erste Septime vorkommt.
Sie beträgt $968{,}83\,\on{ct}$, ist also fast $\frac13\,\text{Hts}$ kleiner als
die gleichstufige kleine Septime und klingt deshalb auch deutlich anders, wie in
dem bereits in \cref{sec:int} erwähnten Audiobeispiel nachzuhören ist\LY{m-4567}
(zuerst erklingt ein gleichstufiger Dominantseptakkord, dann einer mit dem
Frequenzverhältnis $4:5:6:7$).

\subsection{Das septimale Komma}

Um die Naturseptime in die bestehende Helmholtz-Ellis-Notation zu integrieren,
müssen wir ihre Abweichung zur pythagoreischen kleinen Septime betrachten.  Hier
sehen wir, dass die pythagoreische kleine Septime
$9:16=36:64\approx 996{,}09\,\on{ct}$ um $\frac{64}{63}\approx 27{,}26\,\on{ct}$
größer ist als die Naturseptime. Diesen Unterschied bezeichnen wir als
\emph{septimales Komma}.

Die Tiefalteration einer gegebenen Note um ein septimales Komma notieren wir mit
dem hier neu eingeführten Symbol \septimal.  Dieses wird wie ein
Versetzungszeichen gebraucht. Hat die zu modifizierende Note bereits ein
Versetzungszeichen, so wird \septimal\ links von diesem hinzugefügt. So ist zum
Beispiel \septimal\naturalp f ein f, das um ein syntonisches Komma erhöht und um
ein septimales Komma gesenkt wurde.

Da wir keinen Ton um ein septimales Komma \emph{erhöhen} werden, müssen wir
dafür auch kein Zeichen einführen. (Die naheliegende Wahl wäre, das Symbol einer
horizontalen Achse zu spiegeln; so wird das auch in Situationen, wo dies
erforderlich ist, getan.)

\begin{figure}
  \centering
  \LY{b-sept}
  \includegraphics{ly/b-sept}
  \caption{(1)~Eine Schlusskadenz mit Dominantseptakkord. Die Septime im
    Tenor wird pythagoreisch intoniert.
    \quad(2)~Die gleiche Kadenz mit Naturseptime im Tenor.
    \quad(3)~Ein Beispiel mit Septnonakkord ohne Naturseptime.
    \quad(4)~Das gleiche Beispiel mit Naturseptime.}\label{fig:sept}
\end{figure}

\subsection{Septakkorde}

Bei zwei in \cref{sec:quad} vorgestellten Septakkorden kann es unter Umständen
harmonisch sinnvoll sein, sie unter Verwendung der Naturseptime auszustimmen.
Dabei müssen wir gar nicht so viel ändern:
% Wenn wir in den Septakkorden aus \cref{sec:quad} nun die pythagoreische durch
% eine Naturseptime ersetzen wollen, müssen wir oft nicht viel ändern:

\begin{enumerate}
\item Der Dominantseptakkord (\cref{sec:dom7syn}) bedarf nach dem Absenken der
  pythagoreischen Septime um ein septimales Komma keiner Anpassung mehr, denn er
  hat schon das Verhältnis $4:5:6:7$. In C-Dur hat dieser Akkord dann die Töne
  g\,:\,\naturalm h\,:\,d\,:\,\septimal f.  Die verminderte Quinte \naturalm
  h\,:\,\septimal f entspricht $5:7 \approx 582{,}51\ct$.  Diese setzt sich aus
  zwei kleinen Terzen unterschiedlicher Größe zusammen, nämlich
  $5:6 \approx 315{,}64\ct$ (\naturalm h\,:\,d) und $6:7\approx 266{,}87\ct$
  (d\,:\,\septimal f).  Dieser Akkord ist berüchtigt für seine Omnipräsenz im
  Barbershopgesang (wobei dieses Gerücht einer empirischen Untersuchung
  \cite{Barbershop} nur bedingt standhalten kann).  \looseness-1
  % Wird die Septime abwärts in
  % die Dur- oder Mollterz der Tonika aufgelöst, so entsteht als horizontales
  % Intervall $20:21\approx 84{,}47\ct$ im Falle von Dur bzw.
  % $32:35\approx 155{,}20\ct$ im Falle von Moll,
  % vgl. \cref{sec:septmel}.\looseness-1
% \item\todo{Evtl. weglassen. Zu kompliziert, taucht zu selten auf, und wir lassen
%     ja schon den Herzchenakkord weg.} Im verminderten Septakkord
%   (\cref{sec:dim7syn}) haben wir die kleine None als reine kleine Terz über der
%   Septime definiert. Es liegt also nahe, die None gemeinsam mit der Septime um
%   ein septimales Komma abzusenken. Der Akkord hat das Verhältnis $25:30:35:42$
%   und besteht in C-Dur aus den Tönen \naturalm h\,:\,d\,:\,\septimal
%   f\,:\,\septimal\flatp a. Wie in \cref{sec:dim7syn} erwähnt wurde handelt es
%   sich (laut \cite[S.\,92ff.]{deLaMotte}) bei diesem Akkord in Moll um eine
%   Verschmelzung von s und D. Mit der septimalen Alterierung der Septime und None
%   verlieren wir diese Eigenschaft. In Moll ist dies daher als weiterer Faktor
%   gegen die Naturseptime in der Abwägung für und gegen ihren Einsatz anzuführen.
\item Im Dominantseptnonakkord (\cref{sec:dom79syn}) muss ebenfalls lediglich
  die Septime gesenkt werden; dann entsteht das Verhältnis $5:6:7:9$. In C-Dur
  entspricht dieser Akkord dann \naturalm h\,:\,d\,:\,\septimal f\,:\,a.  Hier
  taucht nun eine sehr große große Terz auf, nämlich $7:9\approx 435{,}08\ct$
  (\septimal f\,:\,a).
\end{enumerate}
Wir halten fest, dass die Absenkung der Septime in diesen beiden Akkorden
deutlich hörbar ist, weil die Naturseptime wirklich \emph{viel} tiefer als die
pythagoreische und erst recht als die mittlerweile wohl sehr gewohnte
gleichstufige ist.

\subsection{Septimale Schritte}
\label{sec:septSteps}

Durch den Einsatz des septimalen Kommas können natürlich auch neue horizontale
Intervalle und vor allem interessante bis unbrauchbare Schritte entstehen:

\begin{enumerate}
\item \emph{Septimaler (diatonischer) Halbton}\\
  Wird die Naturseptime im Dominantseptakkord abwärts in die Durterz der Tonika
  aufgelöst (in C-Dur: \septimal f\,–\,\naturalm e), so entsteht ein
  diatonischer Halbton der Größe $20:21\approx 84{,}47\ct$.  Dies ist der
  kleinste diatonische Halbton, den wir kennen; allerdings ist er größer als der
  kleine chromatische Halbton.
\item \emph{Kleiner septimaler Ganzton}\\
  Wird die Naturseptime im Dominantseptakkord abwärts in die Mollterz der Tonika
  aufgelöst (in c-Moll: \septimal f\,–\,\flatp e), so entsteht ein Ganzton der
  Größe $32:35\approx 155{,}14\ct$.  Dieser ist extrem klein; er liegt fast in der
  \emph{Mitte} zwischen einem gleichstufigen Halbton und einem gleichstufigen
  Ganzton.\looseness-1
\item \emph{Großer septimaler Ganzton}\\
  Wird die Naturseptime von oben (z.\,B. vom Grundton der Dominante) aus
  erreicht (in C-Dur: g\,-\,\septimal f), so entsteht ein Ganzton der Größe
  $7:8\approx 231{,}17\ct$. Dies ist der größte Ganzton, den wir kennen, und für
  die klassische Musik als Schritt eigentlich unbrauchbar,
  siehe \cref{sec:septmel}.
\end{enumerate}

\subsection{Brauchbarkeit}
\label{sec:septmel}

% Da die Naturseptime \emph{sehr} tief ist, können bei ihrem Einsatz melodische
% Probleme auftreten (siehe \cref{sec:melody}).
% Wichtige horizontale Intervalle die vorkommen können sind diese:
% \begin{enumerate}
%   \item Wird die Naturseptime mit einem Halbton von unten erreicht so ist das
%   horizontale Intervall der \emph{septimale (diatonische) Halbton}.\todo{Das
%     \href{https://en.xen.wiki/w/Gallery_of_just_intervals}{XenWiki} führt dieses
%     Intervall als „minor semitone, large septimal chroma“. Ersteres ist nicht
%     wirklich spezifisch und zweiteres ist falsch, weil das Intervall ja
%     diatonisch und nicht chromatisch ist. Weil diese Namen nichts taugen benutze
%     ich diese für uns sinnvolle Bezeichung. TD} Er ist beispielsweise
%   \naturalm e\,–\,\septimal f, hat die Größe $20:21 = 84{,}47\ct$ und ist damit
%   der kleinste diatonische Halbton den wir kennen. Dies ist auch das Intervall
%   mit dem die Septime in Dur (nach unten) aufgelöst wird.
%   \item Wird die Naturseptime mit einem Ganzton von unten erreicht so ist das
%   horizontale Intervall der \emph{kleine septimale Ganzton}.\todo{Auch hier hat
%     das XenWiki eine andere Meinung: „neutral second“. Da aber dieses Intervall
%     für uns ein Ganzton ist und es (auch laut XenWiki) sonst nur den großen
%     $7:8$ septimalen Ganzton gibt erscheint mir „kleiner septimaler Ganzton“
%     sinnvoll. TD} Er ist beispielsweise \flatp e\,–\,\septimal f, hat die Größe
%   $32:35 = 155{,}14\ct$ und ist damit extrem klein für einen Ganzton. Dies ist
%   auch das Intervall mit dem die Septime in Moll (nach unten) aufgelöst wird.
%   \item Erreichen wir die Naturseptime von oben (z.\,B. vom Grundton der
%   Dominante) so erhalten wir den \emph{großen septimalen Ganzton}. Dieser hat
%   die Größe $7:8 = 231{,}17\ct$, ist damit der größte Ganzton den wir kennen und
%   ist z.\,B. \septimal f\,–\,g.
% \end{enumerate}

So schön die beiden oben erwähnten Septakkorde mit Naturseptime vertikal und
mathematisch sein mögen, so problematisch ist ihr Einsatz im Verlauf eines
Stücks. Dies hat im Wesentlichen drei Gründe:
\begin{enumerate}
\item Der Einsatz des septimalen Kommas verstärkt in melodisch gedachten
  Passagen das in \cref{sec:balance} besprochene Problem unterschiedlich großer
  Schritte.
\item Septimale Ganztöne sind melodisch sehr unpassend, da unser Gehör sie nicht
  einfach als das erkennen kann, was sie sein sollen. Der kleine septimale
  Ganzton ist so klein, dass er fast einen Viertelton tiefer als der
  gleichstufige Ganzton und damit weder als Halb- noch als Ganzton erkennbar
  ist. Der große septimale Ganzton ist $231{,}17\ct$ groß, was sehr nah an den
  $274{,}58\ct$ der reinen übermäßigen Sekunde (in a-Moll etwa \naturalp
  f\,–\,\sharpm g zwischen der Terz von d-Moll und der Terz von E-Dur)
  liegt. Zum Vergleich: Der jeweils kleinstmögliche Unterschied zwischen reiner
  Prime und Halbton, Halbton und Ganzton bzw. kleiner und großer Terz beträgt
  wenigstens $70\ct$\footnote{Die einzige Ausnahme zu dieser Regel ist die
    Differenz zwischen dem großen diatonischen Halbton ($133{,}24\ct$) und dem
    kleinen Ganzton ($182{,}40\ct$). Dies ist aber auch der Grund weshalb wir
    den großen diatonischen Halbton möglichst nicht verwenden wollen.}; hier
  sind es gerade einmal $43,41\ct$.  Daher sollten septimale Ganztöne nach
  Möglichkeit vermieden werden.\looseness-1
\item Wenn der häufige Fall auftritt, in der die Auflösung der Septime
  vorenthalten und ohne Anpassung zum Quartvorhalt wird, ist die entstehende
  Quarte $16:21 \approx 470{,}78\ct$ sehr unrein.  Es ist dann also
  unerlässlich, die Tonhöhe um ein septimales Komma nach \emph{oben}
  anzupassen. Wie schon in \cref{sec:49} argumentiert unterläuft dies aber die
  Idee einer konsonanten Einführung von Vohaltetönen und sollte daher vermieden
  werden.\looseness-1
\end{enumerate}
Im Gegensatz zu septimalen Ganztönen ist der septimale Halbton hingegen zwar
klein, aber nicht so klein, dass er nicht mehr als solcher erkennbar ist. Die
Auflösung der Naturseptime mit diesem Halbton in eine Durterz ist, weil der
Schritt kleiner ist als sonst, eventuell etwas spannungsreicher (vergleichbar
mit dem höheren Leitton in \cref{sec:ln}, nur von oben und ohne harmonische
Abstriche) und deshalb bei eindeutiger Kadenzbildung womöglich sogar willkommen.

Wenn wir Naturseptimen einsetzen wollen, sollten also folgende Bedingungen
erfüllt sein (priorisiert von oben nach unten):

\begin{enumerate}
\item Wir befinden uns in einer eindeutigen vertikal gedachten Kadenz.
\item Die Septime wird durch einen Halbton von unten oder durch einen Sprung
  erreicht. (Dies heißt gerade, dass durch den Einsatz der Naturseptime an
  dieser Stelle kein septimaler Ganzton entsteht.)
\item Die Septime wird in eine Durterz aufgelöst. (Das heißt gerade, dass durch
  den Einsatz der Naturseptime zum Zeitpunkt die Auflösung kein septimaler
  Ganzton entsteht.)
  % (in Moll z.\,B. am Ende in
  % eine \emph{picardische Terz}\todo{aka \acr{PSG} (nicht etwa Paris
  %   Saint-Germain, sondern Philip-Schwartz-Gedächtnisterz) TD}).
%\item Es handelt sich nicht um den verminderten Septakkord in Moll.
\item Die Septime wird nicht gehalten und zum Quartvorhalt.
\item Die Stimme mit der Septime ist nicht die Melodiestimme.
\end{enumerate}

Insgesamt stellen wir also fest, dass es in der klassischen Musik nur sehr
wenige Situationen gibt, in denen man den Einsatz der Naturseptime erwägen
könnte.  Es wird wohl nicht wenige Musiker:innen geben, die von ihrer Verwendung
im klassischen Kontext komplett abraten. 

%\begin{table}
%  \centering
%  \begin{tabular}{lrrrrr}
%    \toprule
%    \textsc{akkord} & \multicolumn{4}{l}{\textsc{frequenzverhältnis}}\\
%    \midrule
%    Dur                               & $4$  & $5$   & $6$                   \\
%    Moll                              & $10$ & $12$  & $15$                  \\
%    Quartvorhalt                      & $6$  & $8$   & $9$                   \\
%    Erweiterter Nonakkord (Dur)       & $4$  & $5$   & $6$    & $9$          \\
%    Erweiterter Nonakkord (Moll)      & $20$ & $24$  & $30$   & $45$         \\
%    Sixte ajoutée (Dur)               & $12$ & $15$  & $18$   & $20$         \\
%    Sixte ajoutée (Moll)              & $80$ & $96$  & $120$  & $135$        \\
%    Dominantseptakkord (7-limit)      & $4$  & $5$   & $6$    & $7$          \\
%    Dominantseptakkord (5-limit)      & $36$ & $45$  & $54$   & $64$         \\
%    Verminderter Septakkord (7-limit) &      & $25$  & $30$   & $35$  & $42$ \\
%    Verminderter Septakkord (5-limit) &      & $225$ & $270$  & $320$ & $384$\\
%    Septnonakkord (7-limit)           &      & $5$   & $6$    & $7$   & $9$  \\
%    Septnonakkord (5-limit)           &      & $45$  & $54$   & $64$  & $81$ \\
%    \bottomrule
%  \end{tabular}
%\end{table}

%%% Local Variables:
%%% mode: latex
%%% TeX-master: "../main"
%%% End:


\appendix
\section{Das kgV-Kriterium}
\label{sec:kgv}
\todo{Dieser Abschnitt wurde geschrieben, als er noch Teil von §3 war. Wir
  müssen das wohl noch anpassen. FK} Wir bemerken, dass es in pythagoreischer
Stimmung für jeden Ton (z.\,B. \flat a’) eindeutig bestimmte ganze Zahlen
$a,b\in \Z$ gibt, für die das Frequenzverhältnis des gegebenen Tones zum
Kammerton a’ genau $2^a\cdot 3^b$ ist. Umgekehrt gibt es auch für jede dieser
Zahlen $a$ und $b$ einen Ton, der dieses Verhältnis
realisiert.\footnote{Mathematisch gesprochen ist also die Abbildung
  $\Z^2\to \{\text{Töne in pythagoreischer Stimmung}\}$ mit
  $(a,b)\mapsto 2^a\cdot 3^b$ eine Bijektion. Dass die Quintenspirale selbst
  $1$-dimensional ist, liegt daran, dass wir aus ihr die Oktave „rausgerechnet“
  haben, also z.\,B. nicht zwischen \flat a\ und \flat a'\ unterscheiden.}  Um
anzudeuten, dass in pythagoreischer Stimmung nur die Primzahlen bis
einschließlich $3$ vorkommen, bezeichnet man sie auch als \emph{durch $3$
  begrenzt} (engl. „$3$-limit“).  Diese Eigenschaft impliziert, dass
Verhältnisse $k_1:\dotsb:k_r$ in pythagoreischer Stimmung stets der Form
$2^{a_1}3^{b_1}:\dotsb:2^{a_r}3^{b_r}$ mit $a_i,b_i\in\Z_{\ge 0}$ sind. So ist
zum Beispiel $4:5:6$ in pythagoreischer Stimmung nicht realisierbar, da hier die
Primzahl $5$ auftaucht.

Dieser Logik folgend sind die Töne, die unter Zuhilfenahme des syntonischen
Kommas (also mit Helmholtz-Ellis-Versetzungszeichen) beschrieben werden können,
\emph{durch $5$ begrenzt}: Hier gibt es für jeden Ton (z.\,B. \flatp a')
eindeutig bestimmte ganze Zahlen $a,b,c\in\Z$, für die das Frequenzverhältnis
des gegebenen Tones zum Kammerton a’ genau $2^a\cdot 3^b\cdot 5^c$ ist, und
umgekehrt gibt es auch für jede dieser Zahlen $a$, $b$ und $c$ einen Ton, der
dieses Verhältnis realisiert.\footnote{Mathematisch gesprochen ist also die
  Abbildung $\Z^3\to \{\text{Töne mit syntotischem Komma}\}$ mit
  $(a,b,c)\mapsto 2^a\cdot 3^b\cdot 5^c$ eine Bijektion. Dass das Tonnetz selbst
  $2$-dimensional ist, liegt wieder daran, dass wir aus ihm die Oktave
  „rausgerechnet“ haben, also z.\,B. nicht zwischen \flatp a und \flatp a’
  unterscheiden.} Wieder impliziert dies, dass in Frequenzverhältnissen nur
Zahlen der Form $2^a\cdot 3^b\cdot 5^c$ auftreten können, z.\,B. ist $4:5:6:7$ im Tonnetz
nicht realisierbar, da hier die nächste Primzahl $7$ auftaucht.

Man kann sich nun fragen, warum man bei $5$ aufhören sollte. Mathematisch
besteht hierfür kein Grund, und tatsächlich gibt es moderne Musik, die z.\,B. in
„$31$-limit tuning“ geschrieben ist.\todo{Evtl. auf etwas verweisen? War nicht
  z.\,B. dieses eine saggital.org Stück \emph{high limit}? TD}  Für die Belange
klassischer Vokalmusik hingegen genügen die Primzahlen $2$, $3$ und $5$ fast
vollständig; nur ganz selten wollen wir die nächste Primzahl $7$ benutzen, um
die schon angedeutete (sehr) kleine Septime $4:7$ verwenden zu können – dies ist
Gegenstand von \cref{sec:sept}.


\section{Primzahlen}
\label{sec:primes}
Blicken darauf zurück, in welcher Reihenfolge wir unseren Tonvorrat schrittweise
erweitert haben, so bemerken wir, dass in jedem Schritt die nächstgrößere
Primzahl ins Spiel gekommen ist. Um dies präzise zu machen, fixieren wir die
Frequenz $f_{\text K}$ des Kammertons a’ (z.\,B. $440\Hz$) und nennen für einen Ton mit
Frequenz $f$ den Quotienten $f/f_{\text K}$ den \emph{Faktor} des Tons. Dann sehen wir
folgendes:
\begin{enumerate}
	\item[2.] Für jede Oktavierung des Kammertons gibt es eine eindeutig bestimmte
	ganze Zahl $a_1\in\Z$, für die der Faktor des gegebenen Tons genau $2^{a_1}$
	beträgt. Mathematisch gesprochen definiert $a_1\mapsto 2^{a_1}$ eine Bijektion
	zwischen $\Z$ und den Faktoren der Oktavierungen des Kammertons.
	\item[3.] Für jeden Ton in pythagoreischer Stimmung gibt es eindeutig bestimmte
	ganze Zahlen $a_1,a_2\in\Z$, für die der Faktor des gegebenen Tons genau
	$2^{a_1}\cdot 3^{a_2}$ beträgt. Mathematisch gesprochen definiert
	$(a_1,a_2)\mapsto 2^{a_1}\cdot 3^{a_2}$ eine Bijektion zwischen $\Z^2$ und den
	Faktoren der Töne in pythagoreischer Stimmung.\footnote{Dass die
		Quintenspirale selbst $1$-dimensional ist, liegt daran, dass wir aus ihr die
		Oktave „rausgerechnet“ haben, also z.\,B. nicht zwischen \flat a und \flat
		a’ unterscheiden.}
	\item[5.] Für jeden Ton im Eulerschen Tonnetz (zusammen mit einer Oktavlage,
	z.\,B. \flatp a’) gibt es eindeutig bestimmte ganze Zahlen $a_1,a_2,a_3\in\Z$,
	für die der Faktor des gegebenen Tons genau
	$2^{a_1}\cdot 3^{a_2}\cdot 5^{a_3}$ beträgt.  Mathematisch gesprochen
	definiert $(a_1,a_2,a_3)\mapsto 2^{a_1}\cdot 3^{a_2}\cdot 5^{a_3}$ eine
	Bijektion zwischen $\Z^3$ und den Faktoren der Töne im Eulerschen
	Tonnetz.\footnote{Dass das Tonnetz selbst $2$-dimensional ist, liegt wieder
		daran, dass wir aus ihm die Oktave „rausgerechnet“ haben, also z.\,B. nicht
		zwischen \flatp a und \flatp a’ unterscheiden.}
	\item[7.] Das septimale Kommas bringt die Primzahl $7$ ins Spiel. Wir haben in
	\cref{sec:sept} nur die $0$- oder $(-1)$-fache Anwendung des Kommas
	eingeführt. Würde man dies für jede ganze Zahl $a_4$ tun, so ergäbe sich ein
	$4$-dimensionales Gitter an Tönen, das durch
	$(a_1,a_2,a_3,a_4)\mapsto 2^{a_1}\cdot 3^{a_2}\cdot 5^{a_3}\cdot 7^{a_4}$
	parametrisiert ist.\looseness-1\end{enumerate}
Allgemein bezeichnen wir eine Menge $T\subset \Q_{>0}$ von positiven Brüchen als
\emph{rationalen Tonvorrat}.  Unabhängig davon, wie wir bestimmte Töne nennen,
kann man für ein gegebenes Stück mathematisch beschreiben, welcher rationale
Tonvorrat verwendet wird.  Sei $p$ eine Primzahl und seien $p_1<\dotsb<p_r=p$
alle Primzahlen bis $p$. Dann ist der \emph{durch $p$ begrenzte} Tonvorrat
(engl. „$p$-limit tuning“) definiert als
\[T_p\coloneqq\{p_1^{a_1}\dotsm p_r^{a_r};\,a_i\in\Z\}\subset \Q_{>0}.\]%
Wir bemerken, dass $(a_1,\dotsc,a_r)\mapsto p_1^{a_1}\dotsm p_r^{a_r}$ eine
Bijektion zwischen $\Z^r$ und $T_p$ ist, sodass wir die Töne in $T_p$ in einem
$r$-dimensionalen Gitter anordnen können. Ferner sind im durch $p$ begrenzten
Tonvorrat alle auftretenden Frequenzverhältnisse zwischen $s$ gegebenen Tönen
der Form
\[p_1^{a_{11}}\dotsm p_r^{a_{r1}}:\dotsb:p_1^{a_{1s}}\dotsm p_r^{a_{rs}},\]%
wobei wir durch Multiplikation mit dem Hauptnenner erreichen können, dass
$a_{ij}\ge 0$ gilt. Ein Verhältnis von $s$ Tönen ist also beschrieben durch eine
$(r\times s)$-Matrix mit nicht-negativen ganzzahligen
Einträgen.\footnote{Gekürzt ist das Verhältnis genau dann, wenn es für jedes
	$1\le i\le r$ ein $1\le j\le s$ gibt mit $a_{ij}=0$, also wenn in jeder Zeile
	der Matrix eine $0$ auftaucht.} Dies erklärt zum Beispiel, warum das
Verhältnis $4:7$ im Eulerschen Tonnetz nicht auftreten kann.

Der für klassische Musik geeignete ist im Wesentlichen der durch $5$ begrenzte
Tonvorrat $T_5$, mit vereinzelten, in \cref{sec:sept} vorgestellten, Ausnahmen,
in denen man einen Schritt nach $T_7$ erwägen kann.  Dies macht natürlich die
daruffolgenden Primzahlen und die durch sie beschreibbaren Tonvorräte nicht
weniger interessant. Zum einen gibt es für jede Primzahl $p$ ein entsprechendes
Komma, um sie in unser Notensystem zu integrieren. Für viele davon wurden in
\cite{HEJI} Glyphen entworfen, die ähnlich wie \septimal\ die vorgestelle
Helmholtz-Ellis-Notation fortsetzen. Zum anderen wurde in den zurückliegenden
Jahrzehnten eine ganze Reihe sogenannter „mikrotonaler Musik“ geschrieben, die
auf einen durch eine große Primzahl (z.\,B. $67$) begrenzten Tonvorrat
zurückgreift, z.\,B. \cite{Nicholson}.

%%% Local Variables:
%%% mode: latex
%%% TeX-master: "../main"
%%% End:
 



\printbibliography[heading=bibintoc]

\end{document}
