\documentclass[ngerman,11pt]{scrartcl}

\title     {Reine Intonation und Microtuning}
\author    {Thomas Drögemüller\and Florian Kranhold}
\subtitle  {Eine Abhandlung}

%% KOMA-Optionen (scrlttr2, sonst: scrartcl, scrbook)
\KOMAoptions{headings=standardclasses,numbers=enddot}

%% Encoding
\usepackage[utf8]{inputenc}
\usepackage[T1]{fontenc}

%% Sprache
\usepackage{babel}
\usepackage{csquotes} % Correct quotiation marks „…“ bzw. ‘…’

%% Datumsformat (ngerman, british)
\usepackage[cleanlook]{isodate}

%% LModern ist ein besserer Fallback als Typewriter 
\usepackage{lmodern}

%% Palatino als Brotschrift
%% (Um die Serifenlose kümmern wir uns unten)
\usepackage[osf,sc]{mathpazo}

%% In einigen Fällen passen Lining figures besser
\newcommand{\LF}[1]{{\fontfamily{pplx}\selectfont #1}}

%% Classico als Serifenlose, falls vorhanden
%% Andernfalls ist der Fallback die LM Sans Serif,
%% die wir unten nach Möglichkeit vermeiden werden.
\IfFileExists{classico.sty}{\usepackage{classico}}{}

%% Beramono als Dicktengleiche
\usepackage[scaled=.85]{beramono}

%% Palatino braucht mehr Platz
\usepackage{setspace}
\setstretch{1.07}
\renewcommand{\arraystretch}{1.07} % Für tabular- und array-Umgebungen

%% Mikrotypographie!
\usepackage{ellipsis} % More space after dots
\usepackage[babel,tracking=true]{microtype}

%% microtype: Disable Footnote-Patching for ≥3.0
\makeatletter
\@ifpackagelater{microtype}{2020/12/08}
  {\microtypesetup{nopatch=footnote}}
  {}
\makeatother

%% microtype: Settings
\UseMicrotypeSet[tracking]{smallcaps}
\SetTracking{encoding=*,shape=sc}{30}

%% Hurenkinder verbieten
\makeatletter
  \widowpenalty=\@M
\makeatother

%% Inline-Formeln NIE umbrechen
\relpenalty=10000
\binoppenalty=10000

%% Less Looseness
\everypar=\expandafter{\the\everypar\looseness=-1}

%% Text nicht vertikal strecken oder stauchen
\raggedbottom

%% Akronyme
\usepackage{relsize}
\newcommand{\acr}[1]{\texorpdfstring{\textsmaller{\LF{#1}}}{#1}}

%% Eigene Listing styles mit kurzer Definition
\usepackage[shortlabels]{enumitem}

% hyperref<7.00h replaces \C with the cyrillic letter \CYRDZE
% See https://github.com/latex3/hyperref/issues/170
% We want it to be the complex numbers, as previously defined.
\let\COld\C

%% Figures
\usepackage{graphicx}
\usepackage[margin    =20pt,
            font      ={footnotesize,stretch=1.06},
            format    =plain,
            labelsep  =period,
            labelfont =bf]{caption}
  
%% See above
\let\C\COld

%% Fancy tables
\usepackage{booktabs}
  
%% Colours, e.g. for References
\usepackage{xcolor}

%% Einbinden anderen Präambeln (die auch Pakete einbinden),
%% außer -env, was cleveref einbindet und deshalb zuletzt kommt.
\IfFileExists{preamble-math.tex}  {%%% commit: dc6a172

%% Grundlegende mathematische Symbole
\usepackage{amssymb,amsmath} % Standard
\usepackage{mathtools}       % \coloneqq, \eqqcolon
\usepackage{stmaryrd}        % \sslash, \Ydown, \llbracket

%% TikZ and TikZ-CD
\usepackage{tikz-cd}
\usepackage{tikz}
\usetikzlibrary{arrows,calc}

%% Mini-Indizes (\scaleto{…}{3pt})
\usepackage{scalerel}

%% Some choices
\let \le       \leqslant
\let \ge       \geqslant
\let \leq      \leqslant
\let \geq      \geqslant
\let \setminus \smallsetminus
\let \emptyset \varnothing
\let \epsilon  \varepsilon
%\let \theta    \vartheta

%% Bold
\newcommand{\bm} [1]{\mathbold{#1}}

%% Short for overline and operatorname
\newcommand{\ol} [1]{\overline{#1}}
\newcommand{\on} [1]{\mathrm{#1}}

%% Fix the spacing of \left and \right
\let\originalleft\left
\let\originalright\right
\renewcommand{\left}{\mathopen{}\mathclose\bgroup\originalleft}
\renewcommand{\right}{\aftergroup\egroup\originalright}
\renewcommand{\bigl}{\mathopen\big}
\renewcommand{\bigr}{\mathclose\big}

%% Brackets
\newcommand{\lla}[1]{\mathopen{}\left\langle #1 \right\rangle\mathclose{}}      % <…>
\newcommand{\ula}[1]{\langle #1\rangle}                                         % <…>   sm
\newcommand{\llb}[1]{\mathopen{}\left\llbracket #1\right\rrbracket\mathclose{}} % [[…]]
\newcommand{\ulb}[1]{\mathopen{}\left\llbracket #1\right\rrbracket\mathclose{}} % [[…]] sm

%% Some operatornames
\newcommand{\GL} {\on{GL}}
\newcommand{\pr} {\on{pr}}
\newcommand{\ob} {\on{ob}}
\newcommand{\sg} {\on{sg}}
\newcommand{\im} {\on{im}}
\newcommand{\op} {{\on{op}}} % It is nearly always an exponent
\newcommand{\Aut}{\on{Aut}}
\newcommand{\Hom}{\on{Hom}}
\newcommand{\map}{\on{map}}
\newcommand{\Id} {\on{id}}   % The only one where the capitalisation has changed

% colon on the right side
\newcommand*\lon{%
        \nobreak
        \mskip6mu plus1mu
        \mathpunct{}%
        \nonscript
        \mkern-\thinmuskip
        {:}%
        \mskip2mu
        \relax
}

%% Some categories
\newcommand{\Top}{\mathbf{Top}}
\newcommand{\Sig}{\mathbf{\Sigma}}
\newcommand{\Del}{\mathbf{\Delta}}
\newcommand{\Set}{\mathbf{Set}}
\newcommand{\Mod}{\mathbf{Mod}}
\newcommand{\Alg}{\mathbf{Alg}}

%% Miscellanea 
\newcommand{\bEn}{\mathbb{1}}
\newcommand{\RP}{\R\mathrm{P}}
\newcommand{\CP}{\C\mathrm{P}}
\newcommand{\Diff}{\on{Diff}{}}
\newcommand{\hAut}{\on{hAut}}
\newcommand{\dcup}{\,\dot\cup\,}
\newcommand{\swr}{\kern.6px\wr\kern.6px}

%% Integral sign
\DeclareSymbolFont{eulargesymbols}{U}{zeuex}{m}{n}
\DeclareMathSymbol{\intop}{\mathop}{eulargesymbols}{"52}

%% TikZ: centerarc
\def\centerarc[#1](#2)(#3:#4:#5)% Syntax: [draw options] (center) (initial angle:final angle:radius)
    { \draw[#1] ($(#2)+({#5*cos(#3)},{#5*sin(#3)})$) arc (#3:#4:#5); }

%\usepackage[scr=boondoxo,scrscaled=1,cal=euler]{mathalfa} % Calligraphic fonts

% Zahlenbereiche
\newcommand{\N}{\mathbb{N}}
\newcommand{\Z}{\mathbb{Z}}
\newcommand{\Q}{\mathbb{Q}}
\newcommand{\R}{\mathbb{R}}
\newcommand{\C}{\mathbb{C}}

%% Small \vdots
\usepackage{letltxmacro}
\LetLtxMacro\orgvdots\vdots
\LetLtxMacro\orgddots\ddots
\makeatletter
\DeclareRobustCommand\vdots{%
  \mathpalette\@vdots{}%
}
\newcommand*{\@vdots}[2]{%
  % #1: math style
  % #2: unused
  \sbox0{$#1\cdotp\cdotp\cdotp\m@th$}%
  \sbox2{$#1.\m@th$}%
  \vbox{%
    \dimen@=\wd0 %
    \advance\dimen@ -3\ht2 %
    \kern.5\dimen@
    % remove side bearings
    \dimen@=\wd2 %
    \advance\dimen@ -\ht2 %
    \dimen2=\wd0 %
    \advance\dimen2 -\dimen@
    \vbox to \dimen2{%
      \offinterlineskip
      \copy2 \vfill\copy2 \vfill\copy2 %
    }%
  }%
}
\makeatother

% Correct fracs
\DeclareFontFamily{OMS}{zplm}{
   \skewchar\font=48 % kept from the original file
   \fontdimen 9\font=0.34\fontdimen6\font
   \fontdimen 8\font=0.84\fontdimen6\font
}
\DeclareFontShape{OMS}{zplm}{m}{n}{<-> zplmr7y}{}
\DeclareFontShape{OMS}{zplm}{b}{n}{<-> zplmb7y}{}
\DeclareFontShape{OMS}{zplm}{bx}{n}{<->ssub * zplm/b/n}{}

%%% The rest of this header file shortens the following letter sets:
%%% bbA, bA, caA, scA, sca, bfA, bmA, frA, fra, tA

%% Blackboard bold
\newcommand{\bbA}{\mathbb{A}}
\newcommand{\bbB}{\mathbb{B}}
\newcommand{\bbC}{\mathbb{C}}
\newcommand{\bbD}{\mathbb{D}}
\newcommand{\bbE}{\mathbb{E}}
\newcommand{\bbF}{\mathbb{F}}
\newcommand{\bbG}{\mathbb{G}}
\newcommand{\bbH}{\mathbb{H}}
\newcommand{\bbI}{\mathbb{I}}
\newcommand{\bbJ}{\mathbb{J}}
\newcommand{\bbK}{\mathbb{K}}
\newcommand{\bbL}{\mathbb{L}}
\newcommand{\bbM}{\mathbb{M}}
\newcommand{\bbN}{\mathbb{N}}
\newcommand{\bbO}{\mathbb{O}}
\newcommand{\bbP}{\mathbb{P}}
\newcommand{\bbQ}{\mathbb{Q}}
\newcommand{\bbR}{\mathbb{R}}
\newcommand{\bbS}{\mathbb{S}}
\newcommand{\bbT}{\mathbb{T}}
\newcommand{\bbU}{\mathbb{U}}
\newcommand{\bbV}{\mathbb{V}}
\newcommand{\bbW}{\mathbb{W}}
\newcommand{\bbX}{\mathbb{X}}
\newcommand{\bbY}{\mathbb{Y}}
\newcommand{\bbZ}{\mathbb{Z}}

%% Blackboard bold (shorter)
\newcommand{\bA}{\mathbb{A}}
\newcommand{\bB}{\mathbb{B}}
\newcommand{\bC}{\mathbb{C}}
\newcommand{\bD}{\mathbb{D}}
\newcommand{\bE}{\mathbb{E}}
\newcommand{\bF}{\mathbb{F}}
\newcommand{\bG}{\mathbb{G}}
\newcommand{\bH}{\mathbb{H}}
\newcommand{\bI}{\mathbb{I}}
\newcommand{\bJ}{\mathbb{J}}
\newcommand{\bK}{\mathbb{K}}
\newcommand{\bL}{\mathbb{L}}
\newcommand{\bM}{\mathbb{M}}
\newcommand{\bN}{\mathbb{N}}
\newcommand{\bO}{\mathbb{O}}
\newcommand{\bP}{\mathbb{P}}
\newcommand{\bQ}{\mathbb{Q}}
\newcommand{\bR}{\mathbb{R}}
\newcommand{\bS}{\mathbb{S}}
\newcommand{\bT}{\mathbb{T}}
\newcommand{\bU}{\mathbb{U}}
\newcommand{\bV}{\mathbb{V}}
\newcommand{\bW}{\mathbb{W}}
\newcommand{\bX}{\mathbb{X}}
\newcommand{\bY}{\mathbb{Y}}
\newcommand{\bZ}{\mathbb{Z}}

%% Calligraphic
\newcommand{\caA}{\mathcal{A}}
\newcommand{\caB}{\mathcal{B}}
\newcommand{\caC}{\mathcal{C}}
\newcommand{\caD}{\mathcal{D}}
\newcommand{\caE}{\mathcal{E}}
\newcommand{\caF}{\mathcal{F}}
\newcommand{\caG}{\mathcal{G}}
\newcommand{\caH}{\mathcal{H}}
\newcommand{\caI}{\mathcal{I}}
\newcommand{\caJ}{\mathcal{J}}
\newcommand{\caK}{\mathcal{K}}
\newcommand{\caL}{\mathcal{L}}
\newcommand{\caM}{\mathcal{M}}
\newcommand{\caN}{\mathcal{N}}
\newcommand{\caO}{\mathcal{O}}
\newcommand{\caP}{\mathcal{P}}
\newcommand{\caQ}{\mathcal{Q}}
\newcommand{\caR}{\mathcal{R}}
\newcommand{\caS}{\mathcal{S}}
\newcommand{\caT}{\mathcal{T}}
\newcommand{\caU}{\mathcal{U}}
\newcommand{\caV}{\mathcal{V}}
\newcommand{\caW}{\mathcal{W}}
\newcommand{\caX}{\mathcal{X}}
\newcommand{\caY}{\mathcal{Y}}
\newcommand{\caZ}{\mathcal{Z}}

%% Script large
\newcommand{\scA}{\mathscr{A}}
\newcommand{\scB}{\mathscr{B}}
\newcommand{\scC}{\mathscr{C}}
\newcommand{\scD}{\mathscr{D}}
\newcommand{\scE}{\mathscr{E}}
\newcommand{\scF}{\mathscr{F}}
\newcommand{\scG}{\mathscr{G}}
\newcommand{\scH}{\mathscr{H}}
\newcommand{\scI}{\mathscr{I}}
\newcommand{\scJ}{\mathscr{J}}
\newcommand{\scK}{\mathscr{K}}
\newcommand{\scL}{\mathscr{L}}
\newcommand{\scM}{\mathscr{M}}
\newcommand{\scN}{\mathscr{N}}
\newcommand{\scO}{\mathscr{O}}
\newcommand{\scP}{\mathscr{P}}
\newcommand{\scQ}{\mathscr{Q}}
\newcommand{\scR}{\mathscr{R}}
\newcommand{\scS}{\mathscr{S}}
\newcommand{\scT}{\mathscr{T}}
\newcommand{\scU}{\mathscr{U}}
\newcommand{\scV}{\mathscr{V}}
\newcommand{\scW}{\mathscr{W}}
\newcommand{\scX}{\mathscr{X}}
\newcommand{\scY}{\mathscr{Y}}
\newcommand{\scZ}{\mathscr{Z}}

%% Script small
\newcommand{\sca}{\mathscr{a}}
\newcommand{\scb}{\mathscr{b}}
\newcommand{\scc}{\mathscr{c}}
\newcommand{\scd}{\mathscr{d}}
\newcommand{\sce}{\mathscr{e}}
\newcommand{\scf}{\mathscr{f}}
\newcommand{\scg}{\mathscr{g}}
\newcommand{\sch}{\mathscr{h}}
\newcommand{\sci}{\mathscr{i}}
\newcommand{\scj}{\mathscr{j}}
\newcommand{\sck}{\mathscr{k}}
\newcommand{\scl}{\mathscr{l}}
\newcommand{\scm}{\mathscr{m}}
\newcommand{\scn}{\mathscr{n}}
\newcommand{\sco}{\mathscr{o}}
\newcommand{\scp}{\mathscr{p}}
\newcommand{\scq}{\mathscr{q}}
\newcommand{\scr}{\mathscr{r}}
\newcommand{\scs}{\mathscr{s}}
\newcommand{\sct}{\mathscr{t}}
\newcommand{\scu}{\mathscr{u}}
\newcommand{\scv}{\mathscr{v}}
\newcommand{\scw}{\mathscr{w}}
\newcommand{\scx}{\mathscr{x}}
\newcommand{\scy}{\mathscr{y}}
\newcommand{\scz}{\mathscr{z}}

%% Bold (but not italic)
\newcommand{\bfA}{\mathbf{A}}
\newcommand{\bfB}{\mathbf{B}}
\newcommand{\bfC}{\mathbf{C}}
\newcommand{\bfD}{\mathbf{D}}
\newcommand{\bfE}{\mathbf{E}}
\newcommand{\bfF}{\mathbf{F}}
\newcommand{\bfG}{\mathbf{G}}
\newcommand{\bfH}{\mathbf{H}}
\newcommand{\bfI}{\mathbf{I}}
\newcommand{\bfJ}{\mathbf{J}}
\newcommand{\bfK}{\mathbf{K}}
\newcommand{\bfL}{\mathbf{L}}
\newcommand{\bfM}{\mathbf{M}}
\newcommand{\bfN}{\mathbf{N}}
\newcommand{\bfO}{\mathbf{O}}
\newcommand{\bfP}{\mathbf{P}}
\newcommand{\bfQ}{\mathbf{Q}}
\newcommand{\bfR}{\mathbf{R}}
\newcommand{\bfS}{\mathbf{S}}
\newcommand{\bfT}{\mathbf{T}}
\newcommand{\bfU}{\mathbf{U}}
\newcommand{\bfV}{\mathbf{V}}
\newcommand{\bfW}{\mathbf{W}}
\newcommand{\bfX}{\mathbf{X}}
\newcommand{\bfY}{\mathbf{Y}}
\newcommand{\bfZ}{\mathbf{Z}}

%% Bold (but not italic)
\newcommand{\bfa}{\mathbf{a}}
\newcommand{\bfb}{\mathbf{b}}
\newcommand{\bfc}{\mathbf{c}}
\newcommand{\bfd}{\mathbf{d}}
\newcommand{\bfe}{\mathbf{e}}
\newcommand{\bff}{\mathbf{f}}
\newcommand{\bfg}{\mathbf{g}}
\newcommand{\bfh}{\mathbf{h}}
\newcommand{\bfi}{\mathbf{i}}
\newcommand{\bfj}{\mathbf{j}}
\newcommand{\bfk}{\mathbf{k}}
\newcommand{\bfl}{\mathbf{l}}
\newcommand{\bfm}{\mathbf{m}}
\newcommand{\bfn}{\mathbf{n}}
\newcommand{\bfo}{\mathbf{o}}
\newcommand{\bfp}{\mathbf{p}}
\newcommand{\bfq}{\mathbf{q}}
\newcommand{\bfr}{\mathbf{r}}
\newcommand{\bfs}{\mathbf{s}}
\newcommand{\bft}{\mathbf{t}}
\newcommand{\bfu}{\mathbf{u}}
\newcommand{\bfv}{\mathbf{v}}
\newcommand{\bfw}{\mathbf{w}}
\newcommand{\bfx}{\mathbf{x}}
\newcommand{\bfy}{\mathbf{y}}
\newcommand{\bfz}{\mathbf{z}}

%% Bold and italic
\newcommand{\bmA}{\bm{A}}
\newcommand{\bmB}{\bm{B}}
\newcommand{\bmC}{\bm{C}}
\newcommand{\bmD}{\bm{D}}
\newcommand{\bmE}{\bm{E}}
\newcommand{\bmF}{\bm{F}}
\newcommand{\bmG}{\bm{G}}
\newcommand{\bmH}{\bm{H}}
\newcommand{\bmI}{\bm{I}}
\newcommand{\bmJ}{\bm{J}}
\newcommand{\bmK}{\bm{K}}
\newcommand{\bmL}{\bm{L}}
\newcommand{\bmM}{\bm{M}}
\newcommand{\bmN}{\bm{N}}
\newcommand{\bmO}{\bm{O}}
\newcommand{\bmP}{\bm{P}}
\newcommand{\bmQ}{\bm{Q}}
\newcommand{\bmR}{\bm{R}}
\newcommand{\bmS}{\bm{S}}
\newcommand{\bmT}{\bm{T}}
\newcommand{\bmU}{\bm{U}}
\newcommand{\bmV}{\bm{V}}
\newcommand{\bmW}{\bm{W}}
\newcommand{\bmX}{\bm{X}}
\newcommand{\bmY}{\bm{Y}}
\newcommand{\bmZ}{\bm{Z}}

%% Fraktur large
\newcommand{\frA}{\mathfrak{A}}
\newcommand{\frB}{\mathfrak{B}}
\newcommand{\frC}{\mathfrak{C}}
\newcommand{\frD}{\mathfrak{D}}
\newcommand{\frE}{\mathfrak{E}}
\newcommand{\frF}{\mathfrak{F}}
\newcommand{\frG}{\mathfrak{G}}
\newcommand{\frH}{\mathfrak{H}}
\newcommand{\frI}{\mathfrak{I}}
\newcommand{\frJ}{\mathfrak{J}}
\newcommand{\frK}{\mathfrak{K}}
\newcommand{\frL}{\mathfrak{L}}
\newcommand{\frM}{\mathfrak{M}}
\newcommand{\frN}{\mathfrak{N}}
\newcommand{\frO}{\mathfrak{O}}
\newcommand{\frP}{\mathfrak{P}}
\newcommand{\frQ}{\mathfrak{Q}}
\newcommand{\frR}{\mathfrak{R}}
\newcommand{\frS}{\mathfrak{S}}
\newcommand{\frT}{\mathfrak{T}}
\newcommand{\frU}{\mathfrak{U}}
\newcommand{\frV}{\mathfrak{V}}
\newcommand{\frW}{\mathfrak{W}}
\newcommand{\frX}{\mathfrak{X}}
\newcommand{\frY}{\mathfrak{Y}}
\newcommand{\frZ}{\mathfrak{Z}}

%% Fraktur small
\newcommand{\fra}{\mathfrak{a}}
\newcommand{\frb}{\mathfrak{b}}
\newcommand{\frc}{\mathfrak{c}}
\newcommand{\frd}{\mathfrak{d}}
\newcommand{\fre}{\mathfrak{e}}
\newcommand{\frf}{\mathfrak{f}}
\newcommand{\frg}{\mathfrak{g}}
\newcommand{\frh}{\mathfrak{h}}
\newcommand{\fri}{\mathfrak{i}}
\newcommand{\frj}{\mathfrak{j}}
\newcommand{\frk}{\mathfrak{k}}
\newcommand{\frl}{\mathfrak{l}}
\newcommand{\frm}{\mathfrak{m}}
\newcommand{\frn}{\mathfrak{n}}
\newcommand{\fro}{\mathfrak{o}}
\newcommand{\frp}{\mathfrak{p}}
\ifdefined\frq\renewcommand{\frq}{\mathfrak{q}}\else \newcommand{\frq}{\mathfrak{q}}\fi
\newcommand{\frr}{\mathfrak{r}}
\newcommand{\frs}{\mathfrak{s}}
\newcommand{\frt}{\mathfrak{t}}
\newcommand{\fru}{\mathfrak{u}}
\newcommand{\frv}{\mathfrak{v}}
\newcommand{\frw}{\mathfrak{w}}
\newcommand{\frx}{\mathfrak{x}}
\newcommand{\fry}{\mathfrak{y}}
\newcommand{\frz}{\mathfrak{z}}

%% Tilde
\newcommand{\tA}{\tilde{A}}
\newcommand{\tB}{\tilde{B}}
\newcommand{\tC}{\tilde{C}}
\newcommand{\tD}{\tilde{D}}
\newcommand{\tE}{\tilde{E}}
\newcommand{\tF}{\tilde{F}}
\newcommand{\tG}{\tilde{G}}
\newcommand{\tH}{\tilde{H}}
\newcommand{\tI}{\tilde{I}}
\newcommand{\tJ}{\tilde{J}}
\newcommand{\tK}{\tilde{K}}
\newcommand{\tL}{\tilde{L}}
\newcommand{\tM}{\tilde{M}}
\newcommand{\tN}{\tilde{N}}
\newcommand{\tO}{\tilde{O}}
\newcommand{\tP}{\tilde{P}}
\newcommand{\tQ}{\tilde{Q}}
\newcommand{\tR}{\tilde{R}}
\newcommand{\tS}{\tilde{S}}
\newcommand{\tT}{\tilde{T}}
\newcommand{\tU}{\tilde{U}}
\newcommand{\tV}{\tilde{V}}
\newcommand{\tW}{\tilde{W}}
\newcommand{\tX}{\tilde{X}}
\newcommand{\tY}{\tilde{Y}}
\newcommand{\tZ}{\tilde{Z}}
}  {}
\IfFileExists{preamble-bib.tex}   {%%% commit: dc6a172

\usepackage[backend      = biber,
            style        = numeric,
            maxbibnames  = 9,
            maxcitenames = 9,
            %doi          = false,
            bibencoding  = utf8,
            datamodel    = ext-eprint]{biblatex}

% Oxford comma in Bibliography
\DefineBibliographyExtras{british}{%
  \renewcommand{\finalandcomma}{,}
}

\DeclareFieldFormat{doi}{\textsf{\href{https://doi.org/#1}{[DOI]}}}
%\DeclareFieldFormat[misc]{title}{‘#1’}  % plain article title for arXiv

%% Falls die Serifenlose zur Verfügung steht,
%% dann setze auch die arXiv-Klasse [math.AT] damit.
\makeatletter
\IfFileExists{classico.sty}{
\DeclareFieldFormat{eprint:arxiv}{%
  arXiv\addcolon\space
  \ifhyperref
    {\href{http://arxiv.org/\abx@arxivpath/##1}{%
       \nolinkurl{##1}%
       \iffieldundef{eprintclass}
     {}
     {\addspace\textsf{\mkbibbrackets{\thefield{eprintclass}}}}}}
    {\nolinkurl{##1}
     \iffieldundef{eprintclass}
       {}
       {\addspace\textsf{\mkbibbrackets{\thefield{eprintclass}}}}}}
}{}
\makeatother

%% Nicht in Von-Bis-Strichen umbrechen,
%% z.B. bei den Seitenzahlen
\DefineBibliographyExtras{british}{\protected\def\bibrangedash{\nobreakdash--}}
\DefineBibliographyExtras{ngerman}{\protected\def\bibrangedash{\nobreakdash--}}

%\AtBeginBibliography{\small}

\bibliography{bibliography.bib}
}   {}
\IfFileExists{preamble-music.tex} {\input{preamble-music.tex}} {}
\IfFileExists{preamble-custom.tex}{\usepackage[size=tiny]{todonotes}
\usepackage{multicol}
%\setlength\columnseprule{.4pt}

\usepackage{marginnote}
\newcommand{\AS}[1]{\marginnote{\texttt{#1}}}
\newcommand{\LY}[1]{\marginnote{\texttt{#1}}}

%\usepackage{pgfplots}

\usepackage{xparse}% http://ctan.org/pkg/xparse
\usepackage{etoolbox}% http://ctan.org/pkg/etoolbox
\newcounter{listtotal2}\newcounter{listcntr}%
\NewDocumentCommand{\pitch}{o}{%
  \setcounter{listtotal2}{0}\setcounter{listcntr}{-1}%
  \renewcommand*{\do}[1]{\stepcounter{listtotal2}}%
  \expandafter\docsvlist\expandafter{\pitcharray}%
  \IfNoValueTF{#1}
    {\pitcharray}% \pitch
    {% \pitch[<index>]
     \renewcommand*{\do}[1]{\stepcounter{listcntr}\ifnum\value{listcntr}=#1\relax##1\fi}%
     \expandafter\docsvlist\expandafter{\pitcharray}}%
}

\newcommand{\rC}{\text{C}}
\newcommand{\rD}{\text{D}}
\newcommand{\rE}{\text{E}}
\newcommand{\rF}{\text{F}}
\newcommand{\rG}{\text{G}}
\newcommand{\rA}{\text{A}}
\newcommand{\rB}{\text{B}}

\newcommand{\uP}{{\text{+}}}
\newcommand{\uM}{{\text{\textminus}}}
\newcommand{\uMM}{{\text{\textminus\textminus}}}

\newcommand{\Hz}{\,\text{Hz}}
\newcommand{\ct}{\,\text{ct}}

\usepackage{heji2}
\newcommand{\om}[1]{\mathrm{#1}}

% Format for the table headline
\newcommand{\thl}[1]{\textsc{\textls[40]{\MakeLowercase{#1}}}}
}{}

%% Footnote-Design
\usepackage{footmisc}
\renewcommand{\footnotelayout}{\setstretch{1.06}}
\setlength{\footnotemargin}{1em}

%% Kopfzeilen in geneigtem statt kursivem Palatino
\addtokomafont{pagehead}{\fontfamily{ppl}\selectfont}

%% Description mit SC, vor allem nicht als Sans
\setkomafont{descriptionlabel}{\scshape}

%% pazofrac (ähnlich nicefrac, aber besser auf mathpazo abgestimmt)
%% z.B. um „4/4-Takt“ oder „1/2 Liter“ zu schreiben, deshalb
%% von allgemeinem Interesse und nicht in header-math.tex
\newcommand{\pazofrac}[2]%
  {\leavevmode\kern.1em\raise.65ex\hbox{\scriptsize #1}%
   \kern-.125em/\kern-.125em\lower-.05ex\hbox{\scriptsize #2}}

%% Abstract ohne Einrückung des ersten Absatzes
\usepackage{etoolbox}
\ifundef{\abstract}{}{
  \patchcmd{\abstract}{\quotation}{\quotation\noindent\ignorespaces}{}{}}

%% hyperref should be the last package, except for cleveref.
%% We load cleveref and amsthm later.
\usepackage[pdfusetitle,%
            bookmarksnumbered=true,%
            unicode,%
            hidelinks]{hyperref}
            
%% Wenn Classico nicht verfügbar ist, dann umgehen wir Serifenlose
%% für URLs und Mailadressen und weichen auf Dicktengleiche aus.
\IfFileExists{classico.sty}{
  \let\mailfont\textsf
  \urlstyle{sf}
}{
  \let\mailfont\texttt
  \urlstyle{tt}
}

%% Mails
\newcommand{\mail}[1]{\upshape\href{mailto:#1}{\mailfont{#1}}}

%% Farben für Referenzen
\hypersetup{
  colorlinks,
  linkcolor={black!45!red},   % Interne Referenzen
  citecolor={black!55!green}, % Bibliographie
  urlcolor ={black!45!blue}   % Externe Referenzen
}

%% Umgebungen und cleveref (lädt: amsthm, cleveref)
\IfFileExists{preamble-env.tex}{%%% commit: dc6a172

%% Spätestens hier brauchen wir das amsthm-Paket.
%% Womöglich wurde es aber schon über die header-custom
%% eingebunden, etwa wenn individuelle Umgebungen definiert wurden.
\makeatletter
  \@ifpackageloaded{amsthm}{}{\usepackage{amsthm}}
\makeatother

%% cleveref has to be the last package,
%% but it has to be introduced before we define
%% individual theorem environments.
\usepackage[capitalise,noabbrev]{cleveref}

\theoremstyle{definition}
\newtheorem{aufg}{Aufgabe}[section]
\crefname{aufg}{Aufgabe}{Aufgaben}

%% A command between two environments, which takes effect
%% somewhere else (e.g. a figure that is placed on top of
%% a page, or an \enlargethispage) will cause additional
%% space between the environments. The following modified
%% commands prevent this.

%% Measure the intended distance
\newlength{\intendedDistance}

%% Wrap (e.g. \wrap{\enlargethispage{\baselineskip}})
\newcommand{\wrap}[1]{
  \setlength{\intendedDistance}{\lastskip}\addvspace{-\lastskip}
  #1
  \addvspace{\intendedDistance}
}

%% Repair top figures
\newenvironment{t-figure}{
  \setlength{\intendedDistance}{\lastskip}\addvspace{-\lastskip}
  \begin{figure}[t]
  }{
  \end{figure}
  \addvspace{\intendedDistance}
}

%% Repair top tables
\newenvironment{t-table}{
  \setlength{\intendedDistance}{\lastskip}\addvspace{-\lastskip}  
  \begin{table}[t]
  \centering
  }{
  \end{table}
  \addvspace{\intendedDistance}
}
}{}

%% Adressen am Ende eines Artikels mit Lining Figures
\newcommand{\addr}[2]{%
  \paragraph{#1}
  \LF{#2}
} 

%% Wir lassen KOMA den Satzspiegel selbst errechnen.
\KOMAoptions{DIV=calc}


% % Kopfzeilen können wie folgt ergänzt werden
\usepackage{scrlayer-scrpage}
\lohead{T. Drögemüller, F. Kranhold}
\cohead{}
\rohead{Reine Intonation und Microtuning}
\pagestyle{scrheadings}

\begin{document}

\maketitle

\begin{abstract}
  In dieser Arbeit entwickeln wir die mathematischen und musikalischen
  Grundlagen  der reinen Intonation für die klassische westliche Musik,
  einschließlich des pythagoreischen, syntonischen und septimalen Kommas, sowie
  des Eulerschen Tonnetzes. Darauf aufbauend stellen wir ein System zum
  „Microtunen“ von Vokalmusik vor. Unsere Herangehensweise ist theoretisch,
  d.\,h. wir wollen uns nicht auf ein Instrument mit einer endlichen Anzahl an
  Tasten beschränken. Dieser Aufsatz wird von mehreren Audiobeispielen
  begleitet, die für den Open-Source-Interpreter LilyPond geschrieben sind. Wir
  erklären im Anhang, wie man sie kompiliert.
\end{abstract}

\section{Einleitung und Überblick}
\label{sec:int}

Wir setzen folgende Grundkonzepte klassischer Musiktheorie als bekannt voraus:
Töne, Intervallnamen, diatonische Skalen, Akkorde, Tonarten und der
Quintenzirkel. Wir verwenden hierbei die Sprache von \cite{Skript}.

Wenn ein Instrument (z.\,B. die menschliche Stimme) einen Ton erzeugt, schwingt
ein „Material“ (wie eine Saite oder eine Luftsäule) mit einer gewissen Frequenz.
Je höher die Frequenz ist, desto höher ist der Ton. Ein \emph{Stimmungssystem}
weist jedem Ton eine Frequenz zu. Typischerweise beginnen wir mit der Fixierung
des \emph{Kammertons}, der Frequenz des a’, heutzutage bei $440$\,Hz. Von dort
leiten wir die anderen Tonhöhen wie folgt ab: Die einfachste Beziehung zwischen
Tönen ist die der \emph{Oktave}, die einem Frequenzverhältnis von $1:2$
entspricht, z.\,B. bekommt das a’’ die Frequenz $880$\,Hz zugewiesen. In dem
modernen Stimmungssystem der \emph{gleichstufigen Stimmung} wird die Oktave in
zwölf gleich große Schritte – genannt \emph{Halbtonschritte} oder \acr{HTS} – 
aufgeteilt. Ein Halbtonschritt entspricht also dem Frequenzverhältnis
$1:2^{\smash{\frac1{12}}}$. Mit diesem Verhältnis kann nun jedem Ton eine
Frequenz zugewiesen werden; beispielsweise wird dem e’’ sieben \acr{HTS}
über dem a’ eine Frequenz von \looseness-1
\[440\,\text{Hz}\cdot 2^{\frac7{12}} \approx 659{,}26\,\text{Hz}\]%
zugewiesen. Hier könnte man erwarten, dass die Erzählung, und damit auch dieses
Skript™, ihr Ende haben. Allerdings stellt sich dieses Stimmungssystem als
suboptimal heraus sobald mehrere Töne gleichzeitig erklingen. Die Physik™ lehrt
uns, dass mehrere Wellen konsonieren, wenn ihre Frequenzen im Verhältnis
kleiner, paarweise teilerfremder Ganzzahlen stehen, z.\,B. $4:5:6:7$. Sehen wir
uns nun einen Durakkord mit kleiner Septime bestehend aus Grundton, großer Terz,
reiner Quinte und kleiner Septime an. Da eine große Terz aus vier, eine reine
Quinte aus sieben und eine kleine Septime aus zehn \acr{HTS} bestehen, ist das
Frequenverhältnis dieser vier gleichzeitig erklingenden Töne \looseness-1%
\[1:2^{\frac4{12}}:2^{\frac7{12}}:2^{\frac{10}{12}}\approx
  1:1{,}260:1{,}498:1{,}781.\]%
Die Beobachtung,\footnote{Streng genommen erzählen wir hier die Geschichte nicht
  in historisch korrekter Reihenfolge, denn es ist mitnichten so, dass im Laufe
  der Musikgeschichte Defizite der gleichstufigen Stimmung erkannt und angepasst
  wurden. Vielmehr kann die Idee ganzzahliger Frequenzverhältnisse als Beginn
  der Musiktheorie gesehen werden, während heutzutage z.\,B. für
  Tasteninstrumente die gleichstufige Stimmung aus Gründen der Praktikabilität
  verwendet wird. Wir haben uns dennoch für diese Darstellung entschieden, weil
  sie systematischer und zudem leichter zu verstehen ist.} die der Theorie der
reinen Stimmung zugrunde liegt, ist nun, dass diese Verhältnis zwar
\emph{ungefähr}, aber nicht \emph{exakt} $4:5:6:7$ ist. Die „Unreinheiten“, die
wir hier sehen, sind deutlich kleiner als ein \acr{HTS}, weswegen wir ihre Größe
mit Hilfe der Einheit \emph{Cent} ausdrücken, wobei $100\,\on{ct}$ einem
gleichstufigen \acr{HTS} entsprechen. Anders ausgedrückt: Die Oktave (d.\,h. die
Multiplikation mit $2$) umfasst $1{.}200\,\on{ct}$. Wir können damit
beispielsweise sehen, dass das Verhältnis $4:5$ einem Unterschied von
\[1{.}200\,\on{ct}\cdot \log_2(\tfrac54) \approx 386,31\,\on{ct}\]%
entspricht, es also von dem vier \acr{HTS} ($400\,\on{ct}$) großen Intervall um
$13.69\,\on{ct}$ abweicht. Diese Unterschiede „korrigiert“ \emph{reine
  Intonation}, indem sie Akkorde systematisch
\emph{microtunt},\footnote{Deutsche Konjugation von Anglismen, immer wieder ein
  Graus.} also die Töne je nach harmonischem Kontext ein kleines bisschen höher
oder tiefer spielt. Das ist natürlich auf Musikinstrumenten mit zwölf Tasten pro
Oktave unmöglich; dort müssen andere Stimmungssysteme eingesetzt werden. Nur an
der Theorie und nicht an praktischen Lösungsansätzen interessiert, werden wir
letztere hier nicht ausführen. Unabhängig davon \emph{ist} reine Intonation in
der musikalischen Praxis durchaus einsetzbar, z.\,B. im A-cappella-Gesang:
Professionell ausgebildete Chorsänger:innen hören aufeinander, um möglichst
konsonant zu singen, und microtunen daher ihre Akkorde
\emph{intuitiv}, siehe etwa \cite{Maria} für einige empirische Untersuchungen.

Dieses Skript gibt die erforderlichen Microtunings explizit an und zeigt, wie
sie konsistent durch ein Stück hinweg eingesetzt werden können. Dabei gliedert
es sich in folgende Abschnitte:\looseness-1

\begin{enumerate}
  \setcounter{enumi}{1}
\item Mit der Einführung der \emph{pythagoreischen Stimmung} erreichen wir,
  dass alle reinen Quinten das Frequenzverhältnis von $2:3$ haben. Dies wird
  dazu führen, dass beispielsweise das \sharp G etwas höher wird als das
  \flat A; dieser Unterschied heißt \emph{pythagoreisches Komma}. Da wir aber 
  nicht an ein Tasteninstument \todo{(oder die Realität überhaupt, wie wir ja gerade erwähnt haben) –T}
  gebunden sind, ist das kein Problem; es zwingt uns nur dazu, Enharmonik ernst
  zu nehmen. Der Quintenzirkel wird zu einer unendlichen \emph{Quintenreihe}.
  Von hier an ist unser Standardstimmungssystem das pytagoreische, das e’’ hat
  also eine Frequenz von $\frac32\cdot 440\,\text{Hz}=660\,\text{Hz}$ und nicht
  mehr $659{,}26\,\text{Hz}$.\looseness-1
\item Selbst im pythagoreischen System sind Dreiklänge nicht optimal gestimmt,
  z.\,B. wäre ein Dur-Dreiklang $64:81:96$ statt $4:5:6$. Die große Terz ist ein
  bisschen zu hoch; der Unterschied $\frac{81}{80}$ wird als \emph{syntonisches
  Komma} bezeichnet. Folglich bedeutet das Microtunen des Akkords eine
  Absenkung der großen Terz um ein syntonisches Komma. Wir erhalten eine zweite
  Quintenreihe, die um ein syntonisches Komma tiefer liegt. Wenn wir diesen
  Prozess in beide Richtungen wiederholen, können die entstehenden Reihen zum
  \emph{Eulerschen Tonnetz} zusammengesetzt werden, in dem die Bestandteile von (Dur-
  und Moll-)Dreiklängen nebeneinander liegen und Dreiecke bilden. Da
  \mbox{(gegen-)}parallele Dreiklänge benachbarten Dreiecken entsprechen, eignet sich
  dieses Gitter gut für die klassische Harmonik, erklärt aber auch die so
  genannte \emph{Kommafalle} für bestimmte Akkordfolgen: eine häufige Quelle
  für Detonationen™ (das ist der Verlust der ursprünglichen Grundfrequenz).
\item Mithilfe des Eulerschen Tonnetzes lassen sich auch andere für die
  klassische Musik übliche Akkorde (etwa die Subdominante mit sixte ajoutée)
  sinnvoll ausstimmen. Wir werden den Versuch unternehmen, dies konsistent für
  die meisten relevanten Harmonien zu tun.
\item Zwar können wir nun sämtliche \emph{vertikale} Klänge über ein Stück
  hinweg konsistent rein ausstimmen, allerdings ist dies an solchen
  Stellen praxisfern, die \emph{horizontal} verstanden und musiziert werden. Wir
  werden daher einige Situationen problematisieren, in denen sich diese beiden
  Perspektiven widersprechen.
\item Selbst innerhalb des Eulerschen Tonnetzes kann der oben beschriebene
  Vierklang $4:5:6:7$ nicht realisiert werden – im Wesentlichen weil das Tonnetz
  nur die Primzahlen $2$, $3$ und $5$ kennt. Ein Dur-Dreiklang mit kleiner
  Septime im Eulerschen Tonnetz hat stattdessen das Verhältnis $36:45:54:64$.
  Die kleine Septime ist also ein bisschen zu hoch; der Fehler $\frac{64}{63}$
  wird als \emph{septimales Komma} bezeichnet. Wir werden einige (allerdings
  ausgesprochen rar gesäte) Situationen in der klassischen Vokalmusik finden,
  in denen durch die Zuhilfenahme des septimalen Kommas ein „besseres“
  Klangergebnis erreicht werden kann.
\end{enumerate}

% Dieser Ansatz unterscheidet sich in zwei Aspekten von der
% Standardliteratur:\todo{Welche? – \{Das würde mich auch interessieren.\} –T} 
% Erstens machen wir das pythagoreische System zum bevorzugten und verfolgen alle
% Abweichungen davon durch syntonische und septimale Kommata, während
% klassischerweise für jede Tonart eine sogenannte ptolemäische Skala verwendet
% wird, in der die Terz, die Sexte und die Septime standardmäßig um ein
% syntonisches Komma erniedrigt werden. Die Ptolemäische Skala hat den Vorteil,
% dass weniger Kommata verfolgt werden müssen; allerdings, werden wir darlegen,
% aus welchen systematischen Gründen wir uns gegen ihre Verwendung entschieden
% haben. Zweitens wird das septimale Komma in der Regel weggelassen,
% wahrscheinlich weil die durch den Tritonus verursachte Spannung (zwischen der
% kleinen Septime und der großen Terz) hörbar sein soll. Wir sind jedoch
% überzeugt, dass die Beachtung des septimalen Kommas zu einem saubereren System
% führt.

Bevor wir diesen Prozess beginnen, weisen wir darauf hin, dass diese Abhandlung
mit mehreren musikalischen Beispielen versehen ist, die am Seitenrand dieses
Dokuments angegeben sind\LY{m-4567} (dies ist ein erstes Beispiel, in dem ein
Dur-Akkord mit kleiner Septime zu hören ist, einmal in gleichstufiger Stimmung
und einmal genau $4:5:6:7$, wie oben beschrieben). Diese Beispiele wurden mit
dem Notensatzprogramm \verb!lilypond! geschrieben und kodieren entweder eine
\acr{PDF}-Datei (dann beginnt ihr Dateiname mit \verb!p-!), eine
\acr{MIDI}-Datei (\verb!m-!) oder beides (\verb!b-!). Auf
\href{https://github.com/fkranhold/jiam/}{\textsf{github.com/fkranhold/jiam}}
sind die Quelldateien, die Kompilate und, falls eine \acr{MIDI}-Datei unter
diesen ist, eine Audio-Konvertierung nach \acr{OGG} Vorbis zu finden. Wir
ermutigen alle, mit diesen Quelldateien selbst zu experimentieren.

\section{Reine Quinten und die pythagoreische Stimmung}
\label{sec:pyth}

Wie bereits in \cref{sec:int} erwähnt, wollen wir uns zunächst mit dem genauen
Verhältnis einer reinen Quinte ($2:3$) beschäftigen. Die Differenz zur
gleichstufigen Quinte, nämlich
\[1{,}200\,\on{ct}\cdot \log_2(\tfrac32) - 700\,\on{ct} \approx 1.96\,\on{ct}\]%
wird manchmal auch \emph{Grad} genannt\todo{Man müsste eigentlich irgendwo mal sagen, dass das Grad $\frac{1}{12}$ des Pyth. Kommas ist. Und dass 12-EDO genau das tut: Das Pyth. Komma auf 12 Quinten aufteilen. –T}
und ist sehr klein, aber dennoch etwas das wir korrigieren wollen. Ohne die
Einschränkungen eines Instruments mit diskreten Tonhöhen kann dies leicht
erreicht werden, wenn wir akzeptieren, dass enharmonisch unterschiedene
Tonhöhen verschiedenen Frequenzen entsprechen können.

Dies mag auf den ersten Blick seltsam erscheinen, aber man kann sich davon
überzeugen, indem man eine übermäßige Sekunde (in einem angemessenen
harmonischen Kontext) mit einer kleinen Terz vergleicht und versucht, sie zu
singen: Wir sehen keinen Grund, warum sie genau das gleiche Frequenzverhältnis
haben sollten. Ein weiteres eindrucksvolles Beispiel ist in
\cref{fig:enhContext} zu sehen (das Audiobeispiel ist in gleichstufiger
Stimmung): In beiden Fällen ist das resultierende Intervall acht \acr{HTS} groß
(\flat e’\,-\,h’ und \sharp d’\,-\,h’), aber im ersten Fall klingt es
dissonant, da der Kontext nahelegt, dass es aus vier Ganztönen besteht, während
es im zweiten Fall konsonant klingt.

\begin{figure}[h]
  \centering
  \LY{b-enhContext}
  \includegraphics{ly/b-enhContext.pdf}
  \caption{Eine übermäßige Quinte ist keine kleine Sexte.}\label{fig:enhContext}
\end{figure}

Akzeptierend, dass Enharmonik eine Rolle spielt, machen wir uns die Tatsache
zunutze, dass es für jede enharmonisch verschiedene Tonhöhe (c’, \sharp h und
\dflat d’ sind drei unterschiedliche Beispiele!) eindeutig bestimmte ganze
Zahlen $p,q\in\Z$ gibt, so dass die Tonhöhe von a’ aus erreicht werden kann,
indem man $p$ reine Quinten und $q$ Oktaven nach oben geht (eine negative Zahl
bedeutet, dass man nach unten geht). Setzen wir den Kammerton auf $440\Hz$, so
ordnen wir der Tonhöhe die Frequenz $(\frac32)^p\cdot 2^q\cdot 440\Hz$ zu. Ein
Beispiel:\looseness-1
\begin{align*}
  \text{\sharp g’}&\equiv(\tfrac32)^5\cdot 2^{-3}\cdot 440\Hz\approx 417.66\Hz\\
  \text{\flat a’} &\equiv(\tfrac32)^{-7}\cdot 2^4\cdot 440\Hz\approx 412.03\Hz
\end{align*}
Der Unterschied zwischen diesen beiden Frequenzen heißt
\emph{pythagoreisches Komma} und ist
$3^{12}\cdot 2^{-19}\approx 23.46\,\on{ct}$. Eine verminderte Sexte
unterscheidet sich von einer reinen Quinte notwendigerweise um genau dieses
pythagoreische Komma. Eine verminderte Sexte wird in der pythagoreischen
Stimmung manchmal als „Wolfsintervall“ bezeichnet, da sie sehr verstimmt
klingt, siehe \cref{fig:pythComma}. In \cref{fig:spiral5}, ist eine
enharmonische Variante des Quintenzirkels zu sehen, die wir als
\emph{Quintenspirale} bezeichnen können. Mit jeder vollen Runde in dieser
Spirale, verschiebt sich die Tonhöhe um ein pythagoreisches Komma.\todo{Hier ein Wort zu Skalen?}

\begin{figure}[h]
  \centering
  \includegraphics{ly/b-pythComma}
  \LY{b-pythComma}
  \caption{(1) Der Unterschied zwischen \sharp g’ and \flat a’. (2) Das
    Wolfsintervall \sharp g’\,–\,\flat e’, gefolgt von der reinen Quinte \sharp
    g’\,–\,\sharp d’.}\label{fig:pythComma}
\end{figure}

Für jedes Intervall können wir das entsprechende Verhältnis berechnen, siehe
\cref{tab:1}. Man beachte, dass, wenn ein bestimmtes Intervall das Verhältnis
$a:b$ hat, sein \emph{Komplementärintervall} das Verhältnis $b:2a$ hat, z.\,B.\
eine kleine Sexte entspricht dem Verhältnis $81:128$. Ähnlich verhält es sich,
wenn ein Intervall die Größe $c$ in Cent hat: Dann hat sein
Komplementärintervall die Größe $1{,}200\,\on{ct}-c$, z.\,B. eine kleine Sexte
entspricht $792.18\,\on{ct}$.

\begin{figure}
  \centering%
  \newcommand{\pitcharray}{%
  \dflat d, \dflat a, \dflat e, \dflat h, \flat f,
  \flat c, \flat g, \flat d, \flat a, \flat e,
  \flat h, f, c, g, d, a, e, h,
  \sharp f, \sharp c, \sharp g, \sharp d, \sharp a, \sharp e, \sharp h}
\begin{tikzpicture}[xscale=-1]
  \def\xA{1.4}
  \def\xB{.1}
  \draw[dotted,domain=.35*pi:.4*pi,smooth,samples=100,variable=\t]
  plot ({(\xA+\xB*\t)*cos(\t r)},{(\xA+\xB*\t)*sin(\t r)});
  \draw[domain=.4*pi:4.6*pi,smooth,samples=100,variable=\t]
  plot ({(\xA+\xB*\t)*cos(\t r)},{(\xA+\xB*\t)*sin(\t r)});
  \draw[dotted,domain=4.6*pi:4.63*pi,smooth,samples=100,variable=\t]
  plot ({(\xA+\xB*\t)*cos(\t r)},{(\xA+\xB*\t)*sin(\t r)});
  \foreach \i in {0,...,24}{
    \node[fill=white,inner sep=1.5px]
    at ({(\xA+\xB*(\i+3)*pi/6)*cos((\i+3)*pi/6 r)},
    {(\xA+\xB*(\i+3)*pi/6)*sin((\i+3)*pi/6 r)})
    {\footnotesize \pitch[\i]$\strut$};
  };
\end{tikzpicture}

  \caption{Die enharmonische Quintenspirale. Mit jeder vollen Rotation im
  	Uhrzeigersinn, gewinnen wir ein pythagoreisches Komma.}\label{fig:spiral5}
\end{figure}

\begin{table}
  \centering
  \begin{tabular}{lr@{\hspace*{2.4px}}lr}
    \toprule
    \textsc{Intervall} & \multicolumn{2}{c}{\textsc{Verhältnis}} & \textsc{Größe in ct}\\
    \midrule
    kleine Sekunde  & $243$ & $:256$ &  $90.22$\\
    große Sekunde   & $8$   & $:9$   & $203.91$\\
    kleine Terz     & $27$  & $:32$  & $294.13$\\
    große Terz      & $64$  & $:81$  & $407.82$\\
    reine Quarte    & $3$   & $:4$   & $498.04$\\
    \bottomrule
  \end{tabular}
  \caption{Ausgewählte Intervalle in pythagoreischer Stimmung.}\label{tab:1}
\end{table}

Von nun an interpretieren wir geschriebene Noten in pythagoreischer
Stimmung und bezeichnen alle Abweichungen davon mit bestimmten Symbolen, die
wir einführen, wenn sie notwendig werden. Man beachte jedoch, dass der
Unterschied zu der gleichstufigen Stimmung eher gering ist.\todo{(Außer man schreibt verrückte Sachen wie \dflat h und \dsharp g. Die unterscheiden sich in 12-EDO nicht, aber in pyth. Stimmung um $46.9\,\on{ct}$.) –T}
Während alle noch kommenden Mikroalterationen (die darauf abzielen, ganze
Akkorde zu stimmen) den Preis haben werden, dass es mehrere Frequenzen für
denselben Tonnamen gibt, verwendet die bare pythagoreische Stimmung nur die
Informationen, die uns durch die geschriebenen Noten gegeben werden. Sie verhält
sich aus \emph{melodischer} Perspektive also gut™, da die Größe der 
(horizontalen) Intervalle nicht von der vertikalen Struktur abhängt.\todo{Also ein Beweis ist das nicht, oder? Falsche Schlussrichtung oder so … Einen Grund dafür, dass pythagoreische Stimmung für sinnvolle Melodien sorgt, habe ich in der Rotation des Mikrotöne-Kurses auf der zweiten Hälfte vorgeführt bekommen: In pythagoreischer Stimmung gibt es \emph{diatonische} Skalen, also Skalen die nur zwei unterschiedliche Schrittweiten haben (i.e. Halbton und Ganzton). In der Rotation hat Georg so ein 22-EDO Beispiel vorgespielt mit $\acr{GT}:\acr{HT} = 4:1$. Das Beispiel produziert grauenhafte Dreiklänge, aber eine Tonleiter mit LLSLLLS hörte sich relativ melodisch an und war eindeutig Dur. –T}

Der einzige Nachteil der pythagoreischen Stimmung, der nichts mit der
Realisierbarkeit auf einem diskretisierenden Instrument zu tun hat, sondern rein
theoretisch ist, ist die Tatsache, dass sie die Unreinheit der Terzen in einem
Dreiklang im Vergleich zur gleichstufigen Stimmung sogar noch erhöht: Der
Unterschied zwischen einer pythagoreischen großen Terz ($407.82\,\on{ct}$) und
dem oben erwähnten Verhältnis $4:5$ ($386.31\,\on{ct}$) beträgt
$21.51\,\on{ct}$. Dieser Fehler ist Gegenstand des nächsten Abschnitts.

% We close this section by considering a three harmonic situations, which could
% appear in western classical music and which require enharmonic sensitivity, see
% \cref{fig:reint}: In the first case, we have a dominant seventh in \flat E major,
% leading to the tonic, while in the second case, we see the German augmented
% sixth chord (which might be seen as a variation of a dominant chord based on E),
% leading to A major. The third example shows how a diminished seventh chord can
% be enharmonically reinterpreted: It is introduced as the dominant of A minor,
% but then reinterpreted as the dominant of C minor, leading to the new tonic. We
% play these chords in Pythagorean tuning.

% \begin{figure}[h]
%   \centering%
%   \includegraphics{ly/b-reint.pdf}%
%   \LY{b-reint}%
%   \caption{Three situations showing the different rôles of \sharp G and \flat
%     A.}\label{fig:reint}
% \end{figure}

\section{Dreiklänge und das Eulersche Tonnetz}
\label{sec:tri}

\subsection{Das syntonische Komma}

Wir beginnen mit der Beobachtung, dass in der pythagoreischen Stimmung ein
Dur-Dreiklang dem Verhältnis $64:81:96$ statt $4:5:6=64:80:96$ entspricht,
d.\,h. die große Terz etwas zu hoch ist (im Hörbeispiel hören wir zuerst einen
pythagoreischen Dur-Dreiklang und dann $4:5:6$).\LY{m-pyth456} Der Unterschied,
$\frac{81}{80}\approx 21.51\,\on{ct}$, heißt das
\emph{syntonische Komma}.\todo{Hier wird einigen Leuten auffallen (naja, vllt. auch nicht, aber egal), dass das syntonische Komma und das pythagoreische fast gleich groß sind und sich nur um das \emph{Schisma} $\approx 1.95 \,\on{ct}$ unterscheiden (insbesondere ist d – \flat g – a ein fast reiner Durdreiklang). Da könnte man nun denken, dass das Schisma genau das \emph{Grad} ist, aber das stimmt nicht. Sie unterscheiden sich um $\approx 0.00128 \,\on{ct}$. Das heißt, dass das gleichstufige e’’ (a’ + Quinte - Grad) und das \flatp f’’ (a’ + Quinte - Schisma) eine Schwebungsperiode von ~34 Minuten haben. –T}
In anderen Worten: Die große pythagoreische Terz $64:81$ ist um ein syntonisches
Komma größer als $4:5$, und die kleine pythagoreische Terz $27:32$ ist ein um
ein syntonisches Komma kleiner als $5:6$, denn $5:6$ ist das Komplement von
$4:5$ in $2:3$ und damit die gewünschte kleine Terz. Von nun an nennen wir das
Verhältnis $4:5$ eine \emph{reine} große Terz und $5:6$ eine \emph{reine} kleine
Terz. Folglich nennen wir ihre zur Oktave komplementären Intervalle $5:8$ und
$3:5$ die \emph{reine} kleine bzw. große Sexte.\looseness-1

Wenn man versucht, Tonhöhen um ein syntonisches Komma zu alterieren, um die
Stimmung zu verbessern, stößt man auf das Problem, dass jeder Ton prinzipiell
die große Terz, aber auch der Grundton eines Dreiklangs sein kann. Das bedeutet:
Es hängt vom harmonischen Kontext ab, ob eine Tonhöhe alteriert werden muss
oder nicht, und das lässt sich leider nicht aus dem bloßen (Enharmonik
beachtenden) Tonnamen herauslesen.\looseness-1

Bei einer Partitur fügen wir die zusätzliche Information, dass bestimmte Noten
um ein oder mehrere syntonische Kommata abgesenkt oder angehoben werden müssen,
durch die Einführung weiterer Vorzeichen in der Tradition von Helmholtz und
Ellis \cite{HE} hinzu: Die Vorzeichen \dsharpp, \sharpp, \naturalp, \flatp, und
\dflatp bedeuten die Erhöhung des Tons um ein syntonisches Komma; genauso
bedeuten \dsharppp, \sharppp, \naturalpp, \flatpp, und \dflatpp zwei Kommata
und so weiter. In die andere Richtung benutzen wir \dsharpm, \sharpm, \naturalm,
\flatm, und \dflatm für Noten die um ein syntonisches Komma tiefer erklingen
sollen. Ein Beispiel ist in \cref{fig:HE} zu sehen. Wie üblich gelten Vorzeichen
für einen ganzen Takt; wenn also ein einzelner Takt erst \naturalm a' und dann
\natural a' enthält, sind beide Vorzeichen verpflichtend. Aus Gründen, die im
nächsten Unterabschnitt erläutert werden, verwenden wir überhaupt keine
Generalvorzeichen.\looseness-1

\begin{figure}\centering
  \includegraphics{ly/p-HE}
  \LY{p-HE}
  \caption{Im ersten Takt ist das \sharp f’ um ein syntonisches Komma 
    tiefalteriert, sodass wir einen Dur-Dreiklang $4:5:6$ erhalten. Im zweiten
    Takt wird das g’ um ein, das \flat e’’ um zwei syntonische Kommata hoch
    alteriert, sodass sich eine reine kleine Sexte $5:8$ ergibt.}\label{fig:HE}
\end{figure}

\subsection{Das Eulersche Tonnetz}

Versuchen wir nun, syntonische Kommata konsequent in einem ganzen Musikstück
einzusetzen: Ein klassischer Ansatz besteht darin, die Dur-Tonleiter zu
verändern indem die Töne, die am ehesten als große Terzen eines Dreiklangs
auftreten, nämlich Terz, Sexte und Septime, um ein syntonisches Komma erniedrigt
werden. Das Ergebnis nennt man die \emph{ptolemäische Skala}. Legt man die
Frequenz des Grundtons der Skala als $1$ fest, lassen sich die Frequenzen aller
in der Skala vorkommenden Noten als 
$1,\frac98,\frac54,\frac43,\frac32,\frac53,\frac{15}8,2$ berechnen. Wir haben
uns aus den folgenden Gründen gegen die Verwendung dieser Skala entschieden
(wir vermeiden das Wort „Skala“ sogar ganz):
\begin{enumerate}
\item In der ptolemäischen Skala beträgt die Quinte zwischen dem zweiten und dem
  sechsten Ton nicht $2:3$, sondern $27:40$. Um die Subdominantparallele
  (z.\,B. d-Moll in C-Dur) zu spielen, müssen wir wieder eine der Noten um ein
  syntonisches Komma ändern.\todo{Ich würde sagen, dass hier Wichtig ist, dass die Sp ähnlich häufig ist wie die D. Wenn das nicht so wäre könnte man einfach sagen: „Naja, dafür sind halt Versetzungszeichen da. Denn das \flat E ist in F-Dur halt seltener als das E. Genauso ist das \natural E seltener als \naturalm E, weshalb es sinnvoll ist \naturalm E per default zu haben.“ Das Problem ist halt, dass das Argument bei dem zweiten Ton der Durskala nicht funktioniert weil zwei unterschiedliche Syntonisierungen ähnlich häufig sind. –T}
\item Die ptolemäische Skala weist nur einigen Tonhöhen eine Frequenz zu. Es 
  gibt einige Tonhöhen, die mit einer Tonleiter „eng verwandt“ sind, auch wenn
  sie technisch gesehen nicht zu ihr gehören (z.\,B. das \sharp F in C-Dur, da
  es™ die große Terz der Doppeldominante ist).
\item Unser Stimmungssystem sollte nicht von der Tonart des Stücks, sondern nur
  vom Kammerton abhängen.\todo{Ich bin mir nicht sicher ob ich diesen Punkt verstehe. Das Stimmungssystem ändert sich ja durch die Vorzeichen nicht, oder? Ich bin verwirrt. –T}
\end{enumerate}
Stattdessen wollen wir alle möglichen Frequenzen eines Stücks in einer Weise
anordnen, die ihre harmonische Nähe zueinander widerspiegelt. Dies wird
typischerweise durch das \emph{Eulersche Tonnetz} erreicht, siehe
\cref{fig:latticeExcerpt}: Wir beginnen mit der Quintenspirale aus
\cref{sec:pyth}, die wir jetzt als horizontale Gerade schreiben, und unterteilen
jeden Schritt ($2:3$) in eine reine große Terz ($4:5$) und eine reine kleine
Terz ($5:6$). Dieser Zwischenton wird ersten und dem zweiten aber etwas weiter
unten notiert. Wenn wir alle Schritte unterteilen, erhalten wir eine
vollständige neue Zeile, die wieder aus reinen Quinten besteht, aber jetzt sind
die Frequenzen in dieser Zeile um ein syntonisches Komma erniedrigt. Dasselbe
lässt sich auch in der anderen Richtung machen, indem man die Reihenfolge von
kleiner und großer Terz vertauscht und die Zwischentöne etwas höher als die
Ursprungsgerade schreibt. Wir erhalten eine neue Quintenreihe, in der alle
Frequenzen um ein syntonisches Komma erhöht sind. Wenn wir diesen Prozess in
beide Richtungen wiederholen, erhalten wir ein Tonhöhengitter (siehe
\cref{fig:latticeExcerpt}), bei dem eine Bewegung der Form ‚$\to$‘ einer reinen
Quinte, ‚$\searrow$‘ einer reinen großen Terz und ‚$\nearrow$’ einer reinen
kleinen Terz entspricht. Jede (Enharmonik beachtende) Tonklasse (d.\,h. ihre
Oktave ignorierend) kommt in jeder Zeile genau einmal vor, so dass ein Knoten im
Gitter einer Tonklasse zusammen mit einer Anzahl von darauf angewandten
syntonischen Kommata entspricht.

\begin{figure}[h]
  \[
  \begin{tikzcd}[column sep=-1em,labels=description]
    & \text{\flatp D}\,(\frac{16}{15})\ar[rr,dash]\ar[dr,dash] && \text{\flatp A}\,(\frac45)\ar[rr,dash]\ar[dr,dash] && \text{\flatp E}\,(\frac{6}{5})\ar[dr,dash]\ar[rr,dash] && \text{\flatp B}\,(\frac95)\ar[dr,dash]\ar[rr,dash] && \text{\naturalp F}\,(\frac{27}{20})\ar[dr,dash]\\
    \text{\flat B}\,(\frac{16}9)\ar[rr,dash]\ar[dr,dash]\ar[ur,dash] && \rF\,(\frac43)\ar[ur,dash]\ar[rr,dash]\ar[dr,dash] && \rC\,(1)\ar[dash]{ur}{5:6}\ar[dash]{rr}{2:3}\ar[dash]{dr}{4:5} && \rG\,(\frac32)\ar[ur,dash]\ar[rr,dash]\ar[dr,dash] && \rD\,(\frac98)\ar[ur,dash]\ar[rr,dash]\ar[dr,dash] && \rA\,(\frac{27}{16})\\
    & \text{\naturalm D}\,(\frac{10}9)\ar[rr,dash]\ar[dr,dash]\ar[ur,dash] && \text{\naturalm A}\,(\frac53)\ar[rr,dash]\ar[dr,dash]\ar[ur,dash] && \text{\naturalm E}\,(\frac54)\ar[rr,dash]\ar[dr,dash]\ar[ur,dash] && \text{\naturalm B}\,(\frac{15}{16})\ar[rr,dash]\ar[dr,dash]\ar[ur,dash] && \text{\sharpm F}\,(\frac{45}{32})\ar[dr,dash]\ar[ur,dash]\\
    \text{\naturalmm B}\,(\frac{27}{25})\ar[rr,dash]\ar[ur,dash] && \text{\sharpmm F}\,(\frac{50}{36})\ar[rr,dash]\ar[ur,dash] && \text{\sharpmm C}\,(\frac{25}{24})\ar[rr,dash]\ar[ur,dash] && \text{\sharpmm G}\,(\frac{25}{16})\ar[rr,dash]\ar[ur,dash] && \text{\sharpmm D}\,(\frac{75}{64})\ar[rr,dash]\ar[ur,dash] && \text{\sharpmm A}\,(\frac{225}{128})
  \end{tikzcd}
\]

  \caption{Ein Ausschnitt des Eulerschen Tonnetzes. Wir berechnen
  	Frequenzfaktoren relativ zum C; oktav-normiert, sodass sie zwischen $1$
  	und $2$ liegen.}\label{fig:latticeExcerpt}
\end{figure}

In diesem Netz umfasst ein Dreieck der Form $\triangledown$ die Bestandteile 
eines rein intonierten Dur-Akkords ($4:5:6$), während ein Dreieck $\vartriangle$
die Bestandteile eines Mollakkords ($10:12:15$) erfasst. Benachbarte Dreiecke
teilen entweder eine Quinte, d.\,h. sie sind \emph{Varianten} voneinander, oder
eine Terz, d.\,h. sie stehen in \emph{paralleler}
($\vartriangle\!\!\!\triangledown$) oder \emph{gegenparalleler}
($\triangledown\!\!\!\vartriangle$) Beziehung zueinander. Legt man ein Dreieck
als Tonika eines Stücks fest, lassen sich die Standardfunktionen wie
(Sub-)Dominante sowie ihre (Gegen-)Parallelen und sogar Zwischendominanten
leicht ermitteln: Für ein gegebenes Dreieck ist seine Zwischendominante das
Dur-Dreieck, das die Quinte rechts neben dem gegebenen Dreieck enthält (z.\,B.\
ist die Doppeldominante von C-Dur der D-Dur-Dreiklang in der Mitte rechts in
\cref{fig:latticeExcerpt}, nicht der unten links).

\subsection{Mikrotuning funktionaler Harmonien}

Bei einem Musikstück (z.\,B. \cref{fig:triadPiece}) bestimmt man zunächst seine
Tonart (hier: e-Moll). Dann wird das Tonika-Dreieck im Eulerschen Tonnetz
eindeutig durch den Modus (Dur oder Moll) und die Bedingung bestimmt, dass der
Grundton (hier: E) nicht durch ein Komma verändert wird. Für jeden Ton bestimmen
wir dann, indem wir alle Stimmen betrachten, den Dreiklang, zu dem er gehört
(der vierte Ton des Tenors, a, ist Teil eines D-Dur-Dreiklangs), bestimmen die
Funktion des Dreiklangs (Zwischendominante zur Tonikaparallele) und finden das
entsprechende Dreieck im Eulerschen Tonnetz (siehe \cref{fig:chordsLattice} für
die Verortung aller in \cref{fig:triadPiece} vorkommen Dreiklänge im Eulerschen
Tonnetz). Daraus ergibt sich dann genau, wie viele Kommata wir auf die Note
anwenden müssen (hier: a muss um eins erhöht werden). Seine absolute Frequenz
beträgt $\frac{81}{80}\cdot 220\,\text{Hz}=222.75\,\text{Hz}$, und relativ zum
nächsttieferen Grundtons, e, entspricht er dem Faktor
$\frac{81}{80}\cdot\frac{4}{3}=\frac{27}{20}$.

Wir weisen darauf hin, dass bei einem Stück in C-Dur der Grundton eines
a-Moll-Dreiklangs um ein syntonisches Komma herabgesetzt wird, während dies bei
einem Stück in a-Moll eindeutig nicht der Fall ist - auch wenn das Stück
dieselben Vorzeichen hat. Daher hat bei einem Stück mit gegebenen
Generalvorzeichen die Entscheidung, welcher der beiden möglichen Dreiklänge die
Tonika ist (z.\,B. C-Dur oder a-Moll), Auswirkungen darauf, wie hoch das Stück
intoniert werden soll. Das mag seltsam klingen, ergibt sich aber, wenn wir uns
darauf einigen, dass der Grundton der Tonika als Zentrum des Stücks direkt vom
Kammerton abgeleitet werden sollte.\todo{Ein weiterer Grund: Wenn wir ein F-Dur Stück in \naturalm F schreiben, hat fast jeder Ton ein Versetzungszeichen und es wird unlesbar. Ich bin mir nicht sicher, ob ich den Grund mit dem Kammerton mag, aber wahrscheinlich denke ich zu praktisch darüber nach: Wenn ich für ein Stück in \flat H-Dur die Töne mit einer Stimmgabel angebe, dann kommt dabei eher sowas wie \flatp B-Dur heraus, denn ich werde an den $15:16$ Leitton denken und nicht das $243:256$ Limma. –T}

\begin{figure}
  \centering
  \LY{b-triadPiece}
  \includegraphics{ly/b-triadPiece}
  \caption{Ein schematisches vierstimmiges Stück in e-Moll (nur aus
  	Dreiklängen), mit eingebauten syntonischen Kommata. Es gibt zwei
  	Audioausgaben des Stücks, eine in reiner pythagoreischer Stimmung und eine,
  	die die syntonischen Kommata zur Korrektur
  	verwendet.\looseness-1}\label{fig:triadPiece}
\end{figure}

\begin{figure}
  \[
  \begin{tikzcd}[row sep=.7em,column sep=.7em]
    & \text{\naturalp c}\ar[rr,dash]\ar[ddr,dash] && \text{\naturalp g}\ar[rr,dash]\ar[ddr,dash] && \text{\naturalp d}\ar[rr,dash]\ar[ddr,dash] && \text{\naturalp a}\ar[ddr,dash] &\\
    & 7 & 10 & 1,3  & 5,9 & & 4,8 & & \\
    \text{a}\ar[rr,dash]\ar[uur,dash]\ar[ddr,dash] && \text{e}\ar[rr,dash]\ar[uur,dash]\ar[ddr,dash] && \text{h}\ar[rr,dash]\ar[uur,dash]\ar[ddr,dash] && \text{\sharp f}\ar[rr,dash]\ar[uur,dash]\ar[ddr,dash] && \text{\sharp c}\\
    &   &    & 6,13 &     & 2,12     & & 11\\
    & \text{\sharpm c}\ar[rr,dash]\ar[uur,dash] && \text{\sharpm g}\ar[rr,dash]\ar[uur,dash] &&
    \text{\sharpm d}\ar[rr,dash]\ar[uur,dash] && \text{\sharpm a}\ar[uur,dash]
  \end{tikzcd}
\]

  \caption{Die Dreiklänge die in \cref{fig:triadPiece} auftauchen, verortet im
  	Eulerschen Tonnetz.}\label{fig:chordsLattice}
\end{figure}

\subsection{Die Kommafalle}

Wenn wir einen erneuten Blick auf den Ausschnitt des Eulerschen Tonnetzes in
\cref{fig:latticeExcerpt} werfen, so finden wir den Ton D an zwei Stellen in der
Nähe von C, nämlich \naturalm $\rD\,(\frac{10}9)$ and $\rD\,(\frac98)$. Beide
erscheinen als Eckpunkte von Dreiecken, die Funktionen in C-Dur entsprechen: Der
erste als Grundton von d-Moll (Subdominantparallele, oder \textsc{ii} in der
Stufentheorie) und der zweite als Quinte von G-Dur (Dominante, oder \textsc{v}).
Stellt man sich nun ein Stück vor, das die Funktionen \textsc{ii} und \textsc{v}
direkt hintereinander enthält, und eine Stimme die den Grundton der \textsc{ii}
hat und hält, während er zur Quinte von \textsc{v} wird (siehe
\cref{fig:drift}). Dann gibt es zwei Möglichkeiten:


\begin{enumerate}[itemsep=0em]
\item Einer der Akkorde wird um ein syntonisches Komma transponiert.
\item Die haltende Stimme muss die Tonhöhe beim Harmoniewechsel 
  \emph{nachjustieren}.
\end{enumerate}%
%
In der Vokalmusik geschieht typischerweise das Erste, weil die übliche
Interpretation des Haltens einer Note darin besteht, sie—offensichtlich—nicht zu
verändern, und die übrigen Stimmen, die sich bewegen müssen, ihre richtige
Intonation im Verhältnis zum gehaltenen Ton suchen. Die Folge ist, dass der
zweite Akkord (\textsc{v}) um ein syntonisches Komma absinkt und das Ensemble
an Intonation verliert, d.\,h. es detoniert, siehe den ersten Teil von
\cref{fig:drift}. Dieser Effekt wird als \emph{Kommafalle} bezeichnet. Wenn
dieses Phänomen fünfmal auftritt, beträgt die Detonation
$5\cdot 21.51\,\on{ct}=107.55\,\on{ct}$, d.\,h. das Ensemble ist um mehr als
einen gleichstufigen Halbton gesackt.

Die wünschenswertere Lösung, die jedoch von den Singenden ein Verständnis der
harmonischen Situation verlangt, ist die zweite, die im zweiten Teil von
\cref{fig:drift} dargestellt ist. Dies ist ein eindrucksvolles Beispiel, das
zeigt, dass das theoretische Bewusstsein der reinen Intonation (das, wie wir
behauptet haben, Chorsänger:innen eher intuitiv nutzen) dabei helfen kann, die
Ursache für ein sehr praktisches und häufiges Problem aufzuspüren, nämlich das
des Intonationsverlusts.

Die Akkordfolge \textsc{ii}\kern1px–\kern1px \textsc{v} ist nicht die einzige,
die Kommaabweichungen verursachen kann. Der dritte Teil von \cref{fig:drift}
zeigt eine modulierende Akkordfolge, die die Doppeldominante einbezieht. Hier
müssen sogar zwei Stimmen, die eine Quinte voneinander entfernt sind, ihre
Intonation anpassen. Die gleichzeitigen Anpassungen ergeben zusammen eine
„Bewegung“ paralleler Quinten; diese verstößt aber eindeutig nicht gegen die
Regeln des klassischen Kontrapunkts.\todo{(Für jemanden wie mich, der sich mit klassischem Kontrapunkt nicht so gut auskennt, wäre es \emph{vielleicht} hilfreich, es trotzdem zu erklären). –T}

Wir weisen darauf hin, dass die „richtige“ Intonation von der Interpretation der
Funktion abhängt, die der Dreiklang darstellt. Wenn man zum Beispiel im ersten
Teil von \cref{fig:triadPiece} den d-Moll-Akkord als Variante der
Doppeldominante interpretiert (was absurd wäre,\todo{Ja, das wäre absurd. Können wir ein Beispiel konstruieren, bei dem es \emph{wirklich} zur Debatte steht, welcher syntonischen Ebene ein Akkord zuzuordnen ist? –T}
aber nehmen wir es nur der Argumentation halber an), dann müsste der Akkord ein
syntonisches Komma höher gespielt werden, und die Kommaanpassungen gelten für
den Akkordwechsel \textsc{vi}\kern1px–\kern1px\textsc{ii} einen Schlag früher.

\begin{figure}
  \centering
  \LY{b-drift}
  \includegraphics{ly/b-drift}
  \caption{(1) Eine Akkordfolge, bei der alle wiederholten Tonhöhen auf die   
  	gleiche Weise intoniert werden, wodurch eine Detonation um ein syntonisches
  	Komma entsteht.%
  	\quad(2) Dieselbe Akkordfolge mit der notwendigen Anpassung (durch eine
  	Zick-Zack-Linie gekennzeichnet): Zwischen dem dritten und dem vierten Akkord
  	muss der Sopran das d’’ anpassen.%
  	\quad(3) Eine ähnliche Situation, in der sogar zwei Stimmen eine Anpassung
  	vornehmen müssen, was eine Komma-Bewegung in „parallelen Quinten“
  	verursacht.}\label{fig:drift}
\end{figure}

\subsection{Ein Wort zu Primzahlen}

…\todo{Oh! Was soll denn hier hin kommen? Eine Erklärung was $p$-limit harmony heißt und bedeutet? –T}

\section{Erweiterte Harmonien}
\label{sec:quad}

In \cref{sec:tri} haben wir gelernt, wie man Stücke stimmt, bei denen jede Note
Teil eines Dur- oder Molldreiklangs ist. Akkorde haben jedoch sehr oft andere
Bestandteile als nur Grundton, Terz und Quinte. In diesem Abschnitt besprechen
wir die wichtigsten Variationen von Dur- und Moll-Akkorden und wie sie gestimmt
werden können.

\subsection{Quartvorhalt and hinzugefügte None}
\label{sec:49}

Der Akkord mit Quartvorhalt besteht aus dem Grundton, der Quarte und der Quinte,
wobei sich die Quarte in der Regel auf dem nächsten Schlag in die (große oder
kleine) Terz auflöst. Der Quartvorhalt braucht keine Korrektur durch ein
syntonisches Komma relativ zum Grundton; der Akkord entspricht $6:8:9$ in der
pythagoreischen Stimmung.\todo{Man beachte, dass die Quarte in der pythagoreischen Stimmung ($3:4$) \emph{rein} ist, außerdem bedarf es keiner Korrektur, wenn die vorenthaltene Quarte aus der Subdominante  entsteht, wo sie der Grundton war. –T}
Die Auflösung der Quarte in die große Terz ist ein Halbton im Verhältnis
$15:16\approx 111.73\,\on{ct}$,\todo{Genau der Schritt den wir auch bei der Auflösung des Leittons haben; auf \cref{sec:ln} verweisen? –T}
die Auflösung der Quarte in die kleine Terz ist ein Ganzton im Verhältnis
$9:10\approx 182.40\,\on{ct}$.\todo{Überbindung ohne korrektur.}


Der \emph{erweiterte Nonakkord}\todo{Ich habe den jetzt manchmal mit „erweiterter Nonakkord“, manchmal mit „[Akkord mit] hinzugefügte[r] None“ und manchmal nur mit „Nonakkord“ übersetzt, weil ich tatsächlich nicht weiß wie der im Deutschen heißt. Wenn irgendwas davon falsch ist, kann ich das noch mal korrigieren. –T}
besteht aus einem (Dur- oder Moll-)Akkord und einer zusätzlichen großen None.
Die None liegt eine reine Quinte über der Quinte des Dreiklangs und hat daher
die gleiche Anzahl an Kommata wie der Grundton und die Quinte. Dieser Akkord
entspricht dem Verhältnis $4:5:6:9$ in Dur und $20:24:30:45$ in Moll. Wir
stellen fest, dass der Halbton zwischen der großen None und der kleinen Terz
(plus eine Oktave) wieder $15:16$ ist.

In \cref{fig:chordLines} haben wir die obigen Akkorde als Unterkomplexe des
Eulerschen Tonnetzes dargestellt. Wir markieren nicht nur die Eckpunkte, sondern
auch alle direkten Kanten zwischen ihnen, da jede Kante für eine Beziehung der
Form $2:3$, $4:5$ oder $5:6$ innerhalb des jeweiligen Akkordes steht.

\subsection{Sixte ajoutée}

Sehr oft kommt der Subdominantakkord (sowohl in Dur als auch in Moll) mit der
großen Sexte über dem Grundton des Akkords zusammen vor
(z.\,B. $\rF:\rA:\rC:\rD$ in C-Dur oder $\rF:\text{\flat A}:\rC:\rD$ in c-Moll).
Im Fall von Dur ist dieser Akkord eine Verschmelzung von zwei Dreiklängen, die
sich eine Terz teilen (im Beispiel: F-Dur und d-Moll). Dementsprechend verwenden
wir die Ecken der benachbarten Dreiecke $\vartriangle\!\!\!\triangledown$, um
die Stimmung der Einzeltöne abzuleiten (siehe \cref{fig:chordLines}). Wir
erhalten das Verhältnis $12:15:18:20$. Wenn auf den Sixte ajoutée die Dominante
folgt, muss die große Sexte beim Harmoniewechsel angepasst werden (siehe
\cref{fig:496}).\todo{Das andere aber wichtiger.}

In Moll ist die Situation komplizierter, da der Sixte ajoutée keine
Verschmelzung wie oben ist. Wir argumentieren, warum wir die große Sexte
verwenden sollten die auf der \emph{gleichen} Kommastufe wie der Grundton und
die Quinte und nicht ein Komma tiefer:

\begin{itemize}
\item \naturalm $6$ ist nicht in einem Dreiklang in der harmonischen Nähe der 
  Tonika enthalten; dies war beim Dur-Fall\footnote{Höhö Dur-„Kadenz“} anders.
\item Die \natural $6$ der Subdominante ist genau die \natural $5$ der 
  Dominante, also ist sogar\todo{Warum „sogar“? Habe ich irgendwas verpasst? –T}
  wenn auf den Sixte ajoutée die Dominante folgt, bei der eine Stimme den
  gemeinsamen Ton hält, keine Anpassung nötig.
\item The Tritonus zwischen \flatp $3$ und \naturalm $6$ ist sehr klein, nämlich
  $18:25\approx 568.71\,\on{ct}$, also $31.29\,\on{ct}$ kleiner als der
  gleichstufige.\todo{Warum ist das ein Argument? Gleichstufige Stimmung war vorher noch nie ein Argument. Ist das evtl. weil der Tritonus anders als andere Intervalle dissonant sein \emph{soll} und deshalb Nähe zu $1:\sqrt{2}$ erwünscht ist? –T}
\item Der Halbton zwischen der \flatp $3$ der Tonika und der \naturalm $6$
  der Subdominante ist $25:27\approx 133.28\,\on{ct}$, was viel zu groß ist. In
  dem dritten Teil von \cref{fig:496}, taucht dieser große Halbton als
  horizontales Intervall im Tenor auf.
\end{itemize}
Der sich daraus ergebende Akkord hat das Verhältnis $80:96:120:135$, und der
Tritonus zwischen der kleinen Terz und der großen Sexte beträgt
$32:45\approx 590.22\,\on{ct}$, was der gleichstufigen Stimmung
($600\,\on{ct}$) viel näher kommt als $18:25$.\looseness-1

\begin{figure}
  \centering
  \LY{b-496}
  \includegraphics{ly/b-496}
  \caption{Vier Beispiele, darunter der Quartvorhalt, der Nonakkord und der
  	Sixte ajoutée. Der Unterschied zwischen dem zweiten und dem dritten Beispiel
  	ist nur eine einzige Note im Tenor: In der ersten Version wird die von uns
  	vorgeschlagene große Sexte verwendet, in der zweiten Version die ein
  	syntonisches Komma tiefere, die zu dem horizontalen Intervall $25:27$
  	führt.}\label{fig:496}
\end{figure}

\begin{figure}
  \centering
  \begin{tikzpicture}[scale=.7]
  %% Suspended fourth major
  %  Euler lattice
  \draw[black!50] (0, 0) -- (4, 0);
  \draw[black!50] (1, {sqrt(3)}) -- (3, {sqrt(3)});
  \draw[black!50] (1,-{sqrt(3)}) -- (3,-{sqrt(3)});
  \draw[black!50] (0,0) -- (1, {sqrt(3)}) -- (2,0) -- (3, {sqrt(3)}) -- (4,0);
  \draw[black!50] (0,0) -- (1,-{sqrt(3)}) -- (2,0) -- (3,-{sqrt(3)}) -- (4,0);
  %  Just third and fifth relations
  \draw[black,line width=1mm] (0,0) -- (4,0);
  %  Circles for chord notes labels
  \draw[fill=white] (0,0)          circle (.3);
  \draw[fill=white] (2,0)          circle (.3);
  \draw[fill=white] (4,0)          circle (.3);
  \draw[fill=white] (3,-{sqrt(3)}) circle (.3);
  %  Chord notes labels
  \node at (0,0)              {\scriptsize $4$};
  \node at (2,0)              {\scriptsize $1$};
  \node at (4,0)              {\scriptsize $5$};
  \node at (3,-{sqrt(3)})     {\scriptsize $3$};
  %  Semitone resolutions
  \draw[white,line width=1mm] (.3,{-sqrt(3)*1/10}) -- (2.7,{-sqrt(3)*9/10});
  \draw[-stealth]             (.3,{-sqrt(3)*1/10}) -- (2.7,{-sqrt(3)*9/10});
  %  Label of chord
  \node[rotate=0] at (-1.5,0) {\small $\strut$Dur}; % For rotated text set rotate option to 90 instead of 0
  \node at (2,2.5)            {\small $\strut$Quartvorhalt};
  
  
  %% Extended ninth major
  %  Euler lattice
  \draw[shift={(5.5,0)},black!50] (0, 0) -- (4, 0);
  \draw[shift={(5.5,0)},black!50] (1, {sqrt(3)}) -- (3, {sqrt(3)});
  \draw[shift={(5.5,0)},black!50] (1,-{sqrt(3)}) -- (3,-{sqrt(3)});
  \draw[shift={(5.5,0)},black!50] (0,0) -- (1, {sqrt(3)}) -- (2,0) -- (3, {sqrt(3)}) -- (4,0);
  \draw[shift={(5.5,0)},black!50] (0,0) -- (1,-{sqrt(3)}) -- (2,0) -- (3,-{sqrt(3)}) -- (4,0);
  %  Just third and fifth relations
  \draw[shift={(5.5,0)},line width=1mm] (0,0) -- (4,0);
  \draw[shift={(5.5,0)},line width=1mm] (0,0) -- (1,-{sqrt(3)}) -- (2,0);
  %  Circles for chord notes labels
  \draw[shift={(5.5,0)},fill=white] (0,0)          circle (.3);
  \draw[shift={(5.5,0)},fill=white] (2,0)          circle (.3);
  \draw[shift={(5.5,0)},fill=white] (1,-{sqrt(3)}) circle (.3);
  \draw[shift={(5.5,0)},fill=white] (4,0)          circle (.3);
  %  Chord notes labels
  \node[shift={(3.85,0)}] at (0,0)          {\scriptsize $1$};
  \node[shift={(3.85,0)}] at (1,-{sqrt(3)}) {\scriptsize $3$};
  \node[shift={(3.85,0)}] at (2,0)          {\scriptsize $5$};
  \node[shift={(3.85,0)}] at (4,0)          {\scriptsize $9$};
  %  Label of chord
  \node[shift={(3.85,0)}] at (2,2.5)        {\small $\strut$Hinzugefügte None};
  
  
  %% Sixte ajoutée major
  %  Euler lattice
  \draw[shift={(11,0)},black!50] (0, 0) -- (4, 0);
  \draw[shift={(11,0)},black!50] (1, {sqrt(3)}) -- (3, {sqrt(3)});
  \draw[shift={(11,0)},black!50] (1,-{sqrt(3)}) -- (3,-{sqrt(3)});
  \draw[shift={(11,0)},black!50] (0,0) -- (1, {sqrt(3)}) -- (2,0) -- (3, {sqrt(3)}) -- (4,0);
  \draw[shift={(11,0)},black!50] (0,0) -- (1,-{sqrt(3)}) -- (2,0) -- (3,-{sqrt(3)}) -- (4,0);
  %  Just third and fifth relations
  \draw[shift={(11,0)},line width=1mm] (2,0) -- (4,0);
  \draw[shift={(11,0)},line width=1mm] (1,-{sqrt(3)}) -- (3,-{sqrt(3)});
  \draw[shift={(11,0)},line width=1mm] (1,-{sqrt(3)}) -- (2,0) -- (3,-{sqrt(3)}) -- (4,0);
  %  Circles for chord notes labels
  \draw[shift={(11,0)},fill=white] (1,-{sqrt(3)}) circle (.3);
  \draw[shift={(11,0)},fill=white] (2,0)          circle (.3);
  \draw[shift={(11,0)},fill=white] (3,-{sqrt(3)}) circle (.3);
  \draw[shift={(11,0)},fill=white] (4,0)          circle (.3);
  %  Chord notes labels
  \node[shift={(7.7,0)}] at (2,0)          {\scriptsize $1$};
  \node[shift={(7.7,0)}] at (3,-{sqrt(3)}) {\scriptsize $3$};
  \node[shift={(7.7,0)}] at (4,0)          {\scriptsize $5$};
  \node[shift={(7.7,0)}] at (1,-{sqrt(3)}) {\scriptsize $6$};
  %  Label of chord
  \node[shift={(7.7,0)}] at (2,2.5)        {\small $\strut$Sixte ajoutée};
  
  %% Suspended fourth minor
  %  Euler lattice
  \draw[shift={(0,-4.5)},black!50] (0, 0) -- (4, 0);
  \draw[shift={(0,-4.5)},black!50] (1, {sqrt(3)}) -- (3, {sqrt(3)});
  \draw[shift={(0,-4.5)},black!50] (1,-{sqrt(3)}) -- (3,-{sqrt(3)});
  \draw[shift={(0,-4.5)},black!50] (0,0) -- (1, {sqrt(3)}) -- (2,0) -- (3, {sqrt(3)}) -- (4,0);
  \draw[shift={(0,-4.5)},black!50] (0,0) -- (1,-{sqrt(3)}) -- (2,0) -- (3,-{sqrt(3)}) -- (4,0);
  %  Just third and fifth relations
  \draw[shift={(0,-4.5)},black,line width=1mm] (0,0) -- (4,0);
  %  Circles for chord notes labels
  \draw[shift={(0,-4.5)},fill=white] (0,0)          circle (.3);
  \draw[shift={(0,-4.5)},fill=white] (2,0)          circle (.3);
  \draw[shift={(0,-4.5)},fill=white] (4,0)          circle (.3);
  \draw[shift={(0,-4.5)},fill=white] (3,+{sqrt(3)}) circle (.3);
  %  Semitone resolutions
  \draw[shift={(0,-4.5)},white,line width=1mm] (.3,{+sqrt(3)*1/10}) -- (2.7,{+sqrt(3)*9/10});
  \draw[shift={(0,-4.5)},-stealth]             (.3,{+sqrt(3)*1/10}) -- (2.7,{+sqrt(3)*9/10});
  %  Chord notes labels
  \node[shift={(0,-3.15)}] at (0,0)          {\scriptsize $4$};
  \node[shift={(0,-3.15)}] at (2,0)          {\scriptsize $1$};
  \node[shift={(0,-3.15)}] at (4,0)          {\scriptsize $5$};
  \node[shift={(0,-3.15)}] at (3,+{sqrt(3)}) {\scriptsize \flat $3$};
  %  Label of chord
  \node[rotate=0]          at (-1.5,-4.5)    {\small $\strut$Moll}; % For rotated text set rotate option to 90 instead of 0
  
  
  %% Extended ninth minor
  %  Euler lattice
  \draw[shift={(5.5,-4.5)},black!50] (0, 0) -- (4,0);
  \draw[shift={(5.5,-4.5)},black!50] (1, {sqrt(3)}) -- (3, {sqrt(3)});
  \draw[shift={(5.5,-4.5)},black!50] (1,-{sqrt(3)}) -- (3,-{sqrt(3)});
  \draw[shift={(5.5,-4.5)},black!50] (0,0) -- (1, {sqrt(3)}) -- (2,0) -- (3, {sqrt(3)}) -- (4,0);
  \draw[shift={(5.5,-4.5)},black!50] (0,0) -- (1,-{sqrt(3)}) -- (2,0) -- (3,-{sqrt(3)}) -- (4,0);
  %  Just third and fifth relations
  \draw[shift={(5.5,-4.5)},line width=1mm] (0,0) -- (4,0);
  \draw[shift={(5.5,-4.5)},line width=1mm] (0,0) -- (1,+{sqrt(3)}) -- (2,0);
  %  Circles for chord notes labels
  \draw[shift={(5.5,-4.5)},fill=white] (0,0)          circle (.3);
  \draw[shift={(5.5,-4.5)},fill=white] (2,0)          circle (.3);
  \draw[shift={(5.5,-4.5)},fill=white] (1,+{sqrt(3)}) circle (.3);
  \draw[shift={(5.5,-4.5)},fill=white] (4,0)          circle (.3);
  %  Chord notes labels
  \node[shift={(3.85,-3.15)}] at (0,0)          {\scriptsize $1$};
  \node[shift={(3.85,-3.15)}] at (1,+{sqrt(3)}) {\scriptsize \flat $3$};
  \node[shift={(3.85,-3.15)}] at (2,0)          {\scriptsize $5$};
  \node[shift={(3.85,-3.15)}] at (4,0)          {\scriptsize $9$};
  
  
  %% Sixte ajoutée minor
  %  Euler lattice
  \draw[shift={(11,-4.5)},black!50] (0, 0) -- (6, 0);
  \draw[shift={(11,-4.5)},black!50] (1, {sqrt(3)}) -- (5, {sqrt(3)});
  \draw[shift={(11,-4.5)},black!50] (1,-{sqrt(3)}) -- (5,-{sqrt(3)});
  \draw[shift={(11,-4.5)},black!50] (0,0) -- (1, {sqrt(3)}) -- (2,0) -- (3, {sqrt(3)}) -- (4,0) -- (5, {sqrt(3)}) -- (6,0);
  \draw[shift={(11,-4.5)},black!50] (0,0) -- (1,-{sqrt(3)}) -- (2,0) -- (3,-{sqrt(3)}) -- (4,0) -- (5,-{sqrt(3)}) -- (6,0);
  %  Just third and fifth relations
  \draw[shift={(11,-4.5)},line width=1mm]                (2,0) -- (0,0);
  \draw[shift={(11,-4.5)},line width=1mm]                (2,0) -- (1,+{sqrt(3)}) -- (0,0);
  %  8:9 whole tone relations
  \draw[shift={(11,-4.5)},line width=1mm,densely dashed] (2,0) -- (6,0);
  %  Circles for chord notes labels
  \draw[shift={(11,-4.5)},fill=white] (6,0)          circle (.3);
  \draw[shift={(11,-4.5)},fill=white] (0,0)          circle (.3);
  \draw[shift={(11,-4.5)},fill=white] (1,+{sqrt(3)}) circle (.3);
  \draw[shift={(11,-4.5)},fill=white] (2,0)          circle (.3);
  %  Chord notes labels
  \node[shift={(7.7,-3.15)}] at (0,0)          {\scriptsize $1$};
  \node[shift={(7.7,-3.15)}] at (1,+{sqrt(3)}) {\scriptsize \flat $3$};
  \node[shift={(7.7,-3.15)}] at (2,0)          {\scriptsize $5$};
  \node[shift={(7.7,-3.15)}] at (6,0)          {\scriptsize $6$};
\end{tikzpicture}%

%%% Local Variables:
%%% mode: latex
%%% TeX-master: "../main"
%%% End:

  \caption{Quartvorhalt, Nonakkord und Sixte ajoutée in Dur
    (oben) und Moll (unten), visualisiert als Unterkomplexe des Eulerschen
    Tonnetzes.}\label{fig:chordLines}
\end{figure}

\subsection{Dominantseptakkord}

Kein Vierklang im Eulerschen Tonnetz kann das gewünschte Intervall $4:7$
enthalten (weil nur die Primzahlen $2$, $3$ und $5$ beteiligt
sind).\todo{Allerdings haben wir (insbesondere du) auch mal gesagt, dass die Naturseptime nicht immer gut klingt. Deshalb sollten wir trotzdem die $9:16$ Septime als Alternative einführen. Ich schreibe mal ein Beispiel hier: –T}

Dennoch wollen wir eine rein im Eulerschen Tonnetz beheimatete
Intonationsmöglichkeit vorstellen. Der Dominantseptakkord besteht aus der
Dominante mit einer kleinen Septime, die unabhängig vom Tongeschlecht
leitereigen ist. In der harmonischen Nähe des Dominantakkordes (in C-Dur oder
-Moll wäre das G-Dur) gibt es zwar zwei unterschiedliche kleine Septimen (im
Beispiel \natural f und \naturalp f), aber nur die auf der syntonischen Höhe
des Grundtons ist sinnvoll, denn falls die Septime beim Harmoniewechsel in
einem Quartvorhalt vorenthalten wird ist keine Anpassung nötig und die andere
kommt (als Terz der Doppeldominantvariante) in der Tonart praktisch nicht vor.
Der so entstehende Akkord hat das Frequenzverhältnis $36:45:54:64$. Der Tritonus
zwischen der großen Terz und der kleinen Septime ist
$45:64\approx 609.78\,\on{ct}$ groß. Die Auflösung der Septime der Dominante in
die große oder kleine Terz der Tonika erfolgt wie beim Quartvorhalt (siehe
\cref{sec:49}) mit einem diatonischen Halbton $15:16$, oder einem kleinen
Ganzton $9:10$.

\subsection{Verminderter Septakkord}

Auch für den Fall des verminderten Septakkordes werden wir in \cref{sec:sept}
eine Alternative entwickeln, die wir aber nicht immer einsetzen wollen.

Der verminderte Septakkord besteht aus der großen Terz, reinen Quinte, kleinen
Septime und kleinen None eines (Zwischen-)Dominantakkordes (am Beispiel C-Dur
oder -Moll: h, d, f und \flat a). Es ist also im Wesentlichen ein
Dominantseptakkord, bei dem der Grundton (= die Oktave) durch eine kleine
None ausgetauscht wird. Es ergibt also Sinn Terz, Quinte und Septime wie im
vorherigen Fall zu intonieren. Für die None kommt in der harmonischen Nähe nur
ein Kandidat in Frage, nämlich \flatp $9$. Diese None ist eine reine kleine Terz
über der Septime (so, wie auch die Terz eine reine kleine Terz unter der Quinte
ist) und bildet zur Quinte den gleichen Tritonus, den auch die Septime zur Terz
bildet, der Akkord ist also in symmetrischen Terzen geschichtet: Terz zu Quinte
$5:6$, Quinte zu Septime $27:32$ und Septime zu None $5:6$; insgesamt
$45:54:64:\frac{384}5 = 225:270:320:384$. Die None wird ebenfalls mit dem
diatonischen $15:16$ Halbton nach unten aufgelöst.

Im Moll-Fall ist dieser Akkord ähnlich dem Sixte ajoutée in Dur auch eine
Vermischung von zwei Dreiklängen (in c-Moll wäre der Akkord \naturalm h, d, f,
\flatp a). Die Terz und die Quinte sind die Terz und die Quinte der Dominante
(\naturalm h und d) und die Septime und None sind der Grundton und die Terz der
Subdominante (f und \flatp a), was unsere Wahl der syntonischen Ebenen für die
Septime und None ebenfalls bestätigt.

\subsection{Septnonakkord}

+ Vergleich Septnon mit Sixte ajoutée in Moll

\section{Septimen}
\label{sec:sept}

\subsection{Das septimale Komma}

Mit der Einführung des syntonischen Kommas können wir nicht nur Dreiklänge
(\cref{sec:tri}) sondern auch viele Vierklänge (\cref{sec:quad}) rein stimmen.
Allerdings haben wir immer noch keine Möglichkeit die bereits mehrfach erwähnte
kleine Septime mit der Größe $4:7=36:63\approx 968.83\,\on{ct}$ darzustellen.
Dieses Intervall nennen wir \emph{Naturseptime}, da es in der Naturtonfolge als
erste Septime vorkommt; es ist $31.18\,\on{ct}$ also fast ein Drittel \acr{HTS}
kleiner als die gleichstufige kleine Septime und klingt deshalb auch deutlich
anders, wie in dem Audiobeispiel nachzuhören ist\LY{m-4567} (zuerst erklingt
ein gleichstufiger Dominantseptakkord, dann einer mit dem Frequenzverhältnis
$4:5:6:7$). Die pythagoreische kleine Septime
$9:16=36:64\approx 996.09\,\on{ct}$ ist $\frac{64}{63}\approx 27.26\,\on{ct}$
größer als die Naturseptime, diesen Unterschied nennen wir das
\emph{septimale Komma}.

Eine Tiefalteration um ein septimales Komma notieren wir mit dem zusätzlichen
\septimal Vorzeichen, es wird vor andere Versetzungszeichen gesetzt, wenn sie
nötig sind. Wir werden keinen Ton um ein septimales Komma \emph{erhöhen},
deshalb müssen wir dafür auch kein Zeichen einführen. Der Dominantseptakkord in
D-Dur besteht also aus a, \sharpm c, e und \septimal g. Der erste und zweite
Teil von \cref{fig:sept} zeigt ein Beispiel für einen Dominantseptakkord ohne
und mit Naturseptime. Bei der Auflösung der Naturseptime um einen Halbton nach
unten kommt es zu einem $20:21\approx 84.47\,\on{ct}$, bei einer Auflösung um
einen Ganzton nach unten zu einem $32:35\approx 155.14\,\on{ct}$ großen Schritt.

\begin{figure}
	\centering
	\LY{b-sept}
	\includegraphics{ly/b-sept}
	\caption{(1) Eine Schlusskadenz mit Dominantseptakkord. Die Septime im
		Tenor wird pythagoreisch intoniert.
		\quad(2) Die gleiche Kadenz mit Naturseptime im Tenor.
		\quad(3) Ein Beispiel mit vermindertem Septakkord ohne Naturseptime.
		\quad(4) Das gleiche Beispiel mit Naturseptime; die kleine None wird
		auch um ein septimales Komma alteriert.}\label{fig:sept}
\end{figure}

\subsection{Verminderter Septakkord}

Wie in \cref{sec:quad} stimmen wir die None als reine kleine Terz über der
Septime, sodass der Akkord dem Frequenzverhältnis $5:6:7:\frac{42}5 = 25:30:35:42$
entspricht. Der Unterschied zwischen den beiden Varianten ist im dritten und
viertes Beispiel in \cref{fig:sept} zu sehen und zu hören. Diese Version des
Akkords ist immer noch symmetrisch geschichtet, aber die mittlere der drei
Terzen ist nicht mehr pythagoreisch ($27:32$), sondern septimal ($6:7$). In C besteht dieser Akkord beispielsweise aus \naturalm h, d, \septimal f und \septimal \flatp a.

\subsection{Dominantseptnonakkord}

\begin{table}
	\centering
	\begin{tabular}{lrrrrr}
		\toprule
		\textsc{Akkord} & \multicolumn{4}{l}{\textsc{Frequenzverhältnis}}\\
		\midrule
		Dur                         & $4$  & $5$  & $6$\\
		Moll                        & $10$ & $12$ & $15$\\
		Quartvorhalt                & $6$  & $8$  & $9$\\
		Hinzugefügte None (Dur)     & $4$  & $5$  & $6$  & $9$\\
		Hinzugefügte None (Moll)    & $20$ & $24$ & $30$ & $45$\\
		Sixte ajoutée (Dur)         & $12$ & $15$ & $18$ & $20$\\
		Sixte ajoutée (Moll)        & $80$ & $96$ & $120$ & $135$\\
		Dominantseptakkord          & $4$  & $5$  & $6$  & $7$\\
		Verminderter Septakkord     &      & $25$ & $30$ & $35$ & $42$\\
		Septnonakkord               &      & $5$  & $6$  & $7$  & $9$\\
		\bottomrule
	\end{tabular}
\end{table}

\section{Melodik gegen Harmonik}

\subsection{Schritte als horizontale Intervalle}

\subsection{Durchgangs- und Wechselnoten}

Durchgänge, Wechselnoten

\subsection{Leittöne}
\label{sec:ln}

Wir weisen darauf hin, dass die Terz im Dreiklang, die zur Dominante (oder einer
Zwischendominante) gehört, ebenfalls um ein syntonisches Komma abgesenkt wird,
was der gängigen Praxis widerspricht, Leittöne etwas höher zu intonieren, um die
Spannung zu erhöhen, siehe z.\,B. […]\todo{Literatur}. Wir sind der Überzeugung,
dass die Dominante ein eigenständiger, stabiler Dur-Dreiklang sein sollte, aber
in dieser Frage kann man anderer Meinung sein als wir. In diesem Fall schlagen
wir vor, den Leitton um ein syntonisches Komma zu erhöhen, wodurch der Dreiklang
die pythagoreische Stimmung $64:81:96$ erhält. (Im Hörbeispiel hören wir zwei
Versionen einer Kadenz, eine mit einem rein intonierten Leitton, eine mit einem
pythagoreischen Leitton). Wir halten uns jedoch an die Konvention, alle
Dreiklänge, auch solche, die eine Dominantenfunktion haben, so zu intonieren,
dass sie eine reine Terz haben.

\begin{figure}
  \centering
  \LY{b-lead}
  \includegraphics{ly/b-lead}
  \caption{Zwei Arten, eine Vollkadenz zu intonieren: Einmal mit reiner
  	Intonation aller Dur-Dreiklänge und einmal mit einem pythagoreischen
  	Leitton.}\label{fig:lead}
\end{figure}

% \appendix
% \section{Wie der LilyPond Code funktioniert}
% \label{sec:ly}
% We give them as source code, written in the audio
% programming language \texttt{csound}, for which an interpreter is freely
% available \cite{csound}. We explain in \cref{sec:ly} how the code can be
% compiled, but here is already a short explanation how one can make sense of it:
% There are two ways of prescribing specific frequencies: The first way (encoded
% by \verb!i1!) corresponds to \emph{equal temparament}: a pitch is given by a
% \emph{root frequency} (the once fixed base of a piece, e.g. $330$\,Hz for the
% Pythagorean $\text{E}_4$) and the difference to that root, expressed as a number
% of semitones. The second way (encoded by \verb!i2!)  sets frequencies by a root
% frequency and a \emph{fraction}, consisting of a numerator and a denominator,
% with which the root frequency has to be multiplied:
% \begin{quote}\small
%   %\begin{multicols}{2}
%     \verb!t  0 90!\marginnote{\texttt{ex01.sco}}\\
%     \verb!i1 0 6 330  0!\\
%     \verb!i1 1 5 .    4!\\
%     \verb!i1 2 4 .    7!\\
%     \verb!i1 3 3 .   10!\\
%    % ~\\
%     %~\\
%     \verb!i2 6 6 330  1 1!\\
%     \verb!i2 7 5 .    5 4!\\
%     \verb!i2 8 4 .    3 2!\\
%     \verb!i2 9 3 .    7 4!\\
%     \verb!e!
%   %\end{multicols}
% \end{quote}
% %Note that this code is written in two columns; it contains $10$ lines in total.
% The first line sets the tempo to $90$ beats per minute. We will often skip this
% line; then the tempo is, by default, $60$ beats per minute (the reader should
% feel free to adjust it to their need).

% Each of the following lines stands for a single note. In every line, the first
% entry after \verb!i1! or \verb!i2! gives the starting beat, while the second
% entry gives the duration, e.g. the first note starts at time $0$ and lasts $6$
% beats. In total, we hear two tetrads, with a slow arpeggio.

% The third entry sets the root frequency (once mentioned for the first note; then
% repeated, by the dot, for all following lines), and for the final entries, it
% matters whether we use \verb!i1! or \verb!i2!: In the first case, it gives the
% number of semitones we deviate from the base frequency, and in the second case,
% it gives the numerator and denominator of the fraction. Altogether, we should
% hear the major triad with minor seventh, once in equal temparament and once in
% just intonation. The \verb!e! indicates the end of the score.

% As it is generally painful to copy the source code out of the \acr{PDF} file,
% and since we sometimes only indicate the general pattern for a longer audio
% example, we have collected all necessary files on
% \href{https://fkranhold.de/jiam/}{\textsf{fkranhold.de/jiam}}. The name of the
% specific file is denoted at the page margin, as above \texttt{ex01.sco}.

\printbibliography[heading=bibintoc]

\end{document}
